\chapter[\protect\module{pygsl.chebyshev}]
{\protect\module{pygsl.chebyshev}}
\label{cha:statistics-module}

\declaremodule{standard}{pygsl.chebyshev}
\moduleauthor{Pierre Schnizer}{schnizer@users.sourceforge.net}

\begin{classdesc}{cheb_series}{}
  This base class can be instantiated by its name
\end{classdesc}
\begin{verbatim}
import pygsl.chebyshev
s=pygsl.chebyshev.cheb_series()
\end{verbatim}

\begin{methoddesc}{__init__}{n}\index{__init__}
            n ... number of coefficients        
\end{methoddesc}
\begin{methoddesc}{init}{f, a, b}\index{init}
        This function computes the Chebyshev approximation for the
        function F over the range (a,b) to the previously specified order.
        The computation of the Chebyshev approximation is an O($n^2$)
        process, and requires n function evaluations.

            f ... a gsl_function
            a ... lower limit
            b ... upper limit
        
\end{methoddesc}
\begin{methoddesc}{eval}{x}\index{eval}
        This function evaluates the Chebyshev series at a given point X.
\end{methoddesc}
\begin{methoddesc}{eval_err}{x}\index{eval_err}
         This function computes the Chebyshev series  at a given point X,
         estimating both the series RESULT and its absolute error ABSERR.
         The error estimate is made from the first neglected term in the
         series.
\end{methoddesc}
\begin{methoddesc}{eval_n}{n, x}\index{eval_n}
         This function evaluates the Chebyshev series at a given point
         x, to (at most) the given order n
\end{methoddesc}
\begin{methoddesc}{eval_n_err}{n, x}\index{eval_n_err}
        This function evaluates a Chebyshev series at a given point X,
        estimating both the series RESULT and its absolute error ABSERR,
        to (at most) the given order ORDER.  The error estimate is made
        from the first neglected term in the series.
\end{methoddesc}

\begin{methoddesc}{calc_deriv}{}\index{calc_deriv}
        This method computes the derivative of the series CS. It returns
        a new instance of the cheb_series class.
\end{methoddesc}
\begin{methoddesc}{calc_integ}{}\index{calc_integ}
        This method computes the integral of the series CS. It returns
        a new instance of the cheb_series class.
\end{methoddesc}
\begin{methoddesc}{get_a}{}\index{get_a}
        Get the lower boundary of the current representation       
\end{methoddesc}
\begin{methoddesc}{get_b}{}\index{get_b}
        Get the upper boundary of the current representation        
\end{methoddesc}
\begin{methoddesc}{get_coefficients}{}\index{get_coefficients}
        Get the chebyshev coefficients.         
\end{methoddesc}
\begin{methoddesc}{get_f}{}\index{get_f}
        Get the value f (what is it ?) The documentation does not tell anything
        about it.        
\end{methoddesc}
\begin{methoddesc}{get_order_sp}{}\index{get_order_sp}
        Get the value f (what is it ?) The documentation does not tell anything
        about it.        
\end{methoddesc}
\begin{methoddesc}{set_a}{}\index{set_a}
        Set the lower boundary of the current representation        
\end{methoddesc}
\begin{methoddesc}{set_b}{}\index{set_b}
        Set the upper boundary of the current         
\end{methoddesc}
\begin{methoddesc}{set_coefficients}{}\index{set_coefficients}
        Sets the chebyshev coefficients. 
\end{methoddesc}
\begin{methoddesc}{set_f}{f}\index{set_f}
        Set the value f (what is it ?)        
\end{methoddesc}
\begin{methoddesc}{set_order_sp}{...}\index{set_order_sp}
        Set the value f (what is it ?)        
\end{methoddesc}


\begin{funcdesc}{gsl_function}{f, params}\index{gsl_function}

    This class defines the callbacks known as gsl_function to
    gsl.

    e.g to supply the function f:
    
    def f(x, params):
        a = params[0]
        b = parmas[1]
        c = params[3]
        return a * x ** 2 + b * x + c

    to some solver, use

    function = gsl_function(f, params)
    
\end{funcdesc}

%%% Local Variables: 
%%% mode: latex
%%% TeX-master: "ref"
%%% End: 
