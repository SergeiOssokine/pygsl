\declaremodule{extension}{pygsl.statistics}
\moduleauthor{Jochen K\"upper}{jochen@jochen-kuepper.de}
\modulesynopsis{Statistical functions.}

\index{mean}
\index{standard deviation}
\index{variance}
\index{estimated standard deviation}
\index{estimated variance}
\index{t-test}
\index{range}
\index{min}
\index{max}
\index{kurtosis}
\index{skewness}
\index{autocorrelation}
\index{covariance}

This chapter describes the statistical functions in the library.  The basic
statistical functions include routines to compute the mean, variance and
standard deviation. More advanced functions allow you to calculate absolute
deviations, skewness, and kurtosis as well as the median and arbitrary
percentiles.  The algorithms use recurrence relations to compute average
quantities in a stable way, without large intermediate values that might
overflow.

All functions work on any Python sequence (of appropriate data-type), but see
section \ref{sec:stat:speed-considerations} for advantages and drawbacks of
different kinds of input data.

\begin{seealso}
   For details on the underlying implementation of these functions please
   consult the \GSL{} reference manual.
\end{seealso}



\section{Organization of the module}
\label{sec:stat:organization}

Individual parts of the \gsl{} functions names, providing artificial namespaces
in C, are mapped to modules and submodules in \pygsl{}.  That is,
\cfunction{gsl_stats_mean} can be found as \function{pygsl.statistics.mean} and
\cfunction{gsl_stats_long_mean} as \function{pygsl.statistics.long.mean}.

The functions in the module are available in versions for datasets in the
standard and \numpy{} floating-point and integer types. The generic versions
available in the \module{pygsl.statistics} module are using the generic \gsl{}
\ctype{double} versions.  The submodules use \gsl{} functions according to the
submodule name, e.g. long for \module{pygsl.statistics.long}.

Implemented submodules are \module{char}, \module{uchar}, \module{short},
\module{int}, \module{long}, \module{float}, and \module{double}. The latter
one also serves as default and is used whenever you don't expclicitely state a
different datatype. In most cases it is appropriate to simply use the default
implementation as it covers the widest range of the real space, offers high
precision, and as such is simple to use. If you have a sequence of all integer
values it is straightforward to use \module{pygsl.statistics.long} functions as
these use an implementation corresponding to Pythons \class{Float}-type. These
implemented submodules represent all numeric datatypes available in Python
(\class{Int}, \class{Float}) besides \class{Long Int} which has no
representation in standard C, as well as all numeric datatypes available in
\numpy{} that have corresponding implementations in \gsl{} (on 32 bit systems
these are: Character, UnsigendInt8, Int16, Int32, Int, Float32, Float).



\section{Available functions}
\label{sec:stat:available-functions}

\subsection{Mean, Standard Deviation, and Variance}
\label{sec:stat:mean-stand-deva}

\begin{funcdesc}{mean}{x}\index{mean}
   Arithmetic mean (\emph{sample mean}) of \var{x}:
   \begin{equation}
      \hat\mu = \frac{1}{N} \sum x_i
   \end{equation}
\end{funcdesc}

\begin{funcdesc}{variance}{x}\index{variance}
   Estimated (\emph{sample}) variance of \var{x}:
   \begin{equation}
      \hat\sigma^2 = \frac{1}{N-1} \sum (x_i - \hat\mu)^2
   \end{equation}
   This function computes the mean via a call to \function{mean}.  If you have
   already computed the mean then you can pass it directly to
   \function{variance_m}.
\end{funcdesc}

\begin{funcdesc}{variance_m}{x, mean}\index{variance}
   Estimated (\emph{sample}) variance of \var{x} relative to \var{mean}:
   \begin{equation}
      \hat\sigma^2 = \frac{1}{N-1} \sum (x_i - mean)^2
   \end{equation}
\end{funcdesc}

\begin{funcdesc}{sd}{x}
\end{funcdesc}
\begin{funcdesc}{sd_m}{x, mean}\index{sd}\index{mean}
   The standard deviation is defined as the square root of the variance of
   \var{x}.  These functions returns the square root of the respective
   variance-functions above.
\end{funcdesc}

\begin{funcdesc}{variance_with_fixed_mean}{x, mean}\index{variance}\index{mean}
   Compute an unbiased estimate of the variance of \var{x} when the population
   mean \var{mean} of the underlying distribution is known \emph{a priori}.  In
   this case the estimator for the variance uses the factor $1/N$ and the
   sample mean $\hat\mu$ is replaced by the known population mean $\mu$:
   \begin{equation}
      \hat\sigma^2 = \frac{1}{N} \sum (x_i - \mu)^2
   \end{equation}
\end{funcdesc}


\subsection{Mean, Standard Deviation, and Variance}
\label{sec:stat:mean-stand-deva}


\subsection{Absolute deviation}
\label{sec:stat:absolute-deviation}


\subsection{Higher moments (skewness and kurtosis)}
\label{sec:stat:high-moments-skewn}


\subsection{Autocorrelation}
\label{sec:stat:autocorrelation}


\subsection{Covariance}
\label{sec:stat:covariance}


\subsection{Maximum and Minimum values}
\label{sec:stat:max-min-value}


\subsection{Median and Percentiles}
\label{sec:stat:median-percentiles}



\section{Speed considerations}
\label{sec:stat:speed-considerations}

All functions work on any Python sequence type but are optimized for use with
\NUMPY{} (\numpy{}) arrays. It is strongly suggested that you install NumPy
(available from \url{http://www.numpy.org}).

If you pass NumPy arrays of the \emph{correct data-type} as input data to any
of the functions they are passed straight to the C functions along with the
stride information of the data. If you pass generic (non-\numpy{}) Python
sequences or \numpy{} arrays of the wrong data-type a suntactically suitable
copy of the data will be created and passed to the function. No precautions
against data-losses due to downcasts are taken. Esp. when calling functions
from a module of the smaller datatpes (\module{char}, \module{uchar},
\module{short}) you have to make sure all your data fits into the C
representation of the respective C dataype on the machine the program is
\emph{executed}. \footnote{Of course you also should not pass floating point
   data to integer modules.}

In general, when not using \numpy{} you should always use the default
implementation \module{pygsl.statistics} or the special implementation
\module{pygsl.statistics.long} in case you are \emph{sure} you have only
integer values. If your program needs the little speed-up that might be
possible by doing the calculation on a smaller datatype you should probably not
write it in plain Python anyway:)

If you have installed \numpy{} you can of course use arrays of all the
datatypes defined by \numpy{}. In that case you should use the submodule that
corresponds to the \numpy{} datatype you are using, as only that allows us to
pass your \numpy{} array straight to the C function. See the \numpy{}
documentation for details on the available datatypes.

\begin{seealso}
   If you are still unsure whether you should use \numpy{} take a look at
   \url{http://www.numpy.org} and download your personal copy.
\end{seealso}



\section{Further Reading}
\label{sec:stat:further-reading}

See the \gsl{} reference manual for a description of all available functions
and the calculations they perform.

The standard reference for almost any topic in statistics is the multi-volume
\emph{Advanced Theory of Statistics} by Kendall and Stuart.  Many statistical
concepts can be more easily understood by a Bayesian approach.  The book by
Gelman, Carlin, Stern and Rubin gives a comprehensive coverage of the subject.
For physicists the Particle Data Group provides useful reviews of Probability
and Statistics in the "Mathematical Tools" section of its Annual Review of
Particle Physics.
   
\begin{seealso}
   \seetext{Maurice Kendall, Alan Stuart, and J. Keith Ord: \emph{The Advanced
         Theory of Statistics} (multiple volumes) reprinted as \emph{Kendall's
         Advanced Theory of Statistics}.  Wiley, ISBN 047023380X.}
   
   \seetext{Andrew Gelman, John B. Carlin, Hal S. Stern, Donald B.  Rubin:
      \emph{Bayesian Data Analysis}.  Chapman \& Hall, ISBN 0412039915.}
   
   \seetext{R.M. Barnett et al: Review of Particle Properties. \emph{Physical
         Review} \textbf{D54}, 1 (1996).}
   
   \seetext{D.E. Groom et al., \emph{The European Physical Journal}
      \textbf{C15}, 1 (2000) and \emph{2001 off-year partial update for the
         2002 edition} available on the PDG WWW pages (URL:
      \url{http://pdg.lbl.gov/}).}
\end{seealso}


%% Local Variables:
%% mode: LaTeX
%% mode: auto-fill
%% fill-column: 79
%% indent-tabs-mode: nil
%% ispell-dictionary: "american"
%% reftex-fref-is-default: nil
%% TeX-auto-save: t
%% TeX-command-default: "pdfeLaTeX"
%% TeX-master: "pygsl"
%% TeX-parse-self: t
%% End:
