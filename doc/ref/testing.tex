\chapter[\protect\module{pygsl.testing} ---  Modules in Testing]
{\protect\module{pygsl.testing} \\ Modules in Testing}
\label{cha:statistics-module}

\declaremodule{standard}{pygsl.testing}

\moduleauthor{Pierre Schnizer}{schnizer@users.sourceforge.net}
Modules in this package are often reimplementations of an original package
with significant change to the original. The current rng implementation, for
example, started its life here. Usage of these modules is encouraged for tests
to see if they work, but use them with caution in your production code!

\section[\protect\module{pygsl.testing.sf} --- Special UFuncs]
{\protect\module{pygsl.testing.sf} \\ Special Functions as UFuncs}

\declaremodule{standard}{pygsl.testing.sf}
\moduleauthor{Pierre Schnizer}{schnizer@users.sourceforge.net}

This chapter provides mainly \numpy{} UFuncs over the special functions. This means
that all input variable can be arrays, and the UFunc will evaluate the gsl
function for all its inputs. It is meant to replace the sf module later;
please use it and find out if it is useful for you. 
Only the python specific part is described here. For a general description of
the function please see the GSL Reference document.  

\section{UFuncs}
These UFuncs allow to evaluate an array of doubles or an array of floats typically.
\begin{funcdesc}{Chi}{...}\index{Chi}

    Number of Input  Arguments:  1
    Number of Output Arguments:  1
\end{funcdesc}

\begin{funcdesc}{Chi_e}{...}\index{Chi_e}

    Number of Input  Arguments:  1
    Number of Output Arguments:  2

The error flag is discarded.
Return Arguments 1 and 2 resemble a gsl_result argument,
	which is  argument 1 of the C argument list

\end{funcdesc}

\begin{funcdesc}{Ci}{...}\index{Ci}

    Number of Input  Arguments:  1
    Number of Output Arguments:  1
\end{funcdesc}

\begin{funcdesc}{Ci_e}{...}\index{Ci_e}

    Number of Input  Arguments:  1
    Number of Output Arguments:  2

The error flag is discarded.
Return Arguments 1 and 2 resemble a gsl_result argument,
	which is  argument 1 of the C argument list

\end{funcdesc}

\begin{funcdesc}{Shi}{...}\index{Shi}

    Number of Input  Arguments:  1
    Number of Output Arguments:  1
\end{funcdesc}

\begin{funcdesc}{Shi_e}{...}\index{Shi_e}

    Number of Input  Arguments:  1
    Number of Output Arguments:  2

The error flag is discarded.
Return Arguments 1 and 2 resemble a gsl_result argument,
	which is  argument 1 of the C argument list

\end{funcdesc}

\begin{funcdesc}{Si}{...}\index{Si}

    Number of Input  Arguments:  1
    Number of Output Arguments:  1
\end{funcdesc}

\begin{funcdesc}{Si_e}{...}\index{Si_e}

    Number of Input  Arguments:  1
    Number of Output Arguments:  2

The error flag is discarded.
Return Arguments 1 and 2 resemble a gsl_result argument,
	which is  argument 1 of the C argument list

\end{funcdesc}

\begin{funcdesc}{airy_Ai}{...}\index{airy_Ai}

    Number of Input  Arguments:  2
    Number of Output Arguments:  1

 Argument 2 is a gsl_mode_t, valid parameters are:
	sf.PREC_DOUBLE or sf.PREC_SINGLE or sf.PREC_APPROX

\end{funcdesc}

\begin{funcdesc}{airy_Ai_deriv}{...}\index{airy_Ai_deriv}

    Number of Input  Arguments:  2
    Number of Output Arguments:  1

 Argument 2 is a gsl_mode_t, valid parameters are:
	sf.PREC_DOUBLE or sf.PREC_SINGLE or sf.PREC_APPROX

\end{funcdesc}

\begin{funcdesc}{airy_Ai_deriv_e}{...}\index{airy_Ai_deriv_e}

    Number of Input  Arguments:  2
    Number of Output Arguments:  2

 Argument 2 is a gsl_mode_t, valid parameters are:
	sf.PREC_DOUBLE or sf.PREC_SINGLE or sf.PREC_APPROX
The error flag is discarded.
Return Arguments 1 and 2 resemble a gsl_result argument,
	which is  argument 2 of the C argument list

\end{funcdesc}

\begin{funcdesc}{airy_Ai_deriv_scaled}{...}\index{airy_Ai_deriv_scaled}

    Number of Input  Arguments:  2
    Number of Output Arguments:  1

 Argument 2 is a gsl_mode_t, valid parameters are:
	sf.PREC_DOUBLE or sf.PREC_SINGLE or sf.PREC_APPROX

\end{funcdesc}

\begin{funcdesc}{airy_Ai_deriv_scaled_e}{...}\index{airy_Ai_deriv_scaled_e}

    Number of Input  Arguments:  2
    Number of Output Arguments:  2

 Argument 2 is a gsl_mode_t, valid parameters are:
	sf.PREC_DOUBLE or sf.PREC_SINGLE or sf.PREC_APPROX
The error flag is discarded.
Return Arguments 1 and 2 resemble a gsl_result argument,
	which is  argument 2 of the C argument list

\end{funcdesc}

\begin{funcdesc}{airy_Ai_e}{...}\index{airy_Ai_e}

    Number of Input  Arguments:  2
    Number of Output Arguments:  2

 Argument 2 is a gsl_mode_t, valid parameters are:
	sf.PREC_DOUBLE or sf.PREC_SINGLE or sf.PREC_APPROX
The error flag is discarded.
Return Arguments 1 and 2 resemble a gsl_result argument,
	which is  argument 2 of the C argument list

\end{funcdesc}

\begin{funcdesc}{airy_Ai_scaled}{...}\index{airy_Ai_scaled}

    Number of Input  Arguments:  2
    Number of Output Arguments:  1

 Argument 2 is a gsl_mode_t, valid parameters are:
	sf.PREC_DOUBLE or sf.PREC_SINGLE or sf.PREC_APPROX

\end{funcdesc}

\begin{funcdesc}{airy_Ai_scaled_e}{...}\index{airy_Ai_scaled_e}

    Number of Input  Arguments:  2
    Number of Output Arguments:  2

 Argument 2 is a gsl_mode_t, valid parameters are:
	sf.PREC_DOUBLE or sf.PREC_SINGLE or sf.PREC_APPROX
The error flag is discarded.
Return Arguments 1 and 2 resemble a gsl_result argument,
	which is  argument 2 of the C argument list

\end{funcdesc}

\begin{funcdesc}{airy_Bi}{...}\index{airy_Bi}

    Number of Input  Arguments:  2
    Number of Output Arguments:  1

 Argument 2 is a gsl_mode_t, valid parameters are:
	sf.PREC_DOUBLE or sf.PREC_SINGLE or sf.PREC_APPROX

\end{funcdesc}

\begin{funcdesc}{airy_Bi_deriv}{...}\index{airy_Bi_deriv}

    Number of Input  Arguments:  2
    Number of Output Arguments:  1

 Argument 2 is a gsl_mode_t, valid parameters are:
	sf.PREC_DOUBLE or sf.PREC_SINGLE or sf.PREC_APPROX

\end{funcdesc}

\begin{funcdesc}{airy_Bi_deriv_e}{...}\index{airy_Bi_deriv_e}

    Number of Input  Arguments:  2
    Number of Output Arguments:  2

 Argument 2 is a gsl_mode_t, valid parameters are:
	sf.PREC_DOUBLE or sf.PREC_SINGLE or sf.PREC_APPROX
The error flag is discarded.
Return Arguments 1 and 2 resemble a gsl_result argument,
	which is  argument 2 of the C argument list

\end{funcdesc}

\begin{funcdesc}{airy_Bi_deriv_scaled}{...}\index{airy_Bi_deriv_scaled}

    Number of Input  Arguments:  2
    Number of Output Arguments:  1

 Argument 2 is a gsl_mode_t, valid parameters are:
	sf.PREC_DOUBLE or sf.PREC_SINGLE or sf.PREC_APPROX

\end{funcdesc}

\begin{funcdesc}{airy_Bi_deriv_scaled_e}{...}\index{airy_Bi_deriv_scaled_e}

    Number of Input  Arguments:  2
    Number of Output Arguments:  2

 Argument 2 is a gsl_mode_t, valid parameters are:
	sf.PREC_DOUBLE or sf.PREC_SINGLE or sf.PREC_APPROX
The error flag is discarded.
Return Arguments 1 and 2 resemble a gsl_result argument,
	which is  argument 2 of the C argument list

\end{funcdesc}

\begin{funcdesc}{airy_Bi_e}{...}\index{airy_Bi_e}

    Number of Input  Arguments:  2
    Number of Output Arguments:  2

 Argument 2 is a gsl_mode_t, valid parameters are:
	sf.PREC_DOUBLE or sf.PREC_SINGLE or sf.PREC_APPROX
The error flag is discarded.
Return Arguments 1 and 2 resemble a gsl_result argument,
	which is  argument 2 of the C argument list

\end{funcdesc}

\begin{funcdesc}{airy_Bi_scaled}{...}\index{airy_Bi_scaled}

    Number of Input  Arguments:  2
    Number of Output Arguments:  1

 Argument 2 is a gsl_mode_t, valid parameters are:
	sf.PREC_DOUBLE or sf.PREC_SINGLE or sf.PREC_APPROX

\end{funcdesc}

\begin{funcdesc}{airy_Bi_scaled_e}{...}\index{airy_Bi_scaled_e}

    Number of Input  Arguments:  2
    Number of Output Arguments:  2

 Argument 2 is a gsl_mode_t, valid parameters are:
	sf.PREC_DOUBLE or sf.PREC_SINGLE or sf.PREC_APPROX
The error flag is discarded.
Return Arguments 1 and 2 resemble a gsl_result argument,
	which is  argument 2 of the C argument list

\end{funcdesc}

\begin{funcdesc}{airy_zero_Ai}{...}\index{airy_zero_Ai}

    Number of Input  Arguments:  1
    Number of Output Arguments:  1
\end{funcdesc}

\begin{funcdesc}{airy_zero_Ai_deriv}{...}\index{airy_zero_Ai_deriv}

    Number of Input  Arguments:  1
    Number of Output Arguments:  1
\end{funcdesc}

\begin{funcdesc}{airy_zero_Ai_deriv_e}{...}\index{airy_zero_Ai_deriv_e}

    Number of Input  Arguments:  1
    Number of Output Arguments:  2

The error flag is discarded.
Return Arguments 1 and 2 resemble a gsl_result argument,
	which is  argument 1 of the C argument list

\end{funcdesc}

\begin{funcdesc}{airy_zero_Ai_e}{...}\index{airy_zero_Ai_e}

    Number of Input  Arguments:  1
    Number of Output Arguments:  2

The error flag is discarded.
Return Arguments 1 and 2 resemble a gsl_result argument,
	which is  argument 1 of the C argument list

\end{funcdesc}

\begin{funcdesc}{airy_zero_Bi}{...}\index{airy_zero_Bi}

    Number of Input  Arguments:  1
    Number of Output Arguments:  1
\end{funcdesc}

\begin{funcdesc}{airy_zero_Bi_deriv}{...}\index{airy_zero_Bi_deriv}

    Number of Input  Arguments:  1
    Number of Output Arguments:  1
\end{funcdesc}

\begin{funcdesc}{airy_zero_Bi_deriv_e}{...}\index{airy_zero_Bi_deriv_e}

    Number of Input  Arguments:  1
    Number of Output Arguments:  2

The error flag is discarded.
Return Arguments 1 and 2 resemble a gsl_result argument,
	which is  argument 1 of the C argument list

\end{funcdesc}

\begin{funcdesc}{airy_zero_Bi_e}{...}\index{airy_zero_Bi_e}

    Number of Input  Arguments:  1
    Number of Output Arguments:  2

The error flag is discarded.
Return Arguments 1 and 2 resemble a gsl_result argument,
	which is  argument 1 of the C argument list

\end{funcdesc}

\begin{funcdesc}{angle_restrict_pos}{...}\index{angle_restrict_pos}

    Number of Input  Arguments:  1
    Number of Output Arguments:  1
\end{funcdesc}

\begin{funcdesc}{angle_restrict_pos_err_e}{...}\index{angle_restrict_pos_err_e}

    Number of Input  Arguments:  1
    Number of Output Arguments:  2

The error flag is discarded.
Return Arguments 1 and 2 resemble a gsl_result argument,
	which is  argument 1 of the C argument list

\end{funcdesc}

\begin{funcdesc}{angle_restrict_symm}{...}\index{angle_restrict_symm}

    Number of Input  Arguments:  1
    Number of Output Arguments:  1
\end{funcdesc}

\begin{funcdesc}{angle_restrict_symm_err_e}{...}\index{angle_restrict_symm_err_e}

    Number of Input  Arguments:  1
    Number of Output Arguments:  2

The error flag is discarded.
Return Arguments 1 and 2 resemble a gsl_result argument,
	which is  argument 1 of the C argument list

\end{funcdesc}

\begin{funcdesc}{atanint}{...}\index{atanint}

    Number of Input  Arguments:  1
    Number of Output Arguments:  1
\end{funcdesc}

\begin{funcdesc}{atanint_e}{...}\index{atanint_e}

    Number of Input  Arguments:  1
    Number of Output Arguments:  2

The error flag is discarded.
Return Arguments 1 and 2 resemble a gsl_result argument,
	which is  argument 1 of the C argument list

\end{funcdesc}

\begin{funcdesc}{bessel_I0}{...}\index{bessel_I0}

    Number of Input  Arguments:  1
    Number of Output Arguments:  1
\end{funcdesc}

\begin{funcdesc}{bessel_I0_e}{...}\index{bessel_I0_e}

    Number of Input  Arguments:  1
    Number of Output Arguments:  2

The error flag is discarded.
Return Arguments 1 and 2 resemble a gsl_result argument,
	which is  argument 1 of the C argument list

\end{funcdesc}

\begin{funcdesc}{bessel_I0_scaled}{...}\index{bessel_I0_scaled}

    Number of Input  Arguments:  1
    Number of Output Arguments:  1
\end{funcdesc}

\begin{funcdesc}{bessel_I0_scaled_e}{...}\index{bessel_I0_scaled_e}

    Number of Input  Arguments:  1
    Number of Output Arguments:  2

The error flag is discarded.
Return Arguments 1 and 2 resemble a gsl_result argument,
	which is  argument 1 of the C argument list

\end{funcdesc}

\begin{funcdesc}{bessel_I1}{...}\index{bessel_I1}

    Number of Input  Arguments:  1
    Number of Output Arguments:  1
\end{funcdesc}

\begin{funcdesc}{bessel_I1_e}{...}\index{bessel_I1_e}

    Number of Input  Arguments:  1
    Number of Output Arguments:  2

The error flag is discarded.
Return Arguments 1 and 2 resemble a gsl_result argument,
	which is  argument 1 of the C argument list

\end{funcdesc}

\begin{funcdesc}{bessel_I1_scaled}{...}\index{bessel_I1_scaled}

    Number of Input  Arguments:  1
    Number of Output Arguments:  1
\end{funcdesc}

\begin{funcdesc}{bessel_I1_scaled_e}{...}\index{bessel_I1_scaled_e}

    Number of Input  Arguments:  1
    Number of Output Arguments:  2

The error flag is discarded.
Return Arguments 1 and 2 resemble a gsl_result argument,
	which is  argument 1 of the C argument list

\end{funcdesc}

\begin{funcdesc}{bessel_In}{...}\index{bessel_In}

    Number of Input  Arguments:  2
    Number of Output Arguments:  1
\end{funcdesc}

\begin{funcdesc}{bessel_In_e}{...}\index{bessel_In_e}

    Number of Input  Arguments:  2
    Number of Output Arguments:  2

The error flag is discarded.
Return Arguments 1 and 2 resemble a gsl_result argument,
	which is  argument 2 of the C argument list

\end{funcdesc}

\begin{funcdesc}{bessel_In_scaled}{...}\index{bessel_In_scaled}

    Number of Input  Arguments:  2
    Number of Output Arguments:  1
\end{funcdesc}

\begin{funcdesc}{bessel_In_scaled_e}{...}\index{bessel_In_scaled_e}

    Number of Input  Arguments:  2
    Number of Output Arguments:  2

The error flag is discarded.
Return Arguments 1 and 2 resemble a gsl_result argument,
	which is  argument 2 of the C argument list

\end{funcdesc}

\begin{funcdesc}{bessel_Inu}{...}\index{bessel_Inu}

    Number of Input  Arguments:  2
    Number of Output Arguments:  1
\end{funcdesc}

\begin{funcdesc}{bessel_Inu_e}{...}\index{bessel_Inu_e}

    Number of Input  Arguments:  2
    Number of Output Arguments:  2

The error flag is discarded.
Return Arguments 1 and 2 resemble a gsl_result argument,
	which is  argument 2 of the C argument list

\end{funcdesc}

\begin{funcdesc}{bessel_Inu_scaled}{...}\index{bessel_Inu_scaled}

    Number of Input  Arguments:  2
    Number of Output Arguments:  1
\end{funcdesc}

\begin{funcdesc}{bessel_Inu_scaled_e}{...}\index{bessel_Inu_scaled_e}

    Number of Input  Arguments:  2
    Number of Output Arguments:  2

The error flag is discarded.
Return Arguments 1 and 2 resemble a gsl_result argument,
	which is  argument 2 of the C argument list

\end{funcdesc}

\begin{funcdesc}{bessel_J0}{...}\index{bessel_J0}

    Number of Input  Arguments:  1
    Number of Output Arguments:  1
\end{funcdesc}

\begin{funcdesc}{bessel_J0_e}{...}\index{bessel_J0_e}

    Number of Input  Arguments:  1
    Number of Output Arguments:  2

The error flag is discarded.
Return Arguments 1 and 2 resemble a gsl_result argument,
	which is  argument 1 of the C argument list

\end{funcdesc}

\begin{funcdesc}{bessel_J1}{...}\index{bessel_J1}

    Number of Input  Arguments:  1
    Number of Output Arguments:  1
\end{funcdesc}

\begin{funcdesc}{bessel_J1_e}{...}\index{bessel_J1_e}

    Number of Input  Arguments:  1
    Number of Output Arguments:  2

The error flag is discarded.
Return Arguments 1 and 2 resemble a gsl_result argument,
	which is  argument 1 of the C argument list

\end{funcdesc}

\begin{funcdesc}{bessel_Jn}{...}\index{bessel_Jn}

    Number of Input  Arguments:  2
    Number of Output Arguments:  1
\end{funcdesc}

\begin{funcdesc}{bessel_Jn_e}{...}\index{bessel_Jn_e}

    Number of Input  Arguments:  2
    Number of Output Arguments:  2

The error flag is discarded.
Return Arguments 1 and 2 resemble a gsl_result argument,
	which is  argument 2 of the C argument list

\end{funcdesc}

\begin{funcdesc}{bessel_Jnu}{...}\index{bessel_Jnu}

    Number of Input  Arguments:  2
    Number of Output Arguments:  1
\end{funcdesc}

\begin{funcdesc}{bessel_Jnu_e}{...}\index{bessel_Jnu_e}

    Number of Input  Arguments:  2
    Number of Output Arguments:  2

The error flag is discarded.
Return Arguments 1 and 2 resemble a gsl_result argument,
	which is  argument 2 of the C argument list

\end{funcdesc}

\begin{funcdesc}{bessel_K0}{...}\index{bessel_K0}

    Number of Input  Arguments:  1
    Number of Output Arguments:  1
\end{funcdesc}

\begin{funcdesc}{bessel_K0_e}{...}\index{bessel_K0_e}

    Number of Input  Arguments:  1
    Number of Output Arguments:  2

The error flag is discarded.
Return Arguments 1 and 2 resemble a gsl_result argument,
	which is  argument 1 of the C argument list

\end{funcdesc}

\begin{funcdesc}{bessel_K0_scaled}{...}\index{bessel_K0_scaled}

    Number of Input  Arguments:  1
    Number of Output Arguments:  1
\end{funcdesc}

\begin{funcdesc}{bessel_K0_scaled_e}{...}\index{bessel_K0_scaled_e}

    Number of Input  Arguments:  1
    Number of Output Arguments:  2

The error flag is discarded.
Return Arguments 1 and 2 resemble a gsl_result argument,
	which is  argument 1 of the C argument list

\end{funcdesc}

\begin{funcdesc}{bessel_K1}{...}\index{bessel_K1}

    Number of Input  Arguments:  1
    Number of Output Arguments:  1
\end{funcdesc}

\begin{funcdesc}{bessel_K1_e}{...}\index{bessel_K1_e}

    Number of Input  Arguments:  1
    Number of Output Arguments:  2

The error flag is discarded.
Return Arguments 1 and 2 resemble a gsl_result argument,
	which is  argument 1 of the C argument list

\end{funcdesc}

\begin{funcdesc}{bessel_K1_scaled}{...}\index{bessel_K1_scaled}

    Number of Input  Arguments:  1
    Number of Output Arguments:  1
\end{funcdesc}

\begin{funcdesc}{bessel_K1_scaled_e}{...}\index{bessel_K1_scaled_e}

    Number of Input  Arguments:  1
    Number of Output Arguments:  2

The error flag is discarded.
Return Arguments 1 and 2 resemble a gsl_result argument,
	which is  argument 1 of the C argument list

\end{funcdesc}

\begin{funcdesc}{bessel_Kn}{...}\index{bessel_Kn}

    Number of Input  Arguments:  2
    Number of Output Arguments:  1
\end{funcdesc}

\begin{funcdesc}{bessel_Kn_e}{...}\index{bessel_Kn_e}

    Number of Input  Arguments:  2
    Number of Output Arguments:  2

The error flag is discarded.
Return Arguments 1 and 2 resemble a gsl_result argument,
	which is  argument 2 of the C argument list

\end{funcdesc}

\begin{funcdesc}{bessel_Kn_scaled}{...}\index{bessel_Kn_scaled}

    Number of Input  Arguments:  2
    Number of Output Arguments:  1
\end{funcdesc}

\begin{funcdesc}{bessel_Kn_scaled_e}{...}\index{bessel_Kn_scaled_e}

    Number of Input  Arguments:  2
    Number of Output Arguments:  2

The error flag is discarded.
Return Arguments 1 and 2 resemble a gsl_result argument,
	which is  argument 2 of the C argument list

\end{funcdesc}

\begin{funcdesc}{bessel_Knu}{...}\index{bessel_Knu}

    Number of Input  Arguments:  2
    Number of Output Arguments:  1
\end{funcdesc}

\begin{funcdesc}{bessel_Knu_e}{...}\index{bessel_Knu_e}

    Number of Input  Arguments:  2
    Number of Output Arguments:  2

The error flag is discarded.
Return Arguments 1 and 2 resemble a gsl_result argument,
	which is  argument 2 of the C argument list

\end{funcdesc}

\begin{funcdesc}{bessel_Knu_scaled}{...}\index{bessel_Knu_scaled}

    Number of Input  Arguments:  2
    Number of Output Arguments:  1
\end{funcdesc}

\begin{funcdesc}{bessel_Knu_scaled_e}{...}\index{bessel_Knu_scaled_e}

    Number of Input  Arguments:  2
    Number of Output Arguments:  2

The error flag is discarded.
Return Arguments 1 and 2 resemble a gsl_result argument,
	which is  argument 2 of the C argument list

\end{funcdesc}

\begin{funcdesc}{bessel_Y0}{...}\index{bessel_Y0}

    Number of Input  Arguments:  1
    Number of Output Arguments:  1
\end{funcdesc}

\begin{funcdesc}{bessel_Y0_e}{...}\index{bessel_Y0_e}

    Number of Input  Arguments:  1
    Number of Output Arguments:  2

The error flag is discarded.
Return Arguments 1 and 2 resemble a gsl_result argument,
	which is  argument 1 of the C argument list

\end{funcdesc}

\begin{funcdesc}{bessel_Y1}{...}\index{bessel_Y1}

    Number of Input  Arguments:  1
    Number of Output Arguments:  1
\end{funcdesc}

\begin{funcdesc}{bessel_Y1_e}{...}\index{bessel_Y1_e}

    Number of Input  Arguments:  1
    Number of Output Arguments:  2

The error flag is discarded.
Return Arguments 1 and 2 resemble a gsl_result argument,
	which is  argument 1 of the C argument list

\end{funcdesc}

\begin{funcdesc}{bessel_Yn}{...}\index{bessel_Yn}

    Number of Input  Arguments:  2
    Number of Output Arguments:  1
\end{funcdesc}

\begin{funcdesc}{bessel_Yn_e}{...}\index{bessel_Yn_e}

    Number of Input  Arguments:  2
    Number of Output Arguments:  2

The error flag is discarded.
Return Arguments 1 and 2 resemble a gsl_result argument,
	which is  argument 2 of the C argument list

\end{funcdesc}

\begin{funcdesc}{bessel_Ynu}{...}\index{bessel_Ynu}

    Number of Input  Arguments:  2
    Number of Output Arguments:  1
\end{funcdesc}

\begin{funcdesc}{bessel_Ynu_e}{...}\index{bessel_Ynu_e}

    Number of Input  Arguments:  2
    Number of Output Arguments:  2

The error flag is discarded.
Return Arguments 1 and 2 resemble a gsl_result argument,
	which is  argument 2 of the C argument list

\end{funcdesc}

\begin{funcdesc}{bessel_i0_scaled}{...}\index{bessel_i0_scaled}

    Number of Input  Arguments:  1
    Number of Output Arguments:  1
\end{funcdesc}

\begin{funcdesc}{bessel_i0_scaled_e}{...}\index{bessel_i0_scaled_e}

    Number of Input  Arguments:  1
    Number of Output Arguments:  2

The error flag is discarded.
Return Arguments 1 and 2 resemble a gsl_result argument,
	which is  argument 1 of the C argument list

\end{funcdesc}

\begin{funcdesc}{bessel_i1_scaled}{...}\index{bessel_i1_scaled}

    Number of Input  Arguments:  1
    Number of Output Arguments:  1
\end{funcdesc}

\begin{funcdesc}{bessel_i1_scaled_e}{...}\index{bessel_i1_scaled_e}

    Number of Input  Arguments:  1
    Number of Output Arguments:  2

The error flag is discarded.
Return Arguments 1 and 2 resemble a gsl_result argument,
	which is  argument 1 of the C argument list

\end{funcdesc}

\begin{funcdesc}{bessel_i2_scaled}{...}\index{bessel_i2_scaled}

    Number of Input  Arguments:  1
    Number of Output Arguments:  1
\end{funcdesc}

\begin{funcdesc}{bessel_i2_scaled_e}{...}\index{bessel_i2_scaled_e}

    Number of Input  Arguments:  1
    Number of Output Arguments:  2

The error flag is discarded.
Return Arguments 1 and 2 resemble a gsl_result argument,
	which is  argument 1 of the C argument list

\end{funcdesc}

\begin{funcdesc}{bessel_il_scaled}{...}\index{bessel_il_scaled}

    Number of Input  Arguments:  2
    Number of Output Arguments:  1
\end{funcdesc}

\begin{funcdesc}{bessel_il_scaled_e}{...}\index{bessel_il_scaled_e}

    Number of Input  Arguments:  2
    Number of Output Arguments:  2

The error flag is discarded.
Return Arguments 1 and 2 resemble a gsl_result argument,
	which is  argument 2 of the C argument list

\end{funcdesc}

\begin{funcdesc}{bessel_j0}{...}\index{bessel_j0}

    Number of Input  Arguments:  1
    Number of Output Arguments:  1
\end{funcdesc}

\begin{funcdesc}{bessel_j0_e}{...}\index{bessel_j0_e}

    Number of Input  Arguments:  1
    Number of Output Arguments:  2

The error flag is discarded.
Return Arguments 1 and 2 resemble a gsl_result argument,
	which is  argument 1 of the C argument list

\end{funcdesc}

\begin{funcdesc}{bessel_j1}{...}\index{bessel_j1}

    Number of Input  Arguments:  1
    Number of Output Arguments:  1
\end{funcdesc}

\begin{funcdesc}{bessel_j1_e}{...}\index{bessel_j1_e}

    Number of Input  Arguments:  1
    Number of Output Arguments:  2

The error flag is discarded.
Return Arguments 1 and 2 resemble a gsl_result argument,
	which is  argument 1 of the C argument list

\end{funcdesc}

\begin{funcdesc}{bessel_j2}{...}\index{bessel_j2}

    Number of Input  Arguments:  1
    Number of Output Arguments:  1
\end{funcdesc}

\begin{funcdesc}{bessel_j2_e}{...}\index{bessel_j2_e}

    Number of Input  Arguments:  1
    Number of Output Arguments:  2

The error flag is discarded.
Return Arguments 1 and 2 resemble a gsl_result argument,
	which is  argument 1 of the C argument list

\end{funcdesc}

\begin{funcdesc}{bessel_jl}{...}\index{bessel_jl}

    Number of Input  Arguments:  2
    Number of Output Arguments:  1
\end{funcdesc}

\begin{funcdesc}{bessel_jl_e}{...}\index{bessel_jl_e}

    Number of Input  Arguments:  2
    Number of Output Arguments:  2

The error flag is discarded.
Return Arguments 1 and 2 resemble a gsl_result argument,
	which is  argument 2 of the C argument list

\end{funcdesc}

\begin{funcdesc}{bessel_k0_scaled}{...}\index{bessel_k0_scaled}

    Number of Input  Arguments:  1
    Number of Output Arguments:  1
\end{funcdesc}

\begin{funcdesc}{bessel_k0_scaled_e}{...}\index{bessel_k0_scaled_e}

    Number of Input  Arguments:  1
    Number of Output Arguments:  2

The error flag is discarded.
Return Arguments 1 and 2 resemble a gsl_result argument,
	which is  argument 1 of the C argument list

\end{funcdesc}

\begin{funcdesc}{bessel_k1_scaled}{...}\index{bessel_k1_scaled}

    Number of Input  Arguments:  1
    Number of Output Arguments:  1
\end{funcdesc}

\begin{funcdesc}{bessel_k1_scaled_e}{...}\index{bessel_k1_scaled_e}

    Number of Input  Arguments:  1
    Number of Output Arguments:  2

The error flag is discarded.
Return Arguments 1 and 2 resemble a gsl_result argument,
	which is  argument 1 of the C argument list

\end{funcdesc}

\begin{funcdesc}{bessel_k2_scaled}{...}\index{bessel_k2_scaled}

    Number of Input  Arguments:  1
    Number of Output Arguments:  1
\end{funcdesc}

\begin{funcdesc}{bessel_k2_scaled_e}{...}\index{bessel_k2_scaled_e}

    Number of Input  Arguments:  1
    Number of Output Arguments:  2

The error flag is discarded.
Return Arguments 1 and 2 resemble a gsl_result argument,
	which is  argument 1 of the C argument list

\end{funcdesc}

\begin{funcdesc}{bessel_kl_scaled}{...}\index{bessel_kl_scaled}

    Number of Input  Arguments:  2
    Number of Output Arguments:  1
\end{funcdesc}

\begin{funcdesc}{bessel_kl_scaled_e}{...}\index{bessel_kl_scaled_e}

    Number of Input  Arguments:  2
    Number of Output Arguments:  2

The error flag is discarded.
Return Arguments 1 and 2 resemble a gsl_result argument,
	which is  argument 2 of the C argument list

\end{funcdesc}

\begin{funcdesc}{bessel_lnKnu}{...}\index{bessel_lnKnu}

    Number of Input  Arguments:  2
    Number of Output Arguments:  1
\end{funcdesc}

\begin{funcdesc}{bessel_lnKnu_e}{...}\index{bessel_lnKnu_e}

    Number of Input  Arguments:  2
    Number of Output Arguments:  2

The error flag is discarded.
Return Arguments 1 and 2 resemble a gsl_result argument,
	which is  argument 2 of the C argument list

\end{funcdesc}

\begin{funcdesc}{bessel_y0}{...}\index{bessel_y0}

    Number of Input  Arguments:  1
    Number of Output Arguments:  1
\end{funcdesc}

\begin{funcdesc}{bessel_y0_e}{...}\index{bessel_y0_e}

    Number of Input  Arguments:  1
    Number of Output Arguments:  2

The error flag is discarded.
Return Arguments 1 and 2 resemble a gsl_result argument,
	which is  argument 1 of the C argument list

\end{funcdesc}

\begin{funcdesc}{bessel_y1}{...}\index{bessel_y1}

    Number of Input  Arguments:  1
    Number of Output Arguments:  1
\end{funcdesc}

\begin{funcdesc}{bessel_y1_e}{...}\index{bessel_y1_e}

    Number of Input  Arguments:  1
    Number of Output Arguments:  2

The error flag is discarded.
Return Arguments 1 and 2 resemble a gsl_result argument,
	which is  argument 1 of the C argument list

\end{funcdesc}

\begin{funcdesc}{bessel_y2}{...}\index{bessel_y2}

    Number of Input  Arguments:  1
    Number of Output Arguments:  1
\end{funcdesc}

\begin{funcdesc}{bessel_y2_e}{...}\index{bessel_y2_e}

    Number of Input  Arguments:  1
    Number of Output Arguments:  2

The error flag is discarded.
Return Arguments 1 and 2 resemble a gsl_result argument,
	which is  argument 1 of the C argument list

\end{funcdesc}

\begin{funcdesc}{bessel_yl}{...}\index{bessel_yl}

    Number of Input  Arguments:  2
    Number of Output Arguments:  1
\end{funcdesc}

\begin{funcdesc}{bessel_yl_e}{...}\index{bessel_yl_e}

    Number of Input  Arguments:  2
    Number of Output Arguments:  2

The error flag is discarded.
Return Arguments 1 and 2 resemble a gsl_result argument,
	which is  argument 2 of the C argument list

\end{funcdesc}

\begin{funcdesc}{bessel_zero_J0}{...}\index{bessel_zero_J0}

    Number of Input  Arguments:  1
    Number of Output Arguments:  1
\end{funcdesc}

\begin{funcdesc}{bessel_zero_J0_e}{...}\index{bessel_zero_J0_e}

    Number of Input  Arguments:  1
    Number of Output Arguments:  2

The error flag is discarded.
Return Arguments 1 and 2 resemble a gsl_result argument,
	which is  argument 1 of the C argument list

\end{funcdesc}

\begin{funcdesc}{bessel_zero_J1}{...}\index{bessel_zero_J1}

    Number of Input  Arguments:  1
    Number of Output Arguments:  1
\end{funcdesc}

\begin{funcdesc}{bessel_zero_J1_e}{...}\index{bessel_zero_J1_e}

    Number of Input  Arguments:  1
    Number of Output Arguments:  2

The error flag is discarded.
Return Arguments 1 and 2 resemble a gsl_result argument,
	which is  argument 1 of the C argument list

\end{funcdesc}

\begin{funcdesc}{bessel_zero_Jnu}{...}\index{bessel_zero_Jnu}

    Number of Input  Arguments:  2
    Number of Output Arguments:  1
\end{funcdesc}

\begin{funcdesc}{bessel_zero_Jnu_e}{...}\index{bessel_zero_Jnu_e}

    Number of Input  Arguments:  2
    Number of Output Arguments:  2

The error flag is discarded.
Return Arguments 1 and 2 resemble a gsl_result argument,
	which is  argument 2 of the C argument list

\end{funcdesc}

\begin{funcdesc}{beta}{...}\index{beta}

    Number of Input  Arguments:  2
    Number of Output Arguments:  1
\end{funcdesc}

\begin{funcdesc}{beta_e}{...}\index{beta_e}

    Number of Input  Arguments:  2
    Number of Output Arguments:  2

The error flag is discarded.
Return Arguments 1 and 2 resemble a gsl_result argument,
	which is  argument 2 of the C argument list

\end{funcdesc}

\begin{funcdesc}{beta_inc}{...}\index{beta_inc}

    Number of Input  Arguments:  3
    Number of Output Arguments:  1
\end{funcdesc}

\begin{funcdesc}{beta_inc_e}{...}\index{beta_inc_e}

    Number of Input  Arguments:  3
    Number of Output Arguments:  2

The error flag is discarded.
Return Arguments 1 and 2 resemble a gsl_result argument,
	which is  argument 3 of the C argument list

\end{funcdesc}

\begin{funcdesc}{choose}{...}\index{choose}

    Number of Input  Arguments:  2
    Number of Output Arguments:  1
\end{funcdesc}

\begin{funcdesc}{choose_e}{...}\index{choose_e}

    Number of Input  Arguments:  2
    Number of Output Arguments:  2

The error flag is discarded.
Return Arguments 1 and 2 resemble a gsl_result argument,
	which is  argument 2 of the C argument list

\end{funcdesc}

\begin{funcdesc}{clausen}{...}\index{clausen}

    Number of Input  Arguments:  1
    Number of Output Arguments:  1
\end{funcdesc}

\begin{funcdesc}{clausen_e}{...}\index{clausen_e}

    Number of Input  Arguments:  1
    Number of Output Arguments:  2

The error flag is discarded.
Return Arguments 1 and 2 resemble a gsl_result argument,
	which is  argument 1 of the C argument list

\end{funcdesc}

\begin{funcdesc}{conicalP_0}{...}\index{conicalP_0}

    Number of Input  Arguments:  2
    Number of Output Arguments:  1
\end{funcdesc}

\begin{funcdesc}{conicalP_0_e}{...}\index{conicalP_0_e}

    Number of Input  Arguments:  2
    Number of Output Arguments:  2

The error flag is discarded.
Return Arguments 1 and 2 resemble a gsl_result argument,
	which is  argument 2 of the C argument list

\end{funcdesc}

\begin{funcdesc}{conicalP_1}{...}\index{conicalP_1}

    Number of Input  Arguments:  2
    Number of Output Arguments:  1
\end{funcdesc}

\begin{funcdesc}{conicalP_1_e}{...}\index{conicalP_1_e}

    Number of Input  Arguments:  2
    Number of Output Arguments:  2

The error flag is discarded.
Return Arguments 1 and 2 resemble a gsl_result argument,
	which is  argument 2 of the C argument list

\end{funcdesc}

\begin{funcdesc}{conicalP_cyl_reg}{...}\index{conicalP_cyl_reg}

    Number of Input  Arguments:  3
    Number of Output Arguments:  1
\end{funcdesc}

\begin{funcdesc}{conicalP_cyl_reg_e}{...}\index{conicalP_cyl_reg_e}

    Number of Input  Arguments:  3
    Number of Output Arguments:  2

The error flag is discarded.
Return Arguments 1 and 2 resemble a gsl_result argument,
	which is  argument 3 of the C argument list

\end{funcdesc}

\begin{funcdesc}{conicalP_half}{...}\index{conicalP_half}

    Number of Input  Arguments:  2
    Number of Output Arguments:  1
\end{funcdesc}

\begin{funcdesc}{conicalP_half_e}{...}\index{conicalP_half_e}

    Number of Input  Arguments:  2
    Number of Output Arguments:  2

The error flag is discarded.
Return Arguments 1 and 2 resemble a gsl_result argument,
	which is  argument 2 of the C argument list

\end{funcdesc}

\begin{funcdesc}{conicalP_mhalf}{...}\index{conicalP_mhalf}

    Number of Input  Arguments:  2
    Number of Output Arguments:  1
\end{funcdesc}

\begin{funcdesc}{conicalP_mhalf_e}{...}\index{conicalP_mhalf_e}

    Number of Input  Arguments:  2
    Number of Output Arguments:  2

The error flag is discarded.
Return Arguments 1 and 2 resemble a gsl_result argument,
	which is  argument 2 of the C argument list

\end{funcdesc}

\begin{funcdesc}{conicalP_sph_reg}{...}\index{conicalP_sph_reg}

    Number of Input  Arguments:  3
    Number of Output Arguments:  1
\end{funcdesc}

\begin{funcdesc}{conicalP_sph_reg_e}{...}\index{conicalP_sph_reg_e}

    Number of Input  Arguments:  3
    Number of Output Arguments:  2

The error flag is discarded.
Return Arguments 1 and 2 resemble a gsl_result argument,
	which is  argument 3 of the C argument list

\end{funcdesc}

\begin{funcdesc}{cos}{...}\index{cos}

    Number of Input  Arguments:  1
    Number of Output Arguments:  1
\end{funcdesc}

\begin{funcdesc}{cos_e}{...}\index{cos_e}

    Number of Input  Arguments:  1
    Number of Output Arguments:  2

The error flag is discarded.
Return Arguments 1 and 2 resemble a gsl_result argument,
	which is  argument 1 of the C argument list

\end{funcdesc}

\begin{funcdesc}{cos_err_e}{...}\index{cos_err_e}

    Number of Input  Arguments:  2
    Number of Output Arguments:  2

The error flag is discarded.
Return Arguments 1 and 2 resemble a gsl_result argument,
	which is  argument 2 of the C argument list

\end{funcdesc}

\begin{funcdesc}{coulomb_CL_e}{...}\index{coulomb_CL_e}

    Number of Input  Arguments:  2
    Number of Output Arguments:  2

The error flag is discarded.
Return Arguments 1 and 2 resemble a gsl_result argument,
	which is  argument 2 of the C argument list

\end{funcdesc}

\begin{funcdesc}{coulomb_wave_FG_e}{...}\index{coulomb_wave_FG_e}

    Number of Input  Arguments:  4
    Number of Output Arguments: 10

The error flag is discarded.
Return Arguments 1 and 2 resemble a gsl_result argument,
	which is  argument 4 of the C argument list
Return Arguments 3 and 4 resemble a gsl_result argument,
	which is  argument 5 of the C argument list
Return Arguments 5 and 6 resemble a gsl_result argument,
	which is  argument 6 of the C argument list
Return Arguments 7 and 8 resemble a gsl_result argument,
	which is  argument 7 of the C argument list

\end{funcdesc}

\begin{funcdesc}{coupling_3j}{...}\index{coupling_3j}

    Number of Input  Arguments:  6
    Number of Output Arguments:  1
\end{funcdesc}

\begin{funcdesc}{coupling_3j_e}{...}\index{coupling_3j_e}

    Number of Input  Arguments:  6
    Number of Output Arguments:  2

The error flag is discarded.
Return Arguments 1 and 2 resemble a gsl_result argument,
	which is  argument 6 of the C argument list

\end{funcdesc}

\begin{funcdesc}{coupling_6j}{...}\index{coupling_6j}

    Number of Input  Arguments:  6
    Number of Output Arguments:  1
\end{funcdesc}

\begin{funcdesc}{coupling_6j_e}{...}\index{coupling_6j_e}

    Number of Input  Arguments:  6
    Number of Output Arguments:  2

The error flag is discarded.
Return Arguments 1 and 2 resemble a gsl_result argument,
	which is  argument 6 of the C argument list

\end{funcdesc}

\begin{funcdesc}{coupling_9j}{...}\index{coupling_9j}

    Number of Input  Arguments:  9
    Number of Output Arguments:  1
\end{funcdesc}

\begin{funcdesc}{coupling_9j_e}{...}\index{coupling_9j_e}

    Number of Input  Arguments:  9
    Number of Output Arguments:  2

The error flag is discarded.
Return Arguments 1 and 2 resemble a gsl_result argument,
	which is  argument 9 of the C argument list

\end{funcdesc}

\begin{funcdesc}{coupling_RacahW}{...}\index{coupling_RacahW}

    Number of Input  Arguments:  6
    Number of Output Arguments:  1
\end{funcdesc}

\begin{funcdesc}{coupling_RacahW_e}{...}\index{coupling_RacahW_e}

    Number of Input  Arguments:  6
    Number of Output Arguments:  2

The error flag is discarded.
Return Arguments 1 and 2 resemble a gsl_result argument,
	which is  argument 6 of the C argument list

\end{funcdesc}

\begin{funcdesc}{dawson}{...}\index{dawson}

    Number of Input  Arguments:  1
    Number of Output Arguments:  1
\end{funcdesc}

\begin{funcdesc}{dawson_e}{...}\index{dawson_e}

    Number of Input  Arguments:  1
    Number of Output Arguments:  2

The error flag is discarded.
Return Arguments 1 and 2 resemble a gsl_result argument,
	which is  argument 1 of the C argument list

\end{funcdesc}

\begin{funcdesc}{debye_1}{...}\index{debye_1}

    Number of Input  Arguments:  1
    Number of Output Arguments:  1
\end{funcdesc}

\begin{funcdesc}{debye_1_e}{...}\index{debye_1_e}

    Number of Input  Arguments:  1
    Number of Output Arguments:  2

The error flag is discarded.
Return Arguments 1 and 2 resemble a gsl_result argument,
	which is  argument 1 of the C argument list

\end{funcdesc}

\begin{funcdesc}{debye_2}{...}\index{debye_2}

    Number of Input  Arguments:  1
    Number of Output Arguments:  1
\end{funcdesc}

\begin{funcdesc}{debye_2_e}{...}\index{debye_2_e}

    Number of Input  Arguments:  1
    Number of Output Arguments:  2

The error flag is discarded.
Return Arguments 1 and 2 resemble a gsl_result argument,
	which is  argument 1 of the C argument list

\end{funcdesc}

\begin{funcdesc}{debye_3}{...}\index{debye_3}

    Number of Input  Arguments:  1
    Number of Output Arguments:  1
\end{funcdesc}

\begin{funcdesc}{debye_3_e}{...}\index{debye_3_e}

    Number of Input  Arguments:  1
    Number of Output Arguments:  2

The error flag is discarded.
Return Arguments 1 and 2 resemble a gsl_result argument,
	which is  argument 1 of the C argument list

\end{funcdesc}

\begin{funcdesc}{debye_4}{...}\index{debye_4}

    Number of Input  Arguments:  1
    Number of Output Arguments:  1
\end{funcdesc}

\begin{funcdesc}{debye_4_e}{...}\index{debye_4_e}

    Number of Input  Arguments:  1
    Number of Output Arguments:  2

The error flag is discarded.
Return Arguments 1 and 2 resemble a gsl_result argument,
	which is  argument 1 of the C argument list

\end{funcdesc}

\begin{funcdesc}{dilog}{...}\index{dilog}

    Number of Input  Arguments:  1
    Number of Output Arguments:  1
\end{funcdesc}

\begin{funcdesc}{dilog_e}{...}\index{dilog_e}

    Number of Input  Arguments:  1
    Number of Output Arguments:  2

The error flag is discarded.
Return Arguments 1 and 2 resemble a gsl_result argument,
	which is  argument 1 of the C argument list

\end{funcdesc}

\begin{funcdesc}{doublefact}{...}\index{doublefact}

    Number of Input  Arguments:  1
    Number of Output Arguments:  1
\end{funcdesc}

\begin{funcdesc}{doublefact_e}{...}\index{doublefact_e}

    Number of Input  Arguments:  1
    Number of Output Arguments:  2

The error flag is discarded.
Return Arguments 1 and 2 resemble a gsl_result argument,
	which is  argument 1 of the C argument list

\end{funcdesc}

\begin{funcdesc}{ellint_D}{...}\index{ellint_D}

    Number of Input  Arguments:  4
    Number of Output Arguments:  1

 Argument 4 is a gsl_mode_t, valid parameters are:
	sf.PREC_DOUBLE or sf.PREC_SINGLE or sf.PREC_APPROX

\end{funcdesc}

\begin{funcdesc}{ellint_D_e}{...}\index{ellint_D_e}

    Number of Input  Arguments:  4
    Number of Output Arguments:  2

 Argument 4 is a gsl_mode_t, valid parameters are:
	sf.PREC_DOUBLE or sf.PREC_SINGLE or sf.PREC_APPROX
The error flag is discarded.
Return Arguments 1 and 2 resemble a gsl_result argument,
	which is  argument 4 of the C argument list

\end{funcdesc}

\begin{funcdesc}{ellint_E}{...}\index{ellint_E}

    Number of Input  Arguments:  3
    Number of Output Arguments:  1

 Argument 3 is a gsl_mode_t, valid parameters are:
	sf.PREC_DOUBLE or sf.PREC_SINGLE or sf.PREC_APPROX

\end{funcdesc}

\begin{funcdesc}{ellint_E_e}{...}\index{ellint_E_e}

    Number of Input  Arguments:  3
    Number of Output Arguments:  2

 Argument 3 is a gsl_mode_t, valid parameters are:
	sf.PREC_DOUBLE or sf.PREC_SINGLE or sf.PREC_APPROX
The error flag is discarded.
Return Arguments 1 and 2 resemble a gsl_result argument,
	which is  argument 3 of the C argument list

\end{funcdesc}

\begin{funcdesc}{ellint_Ecomp}{...}\index{ellint_Ecomp}

    Number of Input  Arguments:  2
    Number of Output Arguments:  1

 Argument 2 is a gsl_mode_t, valid parameters are:
	sf.PREC_DOUBLE or sf.PREC_SINGLE or sf.PREC_APPROX

\end{funcdesc}

\begin{funcdesc}{ellint_Ecomp_e}{...}\index{ellint_Ecomp_e}

    Number of Input  Arguments:  2
    Number of Output Arguments:  2

 Argument 2 is a gsl_mode_t, valid parameters are:
	sf.PREC_DOUBLE or sf.PREC_SINGLE or sf.PREC_APPROX
The error flag is discarded.
Return Arguments 1 and 2 resemble a gsl_result argument,
	which is  argument 2 of the C argument list

\end{funcdesc}

\begin{funcdesc}{ellint_F}{...}\index{ellint_F}

    Number of Input  Arguments:  3
    Number of Output Arguments:  1

 Argument 3 is a gsl_mode_t, valid parameters are:
	sf.PREC_DOUBLE or sf.PREC_SINGLE or sf.PREC_APPROX

\end{funcdesc}

\begin{funcdesc}{ellint_F_e}{...}\index{ellint_F_e}

    Number of Input  Arguments:  3
    Number of Output Arguments:  2

 Argument 3 is a gsl_mode_t, valid parameters are:
	sf.PREC_DOUBLE or sf.PREC_SINGLE or sf.PREC_APPROX
The error flag is discarded.
Return Arguments 1 and 2 resemble a gsl_result argument,
	which is  argument 3 of the C argument list

\end{funcdesc}

\begin{funcdesc}{ellint_Kcomp}{...}\index{ellint_Kcomp}

    Number of Input  Arguments:  2
    Number of Output Arguments:  1

 Argument 2 is a gsl_mode_t, valid parameters are:
	sf.PREC_DOUBLE or sf.PREC_SINGLE or sf.PREC_APPROX

\end{funcdesc}

\begin{funcdesc}{ellint_Kcomp_e}{...}\index{ellint_Kcomp_e}

    Number of Input  Arguments:  2
    Number of Output Arguments:  2

 Argument 2 is a gsl_mode_t, valid parameters are:
	sf.PREC_DOUBLE or sf.PREC_SINGLE or sf.PREC_APPROX
The error flag is discarded.
Return Arguments 1 and 2 resemble a gsl_result argument,
	which is  argument 2 of the C argument list

\end{funcdesc}

\begin{funcdesc}{ellint_P}{...}\index{ellint_P}

    Number of Input  Arguments:  4
    Number of Output Arguments:  1

 Argument 4 is a gsl_mode_t, valid parameters are:
	sf.PREC_DOUBLE or sf.PREC_SINGLE or sf.PREC_APPROX

\end{funcdesc}

\begin{funcdesc}{ellint_P_e}{...}\index{ellint_P_e}

    Number of Input  Arguments:  4
    Number of Output Arguments:  2

 Argument 4 is a gsl_mode_t, valid parameters are:
	sf.PREC_DOUBLE or sf.PREC_SINGLE or sf.PREC_APPROX
The error flag is discarded.
Return Arguments 1 and 2 resemble a gsl_result argument,
	which is  argument 4 of the C argument list

\end{funcdesc}

\begin{funcdesc}{ellint_RC}{...}\index{ellint_RC}

    Number of Input  Arguments:  3
    Number of Output Arguments:  1

 Argument 3 is a gsl_mode_t, valid parameters are:
	sf.PREC_DOUBLE or sf.PREC_SINGLE or sf.PREC_APPROX

\end{funcdesc}

\begin{funcdesc}{ellint_RC_e}{...}\index{ellint_RC_e}

    Number of Input  Arguments:  3
    Number of Output Arguments:  2

 Argument 3 is a gsl_mode_t, valid parameters are:
	sf.PREC_DOUBLE or sf.PREC_SINGLE or sf.PREC_APPROX
The error flag is discarded.
Return Arguments 1 and 2 resemble a gsl_result argument,
	which is  argument 3 of the C argument list

\end{funcdesc}

\begin{funcdesc}{ellint_RD}{...}\index{ellint_RD}

    Number of Input  Arguments:  4
    Number of Output Arguments:  1

 Argument 4 is a gsl_mode_t, valid parameters are:
	sf.PREC_DOUBLE or sf.PREC_SINGLE or sf.PREC_APPROX

\end{funcdesc}

\begin{funcdesc}{ellint_RD_e}{...}\index{ellint_RD_e}

    Number of Input  Arguments:  4
    Number of Output Arguments:  2

 Argument 4 is a gsl_mode_t, valid parameters are:
	sf.PREC_DOUBLE or sf.PREC_SINGLE or sf.PREC_APPROX
The error flag is discarded.
Return Arguments 1 and 2 resemble a gsl_result argument,
	which is  argument 4 of the C argument list

\end{funcdesc}

\begin{funcdesc}{ellint_RF}{...}\index{ellint_RF}

    Number of Input  Arguments:  4
    Number of Output Arguments:  1

 Argument 4 is a gsl_mode_t, valid parameters are:
	sf.PREC_DOUBLE or sf.PREC_SINGLE or sf.PREC_APPROX

\end{funcdesc}

\begin{funcdesc}{ellint_RF_e}{...}\index{ellint_RF_e}

    Number of Input  Arguments:  4
    Number of Output Arguments:  2

 Argument 4 is a gsl_mode_t, valid parameters are:
	sf.PREC_DOUBLE or sf.PREC_SINGLE or sf.PREC_APPROX
The error flag is discarded.
Return Arguments 1 and 2 resemble a gsl_result argument,
	which is  argument 4 of the C argument list

\end{funcdesc}

\begin{funcdesc}{ellint_RJ}{...}\index{ellint_RJ}

    Number of Input  Arguments:  5
    Number of Output Arguments:  1

 Argument 5 is a gsl_mode_t, valid parameters are:
	sf.PREC_DOUBLE or sf.PREC_SINGLE or sf.PREC_APPROX

\end{funcdesc}

\begin{funcdesc}{ellint_RJ_e}{...}\index{ellint_RJ_e}

    Number of Input  Arguments:  5
    Number of Output Arguments:  2

 Argument 5 is a gsl_mode_t, valid parameters are:
	sf.PREC_DOUBLE or sf.PREC_SINGLE or sf.PREC_APPROX
The error flag is discarded.
Return Arguments 1 and 2 resemble a gsl_result argument,
	which is  argument 5 of the C argument list

\end{funcdesc}

\begin{funcdesc}{elljac_e}{...}\index{elljac_e}

    Number of Input  Arguments:  2
    Number of Output Arguments:  3

The error flag is discarded.

\end{funcdesc}

\begin{funcdesc}{erf}{...}\index{erf}

    Number of Input  Arguments:  1
    Number of Output Arguments:  1
\end{funcdesc}

\begin{funcdesc}{erf_Q}{...}\index{erf_Q}

    Number of Input  Arguments:  1
    Number of Output Arguments:  1
\end{funcdesc}

\begin{funcdesc}{erf_Q_e}{...}\index{erf_Q_e}

    Number of Input  Arguments:  1
    Number of Output Arguments:  2

The error flag is discarded.
Return Arguments 1 and 2 resemble a gsl_result argument,
	which is  argument 1 of the C argument list

\end{funcdesc}

\begin{funcdesc}{erf_Z}{...}\index{erf_Z}

    Number of Input  Arguments:  1
    Number of Output Arguments:  1
\end{funcdesc}

\begin{funcdesc}{erf_Z_e}{...}\index{erf_Z_e}

    Number of Input  Arguments:  1
    Number of Output Arguments:  2

The error flag is discarded.
Return Arguments 1 and 2 resemble a gsl_result argument,
	which is  argument 1 of the C argument list

\end{funcdesc}

\begin{funcdesc}{erf_e}{...}\index{erf_e}

    Number of Input  Arguments:  1
    Number of Output Arguments:  2

The error flag is discarded.
Return Arguments 1 and 2 resemble a gsl_result argument,
	which is  argument 1 of the C argument list

\end{funcdesc}

\begin{funcdesc}{erfc}{...}\index{erfc}

    Number of Input  Arguments:  1
    Number of Output Arguments:  1
\end{funcdesc}

\begin{funcdesc}{erfc_e}{...}\index{erfc_e}

    Number of Input  Arguments:  1
    Number of Output Arguments:  2

The error flag is discarded.
Return Arguments 1 and 2 resemble a gsl_result argument,
	which is  argument 1 of the C argument list

\end{funcdesc}

\begin{funcdesc}{eta}{...}\index{eta}

    Number of Input  Arguments:  1
    Number of Output Arguments:  1
\end{funcdesc}

\begin{funcdesc}{eta_e}{...}\index{eta_e}

    Number of Input  Arguments:  1
    Number of Output Arguments:  2

The error flag is discarded.
Return Arguments 1 and 2 resemble a gsl_result argument,
	which is  argument 1 of the C argument list

\end{funcdesc}

\begin{funcdesc}{eta_int}{...}\index{eta_int}

    Number of Input  Arguments:  1
    Number of Output Arguments:  1
\end{funcdesc}

\begin{funcdesc}{eta_int_e}{...}\index{eta_int_e}

    Number of Input  Arguments:  1
    Number of Output Arguments:  2

The error flag is discarded.
Return Arguments 1 and 2 resemble a gsl_result argument,
	which is  argument 1 of the C argument list

\end{funcdesc}

\begin{funcdesc}{expint_3}{...}\index{expint_3}

    Number of Input  Arguments:  1
    Number of Output Arguments:  1
\end{funcdesc}

\begin{funcdesc}{expint_3_e}{...}\index{expint_3_e}

    Number of Input  Arguments:  1
    Number of Output Arguments:  2

The error flag is discarded.
Return Arguments 1 and 2 resemble a gsl_result argument,
	which is  argument 1 of the C argument list

\end{funcdesc}

\begin{funcdesc}{expint_E1}{...}\index{expint_E1}

    Number of Input  Arguments:  1
    Number of Output Arguments:  1
\end{funcdesc}

\begin{funcdesc}{expint_E1_e}{...}\index{expint_E1_e}

    Number of Input  Arguments:  1
    Number of Output Arguments:  2

The error flag is discarded.
Return Arguments 1 and 2 resemble a gsl_result argument,
	which is  argument 1 of the C argument list

\end{funcdesc}

\begin{funcdesc}{expint_E1_scaled}{...}\index{expint_E1_scaled}

    Number of Input  Arguments:  1
    Number of Output Arguments:  1
\end{funcdesc}

\begin{funcdesc}{expint_E1_scaled_e}{...}\index{expint_E1_scaled_e}

    Number of Input  Arguments:  1
    Number of Output Arguments:  2

The error flag is discarded.
Return Arguments 1 and 2 resemble a gsl_result argument,
	which is  argument 1 of the C argument list

\end{funcdesc}

\begin{funcdesc}{expint_E2}{...}\index{expint_E2}

    Number of Input  Arguments:  1
    Number of Output Arguments:  1
\end{funcdesc}

\begin{funcdesc}{expint_E2_e}{...}\index{expint_E2_e}

    Number of Input  Arguments:  1
    Number of Output Arguments:  2

The error flag is discarded.
Return Arguments 1 and 2 resemble a gsl_result argument,
	which is  argument 1 of the C argument list

\end{funcdesc}

\begin{funcdesc}{expint_E2_scaled}{...}\index{expint_E2_scaled}

    Number of Input  Arguments:  1
    Number of Output Arguments:  1
\end{funcdesc}

\begin{funcdesc}{expint_E2_scaled_e}{...}\index{expint_E2_scaled_e}

    Number of Input  Arguments:  1
    Number of Output Arguments:  2

The error flag is discarded.
Return Arguments 1 and 2 resemble a gsl_result argument,
	which is  argument 1 of the C argument list

\end{funcdesc}

\begin{funcdesc}{expint_Ei}{...}\index{expint_Ei}

    Number of Input  Arguments:  1
    Number of Output Arguments:  1
\end{funcdesc}

\begin{funcdesc}{expint_Ei_e}{...}\index{expint_Ei_e}

    Number of Input  Arguments:  1
    Number of Output Arguments:  2

The error flag is discarded.
Return Arguments 1 and 2 resemble a gsl_result argument,
	which is  argument 1 of the C argument list

\end{funcdesc}

\begin{funcdesc}{expint_Ei_scaled}{...}\index{expint_Ei_scaled}

    Number of Input  Arguments:  1
    Number of Output Arguments:  1
\end{funcdesc}

\begin{funcdesc}{expint_Ei_scaled_e}{...}\index{expint_Ei_scaled_e}

    Number of Input  Arguments:  1
    Number of Output Arguments:  2

The error flag is discarded.
Return Arguments 1 and 2 resemble a gsl_result argument,
	which is  argument 1 of the C argument list

\end{funcdesc}

\begin{funcdesc}{fact}{...}\index{fact}

    Number of Input  Arguments:  1
    Number of Output Arguments:  1
\end{funcdesc}

\begin{funcdesc}{fact_e}{...}\index{fact_e}

    Number of Input  Arguments:  1
    Number of Output Arguments:  2

The error flag is discarded.
Return Arguments 1 and 2 resemble a gsl_result argument,
	which is  argument 1 of the C argument list

\end{funcdesc}

\begin{funcdesc}{fermi_dirac_0}{...}\index{fermi_dirac_0}

    Number of Input  Arguments:  1
    Number of Output Arguments:  1
\end{funcdesc}

\begin{funcdesc}{fermi_dirac_0_e}{...}\index{fermi_dirac_0_e}

    Number of Input  Arguments:  1
    Number of Output Arguments:  2

The error flag is discarded.
Return Arguments 1 and 2 resemble a gsl_result argument,
	which is  argument 1 of the C argument list

\end{funcdesc}

\begin{funcdesc}{fermi_dirac_1}{...}\index{fermi_dirac_1}

    Number of Input  Arguments:  1
    Number of Output Arguments:  1
\end{funcdesc}

\begin{funcdesc}{fermi_dirac_1_e}{...}\index{fermi_dirac_1_e}

    Number of Input  Arguments:  1
    Number of Output Arguments:  2

The error flag is discarded.
Return Arguments 1 and 2 resemble a gsl_result argument,
	which is  argument 1 of the C argument list

\end{funcdesc}

\begin{funcdesc}{fermi_dirac_2}{...}\index{fermi_dirac_2}

    Number of Input  Arguments:  1
    Number of Output Arguments:  1
\end{funcdesc}

\begin{funcdesc}{fermi_dirac_2_e}{...}\index{fermi_dirac_2_e}

    Number of Input  Arguments:  1
    Number of Output Arguments:  2

The error flag is discarded.
Return Arguments 1 and 2 resemble a gsl_result argument,
	which is  argument 1 of the C argument list

\end{funcdesc}

\begin{funcdesc}{fermi_dirac_3half}{...}\index{fermi_dirac_3half}

    Number of Input  Arguments:  1
    Number of Output Arguments:  1
\end{funcdesc}

\begin{funcdesc}{fermi_dirac_3half_e}{...}\index{fermi_dirac_3half_e}

    Number of Input  Arguments:  1
    Number of Output Arguments:  2

The error flag is discarded.
Return Arguments 1 and 2 resemble a gsl_result argument,
	which is  argument 1 of the C argument list

\end{funcdesc}

\begin{funcdesc}{fermi_dirac_half}{...}\index{fermi_dirac_half}

    Number of Input  Arguments:  1
    Number of Output Arguments:  1
\end{funcdesc}

\begin{funcdesc}{fermi_dirac_half_e}{...}\index{fermi_dirac_half_e}

    Number of Input  Arguments:  1
    Number of Output Arguments:  2

The error flag is discarded.
Return Arguments 1 and 2 resemble a gsl_result argument,
	which is  argument 1 of the C argument list

\end{funcdesc}

\begin{funcdesc}{fermi_dirac_inc_0}{...}\index{fermi_dirac_inc_0}

    Number of Input  Arguments:  2
    Number of Output Arguments:  1
\end{funcdesc}

\begin{funcdesc}{fermi_dirac_inc_0_e}{...}\index{fermi_dirac_inc_0_e}

    Number of Input  Arguments:  2
    Number of Output Arguments:  2

The error flag is discarded.
Return Arguments 1 and 2 resemble a gsl_result argument,
	which is  argument 2 of the C argument list

\end{funcdesc}

\begin{funcdesc}{fermi_dirac_int}{...}\index{fermi_dirac_int}

    Number of Input  Arguments:  2
    Number of Output Arguments:  1
\end{funcdesc}

\begin{funcdesc}{fermi_dirac_int_e}{...}\index{fermi_dirac_int_e}

    Number of Input  Arguments:  2
    Number of Output Arguments:  2

The error flag is discarded.
Return Arguments 1 and 2 resemble a gsl_result argument,
	which is  argument 2 of the C argument list

\end{funcdesc}

\begin{funcdesc}{fermi_dirac_m1}{...}\index{fermi_dirac_m1}

    Number of Input  Arguments:  1
    Number of Output Arguments:  1
\end{funcdesc}

\begin{funcdesc}{fermi_dirac_m1_e}{...}\index{fermi_dirac_m1_e}

    Number of Input  Arguments:  1
    Number of Output Arguments:  2

The error flag is discarded.
Return Arguments 1 and 2 resemble a gsl_result argument,
	which is  argument 1 of the C argument list

\end{funcdesc}

\begin{funcdesc}{fermi_dirac_mhalf}{...}\index{fermi_dirac_mhalf}

    Number of Input  Arguments:  1
    Number of Output Arguments:  1
\end{funcdesc}

\begin{funcdesc}{fermi_dirac_mhalf_e}{...}\index{fermi_dirac_mhalf_e}

    Number of Input  Arguments:  1
    Number of Output Arguments:  2

The error flag is discarded.
Return Arguments 1 and 2 resemble a gsl_result argument,
	which is  argument 1 of the C argument list

\end{funcdesc}

\begin{funcdesc}{gamma}{...}\index{gamma}

    Number of Input  Arguments:  1
    Number of Output Arguments:  1
\end{funcdesc}

\begin{funcdesc}{gamma_e}{...}\index{gamma_e}

    Number of Input  Arguments:  1
    Number of Output Arguments:  2

The error flag is discarded.
Return Arguments 1 and 2 resemble a gsl_result argument,
	which is  argument 1 of the C argument list

\end{funcdesc}

\begin{funcdesc}{gamma_inc_P}{...}\index{gamma_inc_P}

    Number of Input  Arguments:  2
    Number of Output Arguments:  1
\end{funcdesc}

\begin{funcdesc}{gamma_inc_P_e}{...}\index{gamma_inc_P_e}

    Number of Input  Arguments:  2
    Number of Output Arguments:  2

The error flag is discarded.
Return Arguments 1 and 2 resemble a gsl_result argument,
	which is  argument 2 of the C argument list

\end{funcdesc}

\begin{funcdesc}{gamma_inc_Q}{...}\index{gamma_inc_Q}

    Number of Input  Arguments:  2
    Number of Output Arguments:  1
\end{funcdesc}

\begin{funcdesc}{gamma_inc_Q_e}{...}\index{gamma_inc_Q_e}

    Number of Input  Arguments:  2
    Number of Output Arguments:  2

The error flag is discarded.
Return Arguments 1 and 2 resemble a gsl_result argument,
	which is  argument 2 of the C argument list

\end{funcdesc}

\begin{funcdesc}{gammainv}{...}\index{gammainv}

    Number of Input  Arguments:  1
    Number of Output Arguments:  1
\end{funcdesc}

\begin{funcdesc}{gammainv_e}{...}\index{gammainv_e}

    Number of Input  Arguments:  1
    Number of Output Arguments:  2

The error flag is discarded.
Return Arguments 1 and 2 resemble a gsl_result argument,
	which is  argument 1 of the C argument list

\end{funcdesc}

\begin{funcdesc}{gammastar}{...}\index{gammastar}

    Number of Input  Arguments:  1
    Number of Output Arguments:  1
\end{funcdesc}

\begin{funcdesc}{gammastar_e}{...}\index{gammastar_e}

    Number of Input  Arguments:  1
    Number of Output Arguments:  2

The error flag is discarded.
Return Arguments 1 and 2 resemble a gsl_result argument,
	which is  argument 1 of the C argument list

\end{funcdesc}

\begin{funcdesc}{gegenpoly_1}{...}\index{gegenpoly_1}

    Number of Input  Arguments:  2
    Number of Output Arguments:  1
\end{funcdesc}

\begin{funcdesc}{gegenpoly_1_e}{...}\index{gegenpoly_1_e}

    Number of Input  Arguments:  2
    Number of Output Arguments:  2

The error flag is discarded.
Return Arguments 1 and 2 resemble a gsl_result argument,
	which is  argument 2 of the C argument list

\end{funcdesc}

\begin{funcdesc}{gegenpoly_2}{...}\index{gegenpoly_2}

    Number of Input  Arguments:  2
    Number of Output Arguments:  1
\end{funcdesc}

\begin{funcdesc}{gegenpoly_2_e}{...}\index{gegenpoly_2_e}

    Number of Input  Arguments:  2
    Number of Output Arguments:  2

The error flag is discarded.
Return Arguments 1 and 2 resemble a gsl_result argument,
	which is  argument 2 of the C argument list

\end{funcdesc}

\begin{funcdesc}{gegenpoly_3}{...}\index{gegenpoly_3}

    Number of Input  Arguments:  2
    Number of Output Arguments:  1
\end{funcdesc}

\begin{funcdesc}{gegenpoly_3_e}{...}\index{gegenpoly_3_e}

    Number of Input  Arguments:  2
    Number of Output Arguments:  2

The error flag is discarded.
Return Arguments 1 and 2 resemble a gsl_result argument,
	which is  argument 2 of the C argument list

\end{funcdesc}

\begin{funcdesc}{gegenpoly_n}{...}\index{gegenpoly_n}

    Number of Input  Arguments:  3
    Number of Output Arguments:  1
\end{funcdesc}

\begin{funcdesc}{gegenpoly_n_e}{...}\index{gegenpoly_n_e}

    Number of Input  Arguments:  3
    Number of Output Arguments:  2

The error flag is discarded.
Return Arguments 1 and 2 resemble a gsl_result argument,
	which is  argument 3 of the C argument list

\end{funcdesc}

\begin{funcdesc}{hydrogenicR}{...}\index{hydrogenicR}

    Number of Input  Arguments:  4
    Number of Output Arguments:  1
\end{funcdesc}

\begin{funcdesc}{hydrogenicR_1}{...}\index{hydrogenicR_1}

    Number of Input  Arguments:  2
    Number of Output Arguments:  1
\end{funcdesc}

\begin{funcdesc}{hydrogenicR_1_e}{...}\index{hydrogenicR_1_e}

    Number of Input  Arguments:  2
    Number of Output Arguments:  2

The error flag is discarded.
Return Arguments 1 and 2 resemble a gsl_result argument,
	which is  argument 2 of the C argument list

\end{funcdesc}

\begin{funcdesc}{hydrogenicR_e}{...}\index{hydrogenicR_e}

    Number of Input  Arguments:  4
    Number of Output Arguments:  2

The error flag is discarded.
Return Arguments 1 and 2 resemble a gsl_result argument,
	which is  argument 4 of the C argument list

\end{funcdesc}

\begin{funcdesc}{hyperg_0F1}{...}\index{hyperg_0F1}

    Number of Input  Arguments:  2
    Number of Output Arguments:  1
\end{funcdesc}

\begin{funcdesc}{hyperg_0F1_e}{...}\index{hyperg_0F1_e}

    Number of Input  Arguments:  2
    Number of Output Arguments:  2

The error flag is discarded.
Return Arguments 1 and 2 resemble a gsl_result argument,
	which is  argument 2 of the C argument list

\end{funcdesc}

\begin{funcdesc}{hyperg_1F1}{...}\index{hyperg_1F1}

    Number of Input  Arguments:  3
    Number of Output Arguments:  1
\end{funcdesc}

\begin{funcdesc}{hyperg_1F1_e}{...}\index{hyperg_1F1_e}

    Number of Input  Arguments:  3
    Number of Output Arguments:  2

The error flag is discarded.
Return Arguments 1 and 2 resemble a gsl_result argument,
	which is  argument 3 of the C argument list

\end{funcdesc}

\begin{funcdesc}{hyperg_1F1_int}{...}\index{hyperg_1F1_int}

    Number of Input  Arguments:  3
    Number of Output Arguments:  1
\end{funcdesc}

\begin{funcdesc}{hyperg_1F1_int_e}{...}\index{hyperg_1F1_int_e}

    Number of Input  Arguments:  3
    Number of Output Arguments:  2

The error flag is discarded.
Return Arguments 1 and 2 resemble a gsl_result argument,
	which is  argument 3 of the C argument list

\end{funcdesc}

\begin{funcdesc}{hyperg_2F0}{...}\index{hyperg_2F0}

    Number of Input  Arguments:  3
    Number of Output Arguments:  1
\end{funcdesc}

\begin{funcdesc}{hyperg_2F0_e}{...}\index{hyperg_2F0_e}

    Number of Input  Arguments:  3
    Number of Output Arguments:  2

The error flag is discarded.
Return Arguments 1 and 2 resemble a gsl_result argument,
	which is  argument 3 of the C argument list

\end{funcdesc}

\begin{funcdesc}{hyperg_2F1}{...}\index{hyperg_2F1}

    Number of Input  Arguments:  4
    Number of Output Arguments:  1
\end{funcdesc}

\begin{funcdesc}{hyperg_2F1_conj}{...}\index{hyperg_2F1_conj}

    Number of Input  Arguments:  4
    Number of Output Arguments:  1
\end{funcdesc}

\begin{funcdesc}{hyperg_2F1_conj_e}{...}\index{hyperg_2F1_conj_e}

    Number of Input  Arguments:  4
    Number of Output Arguments:  2

The error flag is discarded.
Return Arguments 1 and 2 resemble a gsl_result argument,
	which is  argument 4 of the C argument list

\end{funcdesc}

\begin{funcdesc}{hyperg_2F1_conj_renorm}{...}\index{hyperg_2F1_conj_renorm}

    Number of Input  Arguments:  4
    Number of Output Arguments:  1
\end{funcdesc}

\begin{funcdesc}{hyperg_2F1_conj_renorm_e}{...}\index{hyperg_2F1_conj_renorm_e}

    Number of Input  Arguments:  4
    Number of Output Arguments:  2

The error flag is discarded.
Return Arguments 1 and 2 resemble a gsl_result argument,
	which is  argument 4 of the C argument list

\end{funcdesc}

\begin{funcdesc}{hyperg_2F1_e}{...}\index{hyperg_2F1_e}

    Number of Input  Arguments:  4
    Number of Output Arguments:  2

The error flag is discarded.
Return Arguments 1 and 2 resemble a gsl_result argument,
	which is  argument 4 of the C argument list

\end{funcdesc}

\begin{funcdesc}{hyperg_2F1_renorm}{...}\index{hyperg_2F1_renorm}

    Number of Input  Arguments:  4
    Number of Output Arguments:  1
\end{funcdesc}

\begin{funcdesc}{hyperg_2F1_renorm_e}{...}\index{hyperg_2F1_renorm_e}

    Number of Input  Arguments:  4
    Number of Output Arguments:  2

The error flag is discarded.
Return Arguments 1 and 2 resemble a gsl_result argument,
	which is  argument 4 of the C argument list

\end{funcdesc}

\begin{funcdesc}{hyperg_U}{...}\index{hyperg_U}

    Number of Input  Arguments:  3
    Number of Output Arguments:  1
\end{funcdesc}

\begin{funcdesc}{hyperg_U_e}{...}\index{hyperg_U_e}

    Number of Input  Arguments:  3
    Number of Output Arguments:  2

The error flag is discarded.
Return Arguments 1 and 2 resemble a gsl_result argument,
	which is  argument 3 of the C argument list

\end{funcdesc}

\begin{funcdesc}{hyperg_U_e10_e}{...}\index{hyperg_U_e10_e}

    Number of Input  Arguments:  3
    Number of Output Arguments:  3

The error flag is discarded.
Return Arguments 1 - 3 resemble a gsl_result_e10 argument,
	which is argument 3 of the C argument list

\end{funcdesc}

\begin{funcdesc}{hyperg_U_int}{...}\index{hyperg_U_int}

    Number of Input  Arguments:  3
    Number of Output Arguments:  1
\end{funcdesc}

\begin{funcdesc}{hyperg_U_int_e}{...}\index{hyperg_U_int_e}

    Number of Input  Arguments:  3
    Number of Output Arguments:  2

The error flag is discarded.
Return Arguments 1 and 2 resemble a gsl_result argument,
	which is  argument 3 of the C argument list

\end{funcdesc}

\begin{funcdesc}{hyperg_U_int_e10_e}{...}\index{hyperg_U_int_e10_e}

    Number of Input  Arguments:  3
    Number of Output Arguments:  3

The error flag is discarded.
Return Arguments 1 - 3 resemble a gsl_result_e10 argument,
	which is argument 3 of the C argument list

\end{funcdesc}

\begin{funcdesc}{hypot}{...}\index{hypot}

    Number of Input  Arguments:  2
    Number of Output Arguments:  1
\end{funcdesc}

\begin{funcdesc}{hypot_e}{...}\index{hypot_e}

    Number of Input  Arguments:  2
    Number of Output Arguments:  2

The error flag is discarded.
Return Arguments 1 and 2 resemble a gsl_result argument,
	which is  argument 2 of the C argument list

\end{funcdesc}

\begin{funcdesc}{hzeta}{...}\index{hzeta}

    Number of Input  Arguments:  2
    Number of Output Arguments:  1
\end{funcdesc}

\begin{funcdesc}{hzeta_e}{...}\index{hzeta_e}

    Number of Input  Arguments:  2
    Number of Output Arguments:  2

The error flag is discarded.
Return Arguments 1 and 2 resemble a gsl_result argument,
	which is  argument 2 of the C argument list

\end{funcdesc}

\begin{funcdesc}{laguerre_1}{...}\index{laguerre_1}

    Number of Input  Arguments:  2
    Number of Output Arguments:  1
\end{funcdesc}

\begin{funcdesc}{laguerre_1_e}{...}\index{laguerre_1_e}

    Number of Input  Arguments:  2
    Number of Output Arguments:  2

The error flag is discarded.
Return Arguments 1 and 2 resemble a gsl_result argument,
	which is  argument 2 of the C argument list

\end{funcdesc}

\begin{funcdesc}{laguerre_2}{...}\index{laguerre_2}

    Number of Input  Arguments:  2
    Number of Output Arguments:  1
\end{funcdesc}

\begin{funcdesc}{laguerre_2_e}{...}\index{laguerre_2_e}

    Number of Input  Arguments:  2
    Number of Output Arguments:  2

The error flag is discarded.
Return Arguments 1 and 2 resemble a gsl_result argument,
	which is  argument 2 of the C argument list

\end{funcdesc}

\begin{funcdesc}{laguerre_3}{...}\index{laguerre_3}

    Number of Input  Arguments:  2
    Number of Output Arguments:  1
\end{funcdesc}

\begin{funcdesc}{laguerre_3_e}{...}\index{laguerre_3_e}

    Number of Input  Arguments:  2
    Number of Output Arguments:  2

The error flag is discarded.
Return Arguments 1 and 2 resemble a gsl_result argument,
	which is  argument 2 of the C argument list

\end{funcdesc}

\begin{funcdesc}{laguerre_n}{...}\index{laguerre_n}

    Number of Input  Arguments:  3
    Number of Output Arguments:  1
\end{funcdesc}

\begin{funcdesc}{laguerre_n_e}{...}\index{laguerre_n_e}

    Number of Input  Arguments:  3
    Number of Output Arguments:  2

The error flag is discarded.
Return Arguments 1 and 2 resemble a gsl_result argument,
	which is  argument 3 of the C argument list

\end{funcdesc}

\begin{funcdesc}{lambert_W0}{...}\index{lambert_W0}

    Number of Input  Arguments:  1
    Number of Output Arguments:  1
\end{funcdesc}

\begin{funcdesc}{lambert_W0_e}{...}\index{lambert_W0_e}

    Number of Input  Arguments:  1
    Number of Output Arguments:  2

The error flag is discarded.
Return Arguments 1 and 2 resemble a gsl_result argument,
	which is  argument 1 of the C argument list

\end{funcdesc}

\begin{funcdesc}{lambert_Wm1}{...}\index{lambert_Wm1}

    Number of Input  Arguments:  1
    Number of Output Arguments:  1
\end{funcdesc}

\begin{funcdesc}{lambert_Wm1_e}{...}\index{lambert_Wm1_e}

    Number of Input  Arguments:  1
    Number of Output Arguments:  2

The error flag is discarded.
Return Arguments 1 and 2 resemble a gsl_result argument,
	which is  argument 1 of the C argument list

\end{funcdesc}

\begin{funcdesc}{legendre_H3d}{...}\index{legendre_H3d}

    Number of Input  Arguments:  3
    Number of Output Arguments:  1
\end{funcdesc}

\begin{funcdesc}{legendre_H3d_0}{...}\index{legendre_H3d_0}

    Number of Input  Arguments:  2
    Number of Output Arguments:  1
\end{funcdesc}

\begin{funcdesc}{legendre_H3d_0_e}{...}\index{legendre_H3d_0_e}

    Number of Input  Arguments:  2
    Number of Output Arguments:  2

The error flag is discarded.
Return Arguments 1 and 2 resemble a gsl_result argument,
	which is  argument 2 of the C argument list

\end{funcdesc}

\begin{funcdesc}{legendre_H3d_1}{...}\index{legendre_H3d_1}

    Number of Input  Arguments:  2
    Number of Output Arguments:  1
\end{funcdesc}

\begin{funcdesc}{legendre_H3d_1_e}{...}\index{legendre_H3d_1_e}

    Number of Input  Arguments:  2
    Number of Output Arguments:  2

The error flag is discarded.
Return Arguments 1 and 2 resemble a gsl_result argument,
	which is  argument 2 of the C argument list

\end{funcdesc}

\begin{funcdesc}{legendre_H3d_e}{...}\index{legendre_H3d_e}

    Number of Input  Arguments:  3
    Number of Output Arguments:  2

The error flag is discarded.
Return Arguments 1 and 2 resemble a gsl_result argument,
	which is  argument 3 of the C argument list

\end{funcdesc}

\begin{funcdesc}{legendre_P1}{...}\index{legendre_P1}

    Number of Input  Arguments:  1
    Number of Output Arguments:  1
\end{funcdesc}

\begin{funcdesc}{legendre_P1_e}{...}\index{legendre_P1_e}

    Number of Input  Arguments:  1
    Number of Output Arguments:  2

The error flag is discarded.
Return Arguments 1 and 2 resemble a gsl_result argument,
	which is  argument 1 of the C argument list

\end{funcdesc}

\begin{funcdesc}{legendre_P2}{...}\index{legendre_P2}

    Number of Input  Arguments:  1
    Number of Output Arguments:  1
\end{funcdesc}

\begin{funcdesc}{legendre_P2_e}{...}\index{legendre_P2_e}

    Number of Input  Arguments:  1
    Number of Output Arguments:  2

The error flag is discarded.
Return Arguments 1 and 2 resemble a gsl_result argument,
	which is  argument 1 of the C argument list

\end{funcdesc}

\begin{funcdesc}{legendre_P3}{...}\index{legendre_P3}

    Number of Input  Arguments:  1
    Number of Output Arguments:  1
\end{funcdesc}

\begin{funcdesc}{legendre_P3_e}{...}\index{legendre_P3_e}

    Number of Input  Arguments:  1
    Number of Output Arguments:  2

The error flag is discarded.
Return Arguments 1 and 2 resemble a gsl_result argument,
	which is  argument 1 of the C argument list

\end{funcdesc}

\begin{funcdesc}{legendre_Pl}{...}\index{legendre_Pl}

    Number of Input  Arguments:  2
    Number of Output Arguments:  1
\end{funcdesc}

\begin{funcdesc}{legendre_Pl_e}{...}\index{legendre_Pl_e}

    Number of Input  Arguments:  2
    Number of Output Arguments:  2

The error flag is discarded.
Return Arguments 1 and 2 resemble a gsl_result argument,
	which is  argument 2 of the C argument list

\end{funcdesc}

\begin{funcdesc}{legendre_Plm}{...}\index{legendre_Plm}

    Number of Input  Arguments:  3
    Number of Output Arguments:  1
\end{funcdesc}

\begin{funcdesc}{legendre_Plm_e}{...}\index{legendre_Plm_e}

    Number of Input  Arguments:  3
    Number of Output Arguments:  2

The error flag is discarded.
Return Arguments 1 and 2 resemble a gsl_result argument,
	which is  argument 3 of the C argument list

\end{funcdesc}

\begin{funcdesc}{legendre_Q0}{...}\index{legendre_Q0}

    Number of Input  Arguments:  1
    Number of Output Arguments:  1
\end{funcdesc}

\begin{funcdesc}{legendre_Q0_e}{...}\index{legendre_Q0_e}

    Number of Input  Arguments:  1
    Number of Output Arguments:  2

The error flag is discarded.
Return Arguments 1 and 2 resemble a gsl_result argument,
	which is  argument 1 of the C argument list

\end{funcdesc}

\begin{funcdesc}{legendre_Q1}{...}\index{legendre_Q1}

    Number of Input  Arguments:  1
    Number of Output Arguments:  1
\end{funcdesc}

\begin{funcdesc}{legendre_Q1_e}{...}\index{legendre_Q1_e}

    Number of Input  Arguments:  1
    Number of Output Arguments:  2

The error flag is discarded.
Return Arguments 1 and 2 resemble a gsl_result argument,
	which is  argument 1 of the C argument list

\end{funcdesc}

\begin{funcdesc}{legendre_Ql}{...}\index{legendre_Ql}

    Number of Input  Arguments:  2
    Number of Output Arguments:  1
\end{funcdesc}

\begin{funcdesc}{legendre_Ql_e}{...}\index{legendre_Ql_e}

    Number of Input  Arguments:  2
    Number of Output Arguments:  2

The error flag is discarded.
Return Arguments 1 and 2 resemble a gsl_result argument,
	which is  argument 2 of the C argument list

\end{funcdesc}

\begin{funcdesc}{legendre_sphPlm}{...}\index{legendre_sphPlm}

    Number of Input  Arguments:  3
    Number of Output Arguments:  1
\end{funcdesc}

\begin{funcdesc}{legendre_sphPlm_e}{...}\index{legendre_sphPlm_e}

    Number of Input  Arguments:  3
    Number of Output Arguments:  2

The error flag is discarded.
Return Arguments 1 and 2 resemble a gsl_result argument,
	which is  argument 3 of the C argument list

\end{funcdesc}

\begin{funcdesc}{lnbeta}{...}\index{lnbeta}

    Number of Input  Arguments:  2
    Number of Output Arguments:  1
\end{funcdesc}

\begin{funcdesc}{lnbeta_e}{...}\index{lnbeta_e}

    Number of Input  Arguments:  2
    Number of Output Arguments:  2

The error flag is discarded.
Return Arguments 1 and 2 resemble a gsl_result argument,
	which is  argument 2 of the C argument list

\end{funcdesc}

\begin{funcdesc}{lnchoose}{...}\index{lnchoose}

    Number of Input  Arguments:  2
    Number of Output Arguments:  1
\end{funcdesc}

\begin{funcdesc}{lnchoose_e}{...}\index{lnchoose_e}

    Number of Input  Arguments:  2
    Number of Output Arguments:  2

The error flag is discarded.
Return Arguments 1 and 2 resemble a gsl_result argument,
	which is  argument 2 of the C argument list

\end{funcdesc}

\begin{funcdesc}{lncosh}{...}\index{lncosh}

    Number of Input  Arguments:  1
    Number of Output Arguments:  1
\end{funcdesc}

\begin{funcdesc}{lncosh_e}{...}\index{lncosh_e}

    Number of Input  Arguments:  1
    Number of Output Arguments:  2

The error flag is discarded.
Return Arguments 1 and 2 resemble a gsl_result argument,
	which is  argument 1 of the C argument list

\end{funcdesc}

\begin{funcdesc}{lndoublefact}{...}\index{lndoublefact}

    Number of Input  Arguments:  1
    Number of Output Arguments:  1
\end{funcdesc}

\begin{funcdesc}{lndoublefact_e}{...}\index{lndoublefact_e}

    Number of Input  Arguments:  1
    Number of Output Arguments:  2

The error flag is discarded.
Return Arguments 1 and 2 resemble a gsl_result argument,
	which is  argument 1 of the C argument list

\end{funcdesc}

\begin{funcdesc}{lnfact}{...}\index{lnfact}

    Number of Input  Arguments:  1
    Number of Output Arguments:  1
\end{funcdesc}

\begin{funcdesc}{lnfact_e}{...}\index{lnfact_e}

    Number of Input  Arguments:  1
    Number of Output Arguments:  2

The error flag is discarded.
Return Arguments 1 and 2 resemble a gsl_result argument,
	which is  argument 1 of the C argument list

\end{funcdesc}

\begin{funcdesc}{lngamma}{...}\index{lngamma}

    Number of Input  Arguments:  1
    Number of Output Arguments:  1
\end{funcdesc}

\begin{funcdesc}{lngamma_e}{...}\index{lngamma_e}

    Number of Input  Arguments:  1
    Number of Output Arguments:  2

The error flag is discarded.
Return Arguments 1 and 2 resemble a gsl_result argument,
	which is  argument 1 of the C argument list

\end{funcdesc}

\begin{funcdesc}{lngamma_sgn_e}{...}\index{lngamma_sgn_e}

    Number of Input  Arguments:  1
    Number of Output Arguments:  3

The error flag is discarded.
Return Arguments 1 and 2 resemble a gsl_result argument,
	which is  argument 1 of the C argument list

\end{funcdesc}

\begin{funcdesc}{lnpoch}{...}\index{lnpoch}

    Number of Input  Arguments:  2
    Number of Output Arguments:  1
\end{funcdesc}

\begin{funcdesc}{lnpoch_e}{...}\index{lnpoch_e}

    Number of Input  Arguments:  2
    Number of Output Arguments:  2

The error flag is discarded.
Return Arguments 1 and 2 resemble a gsl_result argument,
	which is  argument 2 of the C argument list

\end{funcdesc}

\begin{funcdesc}{lnpoch_sgn_e}{...}\index{lnpoch_sgn_e}

    Number of Input  Arguments:  2
    Number of Output Arguments:  3

The error flag is discarded.
Return Arguments 1 and 2 resemble a gsl_result argument,
	which is  argument 2 of the C argument list

\end{funcdesc}

\begin{funcdesc}{lnsinh}{...}\index{lnsinh}

    Number of Input  Arguments:  1
    Number of Output Arguments:  1
\end{funcdesc}

\begin{funcdesc}{lnsinh_e}{...}\index{lnsinh_e}

    Number of Input  Arguments:  1
    Number of Output Arguments:  2

The error flag is discarded.
Return Arguments 1 and 2 resemble a gsl_result argument,
	which is  argument 1 of the C argument list

\end{funcdesc}

\begin{funcdesc}{log}{...}\index{log}

    Number of Input  Arguments:  1
    Number of Output Arguments:  1
\end{funcdesc}

\begin{funcdesc}{log_1plusx}{...}\index{log_1plusx}

    Number of Input  Arguments:  1
    Number of Output Arguments:  1
\end{funcdesc}

\begin{funcdesc}{log_1plusx_e}{...}\index{log_1plusx_e}

    Number of Input  Arguments:  1
    Number of Output Arguments:  2

The error flag is discarded.
Return Arguments 1 and 2 resemble a gsl_result argument,
	which is  argument 1 of the C argument list

\end{funcdesc}

\begin{funcdesc}{log_1plusx_mx}{...}\index{log_1plusx_mx}

    Number of Input  Arguments:  1
    Number of Output Arguments:  1
\end{funcdesc}

\begin{funcdesc}{log_1plusx_mx_e}{...}\index{log_1plusx_mx_e}

    Number of Input  Arguments:  1
    Number of Output Arguments:  2

The error flag is discarded.
Return Arguments 1 and 2 resemble a gsl_result argument,
	which is  argument 1 of the C argument list

\end{funcdesc}

\begin{funcdesc}{log_abs}{...}\index{log_abs}

    Number of Input  Arguments:  1
    Number of Output Arguments:  1
\end{funcdesc}

\begin{funcdesc}{log_abs_e}{...}\index{log_abs_e}

    Number of Input  Arguments:  1
    Number of Output Arguments:  2

The error flag is discarded.
Return Arguments 1 and 2 resemble a gsl_result argument,
	which is  argument 1 of the C argument list

\end{funcdesc}

\begin{funcdesc}{log_e}{...}\index{log_e}

    Number of Input  Arguments:  1
    Number of Output Arguments:  2

The error flag is discarded.
Return Arguments 1 and 2 resemble a gsl_result argument,
	which is  argument 1 of the C argument list

\end{funcdesc}

\begin{funcdesc}{log_erfc}{...}\index{log_erfc}

    Number of Input  Arguments:  1
    Number of Output Arguments:  1
\end{funcdesc}

\begin{funcdesc}{log_erfc_e}{...}\index{log_erfc_e}

    Number of Input  Arguments:  1
    Number of Output Arguments:  2

The error flag is discarded.
Return Arguments 1 and 2 resemble a gsl_result argument,
	which is  argument 1 of the C argument list

\end{funcdesc}

\begin{funcdesc}{multiply}{...}\index{multiply}

    Number of Input  Arguments:  2
    Number of Output Arguments:  1
\end{funcdesc}

\begin{funcdesc}{multiply_e}{...}\index{multiply_e}

    Number of Input  Arguments:  2
    Number of Output Arguments:  2

The error flag is discarded.
Return Arguments 1 and 2 resemble a gsl_result argument,
	which is  argument 2 of the C argument list

\end{funcdesc}

\begin{funcdesc}{multiply_err_e}{...}\index{multiply_err_e}

    Number of Input  Arguments:  4
    Number of Output Arguments:  2

The error flag is discarded.
Return Arguments 1 and 2 resemble a gsl_result argument,
	which is  argument 4 of the C argument list

\end{funcdesc}

\begin{funcdesc}{poch}{...}\index{poch}

    Number of Input  Arguments:  2
    Number of Output Arguments:  1
\end{funcdesc}

\begin{funcdesc}{poch_e}{...}\index{poch_e}

    Number of Input  Arguments:  2
    Number of Output Arguments:  2

The error flag is discarded.
Return Arguments 1 and 2 resemble a gsl_result argument,
	which is  argument 2 of the C argument list

\end{funcdesc}

\begin{funcdesc}{pochrel}{...}\index{pochrel}

    Number of Input  Arguments:  2
    Number of Output Arguments:  1
\end{funcdesc}

\begin{funcdesc}{pochrel_e}{...}\index{pochrel_e}

    Number of Input  Arguments:  2
    Number of Output Arguments:  2

The error flag is discarded.
Return Arguments 1 and 2 resemble a gsl_result argument,
	which is  argument 2 of the C argument list

\end{funcdesc}

\begin{funcdesc}{polar_to_rect}{...}\index{polar_to_rect}

\end{funcdesc}

\begin{funcdesc}{pow_int}{...}\index{pow_int}

    Number of Input  Arguments:  2
    Number of Output Arguments:  1
\end{funcdesc}

\begin{funcdesc}{pow_int_e}{...}\index{pow_int_e}

    Number of Input  Arguments:  2
    Number of Output Arguments:  2

The error flag is discarded.
Return Arguments 1 and 2 resemble a gsl_result argument,
	which is  argument 2 of the C argument list

\end{funcdesc}

\begin{funcdesc}{psi}{...}\index{psi}

    Number of Input  Arguments:  1
    Number of Output Arguments:  1
\end{funcdesc}

\begin{funcdesc}{psi_1_int}{...}\index{psi_1_int}

    Number of Input  Arguments:  1
    Number of Output Arguments:  1
\end{funcdesc}

\begin{funcdesc}{psi_1_int_e}{...}\index{psi_1_int_e}

    Number of Input  Arguments:  1
    Number of Output Arguments:  2

The error flag is discarded.
Return Arguments 1 and 2 resemble a gsl_result argument,
	which is  argument 1 of the C argument list

\end{funcdesc}

\begin{funcdesc}{psi_1piy}{...}\index{psi_1piy}

    Number of Input  Arguments:  1
    Number of Output Arguments:  1
\end{funcdesc}

\begin{funcdesc}{psi_1piy_e}{...}\index{psi_1piy_e}

    Number of Input  Arguments:  1
    Number of Output Arguments:  2

The error flag is discarded.
Return Arguments 1 and 2 resemble a gsl_result argument,
	which is  argument 1 of the C argument list

\end{funcdesc}

\begin{funcdesc}{psi_e}{...}\index{psi_e}

    Number of Input  Arguments:  1
    Number of Output Arguments:  2

The error flag is discarded.
Return Arguments 1 and 2 resemble a gsl_result argument,
	which is  argument 1 of the C argument list

\end{funcdesc}

\begin{funcdesc}{psi_int}{...}\index{psi_int}

    Number of Input  Arguments:  1
    Number of Output Arguments:  1
\end{funcdesc}

\begin{funcdesc}{psi_int_e}{...}\index{psi_int_e}

    Number of Input  Arguments:  1
    Number of Output Arguments:  2

The error flag is discarded.
Return Arguments 1 and 2 resemble a gsl_result argument,
	which is  argument 1 of the C argument list

\end{funcdesc}

\begin{funcdesc}{psi_n}{...}\index{psi_n}

    Number of Input  Arguments:  2
    Number of Output Arguments:  1
\end{funcdesc}

\begin{funcdesc}{psi_n_e}{...}\index{psi_n_e}

    Number of Input  Arguments:  2
    Number of Output Arguments:  2

The error flag is discarded.
Return Arguments 1 and 2 resemble a gsl_result argument,
	which is  argument 2 of the C argument list

\end{funcdesc}

\begin{funcdesc}{rect_to_polar}{...}\index{rect_to_polar}

\end{funcdesc}

\begin{funcdesc}{sin}{...}\index{sin}

    Number of Input  Arguments:  1
    Number of Output Arguments:  1
\end{funcdesc}

\begin{funcdesc}{sin_e}{...}\index{sin_e}

    Number of Input  Arguments:  1
    Number of Output Arguments:  2

The error flag is discarded.
Return Arguments 1 and 2 resemble a gsl_result argument,
	which is  argument 1 of the C argument list

\end{funcdesc}

\begin{funcdesc}{sin_err_e}{...}\index{sin_err_e}

    Number of Input  Arguments:  2
    Number of Output Arguments:  2

The error flag is discarded.
Return Arguments 1 and 2 resemble a gsl_result argument,
	which is  argument 2 of the C argument list

\end{funcdesc}

\begin{funcdesc}{sinc}{...}\index{sinc}

    Number of Input  Arguments:  1
    Number of Output Arguments:  1
\end{funcdesc}

\begin{funcdesc}{sinc_e}{...}\index{sinc_e}

    Number of Input  Arguments:  1
    Number of Output Arguments:  2

The error flag is discarded.
Return Arguments 1 and 2 resemble a gsl_result argument,
	which is  argument 1 of the C argument list

\end{funcdesc}

\begin{funcdesc}{synchrotron_1}{...}\index{synchrotron_1}

    Number of Input  Arguments:  1
    Number of Output Arguments:  1
\end{funcdesc}

\begin{funcdesc}{synchrotron_1_e}{...}\index{synchrotron_1_e}

    Number of Input  Arguments:  1
    Number of Output Arguments:  2

The error flag is discarded.
Return Arguments 1 and 2 resemble a gsl_result argument,
	which is  argument 1 of the C argument list

\end{funcdesc}

\begin{funcdesc}{synchrotron_2}{...}\index{synchrotron_2}

    Number of Input  Arguments:  1
    Number of Output Arguments:  1
\end{funcdesc}

\begin{funcdesc}{synchrotron_2_e}{...}\index{synchrotron_2_e}

    Number of Input  Arguments:  1
    Number of Output Arguments:  2

The error flag is discarded.
Return Arguments 1 and 2 resemble a gsl_result argument,
	which is  argument 1 of the C argument list

\end{funcdesc}

\begin{funcdesc}{taylorcoeff}{...}\index{taylorcoeff}

    Number of Input  Arguments:  2
    Number of Output Arguments:  1
\end{funcdesc}

\begin{funcdesc}{taylorcoeff_e}{...}\index{taylorcoeff_e}

    Number of Input  Arguments:  2
    Number of Output Arguments:  2

The error flag is discarded.
Return Arguments 1 and 2 resemble a gsl_result argument,
	which is  argument 2 of the C argument list

\end{funcdesc}

\begin{funcdesc}{transport_2}{...}\index{transport_2}

    Number of Input  Arguments:  1
    Number of Output Arguments:  1
\end{funcdesc}

\begin{funcdesc}{transport_2_e}{...}\index{transport_2_e}

    Number of Input  Arguments:  1
    Number of Output Arguments:  2

The error flag is discarded.
Return Arguments 1 and 2 resemble a gsl_result argument,
	which is  argument 1 of the C argument list

\end{funcdesc}

\begin{funcdesc}{transport_3}{...}\index{transport_3}

    Number of Input  Arguments:  1
    Number of Output Arguments:  1
\end{funcdesc}

\begin{funcdesc}{transport_3_e}{...}\index{transport_3_e}

    Number of Input  Arguments:  1
    Number of Output Arguments:  2

The error flag is discarded.
Return Arguments 1 and 2 resemble a gsl_result argument,
	which is  argument 1 of the C argument list

\end{funcdesc}

\begin{funcdesc}{transport_4}{...}\index{transport_4}

    Number of Input  Arguments:  1
    Number of Output Arguments:  1
\end{funcdesc}

\begin{funcdesc}{transport_4_e}{...}\index{transport_4_e}

    Number of Input  Arguments:  1
    Number of Output Arguments:  2

The error flag is discarded.
Return Arguments 1 and 2 resemble a gsl_result argument,
	which is  argument 1 of the C argument list

\end{funcdesc}

\begin{funcdesc}{transport_5}{...}\index{transport_5}

    Number of Input  Arguments:  1
    Number of Output Arguments:  1
\end{funcdesc}

\begin{funcdesc}{transport_5_e}{...}\index{transport_5_e}

    Number of Input  Arguments:  1
    Number of Output Arguments:  2

The error flag is discarded.
Return Arguments 1 and 2 resemble a gsl_result argument,
	which is  argument 1 of the C argument list

\end{funcdesc}

\begin{funcdesc}{zeta}{...}\index{zeta}

    Number of Input  Arguments:  1
    Number of Output Arguments:  1
\end{funcdesc}

\begin{funcdesc}{zeta_e}{...}\index{zeta_e}

    Number of Input  Arguments:  1
    Number of Output Arguments:  2

The error flag is discarded.
Return Arguments 1 and 2 resemble a gsl_result argument,
	which is  argument 1 of the C argument list

\end{funcdesc}

\begin{funcdesc}{zeta_int}{...}\index{zeta_int}

    Number of Input  Arguments:  1
    Number of Output Arguments:  1
\end{funcdesc}

\begin{funcdesc}{zeta_int_e}{...}\index{zeta_int_e}

    Number of Input  Arguments:  1
    Number of Output Arguments:  2

The error flag is discarded.
Return Arguments 1 and 2 resemble a gsl_result argument,
	which is  argument 1 of the C argument list

\end{funcdesc}

\section{Ordinary Functions}

The following array functions have been wrapped. These are supposingly faster
than the equivalent functions from above.
\begin{funcdesc}{bessel_In_array}{...}\index{bessel_In_array}
\end{funcdesc}
\begin{funcdesc}{bessel_Jn_array}{...}\index{bessel_Jn_array}
\end{funcdesc}
\begin{funcdesc}{bessel_Kn_array}{...}\index{bessel_Kn_array}
\end{funcdesc}
\begin{funcdesc}{bessel_Kn_scaled_array}{...}\index{bessel_Kn_scaled_array}
\end{funcdesc}
\begin{funcdesc}{bessel_Yn_array}{...}\index{bessel_Yn_array}
\end{funcdesc}
\begin{funcdesc}{bessel_il_scaled_array}{...}\index{bessel_il_scaled_array}
\end{funcdesc}
\begin{funcdesc}{bessel_jl_array}{...}\index{bessel_jl_array}
\end{funcdesc}
\begin{funcdesc}{bessel_jl_steed_array}{...}\index{bessel_jl_steed_array}
\end{funcdesc}
\begin{funcdesc}{bessel_kl_scaled_array}{...}\index{bessel_kl_scaled_array}
\end{funcdesc}
\begin{funcdesc}{bessel_yl_array}{...}\index{bessel_yl_array}
\end{funcdesc}


%%% Local Variables: 
%%% mode: latex
%%% TeX-master: "ref"
%%% End: 
