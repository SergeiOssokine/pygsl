% Complete documentation on the extended LaTeX markup used for Python
% documentation is available in ``Documenting Python'', which is part
% of the standard documentation for Python.  It may be found online
% at:
%
%     http://www.python.org/doc/current/doc/doc.html

\documentclass[hyperref]{manual}

% latex2html doesn't know [T1]{fontenc}, so we cannot use that:(
\usepackage{amsmath}
\usepackage[latin1]{inputenc}
\usepackage{textcomp}


% this version does not reset module names at section level
%begin{latexonly}
\makeatletter
\let\py@OldOldChapter=\chapter
\renewcommand{\chapter}{\py@reset%
                        \py@OldOldChapter}
\renewcommand{\section}{\@startsection{section}{1}{\z@}%
   {-3.5ex \@plus -1ex \@minus -.2ex}%
   {2.3ex \@plus.2ex}%
   {\reset@font\Large\py@HeaderFamily}}
\makeatother
%end{latexonly}


% some convenience declarations
\newcommand{\gsl}{GSL}
\newcommand{\GSL}{GNU Scientific Library}
\newcommand{\numpy}{NumPy}
\newcommand{\NUMPY}{Numerical Python}
\newcommand{\pygsl}{PyGSL}
\newcommand{\PYGSL}{PyGSL: Python wrapper of the GNU Scientific Library}


\title{PyGSL Reference Manual}

\ifhtml
\author{%
   \ulink{Achim G\"adke}{mailto:achimgaedke@users.sourceforge.net}\\
   Center for Applied Informatics, Cologne \\
   \ulink{Jochen K\"upper}{mailto:jochen@jochen-kuepper.de}\\
   Fritz-Haber-Institut der MPG, Berlin
   \ulink{Sebastien Maret}{mailto:schnizer@users.sourceforge.net}\\
   Gesellschaft f�r Schwerionenforschung Darmstadt.
   \ulink{Pierre Schnizer}{mailto:schnizer@users.sourceforge.net}\\
   Gesellschaft f�r Schwerionenforschung, Darmstadt.
}%
\else
%begin{latexonly}
%% pdfelatex (TeXLive 7) doesn't handle \footnotemark in here...
\author{Achim G\"adke \\ Jochen K\"upper \\ Sebastien Maret \\ Pierre Schnizer}
% Please at least include a long-lived email address!
\authoraddress{
   Center for Applied Informatics, Cologne \\
   \email{achimgaedke@users.sourceforge.net} \\[2mm]
   Fritz-Haber-Institut der MPG, Berlin \\
   \email{jochen@jochen-kuepper.de} \\
      Gesellschaft f�r Schwerionenforschung, Darmstadt\\
   \email{schnizer@users.sourceforge.net}\\
}
%end{latexonly}
\fi

\date{January, 2005}            % update before release!
                                % Use an explicit date so that reformatting
                                % doesn't cause a new date to be used.  Setting
                                % the date to \today can be used during draft
                                % stages to make it easier to handle versions.
\release{0.2}                   % release version; this is used to define the
\setshortversion{0.2}           % \version macro
\makeindex                      % tell \index to actually write the .idx file


\begin{document}

\maketitle

% This makes the contents more accessible from the front page of the HTML.
\ifhtml
\chapter*{Front Matter}
\label{front}
\fi

Copyright \copyright{} 2002 The pygsl Team.

Permission is granted to copy, distribute and/or modify this document under the
terms of the GNU Free Documentation License, Version 1.1 or any later version
published by the Free Software Foundation; with no Invariant Sections, no
Front-Cover Texts, and no Back-Cover Texts.  A copy of the license is included
in section \ref{cha:free-documentation-license} entitled ``GNU Free
Documentation License''.


%% Local Variables:
%% mode: LaTeX
%% mode: auto-fill
%% fill-column: 79
%% indent-tabs-mode: nil
%% ispell-dictionary: "american"
%% reftex-fref-is-default: nil
%% TeX-auto-save: t
%% TeX-command-default: "pdfeLaTeX"
%% TeX-master: "pygsl"
%% TeX-parse-self: t
%% End:


\begin{abstract}
   \noindent
   pygsl grants python users access to the GNU scientific library.  The latest
   version can be found at the project homepage, \url{http://pygsl.sf.net}.

   \textbf{Implemented features:} \\
   \begin{tabular}{ll}
     \module{pygsl.blas}                & basic linear algebra system\\
     \module{pygsl.chebyshev}           & chebyshev approximations\\
     \module{pygsl.combination}         & combinations  \\
     \module{pygsl.const}               & $>200$ often used mathematical and
                                          scientific constants. \\
     \module{pygsl.diff}                & (Deprecated. Use pygsl.deriv instead). \\
     \module{pygsl.deriv}               & Numerical differentiation. \\
     \module{pygsl.eigen}               &\\
     \module{pygsl.fit}                 &\\
     \module{pygsl.histogram}          & 1d and 2d histograms and operations
                                          on histograms. \\
     \module{pygsl.ieee}                & Access to the ieee-arithmetics layer
                                          of gsl. \\ 
     \module{pygsl.integrate}           &\\
     \module{pygsl.interpolation}       &\\ 
     \module{pygsl.linalg}              &\\
     \module{pygsl.math}                &\\
     \module{pygsl.monte}               &\\
     \module{pygsl.minimize}            &\\
     \module{pygsl.multifit}            &\\
     \module{pygsl.multifit_nlin}       &\\    
     \module{pygsl.multimin}            &\\
     \module{pygsl.multiroots}          &\\ 
     \module{pygsl.odeiv}               &\\
     \module{pygsl.permutation}         &\\  
     \module{pygsl.poly}                &\\
     \module{pygsl.qrng}                &\\
     \module{pygsl.rng}                 & random number generators and probability densities. \\
     \module{pygsl.roots}               &\\
     \module{pygsl.siman}               &Simulated anealing\\
     \module{pygsl.sf}                  & $>200$ special functions. \\
     \module{pygsl.statistics}          & Statistical functions. \\
\end{tabular}

\end{abstract}


\tableofcontents


\chapter{System Requirements, Installation}
\label{cha:system-req-installation}
\section{Status}

\paragraph*{Status of GSL-Library}
The gsl-library is since version 1.0 stable and for general use.
More information about it at \url{http://www.gnu.org/software/gsl/}.

\paragraph*{Status of this interface}
Nearly all modules are wrapped. A lot of tests are
covering various functionality. Please report to the mailing list
\url{pygsl-discuss@lists.sourceforge.net} if you find a bug.

The hankel modules have been
wrapped. Please write to the mailing list
\url{pygsl-discuss@lists.sourceforge.net} 
if you require one of the modules
and are willing to help with a simple example. 
If any other function is missing or some other module (e.g. ntuple) or
function, do not hesitate to write to the list.

\paragraph*{Retriving the Interface}
You can download it here: \url{http://sourceforge.net/projects/pygsl}

\section{Requirements}

To build the interface, you will need
\begin{itemize}
\item \ulink{gsl-1.x}{http://sources.redhat.com/gsl},
\item \ulink{python2.6}{http://www.python.org} or better,
\item \ulink{NumPy}{http://numpy.sf.net}, and
\item a c compiler (like \ulink{gcc}{http://gcc.gnu.org}).
\end{itemize}

Supported Platforms are:
\begin{itemize}
\item Linux (Redhat/Debian/SuSE) with python2.* and gsl-1.*
\item Win32
\end{itemize}
It was tested and is tested on an irregular basis on the following platforms
\begin{itemize}
\item SUN
\item Cygwin
\item MacOS X
\end{itemize}
but is supposed to build on any POSIX platforms.

\section{Installing the pygsl interface}

\program{gsl-config} must be on your path:\nopagebreak
\begin{verbatim}
# unpack the source distribution
gzip -d -c pygsl-x.y.z.tar.gz|tar xvf-
cd pygsl-x.y.z
# do this with your prefered python version
# to set the gsl location explicitly use setup.py --gsl-prefix=/path/to/gsl
python setup.py build
# change to an user id, that is allowed to do installation
python setup.py install
\end{verbatim}
Ready....

{\bf Do not test the interface in the distribution root or in the directories
 \file{src} or \file{pygsl}.}

If you find unresolved symbols later on, delete the C source in the
swig_src files. Check that swig can be called from the command line. 
Then start the build process again. 

In this case swig will rebuild the C files. The swig_src files
distributed with pygsl are to an up to date version of GSL (1.16 as of
this writing). Swig parses partly some header header files and builds
the appropriate interface functions. If you have an older GSL version 
locally installed, the sources in the swig_src directory can contain 
links to symbols which are not defined by the locally installed GSL
version.

\subsection{Building on win32}

Windows by default does not allow to run a posix shell. Here a different path
is required. First change into the directory \file{gsl_dist}. Copy the file 
\file{gsl_site_example.py}
and edit it to reflect your installation of GSL and SWIG if you want to run it
yourself. The pygsl windows binaries distributed over 
\url{http://sourceforge.net/projects/pygsl/} are built using the mingw32 
compiler. 

\paragraph*{Uninstall GSL interface}
\code{rm -r }"python install path"\code{/lib/python}"version"\code{/site-packages/pygsl}

\paragraph*{Testing}
the directory \file{tests} contains several testsuites, based on python
\module{unittest}.
The script \file{run_test.py} in this directory will run one after the other.

\paragraph*{Support}
Please send mails to our mailinglist at
\email{pygsl-discuss@lists.sourceforge.net}.

\paragraph*{Developement}
You can browse our cvs tree at
\url{http://cvs.sourceforge.net/cgi-bin/viewcvs.cgi/pygsl/pygsl/}.
\\
Type this to check out the actual version:
\begin{verbatim}
cvs -d:pserver:anonymous@cvs.pygsl.sourceforge.net:/cvsroot/pygsl login
#Hit return for no password.
cvs -z3 -d:pserver:anonymous@cvs.pygsl.sourceforge.net:/cvsroot/pygsl co pygsl
\end{verbatim}
The script \program{tools/extract_tool.py} generates most of the special 
function code.

%\input{install_advanced.tex}
\paragraph*{ToDo}
Implement other parts:


\paragraph*{History}
\begin{itemize}
\item a gsl-interface for python was needed for a project at
\ulink{Center for Applied Informatics Cologne}{http://www.zaik.uni-koeln.de/AFS}.
\item \file{gsl-0.0.3} was released at May 23, 2001
\item \file{gsl-0.0.4} was released at January 8, 2002
\item \file{gsl-0.0.5} is growing since January, 2002
\item \file{gsl-0.2.0} was released at 
\item \file{gsl-0.3.0} was released at 
\item \file{gsl-0.3.1} was released at 
\item \file{gsl-0.3.2} was released at 
\item \file{gsl-0.9.4} was released at 25. October 2008
\end{itemize}

\paragraph*{Thanks}
Jochen K\"upper (\email{jochen@jochen-kuepper.de}) for 
\module{pygsl.statistics} part\\
Fabian Jakobs for \module{pygsl.blas}, \module{pygsl.eigen}
\module{pygsl.linalg}, \module{pygsl.permutation}\\ 
Leonardo Milano for rpm build\\
Eric Gurrola and  Peter Stoltz for testing and supporting the port of pygsl to
the MAC\\
Sebastien Maret for supporting the Fink \url{http://fink.sourceforge.net}
port of pygsl.


\paragraph*{Maintainers}
Achim G\"adke (\email{AchimGaedke@users.sourceforge.net}),\\
Pierre Schnizer (\email{schnizer@users.sourceforge.net})


\paragraph*{Acknowledgment}
\label{sec:acknowledgment}
Parts of this this manual are based on the \GSL{} reference manual.


\chapter[\protect\module{pygsl.const} --- Mathematical and scientific
constants]{\protect\module{pygsl.const} \\ Mathematical and scientific
constants} 
\label{cha:const-module}
\declaremodule{extension}{pygsl.const}
\moduleauthor{Achim G\"adke}{achimgaedke@users.sourceforge.net}

In this module some usefull constants are defined.
There are four groups of constants:

\begin{itemize}
\item mathematical
\item physical in cgs unit system
\item physical in mks unit system
\item physical number constants (e.g. fine structure)
\end{itemize}

The other modules are created during the initialisation of \module{pygsl.const}.
The mathematical, physical mks constants and number constants are available in the namespace of \module{pygsl.const}, e.g.
\begin{verbatim}
import pygsl.const
import pygsl.const.cgs
print pygsl.const.cgs.speed_of_light/pygsl.const.speed_of_light
\end{verbatim}
Of course the result is 100.0.

\section[\protect\module{pygsl.const.math} --- Mathematical constants]
{\protect\module{pygsl.const.math} \\ Mathematical constants} 
\label{cha:const-math-module}

\section[\protect\module{pygsl.const.cgs} --- Scientific constants in cgs units]
{\protect\module{pygsl.const.cgs} \\ Scientific constants in cgs units} 
\label{cha:const-cgs-module}

\section[\protect\module{pygsl.const.mks} --- Scientific constants in mks units]
{\protect\module{pygsl.const.mks} \\ Scientific constants in mks units} 
\label{cha:const-mks-module}

\section[\protect\module{pygsl.const.num} --- Scientific number constants]
{\protect\module{pygsl.const.num} \\ Scientific number constants} 
\label{cha:const-num-module}


\chapter[\protect\module{pygsl.chebyshev}]
{\protect\module{pygsl.chebyshev}}
\label{cha:statistics-module}

\declaremodule{standard}{pygsl.chebyshev}
\moduleauthor{Pierre Schnizer}{schnizer@users.sourceforge.net}

\begin{classdesc}{cheb_series}{}
  This base class can be instantiated by its name
\end{classdesc}
\begin{verbatim}
import pygsl.chebyshev
s=pygsl.chebyshev.cheb_series()
\end{verbatim}

\begin{methoddesc}{__init__}{n}\index{__init__}
            n ... number of coefficients        
\end{methoddesc}
\begin{methoddesc}{init}{f, a, b}\index{init}
        This function computes the Chebyshev approximation for the
        function F over the range (a,b) to the previously specified order.
        The computation of the Chebyshev approximation is an O($n^2$)
        process, and requires n function evaluations.

            f ... a gsl_function
            a ... lower limit
            b ... upper limit
        
\end{methoddesc}
\begin{methoddesc}{eval}{x}\index{eval}
        This function evaluates the Chebyshev series at a given point X.
\end{methoddesc}
\begin{methoddesc}{eval_err}{x}\index{eval_err}
         This function computes the Chebyshev series  at a given point X,
         estimating both the series RESULT and its absolute error ABSERR.
         The error estimate is made from the first neglected term in the
         series.
\end{methoddesc}
\begin{methoddesc}{eval_n}{n, x}\index{eval_n}
         This function evaluates the Chebyshev series at a given point
         x, to (at most) the given order n
\end{methoddesc}
\begin{methoddesc}{eval_n_err}{n, x}\index{eval_n_err}
        This function evaluates a Chebyshev series at a given point X,
        estimating both the series RESULT and its absolute error ABSERR,
        to (at most) the given order ORDER.  The error estimate is made
        from the first neglected term in the series.
\end{methoddesc}

\begin{methoddesc}{calc_deriv}{}\index{calc_deriv}
        This method computes the derivative of the series CS. It returns
        a new instance of the cheb_series class.
\end{methoddesc}
\begin{methoddesc}{calc_integ}{}\index{calc_integ}
        This method computes the integral of the series CS. It returns
        a new instance of the cheb_series class.
\end{methoddesc}
\begin{methoddesc}{get_a}{}\index{get_a}
        Get the lower boundary of the current representation       
\end{methoddesc}
\begin{methoddesc}{get_b}{}\index{get_b}
        Get the upper boundary of the current representation        
\end{methoddesc}
\begin{methoddesc}{get_coefficients}{}\index{get_coefficients}
        Get the chebyshev coefficients.         
\end{methoddesc}
\begin{methoddesc}{get_f}{}\index{get_f}
        Get the value f (what is it ?) The documentation does not tell anything
        about it.        
\end{methoddesc}
\begin{methoddesc}{get_order_sp}{}\index{get_order_sp}
        Get the value f (what is it ?) The documentation does not tell anything
        about it.        
\end{methoddesc}
\begin{methoddesc}{set_a}{}\index{set_a}
        Set the lower boundary of the current representation        
\end{methoddesc}
\begin{methoddesc}{set_b}{}\index{set_b}
        Set the upper boundary of the current         
\end{methoddesc}
\begin{methoddesc}{set_coefficients}{}\index{set_coefficients}
        Sets the chebyshev coefficients. 
\end{methoddesc}
\begin{methoddesc}{set_f}{f}\index{set_f}
        Set the value f (what is it ?)        
\end{methoddesc}
\begin{methoddesc}{set_order_sp}{...}\index{set_order_sp}
        Set the value f (what is it ?)        
\end{methoddesc}


\begin{funcdesc}{gsl_function}{f, params}\index{gsl_function}

    This class defines the callbacks known as gsl_function to
    gsl.

    e.g to supply the function f:
    
    def f(x, params):
        a = params[0]
        b = parmas[1]
        c = params[3]
        return a * x ** 2 + b * x + c

    to some solver, use

    function = gsl_function(f, params)
    
\end{funcdesc}

%%% Local Variables: 
%%% mode: latex
%%% TeX-master: "ref"
%%% End: 

\chapter[\protect\module{pygsl.deriv} --- NumericalDifferentiation]%
{\protect\module{pygsl.deriv} \\ Numerical Differentiation}
\label{cha:diff-module}

\declaremodule{extension}{pygsl.deriv}%
 \moduleauthor{Pierre  Schnizer}{schnizer@users.sourceforge.net}%
 \modulesynopsis{Numerical  Differentiation}%

\begin{quote}
  This chapter describes the available functions for numerical differentiation.
\end{quote}

The functions described in this chapter compute numerical derivatives by finite
differencing.  An adaptive algorithm is used to find the best choice of finite
difference and to estimate the error in the derivative. This module supersedes
the diff module which has been deprecated with the release of GSL-1. XXX


\begin{funcdesc}{central}{func, x, h}
  This function computes the numerical derivative of the function \var{func} at
  the point \var{x} using an adaptive central difference algorithm with a step
  size of h.  A tuple \code{(result, error)} is returned with the derivative
  and its estimated absolute error.
\end{funcdesc}

\begin{funcdesc}{backward}{func, x, h}
  This function computes the numerical derivative of the function \var{func} at
  the point \var{x} using an adaptive backward difference algorithm with a step
  size of h.  The function \var{func} is evaluated only at points smaller than
  \var{x} and at \var{x} itself.  A tuple \code{(result, error)} is returned
  with the derivative and its estimated absolute error.
\end{funcdesc}

\begin{funcdesc}{forward}{func, x, h}
  This function computes the numerical derivative of the function \var{func} at
  the point \var{x} using an adaptive forward difference algorithm with a step
  size of h.  The function \var{func} is evaluated only at points greater than
  \var{x} and at \var{x} itself.  A tuple \code{(result, error)} is returned
  with the derivative and its estimated absolute error.
\end{funcdesc}


\begin{seealso}
  The algorithms used by these functions are described in the following book:
  \seetext{S.D.\ Conte and Carl de Boor, \emph{Elementary Numerical Analysis:
      An Algorithmic Approach}, McGraw-Hill, 1972.}
\end{seealso}



%% Local Variables:
%% mode: LaTeX
%% mode: auto-fill
%% fill-column: 79
%% indent-tabs-mode: nil
%% ispell-dictionary: "british"
%% reftex-fref-is-default: nil
%% TeX-auto-save: t
%% TeX-command-default: "pdfeLaTeX"
%% TeX-master: "pygsl"
%% TeX-parse-self: t
%% End:


\chapter[\protect\module{pygsl.histogram} --- Histogram Types]
{\protect\module{pygsl.histogram} \\ Histogram Types}
\label{cha:histogram-module}



%% Local Variables:
%% mode: LaTeX
%% mode: auto-fill
%% fill-column: 79
%% ispell-dictionary: "american"
%% reftex-fref-is-default: nil
%% TeX-auto-save: t
%% TeX-command-default: "pdfeLaTeX"
%% TeX-master: "pygsl"
%% TeX-parse-self: t
%% End:


\chapter[\protect\module{pygsl.rng} --- Random Number Generators]
{\protect\module{pygsl.rng} \\ Random Number Generators}
\label{cha:rng-module}
\declaremodule{standard}{pygsl.rng}
\moduleauthor{Achim G\"adke}{achimgaedke@users.sourceforge.net}

This chapter introduces the random number generator classes provided by \module{pygsl}.

\section{Random Number Generators}

Each random number generator is a derived sperate class, that returns
a pseudo random number sequence. Methods of the common base class \class{rng}
provide the transformation to different probability distributions and
give access to basic properties of random number generators.
\begin{classdesc}{rng}{\texttt{string} typenamme \code{|} \class{rng} r}
This base class can be instantiated by a name string of the desired generator
\begin{verbatim}
import pygsl.rng
my_ran0=pygsl.rng.rng("ran0")
\end{verbatim}
or a clone of an existing generator can be created by:
\begin{verbatim}
clone_ran0=pygsl.rng.rng(my_ran0)
\end{verbatim}
\end{classdesc}
The type of the allocated generator is given by the method
\begin{methoddesc}{name}{}
which returns its name as string.
\end{methoddesc}
All generators can be seeded with
\begin{methoddesc}{set}{seed}
which sets the internal seed according to the positive integer {\tt seed}. Zero as seed
has a special meaning, please read details in the gsl reference.
\end{methoddesc}
The basic returned number type is integer, these are generated by
\begin{methoddesc}{get}{}
which returns the next number of the pseudo random sequence.
\end{methoddesc}
Basic information about these numbers can be obtained by
\begin{methoddesc}{max}{}
maximum number of this sequence and
\end{methoddesc}
\begin{methoddesc}{min}{}
minimum number of this sequence.
\end{methoddesc}
Implemented uniform probability densities are:
\begin{methoddesc}{uniform}{}
returns a real number between $[0,1)$.
\end{methoddesc}
\begin{methoddesc}{uniform_pos}{}
returns a real number between $(0,1)$ --- this excludes 0.
\end{methoddesc}
\begin{methoddesc}{uniform_int}{upper limit}
returns an integer from 0 to the upper limit (exclusive). If this limit is larger than the
number of return values of the underlying generator, \exception{pygsl.gsl_Error} is raised.
\end{methoddesc}
Furthermore a lot of derived probability densities can be used:
\begin{methoddesc}{gaussian}{}
\end{methoddesc}
\begin{methoddesc}{gaussian\_ratio\_method}{}
\end{methoddesc}
\begin{methoddesc}{ugaussian}{}
\end{methoddesc}
\begin{methoddesc}{ugaussian\_ratio\_method}{}
\end{methoddesc}
\begin{methoddesc}{gaussian\_tail}{}
\end{methoddesc}
\begin{methoddesc}{ugaussian\_tail}{}
\end{methoddesc}
\begin{methoddesc}{bivariate\_gaussian}{}
\end{methoddesc}
\begin{methoddesc}{exponential}{}
\end{methoddesc}
\begin{methoddesc}{laplace}{}
\end{methoddesc}
\begin{methoddesc}{exppow}{}
\end{methoddesc}
\begin{methoddesc}{cauchy}{}
\end{methoddesc}
\begin{methoddesc}{rayleigh}{}
\end{methoddesc}
\begin{methoddesc}{rayleigh\_tail}{}
\end{methoddesc}
\begin{methoddesc}{levy}{}
\end{methoddesc}
\begin{methoddesc}{gamma}{}
\end{methoddesc}
\begin{methoddesc}{gamma\_int}{}
\end{methoddesc}
\begin{methoddesc}{flat}{}
\end{methoddesc}
\begin{methoddesc}{lognormal}{}
\end{methoddesc}
\begin{methoddesc}{chisq}{}
\end{methoddesc}
\begin{methoddesc}{fdist}{}
\end{methoddesc}
\begin{methoddesc}{tdist}{}
\end{methoddesc}
\begin{methoddesc}{beta}{}
\end{methoddesc}
\begin{methoddesc}{logistic}{}
\end{methoddesc}
\begin{methoddesc}{pareto}{}
\end{methoddesc}
\begin{methoddesc}{dir\_2d}{}
\end{methoddesc}
\begin{methoddesc}{dir\_2d\_trig\_method}{}
\end{methoddesc}
\begin{methoddesc}{dir\_3d}{}
\end{methoddesc}
\begin{methoddesc}{dir\_nd}{}
\end{methoddesc}
\begin{methoddesc}{weibull}{}
\end{methoddesc}
\begin{methoddesc}{gumbel1}{}
\end{methoddesc}
\begin{methoddesc}{gumbel2}{}
\end{methoddesc}
\begin{methoddesc}{poisson}{}
\end{methoddesc}
\begin{methoddesc}{bernoulli}{}
\end{methoddesc}
\begin{methoddesc}{binomial}{}
\end{methoddesc}
\begin{methoddesc}{negative\_binomial}{}
\end{methoddesc}
\begin{methoddesc}{pascal}{}
\end{methoddesc}
\begin{methoddesc}{geometric}{}
\end{methoddesc}
\begin{methoddesc}{hypergeometric}{}
\end{methoddesc}
\begin{methoddesc}{logarithmic}{}
\end{methoddesc}
\begin{methoddesc}{landau}{}
\end{methoddesc}
\begin{methoddesc}{erlang}{}
\end{methoddesc}


The different generator classes are created according to the output of \code{gsl_rng_types_setup()}
when the \module{pygsl.rng} is loaded. Here is the list of children from \class{rng} for gsl-1.2:
\newline
\class{rng_borosh13},
\class{rng_coveyou},
\class{rng_cmrg},
\class{rng_fishman18},
\class{rng_fishman20},
\class{rng_fishman2x},
\class{rng_gfsr4},
\class{rng_knuthran},
\class{rng_knuthran2},
\class{rng_lecuyer21},
\class{rng_minstd},
\class{rng_mrg},
\class{rng_mt19937},
\class{rng_mt19937_1999},
\class{rng_mt19937_1998},
\class{rng_r250},
\class{rng_ran0},
\class{rng_ran1},
\class{rng_ran2},
\class{rng_ran3},
\class{rng_rand},
\class{rng_rand48},
\class{rng_random128_bsd},
\class{rng_random128_glibc2},
\class{rng_random128_libc5},
\class{rng_random256_bsd},
\class{rng_random256_glibc2},
\class{rng_random256_libc5},
\class{rng_random32_bsd},
\class{rng_random32_glibc2},
\class{rng_random32_libc5},
\class{rng_random64_bsd},
\class{rng_random64_glibc2},
\class{rng_random64_libc5},
\class{rng_random8_bsd},
\class{rng_random8_glibc2},
\class{rng_random8_libc5},
\class{rng_random_bsd},
\class{rng_random_glibc2},
\class{rng_random_libc5},
\class{rng_randu},
\class{rng_ranf},
\class{rng_ranlux},
\class{rng_ranlux389},
\class{rng_ranlxd1},
\class{rng_ranlxd2},
\class{rng_ranlxs0},
\class{rng_ranlxs1},
\class{rng_ranlxs2},
\class{rng_ranmar},
\class{rng_slatec},
\class{rng_taus},
\class{rng_taus2},
\class{rng_taus113},
\class{rng_transputer},
\class{rng_tt800},
\class{rng_uni},
\class{rng_uni32},
\class{rng_vax},
\class{rng_waterman14}, and
\class{rng_zuf}.
\newline
The default generator of \class{rng} is determined by the environment
variable \envvar{GSL_RNG_TYPE} or defaults to {\tt rng_mt19937}.

\section{Probability Density Functions}


\section{Using probability densities with random number generators}


%% Local Variables:
%% mode: LaTeX
%% mode: auto-fill
%% fill-column: 90
%% indent-tabs-mode: nil
%% ispell-dictionary: "american"
%% reftex-fref-is-default: nil
%% TeX-auto-save: t
%% TeX-command-default: "pdfeLaTeX"
%% TeX-master: "pygsl"
%% TeX-parse-self: t
%% End:


\chapter[\protect\module{pygsl.sf} --- Special Functions]
{\protect\module{pygsl.sf} \\ Special Functions}
\label{cha:sf-module}
\declaremodule{extension}{pygsl.sf}
\moduleauthor{Achim G\"adke}{achimgaedke@users.sourceforge.net}

This chapter shows you the list of implemented special function and explains
details of error handling and return values.

\section{Function list}

\begin{longtableii}{l|l}{texttt}{Function}{Description}
\lineii{}{ToDo}
\end{longtableii}

\section{Return values}

\section{Error handling}

\declaremodule{extension}{pygsl.statistics}
\moduleauthor{Jochen K\"upper}{jochen@jochen-kuepper.de}
\index{mean}
\index{standard deviation}
\index{variance}
\index{estimated standard deviation}
\index{estimated variance}
\index{t-test}
\index{range}
\index{min}
\index{max}

This chapter describes the statistical functions in the library.  The
basic statistical functions include routines to compute the mean,
variance and standard deviation. More advanced functions allow you to
calculate absolute deviations, skewness, and kurtosis as well as the
median and arbitrary percentiles.  The algorithms use recurrence
relations to compute average quantities in a stable way, without large
intermediate values that might overflow. 

All functions work on any Python sequence (of the appropriate
data-type), but see section \ref{sec:stat-speed-considerations} for
advantages and drawbacks of different kinds of input data.


\section{Organization of the module}
\label{sec:stat-organization}

The parts of the GSL functions names, providing artificial name spaces,
are mapped to modules and submodules in pygsl.  That is
\code{gsl_stats_mean} can be found as \code{pygsl.statistics.mean} and
\code{gsl_stats_long_mean} as \code{pygsl.statistics.long.mean}.

The functions in the module are available in versions for datasets in
the standard floating-point and integer types. The generic versions
available in the \code{pygsl.statistics} module are using the generic
GSL \code{double} versions.  The submodules use GSL functions according
to the submodule name, e.g. long for \code{pygsl.statistics.long}.

Currently implemented submodules are \code{pygsl.statistics.double} and
\code{pygsl.statistics.long}.



\section{Speed considerations}
\label{sec:stat-speed-considerations}

All functions work on any Python sequence type but are optimized for use
with NumPy arrays. It is strongly suggested that you install NumPy
(available from \url{http://www.numpy.org})!

If you pass NumPy arrays of the \emph{correct data-type} as input data
to any of the functions they are passed straight to the C functions
along with the stride information of the data.

If you pass generic (non-NumPy) Python sequences or NumPy arrays of the
wrong data-type a suitable copy of the data will be created and passed
to the function.


\section{Further Reading}
\label{sec:stat-further-reading}

See the gsl reference manual for a description of all available
functions and the calculations they perform.


%% Local Variables:
%% mode: LaTeX
%% mode: auto-fill
%% fill-column: 79
%% ispell-dictionary: "american"
%% reftex-fref-is-default: nil
%% TeX-auto-save: t
%% TeX-command-default: "pdfeLaTeX"
%% TeX-master: "pygsl"
%% TeX-parse-self: t
%% End:


\chapter[\protect\module{pygsl.testing} ---  Modules in Testing]
{\protect\module{pygsl.testing} \\ Modules in Testing}
\label{cha:statistics-module}

\declaremodule{standard}{pygsl.testing}

\moduleauthor{Pierre Schnizer}{schnizer@users.sourceforge.net}
Modules in this package are often reimplementations of an original package
with significant change to the original. The current rng implementation, for
example, started its life here. Usage of these modules is encouraged for tests
to see if they work, but use them with caution in your production code!

\section[\protect\module{pygsl.testing.sf} --- Special UFuncs]
{\protect\module{pygsl.testing.sf} \\ Special Functions as UFuncs}

\declaremodule{standard}{pygsl.testing.sf}
\moduleauthor{Pierre Schnizer}{schnizer@users.sourceforge.net}

This chapter provides mainly \numpy{} UFuncs over the special functions. This means
that all input variable can be arrays, and the UFunc will evaluate the gsl
function for all its inputs. It is meant to replace the sf module later;
please use it and find out if it is useful for you. 
Only the python specific part is described here. For a general description of
the function please see the GSL Reference document.  

\section{UFuncs}
These UFuncs allow to evaluate an array of doubles or an array of floats typically.
\begin{funcdesc}{Chi}{...}\index{Chi}

    Number of Input  Arguments:  1
    Number of Output Arguments:  1
\end{funcdesc}

\begin{funcdesc}{Chi_e}{...}\index{Chi_e}

    Number of Input  Arguments:  1
    Number of Output Arguments:  2

The error flag is discarded.
Return Arguments 1 and 2 resemble a gsl_result argument,
	which is  argument 1 of the C argument list

\end{funcdesc}

\begin{funcdesc}{Ci}{...}\index{Ci}

    Number of Input  Arguments:  1
    Number of Output Arguments:  1
\end{funcdesc}

\begin{funcdesc}{Ci_e}{...}\index{Ci_e}

    Number of Input  Arguments:  1
    Number of Output Arguments:  2

The error flag is discarded.
Return Arguments 1 and 2 resemble a gsl_result argument,
	which is  argument 1 of the C argument list

\end{funcdesc}

\begin{funcdesc}{Shi}{...}\index{Shi}

    Number of Input  Arguments:  1
    Number of Output Arguments:  1
\end{funcdesc}

\begin{funcdesc}{Shi_e}{...}\index{Shi_e}

    Number of Input  Arguments:  1
    Number of Output Arguments:  2

The error flag is discarded.
Return Arguments 1 and 2 resemble a gsl_result argument,
	which is  argument 1 of the C argument list

\end{funcdesc}

\begin{funcdesc}{Si}{...}\index{Si}

    Number of Input  Arguments:  1
    Number of Output Arguments:  1
\end{funcdesc}

\begin{funcdesc}{Si_e}{...}\index{Si_e}

    Number of Input  Arguments:  1
    Number of Output Arguments:  2

The error flag is discarded.
Return Arguments 1 and 2 resemble a gsl_result argument,
	which is  argument 1 of the C argument list

\end{funcdesc}

\begin{funcdesc}{airy_Ai}{...}\index{airy_Ai}

    Number of Input  Arguments:  2
    Number of Output Arguments:  1

 Argument 2 is a gsl_mode_t, valid parameters are:
	sf.PREC_DOUBLE or sf.PREC_SINGLE or sf.PREC_APPROX

\end{funcdesc}

\begin{funcdesc}{airy_Ai_deriv}{...}\index{airy_Ai_deriv}

    Number of Input  Arguments:  2
    Number of Output Arguments:  1

 Argument 2 is a gsl_mode_t, valid parameters are:
	sf.PREC_DOUBLE or sf.PREC_SINGLE or sf.PREC_APPROX

\end{funcdesc}

\begin{funcdesc}{airy_Ai_deriv_e}{...}\index{airy_Ai_deriv_e}

    Number of Input  Arguments:  2
    Number of Output Arguments:  2

 Argument 2 is a gsl_mode_t, valid parameters are:
	sf.PREC_DOUBLE or sf.PREC_SINGLE or sf.PREC_APPROX
The error flag is discarded.
Return Arguments 1 and 2 resemble a gsl_result argument,
	which is  argument 2 of the C argument list

\end{funcdesc}

\begin{funcdesc}{airy_Ai_deriv_scaled}{...}\index{airy_Ai_deriv_scaled}

    Number of Input  Arguments:  2
    Number of Output Arguments:  1

 Argument 2 is a gsl_mode_t, valid parameters are:
	sf.PREC_DOUBLE or sf.PREC_SINGLE or sf.PREC_APPROX

\end{funcdesc}

\begin{funcdesc}{airy_Ai_deriv_scaled_e}{...}\index{airy_Ai_deriv_scaled_e}

    Number of Input  Arguments:  2
    Number of Output Arguments:  2

 Argument 2 is a gsl_mode_t, valid parameters are:
	sf.PREC_DOUBLE or sf.PREC_SINGLE or sf.PREC_APPROX
The error flag is discarded.
Return Arguments 1 and 2 resemble a gsl_result argument,
	which is  argument 2 of the C argument list

\end{funcdesc}

\begin{funcdesc}{airy_Ai_e}{...}\index{airy_Ai_e}

    Number of Input  Arguments:  2
    Number of Output Arguments:  2

 Argument 2 is a gsl_mode_t, valid parameters are:
	sf.PREC_DOUBLE or sf.PREC_SINGLE or sf.PREC_APPROX
The error flag is discarded.
Return Arguments 1 and 2 resemble a gsl_result argument,
	which is  argument 2 of the C argument list

\end{funcdesc}

\begin{funcdesc}{airy_Ai_scaled}{...}\index{airy_Ai_scaled}

    Number of Input  Arguments:  2
    Number of Output Arguments:  1

 Argument 2 is a gsl_mode_t, valid parameters are:
	sf.PREC_DOUBLE or sf.PREC_SINGLE or sf.PREC_APPROX

\end{funcdesc}

\begin{funcdesc}{airy_Ai_scaled_e}{...}\index{airy_Ai_scaled_e}

    Number of Input  Arguments:  2
    Number of Output Arguments:  2

 Argument 2 is a gsl_mode_t, valid parameters are:
	sf.PREC_DOUBLE or sf.PREC_SINGLE or sf.PREC_APPROX
The error flag is discarded.
Return Arguments 1 and 2 resemble a gsl_result argument,
	which is  argument 2 of the C argument list

\end{funcdesc}

\begin{funcdesc}{airy_Bi}{...}\index{airy_Bi}

    Number of Input  Arguments:  2
    Number of Output Arguments:  1

 Argument 2 is a gsl_mode_t, valid parameters are:
	sf.PREC_DOUBLE or sf.PREC_SINGLE or sf.PREC_APPROX

\end{funcdesc}

\begin{funcdesc}{airy_Bi_deriv}{...}\index{airy_Bi_deriv}

    Number of Input  Arguments:  2
    Number of Output Arguments:  1

 Argument 2 is a gsl_mode_t, valid parameters are:
	sf.PREC_DOUBLE or sf.PREC_SINGLE or sf.PREC_APPROX

\end{funcdesc}

\begin{funcdesc}{airy_Bi_deriv_e}{...}\index{airy_Bi_deriv_e}

    Number of Input  Arguments:  2
    Number of Output Arguments:  2

 Argument 2 is a gsl_mode_t, valid parameters are:
	sf.PREC_DOUBLE or sf.PREC_SINGLE or sf.PREC_APPROX
The error flag is discarded.
Return Arguments 1 and 2 resemble a gsl_result argument,
	which is  argument 2 of the C argument list

\end{funcdesc}

\begin{funcdesc}{airy_Bi_deriv_scaled}{...}\index{airy_Bi_deriv_scaled}

    Number of Input  Arguments:  2
    Number of Output Arguments:  1

 Argument 2 is a gsl_mode_t, valid parameters are:
	sf.PREC_DOUBLE or sf.PREC_SINGLE or sf.PREC_APPROX

\end{funcdesc}

\begin{funcdesc}{airy_Bi_deriv_scaled_e}{...}\index{airy_Bi_deriv_scaled_e}

    Number of Input  Arguments:  2
    Number of Output Arguments:  2

 Argument 2 is a gsl_mode_t, valid parameters are:
	sf.PREC_DOUBLE or sf.PREC_SINGLE or sf.PREC_APPROX
The error flag is discarded.
Return Arguments 1 and 2 resemble a gsl_result argument,
	which is  argument 2 of the C argument list

\end{funcdesc}

\begin{funcdesc}{airy_Bi_e}{...}\index{airy_Bi_e}

    Number of Input  Arguments:  2
    Number of Output Arguments:  2

 Argument 2 is a gsl_mode_t, valid parameters are:
	sf.PREC_DOUBLE or sf.PREC_SINGLE or sf.PREC_APPROX
The error flag is discarded.
Return Arguments 1 and 2 resemble a gsl_result argument,
	which is  argument 2 of the C argument list

\end{funcdesc}

\begin{funcdesc}{airy_Bi_scaled}{...}\index{airy_Bi_scaled}

    Number of Input  Arguments:  2
    Number of Output Arguments:  1

 Argument 2 is a gsl_mode_t, valid parameters are:
	sf.PREC_DOUBLE or sf.PREC_SINGLE or sf.PREC_APPROX

\end{funcdesc}

\begin{funcdesc}{airy_Bi_scaled_e}{...}\index{airy_Bi_scaled_e}

    Number of Input  Arguments:  2
    Number of Output Arguments:  2

 Argument 2 is a gsl_mode_t, valid parameters are:
	sf.PREC_DOUBLE or sf.PREC_SINGLE or sf.PREC_APPROX
The error flag is discarded.
Return Arguments 1 and 2 resemble a gsl_result argument,
	which is  argument 2 of the C argument list

\end{funcdesc}

\begin{funcdesc}{airy_zero_Ai}{...}\index{airy_zero_Ai}

    Number of Input  Arguments:  1
    Number of Output Arguments:  1
\end{funcdesc}

\begin{funcdesc}{airy_zero_Ai_deriv}{...}\index{airy_zero_Ai_deriv}

    Number of Input  Arguments:  1
    Number of Output Arguments:  1
\end{funcdesc}

\begin{funcdesc}{airy_zero_Ai_deriv_e}{...}\index{airy_zero_Ai_deriv_e}

    Number of Input  Arguments:  1
    Number of Output Arguments:  2

The error flag is discarded.
Return Arguments 1 and 2 resemble a gsl_result argument,
	which is  argument 1 of the C argument list

\end{funcdesc}

\begin{funcdesc}{airy_zero_Ai_e}{...}\index{airy_zero_Ai_e}

    Number of Input  Arguments:  1
    Number of Output Arguments:  2

The error flag is discarded.
Return Arguments 1 and 2 resemble a gsl_result argument,
	which is  argument 1 of the C argument list

\end{funcdesc}

\begin{funcdesc}{airy_zero_Bi}{...}\index{airy_zero_Bi}

    Number of Input  Arguments:  1
    Number of Output Arguments:  1
\end{funcdesc}

\begin{funcdesc}{airy_zero_Bi_deriv}{...}\index{airy_zero_Bi_deriv}

    Number of Input  Arguments:  1
    Number of Output Arguments:  1
\end{funcdesc}

\begin{funcdesc}{airy_zero_Bi_deriv_e}{...}\index{airy_zero_Bi_deriv_e}

    Number of Input  Arguments:  1
    Number of Output Arguments:  2

The error flag is discarded.
Return Arguments 1 and 2 resemble a gsl_result argument,
	which is  argument 1 of the C argument list

\end{funcdesc}

\begin{funcdesc}{airy_zero_Bi_e}{...}\index{airy_zero_Bi_e}

    Number of Input  Arguments:  1
    Number of Output Arguments:  2

The error flag is discarded.
Return Arguments 1 and 2 resemble a gsl_result argument,
	which is  argument 1 of the C argument list

\end{funcdesc}

\begin{funcdesc}{angle_restrict_pos}{...}\index{angle_restrict_pos}

    Number of Input  Arguments:  1
    Number of Output Arguments:  1
\end{funcdesc}

\begin{funcdesc}{angle_restrict_pos_err_e}{...}\index{angle_restrict_pos_err_e}

    Number of Input  Arguments:  1
    Number of Output Arguments:  2

The error flag is discarded.
Return Arguments 1 and 2 resemble a gsl_result argument,
	which is  argument 1 of the C argument list

\end{funcdesc}

\begin{funcdesc}{angle_restrict_symm}{...}\index{angle_restrict_symm}

    Number of Input  Arguments:  1
    Number of Output Arguments:  1
\end{funcdesc}

\begin{funcdesc}{angle_restrict_symm_err_e}{...}\index{angle_restrict_symm_err_e}

    Number of Input  Arguments:  1
    Number of Output Arguments:  2

The error flag is discarded.
Return Arguments 1 and 2 resemble a gsl_result argument,
	which is  argument 1 of the C argument list

\end{funcdesc}

\begin{funcdesc}{atanint}{...}\index{atanint}

    Number of Input  Arguments:  1
    Number of Output Arguments:  1
\end{funcdesc}

\begin{funcdesc}{atanint_e}{...}\index{atanint_e}

    Number of Input  Arguments:  1
    Number of Output Arguments:  2

The error flag is discarded.
Return Arguments 1 and 2 resemble a gsl_result argument,
	which is  argument 1 of the C argument list

\end{funcdesc}

\begin{funcdesc}{bessel_I0}{...}\index{bessel_I0}

    Number of Input  Arguments:  1
    Number of Output Arguments:  1
\end{funcdesc}

\begin{funcdesc}{bessel_I0_e}{...}\index{bessel_I0_e}

    Number of Input  Arguments:  1
    Number of Output Arguments:  2

The error flag is discarded.
Return Arguments 1 and 2 resemble a gsl_result argument,
	which is  argument 1 of the C argument list

\end{funcdesc}

\begin{funcdesc}{bessel_I0_scaled}{...}\index{bessel_I0_scaled}

    Number of Input  Arguments:  1
    Number of Output Arguments:  1
\end{funcdesc}

\begin{funcdesc}{bessel_I0_scaled_e}{...}\index{bessel_I0_scaled_e}

    Number of Input  Arguments:  1
    Number of Output Arguments:  2

The error flag is discarded.
Return Arguments 1 and 2 resemble a gsl_result argument,
	which is  argument 1 of the C argument list

\end{funcdesc}

\begin{funcdesc}{bessel_I1}{...}\index{bessel_I1}

    Number of Input  Arguments:  1
    Number of Output Arguments:  1
\end{funcdesc}

\begin{funcdesc}{bessel_I1_e}{...}\index{bessel_I1_e}

    Number of Input  Arguments:  1
    Number of Output Arguments:  2

The error flag is discarded.
Return Arguments 1 and 2 resemble a gsl_result argument,
	which is  argument 1 of the C argument list

\end{funcdesc}

\begin{funcdesc}{bessel_I1_scaled}{...}\index{bessel_I1_scaled}

    Number of Input  Arguments:  1
    Number of Output Arguments:  1
\end{funcdesc}

\begin{funcdesc}{bessel_I1_scaled_e}{...}\index{bessel_I1_scaled_e}

    Number of Input  Arguments:  1
    Number of Output Arguments:  2

The error flag is discarded.
Return Arguments 1 and 2 resemble a gsl_result argument,
	which is  argument 1 of the C argument list

\end{funcdesc}

\begin{funcdesc}{bessel_In}{...}\index{bessel_In}

    Number of Input  Arguments:  2
    Number of Output Arguments:  1
\end{funcdesc}

\begin{funcdesc}{bessel_In_e}{...}\index{bessel_In_e}

    Number of Input  Arguments:  2
    Number of Output Arguments:  2

The error flag is discarded.
Return Arguments 1 and 2 resemble a gsl_result argument,
	which is  argument 2 of the C argument list

\end{funcdesc}

\begin{funcdesc}{bessel_In_scaled}{...}\index{bessel_In_scaled}

    Number of Input  Arguments:  2
    Number of Output Arguments:  1
\end{funcdesc}

\begin{funcdesc}{bessel_In_scaled_e}{...}\index{bessel_In_scaled_e}

    Number of Input  Arguments:  2
    Number of Output Arguments:  2

The error flag is discarded.
Return Arguments 1 and 2 resemble a gsl_result argument,
	which is  argument 2 of the C argument list

\end{funcdesc}

\begin{funcdesc}{bessel_Inu}{...}\index{bessel_Inu}

    Number of Input  Arguments:  2
    Number of Output Arguments:  1
\end{funcdesc}

\begin{funcdesc}{bessel_Inu_e}{...}\index{bessel_Inu_e}

    Number of Input  Arguments:  2
    Number of Output Arguments:  2

The error flag is discarded.
Return Arguments 1 and 2 resemble a gsl_result argument,
	which is  argument 2 of the C argument list

\end{funcdesc}

\begin{funcdesc}{bessel_Inu_scaled}{...}\index{bessel_Inu_scaled}

    Number of Input  Arguments:  2
    Number of Output Arguments:  1
\end{funcdesc}

\begin{funcdesc}{bessel_Inu_scaled_e}{...}\index{bessel_Inu_scaled_e}

    Number of Input  Arguments:  2
    Number of Output Arguments:  2

The error flag is discarded.
Return Arguments 1 and 2 resemble a gsl_result argument,
	which is  argument 2 of the C argument list

\end{funcdesc}

\begin{funcdesc}{bessel_J0}{...}\index{bessel_J0}

    Number of Input  Arguments:  1
    Number of Output Arguments:  1
\end{funcdesc}

\begin{funcdesc}{bessel_J0_e}{...}\index{bessel_J0_e}

    Number of Input  Arguments:  1
    Number of Output Arguments:  2

The error flag is discarded.
Return Arguments 1 and 2 resemble a gsl_result argument,
	which is  argument 1 of the C argument list

\end{funcdesc}

\begin{funcdesc}{bessel_J1}{...}\index{bessel_J1}

    Number of Input  Arguments:  1
    Number of Output Arguments:  1
\end{funcdesc}

\begin{funcdesc}{bessel_J1_e}{...}\index{bessel_J1_e}

    Number of Input  Arguments:  1
    Number of Output Arguments:  2

The error flag is discarded.
Return Arguments 1 and 2 resemble a gsl_result argument,
	which is  argument 1 of the C argument list

\end{funcdesc}

\begin{funcdesc}{bessel_Jn}{...}\index{bessel_Jn}

    Number of Input  Arguments:  2
    Number of Output Arguments:  1
\end{funcdesc}

\begin{funcdesc}{bessel_Jn_e}{...}\index{bessel_Jn_e}

    Number of Input  Arguments:  2
    Number of Output Arguments:  2

The error flag is discarded.
Return Arguments 1 and 2 resemble a gsl_result argument,
	which is  argument 2 of the C argument list

\end{funcdesc}

\begin{funcdesc}{bessel_Jnu}{...}\index{bessel_Jnu}

    Number of Input  Arguments:  2
    Number of Output Arguments:  1
\end{funcdesc}

\begin{funcdesc}{bessel_Jnu_e}{...}\index{bessel_Jnu_e}

    Number of Input  Arguments:  2
    Number of Output Arguments:  2

The error flag is discarded.
Return Arguments 1 and 2 resemble a gsl_result argument,
	which is  argument 2 of the C argument list

\end{funcdesc}

\begin{funcdesc}{bessel_K0}{...}\index{bessel_K0}

    Number of Input  Arguments:  1
    Number of Output Arguments:  1
\end{funcdesc}

\begin{funcdesc}{bessel_K0_e}{...}\index{bessel_K0_e}

    Number of Input  Arguments:  1
    Number of Output Arguments:  2

The error flag is discarded.
Return Arguments 1 and 2 resemble a gsl_result argument,
	which is  argument 1 of the C argument list

\end{funcdesc}

\begin{funcdesc}{bessel_K0_scaled}{...}\index{bessel_K0_scaled}

    Number of Input  Arguments:  1
    Number of Output Arguments:  1
\end{funcdesc}

\begin{funcdesc}{bessel_K0_scaled_e}{...}\index{bessel_K0_scaled_e}

    Number of Input  Arguments:  1
    Number of Output Arguments:  2

The error flag is discarded.
Return Arguments 1 and 2 resemble a gsl_result argument,
	which is  argument 1 of the C argument list

\end{funcdesc}

\begin{funcdesc}{bessel_K1}{...}\index{bessel_K1}

    Number of Input  Arguments:  1
    Number of Output Arguments:  1
\end{funcdesc}

\begin{funcdesc}{bessel_K1_e}{...}\index{bessel_K1_e}

    Number of Input  Arguments:  1
    Number of Output Arguments:  2

The error flag is discarded.
Return Arguments 1 and 2 resemble a gsl_result argument,
	which is  argument 1 of the C argument list

\end{funcdesc}

\begin{funcdesc}{bessel_K1_scaled}{...}\index{bessel_K1_scaled}

    Number of Input  Arguments:  1
    Number of Output Arguments:  1
\end{funcdesc}

\begin{funcdesc}{bessel_K1_scaled_e}{...}\index{bessel_K1_scaled_e}

    Number of Input  Arguments:  1
    Number of Output Arguments:  2

The error flag is discarded.
Return Arguments 1 and 2 resemble a gsl_result argument,
	which is  argument 1 of the C argument list

\end{funcdesc}

\begin{funcdesc}{bessel_Kn}{...}\index{bessel_Kn}

    Number of Input  Arguments:  2
    Number of Output Arguments:  1
\end{funcdesc}

\begin{funcdesc}{bessel_Kn_e}{...}\index{bessel_Kn_e}

    Number of Input  Arguments:  2
    Number of Output Arguments:  2

The error flag is discarded.
Return Arguments 1 and 2 resemble a gsl_result argument,
	which is  argument 2 of the C argument list

\end{funcdesc}

\begin{funcdesc}{bessel_Kn_scaled}{...}\index{bessel_Kn_scaled}

    Number of Input  Arguments:  2
    Number of Output Arguments:  1
\end{funcdesc}

\begin{funcdesc}{bessel_Kn_scaled_e}{...}\index{bessel_Kn_scaled_e}

    Number of Input  Arguments:  2
    Number of Output Arguments:  2

The error flag is discarded.
Return Arguments 1 and 2 resemble a gsl_result argument,
	which is  argument 2 of the C argument list

\end{funcdesc}

\begin{funcdesc}{bessel_Knu}{...}\index{bessel_Knu}

    Number of Input  Arguments:  2
    Number of Output Arguments:  1
\end{funcdesc}

\begin{funcdesc}{bessel_Knu_e}{...}\index{bessel_Knu_e}

    Number of Input  Arguments:  2
    Number of Output Arguments:  2

The error flag is discarded.
Return Arguments 1 and 2 resemble a gsl_result argument,
	which is  argument 2 of the C argument list

\end{funcdesc}

\begin{funcdesc}{bessel_Knu_scaled}{...}\index{bessel_Knu_scaled}

    Number of Input  Arguments:  2
    Number of Output Arguments:  1
\end{funcdesc}

\begin{funcdesc}{bessel_Knu_scaled_e}{...}\index{bessel_Knu_scaled_e}

    Number of Input  Arguments:  2
    Number of Output Arguments:  2

The error flag is discarded.
Return Arguments 1 and 2 resemble a gsl_result argument,
	which is  argument 2 of the C argument list

\end{funcdesc}

\begin{funcdesc}{bessel_Y0}{...}\index{bessel_Y0}

    Number of Input  Arguments:  1
    Number of Output Arguments:  1
\end{funcdesc}

\begin{funcdesc}{bessel_Y0_e}{...}\index{bessel_Y0_e}

    Number of Input  Arguments:  1
    Number of Output Arguments:  2

The error flag is discarded.
Return Arguments 1 and 2 resemble a gsl_result argument,
	which is  argument 1 of the C argument list

\end{funcdesc}

\begin{funcdesc}{bessel_Y1}{...}\index{bessel_Y1}

    Number of Input  Arguments:  1
    Number of Output Arguments:  1
\end{funcdesc}

\begin{funcdesc}{bessel_Y1_e}{...}\index{bessel_Y1_e}

    Number of Input  Arguments:  1
    Number of Output Arguments:  2

The error flag is discarded.
Return Arguments 1 and 2 resemble a gsl_result argument,
	which is  argument 1 of the C argument list

\end{funcdesc}

\begin{funcdesc}{bessel_Yn}{...}\index{bessel_Yn}

    Number of Input  Arguments:  2
    Number of Output Arguments:  1
\end{funcdesc}

\begin{funcdesc}{bessel_Yn_e}{...}\index{bessel_Yn_e}

    Number of Input  Arguments:  2
    Number of Output Arguments:  2

The error flag is discarded.
Return Arguments 1 and 2 resemble a gsl_result argument,
	which is  argument 2 of the C argument list

\end{funcdesc}

\begin{funcdesc}{bessel_Ynu}{...}\index{bessel_Ynu}

    Number of Input  Arguments:  2
    Number of Output Arguments:  1
\end{funcdesc}

\begin{funcdesc}{bessel_Ynu_e}{...}\index{bessel_Ynu_e}

    Number of Input  Arguments:  2
    Number of Output Arguments:  2

The error flag is discarded.
Return Arguments 1 and 2 resemble a gsl_result argument,
	which is  argument 2 of the C argument list

\end{funcdesc}

\begin{funcdesc}{bessel_i0_scaled}{...}\index{bessel_i0_scaled}

    Number of Input  Arguments:  1
    Number of Output Arguments:  1
\end{funcdesc}

\begin{funcdesc}{bessel_i0_scaled_e}{...}\index{bessel_i0_scaled_e}

    Number of Input  Arguments:  1
    Number of Output Arguments:  2

The error flag is discarded.
Return Arguments 1 and 2 resemble a gsl_result argument,
	which is  argument 1 of the C argument list

\end{funcdesc}

\begin{funcdesc}{bessel_i1_scaled}{...}\index{bessel_i1_scaled}

    Number of Input  Arguments:  1
    Number of Output Arguments:  1
\end{funcdesc}

\begin{funcdesc}{bessel_i1_scaled_e}{...}\index{bessel_i1_scaled_e}

    Number of Input  Arguments:  1
    Number of Output Arguments:  2

The error flag is discarded.
Return Arguments 1 and 2 resemble a gsl_result argument,
	which is  argument 1 of the C argument list

\end{funcdesc}

\begin{funcdesc}{bessel_i2_scaled}{...}\index{bessel_i2_scaled}

    Number of Input  Arguments:  1
    Number of Output Arguments:  1
\end{funcdesc}

\begin{funcdesc}{bessel_i2_scaled_e}{...}\index{bessel_i2_scaled_e}

    Number of Input  Arguments:  1
    Number of Output Arguments:  2

The error flag is discarded.
Return Arguments 1 and 2 resemble a gsl_result argument,
	which is  argument 1 of the C argument list

\end{funcdesc}

\begin{funcdesc}{bessel_il_scaled}{...}\index{bessel_il_scaled}

    Number of Input  Arguments:  2
    Number of Output Arguments:  1
\end{funcdesc}

\begin{funcdesc}{bessel_il_scaled_e}{...}\index{bessel_il_scaled_e}

    Number of Input  Arguments:  2
    Number of Output Arguments:  2

The error flag is discarded.
Return Arguments 1 and 2 resemble a gsl_result argument,
	which is  argument 2 of the C argument list

\end{funcdesc}

\begin{funcdesc}{bessel_j0}{...}\index{bessel_j0}

    Number of Input  Arguments:  1
    Number of Output Arguments:  1
\end{funcdesc}

\begin{funcdesc}{bessel_j0_e}{...}\index{bessel_j0_e}

    Number of Input  Arguments:  1
    Number of Output Arguments:  2

The error flag is discarded.
Return Arguments 1 and 2 resemble a gsl_result argument,
	which is  argument 1 of the C argument list

\end{funcdesc}

\begin{funcdesc}{bessel_j1}{...}\index{bessel_j1}

    Number of Input  Arguments:  1
    Number of Output Arguments:  1
\end{funcdesc}

\begin{funcdesc}{bessel_j1_e}{...}\index{bessel_j1_e}

    Number of Input  Arguments:  1
    Number of Output Arguments:  2

The error flag is discarded.
Return Arguments 1 and 2 resemble a gsl_result argument,
	which is  argument 1 of the C argument list

\end{funcdesc}

\begin{funcdesc}{bessel_j2}{...}\index{bessel_j2}

    Number of Input  Arguments:  1
    Number of Output Arguments:  1
\end{funcdesc}

\begin{funcdesc}{bessel_j2_e}{...}\index{bessel_j2_e}

    Number of Input  Arguments:  1
    Number of Output Arguments:  2

The error flag is discarded.
Return Arguments 1 and 2 resemble a gsl_result argument,
	which is  argument 1 of the C argument list

\end{funcdesc}

\begin{funcdesc}{bessel_jl}{...}\index{bessel_jl}

    Number of Input  Arguments:  2
    Number of Output Arguments:  1
\end{funcdesc}

\begin{funcdesc}{bessel_jl_e}{...}\index{bessel_jl_e}

    Number of Input  Arguments:  2
    Number of Output Arguments:  2

The error flag is discarded.
Return Arguments 1 and 2 resemble a gsl_result argument,
	which is  argument 2 of the C argument list

\end{funcdesc}

\begin{funcdesc}{bessel_k0_scaled}{...}\index{bessel_k0_scaled}

    Number of Input  Arguments:  1
    Number of Output Arguments:  1
\end{funcdesc}

\begin{funcdesc}{bessel_k0_scaled_e}{...}\index{bessel_k0_scaled_e}

    Number of Input  Arguments:  1
    Number of Output Arguments:  2

The error flag is discarded.
Return Arguments 1 and 2 resemble a gsl_result argument,
	which is  argument 1 of the C argument list

\end{funcdesc}

\begin{funcdesc}{bessel_k1_scaled}{...}\index{bessel_k1_scaled}

    Number of Input  Arguments:  1
    Number of Output Arguments:  1
\end{funcdesc}

\begin{funcdesc}{bessel_k1_scaled_e}{...}\index{bessel_k1_scaled_e}

    Number of Input  Arguments:  1
    Number of Output Arguments:  2

The error flag is discarded.
Return Arguments 1 and 2 resemble a gsl_result argument,
	which is  argument 1 of the C argument list

\end{funcdesc}

\begin{funcdesc}{bessel_k2_scaled}{...}\index{bessel_k2_scaled}

    Number of Input  Arguments:  1
    Number of Output Arguments:  1
\end{funcdesc}

\begin{funcdesc}{bessel_k2_scaled_e}{...}\index{bessel_k2_scaled_e}

    Number of Input  Arguments:  1
    Number of Output Arguments:  2

The error flag is discarded.
Return Arguments 1 and 2 resemble a gsl_result argument,
	which is  argument 1 of the C argument list

\end{funcdesc}

\begin{funcdesc}{bessel_kl_scaled}{...}\index{bessel_kl_scaled}

    Number of Input  Arguments:  2
    Number of Output Arguments:  1
\end{funcdesc}

\begin{funcdesc}{bessel_kl_scaled_e}{...}\index{bessel_kl_scaled_e}

    Number of Input  Arguments:  2
    Number of Output Arguments:  2

The error flag is discarded.
Return Arguments 1 and 2 resemble a gsl_result argument,
	which is  argument 2 of the C argument list

\end{funcdesc}

\begin{funcdesc}{bessel_lnKnu}{...}\index{bessel_lnKnu}

    Number of Input  Arguments:  2
    Number of Output Arguments:  1
\end{funcdesc}

\begin{funcdesc}{bessel_lnKnu_e}{...}\index{bessel_lnKnu_e}

    Number of Input  Arguments:  2
    Number of Output Arguments:  2

The error flag is discarded.
Return Arguments 1 and 2 resemble a gsl_result argument,
	which is  argument 2 of the C argument list

\end{funcdesc}

\begin{funcdesc}{bessel_y0}{...}\index{bessel_y0}

    Number of Input  Arguments:  1
    Number of Output Arguments:  1
\end{funcdesc}

\begin{funcdesc}{bessel_y0_e}{...}\index{bessel_y0_e}

    Number of Input  Arguments:  1
    Number of Output Arguments:  2

The error flag is discarded.
Return Arguments 1 and 2 resemble a gsl_result argument,
	which is  argument 1 of the C argument list

\end{funcdesc}

\begin{funcdesc}{bessel_y1}{...}\index{bessel_y1}

    Number of Input  Arguments:  1
    Number of Output Arguments:  1
\end{funcdesc}

\begin{funcdesc}{bessel_y1_e}{...}\index{bessel_y1_e}

    Number of Input  Arguments:  1
    Number of Output Arguments:  2

The error flag is discarded.
Return Arguments 1 and 2 resemble a gsl_result argument,
	which is  argument 1 of the C argument list

\end{funcdesc}

\begin{funcdesc}{bessel_y2}{...}\index{bessel_y2}

    Number of Input  Arguments:  1
    Number of Output Arguments:  1
\end{funcdesc}

\begin{funcdesc}{bessel_y2_e}{...}\index{bessel_y2_e}

    Number of Input  Arguments:  1
    Number of Output Arguments:  2

The error flag is discarded.
Return Arguments 1 and 2 resemble a gsl_result argument,
	which is  argument 1 of the C argument list

\end{funcdesc}

\begin{funcdesc}{bessel_yl}{...}\index{bessel_yl}

    Number of Input  Arguments:  2
    Number of Output Arguments:  1
\end{funcdesc}

\begin{funcdesc}{bessel_yl_e}{...}\index{bessel_yl_e}

    Number of Input  Arguments:  2
    Number of Output Arguments:  2

The error flag is discarded.
Return Arguments 1 and 2 resemble a gsl_result argument,
	which is  argument 2 of the C argument list

\end{funcdesc}

\begin{funcdesc}{bessel_zero_J0}{...}\index{bessel_zero_J0}

    Number of Input  Arguments:  1
    Number of Output Arguments:  1
\end{funcdesc}

\begin{funcdesc}{bessel_zero_J0_e}{...}\index{bessel_zero_J0_e}

    Number of Input  Arguments:  1
    Number of Output Arguments:  2

The error flag is discarded.
Return Arguments 1 and 2 resemble a gsl_result argument,
	which is  argument 1 of the C argument list

\end{funcdesc}

\begin{funcdesc}{bessel_zero_J1}{...}\index{bessel_zero_J1}

    Number of Input  Arguments:  1
    Number of Output Arguments:  1
\end{funcdesc}

\begin{funcdesc}{bessel_zero_J1_e}{...}\index{bessel_zero_J1_e}

    Number of Input  Arguments:  1
    Number of Output Arguments:  2

The error flag is discarded.
Return Arguments 1 and 2 resemble a gsl_result argument,
	which is  argument 1 of the C argument list

\end{funcdesc}

\begin{funcdesc}{bessel_zero_Jnu}{...}\index{bessel_zero_Jnu}

    Number of Input  Arguments:  2
    Number of Output Arguments:  1
\end{funcdesc}

\begin{funcdesc}{bessel_zero_Jnu_e}{...}\index{bessel_zero_Jnu_e}

    Number of Input  Arguments:  2
    Number of Output Arguments:  2

The error flag is discarded.
Return Arguments 1 and 2 resemble a gsl_result argument,
	which is  argument 2 of the C argument list

\end{funcdesc}

\begin{funcdesc}{beta}{...}\index{beta}

    Number of Input  Arguments:  2
    Number of Output Arguments:  1
\end{funcdesc}

\begin{funcdesc}{beta_e}{...}\index{beta_e}

    Number of Input  Arguments:  2
    Number of Output Arguments:  2

The error flag is discarded.
Return Arguments 1 and 2 resemble a gsl_result argument,
	which is  argument 2 of the C argument list

\end{funcdesc}

\begin{funcdesc}{beta_inc}{...}\index{beta_inc}

    Number of Input  Arguments:  3
    Number of Output Arguments:  1
\end{funcdesc}

\begin{funcdesc}{beta_inc_e}{...}\index{beta_inc_e}

    Number of Input  Arguments:  3
    Number of Output Arguments:  2

The error flag is discarded.
Return Arguments 1 and 2 resemble a gsl_result argument,
	which is  argument 3 of the C argument list

\end{funcdesc}

\begin{funcdesc}{choose}{...}\index{choose}

    Number of Input  Arguments:  2
    Number of Output Arguments:  1
\end{funcdesc}

\begin{funcdesc}{choose_e}{...}\index{choose_e}

    Number of Input  Arguments:  2
    Number of Output Arguments:  2

The error flag is discarded.
Return Arguments 1 and 2 resemble a gsl_result argument,
	which is  argument 2 of the C argument list

\end{funcdesc}

\begin{funcdesc}{clausen}{...}\index{clausen}

    Number of Input  Arguments:  1
    Number of Output Arguments:  1
\end{funcdesc}

\begin{funcdesc}{clausen_e}{...}\index{clausen_e}

    Number of Input  Arguments:  1
    Number of Output Arguments:  2

The error flag is discarded.
Return Arguments 1 and 2 resemble a gsl_result argument,
	which is  argument 1 of the C argument list

\end{funcdesc}

\begin{funcdesc}{conicalP_0}{...}\index{conicalP_0}

    Number of Input  Arguments:  2
    Number of Output Arguments:  1
\end{funcdesc}

\begin{funcdesc}{conicalP_0_e}{...}\index{conicalP_0_e}

    Number of Input  Arguments:  2
    Number of Output Arguments:  2

The error flag is discarded.
Return Arguments 1 and 2 resemble a gsl_result argument,
	which is  argument 2 of the C argument list

\end{funcdesc}

\begin{funcdesc}{conicalP_1}{...}\index{conicalP_1}

    Number of Input  Arguments:  2
    Number of Output Arguments:  1
\end{funcdesc}

\begin{funcdesc}{conicalP_1_e}{...}\index{conicalP_1_e}

    Number of Input  Arguments:  2
    Number of Output Arguments:  2

The error flag is discarded.
Return Arguments 1 and 2 resemble a gsl_result argument,
	which is  argument 2 of the C argument list

\end{funcdesc}

\begin{funcdesc}{conicalP_cyl_reg}{...}\index{conicalP_cyl_reg}

    Number of Input  Arguments:  3
    Number of Output Arguments:  1
\end{funcdesc}

\begin{funcdesc}{conicalP_cyl_reg_e}{...}\index{conicalP_cyl_reg_e}

    Number of Input  Arguments:  3
    Number of Output Arguments:  2

The error flag is discarded.
Return Arguments 1 and 2 resemble a gsl_result argument,
	which is  argument 3 of the C argument list

\end{funcdesc}

\begin{funcdesc}{conicalP_half}{...}\index{conicalP_half}

    Number of Input  Arguments:  2
    Number of Output Arguments:  1
\end{funcdesc}

\begin{funcdesc}{conicalP_half_e}{...}\index{conicalP_half_e}

    Number of Input  Arguments:  2
    Number of Output Arguments:  2

The error flag is discarded.
Return Arguments 1 and 2 resemble a gsl_result argument,
	which is  argument 2 of the C argument list

\end{funcdesc}

\begin{funcdesc}{conicalP_mhalf}{...}\index{conicalP_mhalf}

    Number of Input  Arguments:  2
    Number of Output Arguments:  1
\end{funcdesc}

\begin{funcdesc}{conicalP_mhalf_e}{...}\index{conicalP_mhalf_e}

    Number of Input  Arguments:  2
    Number of Output Arguments:  2

The error flag is discarded.
Return Arguments 1 and 2 resemble a gsl_result argument,
	which is  argument 2 of the C argument list

\end{funcdesc}

\begin{funcdesc}{conicalP_sph_reg}{...}\index{conicalP_sph_reg}

    Number of Input  Arguments:  3
    Number of Output Arguments:  1
\end{funcdesc}

\begin{funcdesc}{conicalP_sph_reg_e}{...}\index{conicalP_sph_reg_e}

    Number of Input  Arguments:  3
    Number of Output Arguments:  2

The error flag is discarded.
Return Arguments 1 and 2 resemble a gsl_result argument,
	which is  argument 3 of the C argument list

\end{funcdesc}

\begin{funcdesc}{cos}{...}\index{cos}

    Number of Input  Arguments:  1
    Number of Output Arguments:  1
\end{funcdesc}

\begin{funcdesc}{cos_e}{...}\index{cos_e}

    Number of Input  Arguments:  1
    Number of Output Arguments:  2

The error flag is discarded.
Return Arguments 1 and 2 resemble a gsl_result argument,
	which is  argument 1 of the C argument list

\end{funcdesc}

\begin{funcdesc}{cos_err_e}{...}\index{cos_err_e}

    Number of Input  Arguments:  2
    Number of Output Arguments:  2

The error flag is discarded.
Return Arguments 1 and 2 resemble a gsl_result argument,
	which is  argument 2 of the C argument list

\end{funcdesc}

\begin{funcdesc}{coulomb_CL_e}{...}\index{coulomb_CL_e}

    Number of Input  Arguments:  2
    Number of Output Arguments:  2

The error flag is discarded.
Return Arguments 1 and 2 resemble a gsl_result argument,
	which is  argument 2 of the C argument list

\end{funcdesc}

\begin{funcdesc}{coulomb_wave_FG_e}{...}\index{coulomb_wave_FG_e}

    Number of Input  Arguments:  4
    Number of Output Arguments: 10

The error flag is discarded.
Return Arguments 1 and 2 resemble a gsl_result argument,
	which is  argument 4 of the C argument list
Return Arguments 3 and 4 resemble a gsl_result argument,
	which is  argument 5 of the C argument list
Return Arguments 5 and 6 resemble a gsl_result argument,
	which is  argument 6 of the C argument list
Return Arguments 7 and 8 resemble a gsl_result argument,
	which is  argument 7 of the C argument list

\end{funcdesc}

\begin{funcdesc}{coupling_3j}{...}\index{coupling_3j}

    Number of Input  Arguments:  6
    Number of Output Arguments:  1
\end{funcdesc}

\begin{funcdesc}{coupling_3j_e}{...}\index{coupling_3j_e}

    Number of Input  Arguments:  6
    Number of Output Arguments:  2

The error flag is discarded.
Return Arguments 1 and 2 resemble a gsl_result argument,
	which is  argument 6 of the C argument list

\end{funcdesc}

\begin{funcdesc}{coupling_6j}{...}\index{coupling_6j}

    Number of Input  Arguments:  6
    Number of Output Arguments:  1
\end{funcdesc}

\begin{funcdesc}{coupling_6j_e}{...}\index{coupling_6j_e}

    Number of Input  Arguments:  6
    Number of Output Arguments:  2

The error flag is discarded.
Return Arguments 1 and 2 resemble a gsl_result argument,
	which is  argument 6 of the C argument list

\end{funcdesc}

\begin{funcdesc}{coupling_9j}{...}\index{coupling_9j}

    Number of Input  Arguments:  9
    Number of Output Arguments:  1
\end{funcdesc}

\begin{funcdesc}{coupling_9j_e}{...}\index{coupling_9j_e}

    Number of Input  Arguments:  9
    Number of Output Arguments:  2

The error flag is discarded.
Return Arguments 1 and 2 resemble a gsl_result argument,
	which is  argument 9 of the C argument list

\end{funcdesc}

\begin{funcdesc}{coupling_RacahW}{...}\index{coupling_RacahW}

    Number of Input  Arguments:  6
    Number of Output Arguments:  1
\end{funcdesc}

\begin{funcdesc}{coupling_RacahW_e}{...}\index{coupling_RacahW_e}

    Number of Input  Arguments:  6
    Number of Output Arguments:  2

The error flag is discarded.
Return Arguments 1 and 2 resemble a gsl_result argument,
	which is  argument 6 of the C argument list

\end{funcdesc}

\begin{funcdesc}{dawson}{...}\index{dawson}

    Number of Input  Arguments:  1
    Number of Output Arguments:  1
\end{funcdesc}

\begin{funcdesc}{dawson_e}{...}\index{dawson_e}

    Number of Input  Arguments:  1
    Number of Output Arguments:  2

The error flag is discarded.
Return Arguments 1 and 2 resemble a gsl_result argument,
	which is  argument 1 of the C argument list

\end{funcdesc}

\begin{funcdesc}{debye_1}{...}\index{debye_1}

    Number of Input  Arguments:  1
    Number of Output Arguments:  1
\end{funcdesc}

\begin{funcdesc}{debye_1_e}{...}\index{debye_1_e}

    Number of Input  Arguments:  1
    Number of Output Arguments:  2

The error flag is discarded.
Return Arguments 1 and 2 resemble a gsl_result argument,
	which is  argument 1 of the C argument list

\end{funcdesc}

\begin{funcdesc}{debye_2}{...}\index{debye_2}

    Number of Input  Arguments:  1
    Number of Output Arguments:  1
\end{funcdesc}

\begin{funcdesc}{debye_2_e}{...}\index{debye_2_e}

    Number of Input  Arguments:  1
    Number of Output Arguments:  2

The error flag is discarded.
Return Arguments 1 and 2 resemble a gsl_result argument,
	which is  argument 1 of the C argument list

\end{funcdesc}

\begin{funcdesc}{debye_3}{...}\index{debye_3}

    Number of Input  Arguments:  1
    Number of Output Arguments:  1
\end{funcdesc}

\begin{funcdesc}{debye_3_e}{...}\index{debye_3_e}

    Number of Input  Arguments:  1
    Number of Output Arguments:  2

The error flag is discarded.
Return Arguments 1 and 2 resemble a gsl_result argument,
	which is  argument 1 of the C argument list

\end{funcdesc}

\begin{funcdesc}{debye_4}{...}\index{debye_4}

    Number of Input  Arguments:  1
    Number of Output Arguments:  1
\end{funcdesc}

\begin{funcdesc}{debye_4_e}{...}\index{debye_4_e}

    Number of Input  Arguments:  1
    Number of Output Arguments:  2

The error flag is discarded.
Return Arguments 1 and 2 resemble a gsl_result argument,
	which is  argument 1 of the C argument list

\end{funcdesc}

\begin{funcdesc}{dilog}{...}\index{dilog}

    Number of Input  Arguments:  1
    Number of Output Arguments:  1
\end{funcdesc}

\begin{funcdesc}{dilog_e}{...}\index{dilog_e}

    Number of Input  Arguments:  1
    Number of Output Arguments:  2

The error flag is discarded.
Return Arguments 1 and 2 resemble a gsl_result argument,
	which is  argument 1 of the C argument list

\end{funcdesc}

\begin{funcdesc}{doublefact}{...}\index{doublefact}

    Number of Input  Arguments:  1
    Number of Output Arguments:  1
\end{funcdesc}

\begin{funcdesc}{doublefact_e}{...}\index{doublefact_e}

    Number of Input  Arguments:  1
    Number of Output Arguments:  2

The error flag is discarded.
Return Arguments 1 and 2 resemble a gsl_result argument,
	which is  argument 1 of the C argument list

\end{funcdesc}

\begin{funcdesc}{ellint_D}{...}\index{ellint_D}

    Number of Input  Arguments:  4
    Number of Output Arguments:  1

 Argument 4 is a gsl_mode_t, valid parameters are:
	sf.PREC_DOUBLE or sf.PREC_SINGLE or sf.PREC_APPROX

\end{funcdesc}

\begin{funcdesc}{ellint_D_e}{...}\index{ellint_D_e}

    Number of Input  Arguments:  4
    Number of Output Arguments:  2

 Argument 4 is a gsl_mode_t, valid parameters are:
	sf.PREC_DOUBLE or sf.PREC_SINGLE or sf.PREC_APPROX
The error flag is discarded.
Return Arguments 1 and 2 resemble a gsl_result argument,
	which is  argument 4 of the C argument list

\end{funcdesc}

\begin{funcdesc}{ellint_E}{...}\index{ellint_E}

    Number of Input  Arguments:  3
    Number of Output Arguments:  1

 Argument 3 is a gsl_mode_t, valid parameters are:
	sf.PREC_DOUBLE or sf.PREC_SINGLE or sf.PREC_APPROX

\end{funcdesc}

\begin{funcdesc}{ellint_E_e}{...}\index{ellint_E_e}

    Number of Input  Arguments:  3
    Number of Output Arguments:  2

 Argument 3 is a gsl_mode_t, valid parameters are:
	sf.PREC_DOUBLE or sf.PREC_SINGLE or sf.PREC_APPROX
The error flag is discarded.
Return Arguments 1 and 2 resemble a gsl_result argument,
	which is  argument 3 of the C argument list

\end{funcdesc}

\begin{funcdesc}{ellint_Ecomp}{...}\index{ellint_Ecomp}

    Number of Input  Arguments:  2
    Number of Output Arguments:  1

 Argument 2 is a gsl_mode_t, valid parameters are:
	sf.PREC_DOUBLE or sf.PREC_SINGLE or sf.PREC_APPROX

\end{funcdesc}

\begin{funcdesc}{ellint_Ecomp_e}{...}\index{ellint_Ecomp_e}

    Number of Input  Arguments:  2
    Number of Output Arguments:  2

 Argument 2 is a gsl_mode_t, valid parameters are:
	sf.PREC_DOUBLE or sf.PREC_SINGLE or sf.PREC_APPROX
The error flag is discarded.
Return Arguments 1 and 2 resemble a gsl_result argument,
	which is  argument 2 of the C argument list

\end{funcdesc}

\begin{funcdesc}{ellint_F}{...}\index{ellint_F}

    Number of Input  Arguments:  3
    Number of Output Arguments:  1

 Argument 3 is a gsl_mode_t, valid parameters are:
	sf.PREC_DOUBLE or sf.PREC_SINGLE or sf.PREC_APPROX

\end{funcdesc}

\begin{funcdesc}{ellint_F_e}{...}\index{ellint_F_e}

    Number of Input  Arguments:  3
    Number of Output Arguments:  2

 Argument 3 is a gsl_mode_t, valid parameters are:
	sf.PREC_DOUBLE or sf.PREC_SINGLE or sf.PREC_APPROX
The error flag is discarded.
Return Arguments 1 and 2 resemble a gsl_result argument,
	which is  argument 3 of the C argument list

\end{funcdesc}

\begin{funcdesc}{ellint_Kcomp}{...}\index{ellint_Kcomp}

    Number of Input  Arguments:  2
    Number of Output Arguments:  1

 Argument 2 is a gsl_mode_t, valid parameters are:
	sf.PREC_DOUBLE or sf.PREC_SINGLE or sf.PREC_APPROX

\end{funcdesc}

\begin{funcdesc}{ellint_Kcomp_e}{...}\index{ellint_Kcomp_e}

    Number of Input  Arguments:  2
    Number of Output Arguments:  2

 Argument 2 is a gsl_mode_t, valid parameters are:
	sf.PREC_DOUBLE or sf.PREC_SINGLE or sf.PREC_APPROX
The error flag is discarded.
Return Arguments 1 and 2 resemble a gsl_result argument,
	which is  argument 2 of the C argument list

\end{funcdesc}

\begin{funcdesc}{ellint_P}{...}\index{ellint_P}

    Number of Input  Arguments:  4
    Number of Output Arguments:  1

 Argument 4 is a gsl_mode_t, valid parameters are:
	sf.PREC_DOUBLE or sf.PREC_SINGLE or sf.PREC_APPROX

\end{funcdesc}

\begin{funcdesc}{ellint_P_e}{...}\index{ellint_P_e}

    Number of Input  Arguments:  4
    Number of Output Arguments:  2

 Argument 4 is a gsl_mode_t, valid parameters are:
	sf.PREC_DOUBLE or sf.PREC_SINGLE or sf.PREC_APPROX
The error flag is discarded.
Return Arguments 1 and 2 resemble a gsl_result argument,
	which is  argument 4 of the C argument list

\end{funcdesc}

\begin{funcdesc}{ellint_RC}{...}\index{ellint_RC}

    Number of Input  Arguments:  3
    Number of Output Arguments:  1

 Argument 3 is a gsl_mode_t, valid parameters are:
	sf.PREC_DOUBLE or sf.PREC_SINGLE or sf.PREC_APPROX

\end{funcdesc}

\begin{funcdesc}{ellint_RC_e}{...}\index{ellint_RC_e}

    Number of Input  Arguments:  3
    Number of Output Arguments:  2

 Argument 3 is a gsl_mode_t, valid parameters are:
	sf.PREC_DOUBLE or sf.PREC_SINGLE or sf.PREC_APPROX
The error flag is discarded.
Return Arguments 1 and 2 resemble a gsl_result argument,
	which is  argument 3 of the C argument list

\end{funcdesc}

\begin{funcdesc}{ellint_RD}{...}\index{ellint_RD}

    Number of Input  Arguments:  4
    Number of Output Arguments:  1

 Argument 4 is a gsl_mode_t, valid parameters are:
	sf.PREC_DOUBLE or sf.PREC_SINGLE or sf.PREC_APPROX

\end{funcdesc}

\begin{funcdesc}{ellint_RD_e}{...}\index{ellint_RD_e}

    Number of Input  Arguments:  4
    Number of Output Arguments:  2

 Argument 4 is a gsl_mode_t, valid parameters are:
	sf.PREC_DOUBLE or sf.PREC_SINGLE or sf.PREC_APPROX
The error flag is discarded.
Return Arguments 1 and 2 resemble a gsl_result argument,
	which is  argument 4 of the C argument list

\end{funcdesc}

\begin{funcdesc}{ellint_RF}{...}\index{ellint_RF}

    Number of Input  Arguments:  4
    Number of Output Arguments:  1

 Argument 4 is a gsl_mode_t, valid parameters are:
	sf.PREC_DOUBLE or sf.PREC_SINGLE or sf.PREC_APPROX

\end{funcdesc}

\begin{funcdesc}{ellint_RF_e}{...}\index{ellint_RF_e}

    Number of Input  Arguments:  4
    Number of Output Arguments:  2

 Argument 4 is a gsl_mode_t, valid parameters are:
	sf.PREC_DOUBLE or sf.PREC_SINGLE or sf.PREC_APPROX
The error flag is discarded.
Return Arguments 1 and 2 resemble a gsl_result argument,
	which is  argument 4 of the C argument list

\end{funcdesc}

\begin{funcdesc}{ellint_RJ}{...}\index{ellint_RJ}

    Number of Input  Arguments:  5
    Number of Output Arguments:  1

 Argument 5 is a gsl_mode_t, valid parameters are:
	sf.PREC_DOUBLE or sf.PREC_SINGLE or sf.PREC_APPROX

\end{funcdesc}

\begin{funcdesc}{ellint_RJ_e}{...}\index{ellint_RJ_e}

    Number of Input  Arguments:  5
    Number of Output Arguments:  2

 Argument 5 is a gsl_mode_t, valid parameters are:
	sf.PREC_DOUBLE or sf.PREC_SINGLE or sf.PREC_APPROX
The error flag is discarded.
Return Arguments 1 and 2 resemble a gsl_result argument,
	which is  argument 5 of the C argument list

\end{funcdesc}

\begin{funcdesc}{elljac_e}{...}\index{elljac_e}

    Number of Input  Arguments:  2
    Number of Output Arguments:  3

The error flag is discarded.

\end{funcdesc}

\begin{funcdesc}{erf}{...}\index{erf}

    Number of Input  Arguments:  1
    Number of Output Arguments:  1
\end{funcdesc}

\begin{funcdesc}{erf_Q}{...}\index{erf_Q}

    Number of Input  Arguments:  1
    Number of Output Arguments:  1
\end{funcdesc}

\begin{funcdesc}{erf_Q_e}{...}\index{erf_Q_e}

    Number of Input  Arguments:  1
    Number of Output Arguments:  2

The error flag is discarded.
Return Arguments 1 and 2 resemble a gsl_result argument,
	which is  argument 1 of the C argument list

\end{funcdesc}

\begin{funcdesc}{erf_Z}{...}\index{erf_Z}

    Number of Input  Arguments:  1
    Number of Output Arguments:  1
\end{funcdesc}

\begin{funcdesc}{erf_Z_e}{...}\index{erf_Z_e}

    Number of Input  Arguments:  1
    Number of Output Arguments:  2

The error flag is discarded.
Return Arguments 1 and 2 resemble a gsl_result argument,
	which is  argument 1 of the C argument list

\end{funcdesc}

\begin{funcdesc}{erf_e}{...}\index{erf_e}

    Number of Input  Arguments:  1
    Number of Output Arguments:  2

The error flag is discarded.
Return Arguments 1 and 2 resemble a gsl_result argument,
	which is  argument 1 of the C argument list

\end{funcdesc}

\begin{funcdesc}{erfc}{...}\index{erfc}

    Number of Input  Arguments:  1
    Number of Output Arguments:  1
\end{funcdesc}

\begin{funcdesc}{erfc_e}{...}\index{erfc_e}

    Number of Input  Arguments:  1
    Number of Output Arguments:  2

The error flag is discarded.
Return Arguments 1 and 2 resemble a gsl_result argument,
	which is  argument 1 of the C argument list

\end{funcdesc}

\begin{funcdesc}{eta}{...}\index{eta}

    Number of Input  Arguments:  1
    Number of Output Arguments:  1
\end{funcdesc}

\begin{funcdesc}{eta_e}{...}\index{eta_e}

    Number of Input  Arguments:  1
    Number of Output Arguments:  2

The error flag is discarded.
Return Arguments 1 and 2 resemble a gsl_result argument,
	which is  argument 1 of the C argument list

\end{funcdesc}

\begin{funcdesc}{eta_int}{...}\index{eta_int}

    Number of Input  Arguments:  1
    Number of Output Arguments:  1
\end{funcdesc}

\begin{funcdesc}{eta_int_e}{...}\index{eta_int_e}

    Number of Input  Arguments:  1
    Number of Output Arguments:  2

The error flag is discarded.
Return Arguments 1 and 2 resemble a gsl_result argument,
	which is  argument 1 of the C argument list

\end{funcdesc}

\begin{funcdesc}{expint_3}{...}\index{expint_3}

    Number of Input  Arguments:  1
    Number of Output Arguments:  1
\end{funcdesc}

\begin{funcdesc}{expint_3_e}{...}\index{expint_3_e}

    Number of Input  Arguments:  1
    Number of Output Arguments:  2

The error flag is discarded.
Return Arguments 1 and 2 resemble a gsl_result argument,
	which is  argument 1 of the C argument list

\end{funcdesc}

\begin{funcdesc}{expint_E1}{...}\index{expint_E1}

    Number of Input  Arguments:  1
    Number of Output Arguments:  1
\end{funcdesc}

\begin{funcdesc}{expint_E1_e}{...}\index{expint_E1_e}

    Number of Input  Arguments:  1
    Number of Output Arguments:  2

The error flag is discarded.
Return Arguments 1 and 2 resemble a gsl_result argument,
	which is  argument 1 of the C argument list

\end{funcdesc}

\begin{funcdesc}{expint_E1_scaled}{...}\index{expint_E1_scaled}

    Number of Input  Arguments:  1
    Number of Output Arguments:  1
\end{funcdesc}

\begin{funcdesc}{expint_E1_scaled_e}{...}\index{expint_E1_scaled_e}

    Number of Input  Arguments:  1
    Number of Output Arguments:  2

The error flag is discarded.
Return Arguments 1 and 2 resemble a gsl_result argument,
	which is  argument 1 of the C argument list

\end{funcdesc}

\begin{funcdesc}{expint_E2}{...}\index{expint_E2}

    Number of Input  Arguments:  1
    Number of Output Arguments:  1
\end{funcdesc}

\begin{funcdesc}{expint_E2_e}{...}\index{expint_E2_e}

    Number of Input  Arguments:  1
    Number of Output Arguments:  2

The error flag is discarded.
Return Arguments 1 and 2 resemble a gsl_result argument,
	which is  argument 1 of the C argument list

\end{funcdesc}

\begin{funcdesc}{expint_E2_scaled}{...}\index{expint_E2_scaled}

    Number of Input  Arguments:  1
    Number of Output Arguments:  1
\end{funcdesc}

\begin{funcdesc}{expint_E2_scaled_e}{...}\index{expint_E2_scaled_e}

    Number of Input  Arguments:  1
    Number of Output Arguments:  2

The error flag is discarded.
Return Arguments 1 and 2 resemble a gsl_result argument,
	which is  argument 1 of the C argument list

\end{funcdesc}

\begin{funcdesc}{expint_Ei}{...}\index{expint_Ei}

    Number of Input  Arguments:  1
    Number of Output Arguments:  1
\end{funcdesc}

\begin{funcdesc}{expint_Ei_e}{...}\index{expint_Ei_e}

    Number of Input  Arguments:  1
    Number of Output Arguments:  2

The error flag is discarded.
Return Arguments 1 and 2 resemble a gsl_result argument,
	which is  argument 1 of the C argument list

\end{funcdesc}

\begin{funcdesc}{expint_Ei_scaled}{...}\index{expint_Ei_scaled}

    Number of Input  Arguments:  1
    Number of Output Arguments:  1
\end{funcdesc}

\begin{funcdesc}{expint_Ei_scaled_e}{...}\index{expint_Ei_scaled_e}

    Number of Input  Arguments:  1
    Number of Output Arguments:  2

The error flag is discarded.
Return Arguments 1 and 2 resemble a gsl_result argument,
	which is  argument 1 of the C argument list

\end{funcdesc}

\begin{funcdesc}{fact}{...}\index{fact}

    Number of Input  Arguments:  1
    Number of Output Arguments:  1
\end{funcdesc}

\begin{funcdesc}{fact_e}{...}\index{fact_e}

    Number of Input  Arguments:  1
    Number of Output Arguments:  2

The error flag is discarded.
Return Arguments 1 and 2 resemble a gsl_result argument,
	which is  argument 1 of the C argument list

\end{funcdesc}

\begin{funcdesc}{fermi_dirac_0}{...}\index{fermi_dirac_0}

    Number of Input  Arguments:  1
    Number of Output Arguments:  1
\end{funcdesc}

\begin{funcdesc}{fermi_dirac_0_e}{...}\index{fermi_dirac_0_e}

    Number of Input  Arguments:  1
    Number of Output Arguments:  2

The error flag is discarded.
Return Arguments 1 and 2 resemble a gsl_result argument,
	which is  argument 1 of the C argument list

\end{funcdesc}

\begin{funcdesc}{fermi_dirac_1}{...}\index{fermi_dirac_1}

    Number of Input  Arguments:  1
    Number of Output Arguments:  1
\end{funcdesc}

\begin{funcdesc}{fermi_dirac_1_e}{...}\index{fermi_dirac_1_e}

    Number of Input  Arguments:  1
    Number of Output Arguments:  2

The error flag is discarded.
Return Arguments 1 and 2 resemble a gsl_result argument,
	which is  argument 1 of the C argument list

\end{funcdesc}

\begin{funcdesc}{fermi_dirac_2}{...}\index{fermi_dirac_2}

    Number of Input  Arguments:  1
    Number of Output Arguments:  1
\end{funcdesc}

\begin{funcdesc}{fermi_dirac_2_e}{...}\index{fermi_dirac_2_e}

    Number of Input  Arguments:  1
    Number of Output Arguments:  2

The error flag is discarded.
Return Arguments 1 and 2 resemble a gsl_result argument,
	which is  argument 1 of the C argument list

\end{funcdesc}

\begin{funcdesc}{fermi_dirac_3half}{...}\index{fermi_dirac_3half}

    Number of Input  Arguments:  1
    Number of Output Arguments:  1
\end{funcdesc}

\begin{funcdesc}{fermi_dirac_3half_e}{...}\index{fermi_dirac_3half_e}

    Number of Input  Arguments:  1
    Number of Output Arguments:  2

The error flag is discarded.
Return Arguments 1 and 2 resemble a gsl_result argument,
	which is  argument 1 of the C argument list

\end{funcdesc}

\begin{funcdesc}{fermi_dirac_half}{...}\index{fermi_dirac_half}

    Number of Input  Arguments:  1
    Number of Output Arguments:  1
\end{funcdesc}

\begin{funcdesc}{fermi_dirac_half_e}{...}\index{fermi_dirac_half_e}

    Number of Input  Arguments:  1
    Number of Output Arguments:  2

The error flag is discarded.
Return Arguments 1 and 2 resemble a gsl_result argument,
	which is  argument 1 of the C argument list

\end{funcdesc}

\begin{funcdesc}{fermi_dirac_inc_0}{...}\index{fermi_dirac_inc_0}

    Number of Input  Arguments:  2
    Number of Output Arguments:  1
\end{funcdesc}

\begin{funcdesc}{fermi_dirac_inc_0_e}{...}\index{fermi_dirac_inc_0_e}

    Number of Input  Arguments:  2
    Number of Output Arguments:  2

The error flag is discarded.
Return Arguments 1 and 2 resemble a gsl_result argument,
	which is  argument 2 of the C argument list

\end{funcdesc}

\begin{funcdesc}{fermi_dirac_int}{...}\index{fermi_dirac_int}

    Number of Input  Arguments:  2
    Number of Output Arguments:  1
\end{funcdesc}

\begin{funcdesc}{fermi_dirac_int_e}{...}\index{fermi_dirac_int_e}

    Number of Input  Arguments:  2
    Number of Output Arguments:  2

The error flag is discarded.
Return Arguments 1 and 2 resemble a gsl_result argument,
	which is  argument 2 of the C argument list

\end{funcdesc}

\begin{funcdesc}{fermi_dirac_m1}{...}\index{fermi_dirac_m1}

    Number of Input  Arguments:  1
    Number of Output Arguments:  1
\end{funcdesc}

\begin{funcdesc}{fermi_dirac_m1_e}{...}\index{fermi_dirac_m1_e}

    Number of Input  Arguments:  1
    Number of Output Arguments:  2

The error flag is discarded.
Return Arguments 1 and 2 resemble a gsl_result argument,
	which is  argument 1 of the C argument list

\end{funcdesc}

\begin{funcdesc}{fermi_dirac_mhalf}{...}\index{fermi_dirac_mhalf}

    Number of Input  Arguments:  1
    Number of Output Arguments:  1
\end{funcdesc}

\begin{funcdesc}{fermi_dirac_mhalf_e}{...}\index{fermi_dirac_mhalf_e}

    Number of Input  Arguments:  1
    Number of Output Arguments:  2

The error flag is discarded.
Return Arguments 1 and 2 resemble a gsl_result argument,
	which is  argument 1 of the C argument list

\end{funcdesc}

\begin{funcdesc}{gamma}{...}\index{gamma}

    Number of Input  Arguments:  1
    Number of Output Arguments:  1
\end{funcdesc}

\begin{funcdesc}{gamma_e}{...}\index{gamma_e}

    Number of Input  Arguments:  1
    Number of Output Arguments:  2

The error flag is discarded.
Return Arguments 1 and 2 resemble a gsl_result argument,
	which is  argument 1 of the C argument list

\end{funcdesc}

\begin{funcdesc}{gamma_inc_P}{...}\index{gamma_inc_P}

    Number of Input  Arguments:  2
    Number of Output Arguments:  1
\end{funcdesc}

\begin{funcdesc}{gamma_inc_P_e}{...}\index{gamma_inc_P_e}

    Number of Input  Arguments:  2
    Number of Output Arguments:  2

The error flag is discarded.
Return Arguments 1 and 2 resemble a gsl_result argument,
	which is  argument 2 of the C argument list

\end{funcdesc}

\begin{funcdesc}{gamma_inc_Q}{...}\index{gamma_inc_Q}

    Number of Input  Arguments:  2
    Number of Output Arguments:  1
\end{funcdesc}

\begin{funcdesc}{gamma_inc_Q_e}{...}\index{gamma_inc_Q_e}

    Number of Input  Arguments:  2
    Number of Output Arguments:  2

The error flag is discarded.
Return Arguments 1 and 2 resemble a gsl_result argument,
	which is  argument 2 of the C argument list

\end{funcdesc}

\begin{funcdesc}{gammainv}{...}\index{gammainv}

    Number of Input  Arguments:  1
    Number of Output Arguments:  1
\end{funcdesc}

\begin{funcdesc}{gammainv_e}{...}\index{gammainv_e}

    Number of Input  Arguments:  1
    Number of Output Arguments:  2

The error flag is discarded.
Return Arguments 1 and 2 resemble a gsl_result argument,
	which is  argument 1 of the C argument list

\end{funcdesc}

\begin{funcdesc}{gammastar}{...}\index{gammastar}

    Number of Input  Arguments:  1
    Number of Output Arguments:  1
\end{funcdesc}

\begin{funcdesc}{gammastar_e}{...}\index{gammastar_e}

    Number of Input  Arguments:  1
    Number of Output Arguments:  2

The error flag is discarded.
Return Arguments 1 and 2 resemble a gsl_result argument,
	which is  argument 1 of the C argument list

\end{funcdesc}

\begin{funcdesc}{gegenpoly_1}{...}\index{gegenpoly_1}

    Number of Input  Arguments:  2
    Number of Output Arguments:  1
\end{funcdesc}

\begin{funcdesc}{gegenpoly_1_e}{...}\index{gegenpoly_1_e}

    Number of Input  Arguments:  2
    Number of Output Arguments:  2

The error flag is discarded.
Return Arguments 1 and 2 resemble a gsl_result argument,
	which is  argument 2 of the C argument list

\end{funcdesc}

\begin{funcdesc}{gegenpoly_2}{...}\index{gegenpoly_2}

    Number of Input  Arguments:  2
    Number of Output Arguments:  1
\end{funcdesc}

\begin{funcdesc}{gegenpoly_2_e}{...}\index{gegenpoly_2_e}

    Number of Input  Arguments:  2
    Number of Output Arguments:  2

The error flag is discarded.
Return Arguments 1 and 2 resemble a gsl_result argument,
	which is  argument 2 of the C argument list

\end{funcdesc}

\begin{funcdesc}{gegenpoly_3}{...}\index{gegenpoly_3}

    Number of Input  Arguments:  2
    Number of Output Arguments:  1
\end{funcdesc}

\begin{funcdesc}{gegenpoly_3_e}{...}\index{gegenpoly_3_e}

    Number of Input  Arguments:  2
    Number of Output Arguments:  2

The error flag is discarded.
Return Arguments 1 and 2 resemble a gsl_result argument,
	which is  argument 2 of the C argument list

\end{funcdesc}

\begin{funcdesc}{gegenpoly_n}{...}\index{gegenpoly_n}

    Number of Input  Arguments:  3
    Number of Output Arguments:  1
\end{funcdesc}

\begin{funcdesc}{gegenpoly_n_e}{...}\index{gegenpoly_n_e}

    Number of Input  Arguments:  3
    Number of Output Arguments:  2

The error flag is discarded.
Return Arguments 1 and 2 resemble a gsl_result argument,
	which is  argument 3 of the C argument list

\end{funcdesc}

\begin{funcdesc}{hydrogenicR}{...}\index{hydrogenicR}

    Number of Input  Arguments:  4
    Number of Output Arguments:  1
\end{funcdesc}

\begin{funcdesc}{hydrogenicR_1}{...}\index{hydrogenicR_1}

    Number of Input  Arguments:  2
    Number of Output Arguments:  1
\end{funcdesc}

\begin{funcdesc}{hydrogenicR_1_e}{...}\index{hydrogenicR_1_e}

    Number of Input  Arguments:  2
    Number of Output Arguments:  2

The error flag is discarded.
Return Arguments 1 and 2 resemble a gsl_result argument,
	which is  argument 2 of the C argument list

\end{funcdesc}

\begin{funcdesc}{hydrogenicR_e}{...}\index{hydrogenicR_e}

    Number of Input  Arguments:  4
    Number of Output Arguments:  2

The error flag is discarded.
Return Arguments 1 and 2 resemble a gsl_result argument,
	which is  argument 4 of the C argument list

\end{funcdesc}

\begin{funcdesc}{hyperg_0F1}{...}\index{hyperg_0F1}

    Number of Input  Arguments:  2
    Number of Output Arguments:  1
\end{funcdesc}

\begin{funcdesc}{hyperg_0F1_e}{...}\index{hyperg_0F1_e}

    Number of Input  Arguments:  2
    Number of Output Arguments:  2

The error flag is discarded.
Return Arguments 1 and 2 resemble a gsl_result argument,
	which is  argument 2 of the C argument list

\end{funcdesc}

\begin{funcdesc}{hyperg_1F1}{...}\index{hyperg_1F1}

    Number of Input  Arguments:  3
    Number of Output Arguments:  1
\end{funcdesc}

\begin{funcdesc}{hyperg_1F1_e}{...}\index{hyperg_1F1_e}

    Number of Input  Arguments:  3
    Number of Output Arguments:  2

The error flag is discarded.
Return Arguments 1 and 2 resemble a gsl_result argument,
	which is  argument 3 of the C argument list

\end{funcdesc}

\begin{funcdesc}{hyperg_1F1_int}{...}\index{hyperg_1F1_int}

    Number of Input  Arguments:  3
    Number of Output Arguments:  1
\end{funcdesc}

\begin{funcdesc}{hyperg_1F1_int_e}{...}\index{hyperg_1F1_int_e}

    Number of Input  Arguments:  3
    Number of Output Arguments:  2

The error flag is discarded.
Return Arguments 1 and 2 resemble a gsl_result argument,
	which is  argument 3 of the C argument list

\end{funcdesc}

\begin{funcdesc}{hyperg_2F0}{...}\index{hyperg_2F0}

    Number of Input  Arguments:  3
    Number of Output Arguments:  1
\end{funcdesc}

\begin{funcdesc}{hyperg_2F0_e}{...}\index{hyperg_2F0_e}

    Number of Input  Arguments:  3
    Number of Output Arguments:  2

The error flag is discarded.
Return Arguments 1 and 2 resemble a gsl_result argument,
	which is  argument 3 of the C argument list

\end{funcdesc}

\begin{funcdesc}{hyperg_2F1}{...}\index{hyperg_2F1}

    Number of Input  Arguments:  4
    Number of Output Arguments:  1
\end{funcdesc}

\begin{funcdesc}{hyperg_2F1_conj}{...}\index{hyperg_2F1_conj}

    Number of Input  Arguments:  4
    Number of Output Arguments:  1
\end{funcdesc}

\begin{funcdesc}{hyperg_2F1_conj_e}{...}\index{hyperg_2F1_conj_e}

    Number of Input  Arguments:  4
    Number of Output Arguments:  2

The error flag is discarded.
Return Arguments 1 and 2 resemble a gsl_result argument,
	which is  argument 4 of the C argument list

\end{funcdesc}

\begin{funcdesc}{hyperg_2F1_conj_renorm}{...}\index{hyperg_2F1_conj_renorm}

    Number of Input  Arguments:  4
    Number of Output Arguments:  1
\end{funcdesc}

\begin{funcdesc}{hyperg_2F1_conj_renorm_e}{...}\index{hyperg_2F1_conj_renorm_e}

    Number of Input  Arguments:  4
    Number of Output Arguments:  2

The error flag is discarded.
Return Arguments 1 and 2 resemble a gsl_result argument,
	which is  argument 4 of the C argument list

\end{funcdesc}

\begin{funcdesc}{hyperg_2F1_e}{...}\index{hyperg_2F1_e}

    Number of Input  Arguments:  4
    Number of Output Arguments:  2

The error flag is discarded.
Return Arguments 1 and 2 resemble a gsl_result argument,
	which is  argument 4 of the C argument list

\end{funcdesc}

\begin{funcdesc}{hyperg_2F1_renorm}{...}\index{hyperg_2F1_renorm}

    Number of Input  Arguments:  4
    Number of Output Arguments:  1
\end{funcdesc}

\begin{funcdesc}{hyperg_2F1_renorm_e}{...}\index{hyperg_2F1_renorm_e}

    Number of Input  Arguments:  4
    Number of Output Arguments:  2

The error flag is discarded.
Return Arguments 1 and 2 resemble a gsl_result argument,
	which is  argument 4 of the C argument list

\end{funcdesc}

\begin{funcdesc}{hyperg_U}{...}\index{hyperg_U}

    Number of Input  Arguments:  3
    Number of Output Arguments:  1
\end{funcdesc}

\begin{funcdesc}{hyperg_U_e}{...}\index{hyperg_U_e}

    Number of Input  Arguments:  3
    Number of Output Arguments:  2

The error flag is discarded.
Return Arguments 1 and 2 resemble a gsl_result argument,
	which is  argument 3 of the C argument list

\end{funcdesc}

\begin{funcdesc}{hyperg_U_e10_e}{...}\index{hyperg_U_e10_e}

    Number of Input  Arguments:  3
    Number of Output Arguments:  3

The error flag is discarded.
Return Arguments 1 - 3 resemble a gsl_result_e10 argument,
	which is argument 3 of the C argument list

\end{funcdesc}

\begin{funcdesc}{hyperg_U_int}{...}\index{hyperg_U_int}

    Number of Input  Arguments:  3
    Number of Output Arguments:  1
\end{funcdesc}

\begin{funcdesc}{hyperg_U_int_e}{...}\index{hyperg_U_int_e}

    Number of Input  Arguments:  3
    Number of Output Arguments:  2

The error flag is discarded.
Return Arguments 1 and 2 resemble a gsl_result argument,
	which is  argument 3 of the C argument list

\end{funcdesc}

\begin{funcdesc}{hyperg_U_int_e10_e}{...}\index{hyperg_U_int_e10_e}

    Number of Input  Arguments:  3
    Number of Output Arguments:  3

The error flag is discarded.
Return Arguments 1 - 3 resemble a gsl_result_e10 argument,
	which is argument 3 of the C argument list

\end{funcdesc}

\begin{funcdesc}{hypot}{...}\index{hypot}

    Number of Input  Arguments:  2
    Number of Output Arguments:  1
\end{funcdesc}

\begin{funcdesc}{hypot_e}{...}\index{hypot_e}

    Number of Input  Arguments:  2
    Number of Output Arguments:  2

The error flag is discarded.
Return Arguments 1 and 2 resemble a gsl_result argument,
	which is  argument 2 of the C argument list

\end{funcdesc}

\begin{funcdesc}{hzeta}{...}\index{hzeta}

    Number of Input  Arguments:  2
    Number of Output Arguments:  1
\end{funcdesc}

\begin{funcdesc}{hzeta_e}{...}\index{hzeta_e}

    Number of Input  Arguments:  2
    Number of Output Arguments:  2

The error flag is discarded.
Return Arguments 1 and 2 resemble a gsl_result argument,
	which is  argument 2 of the C argument list

\end{funcdesc}

\begin{funcdesc}{laguerre_1}{...}\index{laguerre_1}

    Number of Input  Arguments:  2
    Number of Output Arguments:  1
\end{funcdesc}

\begin{funcdesc}{laguerre_1_e}{...}\index{laguerre_1_e}

    Number of Input  Arguments:  2
    Number of Output Arguments:  2

The error flag is discarded.
Return Arguments 1 and 2 resemble a gsl_result argument,
	which is  argument 2 of the C argument list

\end{funcdesc}

\begin{funcdesc}{laguerre_2}{...}\index{laguerre_2}

    Number of Input  Arguments:  2
    Number of Output Arguments:  1
\end{funcdesc}

\begin{funcdesc}{laguerre_2_e}{...}\index{laguerre_2_e}

    Number of Input  Arguments:  2
    Number of Output Arguments:  2

The error flag is discarded.
Return Arguments 1 and 2 resemble a gsl_result argument,
	which is  argument 2 of the C argument list

\end{funcdesc}

\begin{funcdesc}{laguerre_3}{...}\index{laguerre_3}

    Number of Input  Arguments:  2
    Number of Output Arguments:  1
\end{funcdesc}

\begin{funcdesc}{laguerre_3_e}{...}\index{laguerre_3_e}

    Number of Input  Arguments:  2
    Number of Output Arguments:  2

The error flag is discarded.
Return Arguments 1 and 2 resemble a gsl_result argument,
	which is  argument 2 of the C argument list

\end{funcdesc}

\begin{funcdesc}{laguerre_n}{...}\index{laguerre_n}

    Number of Input  Arguments:  3
    Number of Output Arguments:  1
\end{funcdesc}

\begin{funcdesc}{laguerre_n_e}{...}\index{laguerre_n_e}

    Number of Input  Arguments:  3
    Number of Output Arguments:  2

The error flag is discarded.
Return Arguments 1 and 2 resemble a gsl_result argument,
	which is  argument 3 of the C argument list

\end{funcdesc}

\begin{funcdesc}{lambert_W0}{...}\index{lambert_W0}

    Number of Input  Arguments:  1
    Number of Output Arguments:  1
\end{funcdesc}

\begin{funcdesc}{lambert_W0_e}{...}\index{lambert_W0_e}

    Number of Input  Arguments:  1
    Number of Output Arguments:  2

The error flag is discarded.
Return Arguments 1 and 2 resemble a gsl_result argument,
	which is  argument 1 of the C argument list

\end{funcdesc}

\begin{funcdesc}{lambert_Wm1}{...}\index{lambert_Wm1}

    Number of Input  Arguments:  1
    Number of Output Arguments:  1
\end{funcdesc}

\begin{funcdesc}{lambert_Wm1_e}{...}\index{lambert_Wm1_e}

    Number of Input  Arguments:  1
    Number of Output Arguments:  2

The error flag is discarded.
Return Arguments 1 and 2 resemble a gsl_result argument,
	which is  argument 1 of the C argument list

\end{funcdesc}

\begin{funcdesc}{legendre_H3d}{...}\index{legendre_H3d}

    Number of Input  Arguments:  3
    Number of Output Arguments:  1
\end{funcdesc}

\begin{funcdesc}{legendre_H3d_0}{...}\index{legendre_H3d_0}

    Number of Input  Arguments:  2
    Number of Output Arguments:  1
\end{funcdesc}

\begin{funcdesc}{legendre_H3d_0_e}{...}\index{legendre_H3d_0_e}

    Number of Input  Arguments:  2
    Number of Output Arguments:  2

The error flag is discarded.
Return Arguments 1 and 2 resemble a gsl_result argument,
	which is  argument 2 of the C argument list

\end{funcdesc}

\begin{funcdesc}{legendre_H3d_1}{...}\index{legendre_H3d_1}

    Number of Input  Arguments:  2
    Number of Output Arguments:  1
\end{funcdesc}

\begin{funcdesc}{legendre_H3d_1_e}{...}\index{legendre_H3d_1_e}

    Number of Input  Arguments:  2
    Number of Output Arguments:  2

The error flag is discarded.
Return Arguments 1 and 2 resemble a gsl_result argument,
	which is  argument 2 of the C argument list

\end{funcdesc}

\begin{funcdesc}{legendre_H3d_e}{...}\index{legendre_H3d_e}

    Number of Input  Arguments:  3
    Number of Output Arguments:  2

The error flag is discarded.
Return Arguments 1 and 2 resemble a gsl_result argument,
	which is  argument 3 of the C argument list

\end{funcdesc}

\begin{funcdesc}{legendre_P1}{...}\index{legendre_P1}

    Number of Input  Arguments:  1
    Number of Output Arguments:  1
\end{funcdesc}

\begin{funcdesc}{legendre_P1_e}{...}\index{legendre_P1_e}

    Number of Input  Arguments:  1
    Number of Output Arguments:  2

The error flag is discarded.
Return Arguments 1 and 2 resemble a gsl_result argument,
	which is  argument 1 of the C argument list

\end{funcdesc}

\begin{funcdesc}{legendre_P2}{...}\index{legendre_P2}

    Number of Input  Arguments:  1
    Number of Output Arguments:  1
\end{funcdesc}

\begin{funcdesc}{legendre_P2_e}{...}\index{legendre_P2_e}

    Number of Input  Arguments:  1
    Number of Output Arguments:  2

The error flag is discarded.
Return Arguments 1 and 2 resemble a gsl_result argument,
	which is  argument 1 of the C argument list

\end{funcdesc}

\begin{funcdesc}{legendre_P3}{...}\index{legendre_P3}

    Number of Input  Arguments:  1
    Number of Output Arguments:  1
\end{funcdesc}

\begin{funcdesc}{legendre_P3_e}{...}\index{legendre_P3_e}

    Number of Input  Arguments:  1
    Number of Output Arguments:  2

The error flag is discarded.
Return Arguments 1 and 2 resemble a gsl_result argument,
	which is  argument 1 of the C argument list

\end{funcdesc}

\begin{funcdesc}{legendre_Pl}{...}\index{legendre_Pl}

    Number of Input  Arguments:  2
    Number of Output Arguments:  1
\end{funcdesc}

\begin{funcdesc}{legendre_Pl_e}{...}\index{legendre_Pl_e}

    Number of Input  Arguments:  2
    Number of Output Arguments:  2

The error flag is discarded.
Return Arguments 1 and 2 resemble a gsl_result argument,
	which is  argument 2 of the C argument list

\end{funcdesc}

\begin{funcdesc}{legendre_Plm}{...}\index{legendre_Plm}

    Number of Input  Arguments:  3
    Number of Output Arguments:  1
\end{funcdesc}

\begin{funcdesc}{legendre_Plm_e}{...}\index{legendre_Plm_e}

    Number of Input  Arguments:  3
    Number of Output Arguments:  2

The error flag is discarded.
Return Arguments 1 and 2 resemble a gsl_result argument,
	which is  argument 3 of the C argument list

\end{funcdesc}

\begin{funcdesc}{legendre_Q0}{...}\index{legendre_Q0}

    Number of Input  Arguments:  1
    Number of Output Arguments:  1
\end{funcdesc}

\begin{funcdesc}{legendre_Q0_e}{...}\index{legendre_Q0_e}

    Number of Input  Arguments:  1
    Number of Output Arguments:  2

The error flag is discarded.
Return Arguments 1 and 2 resemble a gsl_result argument,
	which is  argument 1 of the C argument list

\end{funcdesc}

\begin{funcdesc}{legendre_Q1}{...}\index{legendre_Q1}

    Number of Input  Arguments:  1
    Number of Output Arguments:  1
\end{funcdesc}

\begin{funcdesc}{legendre_Q1_e}{...}\index{legendre_Q1_e}

    Number of Input  Arguments:  1
    Number of Output Arguments:  2

The error flag is discarded.
Return Arguments 1 and 2 resemble a gsl_result argument,
	which is  argument 1 of the C argument list

\end{funcdesc}

\begin{funcdesc}{legendre_Ql}{...}\index{legendre_Ql}

    Number of Input  Arguments:  2
    Number of Output Arguments:  1
\end{funcdesc}

\begin{funcdesc}{legendre_Ql_e}{...}\index{legendre_Ql_e}

    Number of Input  Arguments:  2
    Number of Output Arguments:  2

The error flag is discarded.
Return Arguments 1 and 2 resemble a gsl_result argument,
	which is  argument 2 of the C argument list

\end{funcdesc}

\begin{funcdesc}{legendre_sphPlm}{...}\index{legendre_sphPlm}

    Number of Input  Arguments:  3
    Number of Output Arguments:  1
\end{funcdesc}

\begin{funcdesc}{legendre_sphPlm_e}{...}\index{legendre_sphPlm_e}

    Number of Input  Arguments:  3
    Number of Output Arguments:  2

The error flag is discarded.
Return Arguments 1 and 2 resemble a gsl_result argument,
	which is  argument 3 of the C argument list

\end{funcdesc}

\begin{funcdesc}{lnbeta}{...}\index{lnbeta}

    Number of Input  Arguments:  2
    Number of Output Arguments:  1
\end{funcdesc}

\begin{funcdesc}{lnbeta_e}{...}\index{lnbeta_e}

    Number of Input  Arguments:  2
    Number of Output Arguments:  2

The error flag is discarded.
Return Arguments 1 and 2 resemble a gsl_result argument,
	which is  argument 2 of the C argument list

\end{funcdesc}

\begin{funcdesc}{lnchoose}{...}\index{lnchoose}

    Number of Input  Arguments:  2
    Number of Output Arguments:  1
\end{funcdesc}

\begin{funcdesc}{lnchoose_e}{...}\index{lnchoose_e}

    Number of Input  Arguments:  2
    Number of Output Arguments:  2

The error flag is discarded.
Return Arguments 1 and 2 resemble a gsl_result argument,
	which is  argument 2 of the C argument list

\end{funcdesc}

\begin{funcdesc}{lncosh}{...}\index{lncosh}

    Number of Input  Arguments:  1
    Number of Output Arguments:  1
\end{funcdesc}

\begin{funcdesc}{lncosh_e}{...}\index{lncosh_e}

    Number of Input  Arguments:  1
    Number of Output Arguments:  2

The error flag is discarded.
Return Arguments 1 and 2 resemble a gsl_result argument,
	which is  argument 1 of the C argument list

\end{funcdesc}

\begin{funcdesc}{lndoublefact}{...}\index{lndoublefact}

    Number of Input  Arguments:  1
    Number of Output Arguments:  1
\end{funcdesc}

\begin{funcdesc}{lndoublefact_e}{...}\index{lndoublefact_e}

    Number of Input  Arguments:  1
    Number of Output Arguments:  2

The error flag is discarded.
Return Arguments 1 and 2 resemble a gsl_result argument,
	which is  argument 1 of the C argument list

\end{funcdesc}

\begin{funcdesc}{lnfact}{...}\index{lnfact}

    Number of Input  Arguments:  1
    Number of Output Arguments:  1
\end{funcdesc}

\begin{funcdesc}{lnfact_e}{...}\index{lnfact_e}

    Number of Input  Arguments:  1
    Number of Output Arguments:  2

The error flag is discarded.
Return Arguments 1 and 2 resemble a gsl_result argument,
	which is  argument 1 of the C argument list

\end{funcdesc}

\begin{funcdesc}{lngamma}{...}\index{lngamma}

    Number of Input  Arguments:  1
    Number of Output Arguments:  1
\end{funcdesc}

\begin{funcdesc}{lngamma_e}{...}\index{lngamma_e}

    Number of Input  Arguments:  1
    Number of Output Arguments:  2

The error flag is discarded.
Return Arguments 1 and 2 resemble a gsl_result argument,
	which is  argument 1 of the C argument list

\end{funcdesc}

\begin{funcdesc}{lngamma_sgn_e}{...}\index{lngamma_sgn_e}

    Number of Input  Arguments:  1
    Number of Output Arguments:  3

The error flag is discarded.
Return Arguments 1 and 2 resemble a gsl_result argument,
	which is  argument 1 of the C argument list

\end{funcdesc}

\begin{funcdesc}{lnpoch}{...}\index{lnpoch}

    Number of Input  Arguments:  2
    Number of Output Arguments:  1
\end{funcdesc}

\begin{funcdesc}{lnpoch_e}{...}\index{lnpoch_e}

    Number of Input  Arguments:  2
    Number of Output Arguments:  2

The error flag is discarded.
Return Arguments 1 and 2 resemble a gsl_result argument,
	which is  argument 2 of the C argument list

\end{funcdesc}

\begin{funcdesc}{lnpoch_sgn_e}{...}\index{lnpoch_sgn_e}

    Number of Input  Arguments:  2
    Number of Output Arguments:  3

The error flag is discarded.
Return Arguments 1 and 2 resemble a gsl_result argument,
	which is  argument 2 of the C argument list

\end{funcdesc}

\begin{funcdesc}{lnsinh}{...}\index{lnsinh}

    Number of Input  Arguments:  1
    Number of Output Arguments:  1
\end{funcdesc}

\begin{funcdesc}{lnsinh_e}{...}\index{lnsinh_e}

    Number of Input  Arguments:  1
    Number of Output Arguments:  2

The error flag is discarded.
Return Arguments 1 and 2 resemble a gsl_result argument,
	which is  argument 1 of the C argument list

\end{funcdesc}

\begin{funcdesc}{log}{...}\index{log}

    Number of Input  Arguments:  1
    Number of Output Arguments:  1
\end{funcdesc}

\begin{funcdesc}{log_1plusx}{...}\index{log_1plusx}

    Number of Input  Arguments:  1
    Number of Output Arguments:  1
\end{funcdesc}

\begin{funcdesc}{log_1plusx_e}{...}\index{log_1plusx_e}

    Number of Input  Arguments:  1
    Number of Output Arguments:  2

The error flag is discarded.
Return Arguments 1 and 2 resemble a gsl_result argument,
	which is  argument 1 of the C argument list

\end{funcdesc}

\begin{funcdesc}{log_1plusx_mx}{...}\index{log_1plusx_mx}

    Number of Input  Arguments:  1
    Number of Output Arguments:  1
\end{funcdesc}

\begin{funcdesc}{log_1plusx_mx_e}{...}\index{log_1plusx_mx_e}

    Number of Input  Arguments:  1
    Number of Output Arguments:  2

The error flag is discarded.
Return Arguments 1 and 2 resemble a gsl_result argument,
	which is  argument 1 of the C argument list

\end{funcdesc}

\begin{funcdesc}{log_abs}{...}\index{log_abs}

    Number of Input  Arguments:  1
    Number of Output Arguments:  1
\end{funcdesc}

\begin{funcdesc}{log_abs_e}{...}\index{log_abs_e}

    Number of Input  Arguments:  1
    Number of Output Arguments:  2

The error flag is discarded.
Return Arguments 1 and 2 resemble a gsl_result argument,
	which is  argument 1 of the C argument list

\end{funcdesc}

\begin{funcdesc}{log_e}{...}\index{log_e}

    Number of Input  Arguments:  1
    Number of Output Arguments:  2

The error flag is discarded.
Return Arguments 1 and 2 resemble a gsl_result argument,
	which is  argument 1 of the C argument list

\end{funcdesc}

\begin{funcdesc}{log_erfc}{...}\index{log_erfc}

    Number of Input  Arguments:  1
    Number of Output Arguments:  1
\end{funcdesc}

\begin{funcdesc}{log_erfc_e}{...}\index{log_erfc_e}

    Number of Input  Arguments:  1
    Number of Output Arguments:  2

The error flag is discarded.
Return Arguments 1 and 2 resemble a gsl_result argument,
	which is  argument 1 of the C argument list

\end{funcdesc}

\begin{funcdesc}{multiply}{...}\index{multiply}

    Number of Input  Arguments:  2
    Number of Output Arguments:  1
\end{funcdesc}

\begin{funcdesc}{multiply_e}{...}\index{multiply_e}

    Number of Input  Arguments:  2
    Number of Output Arguments:  2

The error flag is discarded.
Return Arguments 1 and 2 resemble a gsl_result argument,
	which is  argument 2 of the C argument list

\end{funcdesc}

\begin{funcdesc}{multiply_err_e}{...}\index{multiply_err_e}

    Number of Input  Arguments:  4
    Number of Output Arguments:  2

The error flag is discarded.
Return Arguments 1 and 2 resemble a gsl_result argument,
	which is  argument 4 of the C argument list

\end{funcdesc}

\begin{funcdesc}{poch}{...}\index{poch}

    Number of Input  Arguments:  2
    Number of Output Arguments:  1
\end{funcdesc}

\begin{funcdesc}{poch_e}{...}\index{poch_e}

    Number of Input  Arguments:  2
    Number of Output Arguments:  2

The error flag is discarded.
Return Arguments 1 and 2 resemble a gsl_result argument,
	which is  argument 2 of the C argument list

\end{funcdesc}

\begin{funcdesc}{pochrel}{...}\index{pochrel}

    Number of Input  Arguments:  2
    Number of Output Arguments:  1
\end{funcdesc}

\begin{funcdesc}{pochrel_e}{...}\index{pochrel_e}

    Number of Input  Arguments:  2
    Number of Output Arguments:  2

The error flag is discarded.
Return Arguments 1 and 2 resemble a gsl_result argument,
	which is  argument 2 of the C argument list

\end{funcdesc}

\begin{funcdesc}{polar_to_rect}{...}\index{polar_to_rect}

\end{funcdesc}

\begin{funcdesc}{pow_int}{...}\index{pow_int}

    Number of Input  Arguments:  2
    Number of Output Arguments:  1
\end{funcdesc}

\begin{funcdesc}{pow_int_e}{...}\index{pow_int_e}

    Number of Input  Arguments:  2
    Number of Output Arguments:  2

The error flag is discarded.
Return Arguments 1 and 2 resemble a gsl_result argument,
	which is  argument 2 of the C argument list

\end{funcdesc}

\begin{funcdesc}{psi}{...}\index{psi}

    Number of Input  Arguments:  1
    Number of Output Arguments:  1
\end{funcdesc}

\begin{funcdesc}{psi_1_int}{...}\index{psi_1_int}

    Number of Input  Arguments:  1
    Number of Output Arguments:  1
\end{funcdesc}

\begin{funcdesc}{psi_1_int_e}{...}\index{psi_1_int_e}

    Number of Input  Arguments:  1
    Number of Output Arguments:  2

The error flag is discarded.
Return Arguments 1 and 2 resemble a gsl_result argument,
	which is  argument 1 of the C argument list

\end{funcdesc}

\begin{funcdesc}{psi_1piy}{...}\index{psi_1piy}

    Number of Input  Arguments:  1
    Number of Output Arguments:  1
\end{funcdesc}

\begin{funcdesc}{psi_1piy_e}{...}\index{psi_1piy_e}

    Number of Input  Arguments:  1
    Number of Output Arguments:  2

The error flag is discarded.
Return Arguments 1 and 2 resemble a gsl_result argument,
	which is  argument 1 of the C argument list

\end{funcdesc}

\begin{funcdesc}{psi_e}{...}\index{psi_e}

    Number of Input  Arguments:  1
    Number of Output Arguments:  2

The error flag is discarded.
Return Arguments 1 and 2 resemble a gsl_result argument,
	which is  argument 1 of the C argument list

\end{funcdesc}

\begin{funcdesc}{psi_int}{...}\index{psi_int}

    Number of Input  Arguments:  1
    Number of Output Arguments:  1
\end{funcdesc}

\begin{funcdesc}{psi_int_e}{...}\index{psi_int_e}

    Number of Input  Arguments:  1
    Number of Output Arguments:  2

The error flag is discarded.
Return Arguments 1 and 2 resemble a gsl_result argument,
	which is  argument 1 of the C argument list

\end{funcdesc}

\begin{funcdesc}{psi_n}{...}\index{psi_n}

    Number of Input  Arguments:  2
    Number of Output Arguments:  1
\end{funcdesc}

\begin{funcdesc}{psi_n_e}{...}\index{psi_n_e}

    Number of Input  Arguments:  2
    Number of Output Arguments:  2

The error flag is discarded.
Return Arguments 1 and 2 resemble a gsl_result argument,
	which is  argument 2 of the C argument list

\end{funcdesc}

\begin{funcdesc}{rect_to_polar}{...}\index{rect_to_polar}

\end{funcdesc}

\begin{funcdesc}{sin}{...}\index{sin}

    Number of Input  Arguments:  1
    Number of Output Arguments:  1
\end{funcdesc}

\begin{funcdesc}{sin_e}{...}\index{sin_e}

    Number of Input  Arguments:  1
    Number of Output Arguments:  2

The error flag is discarded.
Return Arguments 1 and 2 resemble a gsl_result argument,
	which is  argument 1 of the C argument list

\end{funcdesc}

\begin{funcdesc}{sin_err_e}{...}\index{sin_err_e}

    Number of Input  Arguments:  2
    Number of Output Arguments:  2

The error flag is discarded.
Return Arguments 1 and 2 resemble a gsl_result argument,
	which is  argument 2 of the C argument list

\end{funcdesc}

\begin{funcdesc}{sinc}{...}\index{sinc}

    Number of Input  Arguments:  1
    Number of Output Arguments:  1
\end{funcdesc}

\begin{funcdesc}{sinc_e}{...}\index{sinc_e}

    Number of Input  Arguments:  1
    Number of Output Arguments:  2

The error flag is discarded.
Return Arguments 1 and 2 resemble a gsl_result argument,
	which is  argument 1 of the C argument list

\end{funcdesc}

\begin{funcdesc}{synchrotron_1}{...}\index{synchrotron_1}

    Number of Input  Arguments:  1
    Number of Output Arguments:  1
\end{funcdesc}

\begin{funcdesc}{synchrotron_1_e}{...}\index{synchrotron_1_e}

    Number of Input  Arguments:  1
    Number of Output Arguments:  2

The error flag is discarded.
Return Arguments 1 and 2 resemble a gsl_result argument,
	which is  argument 1 of the C argument list

\end{funcdesc}

\begin{funcdesc}{synchrotron_2}{...}\index{synchrotron_2}

    Number of Input  Arguments:  1
    Number of Output Arguments:  1
\end{funcdesc}

\begin{funcdesc}{synchrotron_2_e}{...}\index{synchrotron_2_e}

    Number of Input  Arguments:  1
    Number of Output Arguments:  2

The error flag is discarded.
Return Arguments 1 and 2 resemble a gsl_result argument,
	which is  argument 1 of the C argument list

\end{funcdesc}

\begin{funcdesc}{taylorcoeff}{...}\index{taylorcoeff}

    Number of Input  Arguments:  2
    Number of Output Arguments:  1
\end{funcdesc}

\begin{funcdesc}{taylorcoeff_e}{...}\index{taylorcoeff_e}

    Number of Input  Arguments:  2
    Number of Output Arguments:  2

The error flag is discarded.
Return Arguments 1 and 2 resemble a gsl_result argument,
	which is  argument 2 of the C argument list

\end{funcdesc}

\begin{funcdesc}{transport_2}{...}\index{transport_2}

    Number of Input  Arguments:  1
    Number of Output Arguments:  1
\end{funcdesc}

\begin{funcdesc}{transport_2_e}{...}\index{transport_2_e}

    Number of Input  Arguments:  1
    Number of Output Arguments:  2

The error flag is discarded.
Return Arguments 1 and 2 resemble a gsl_result argument,
	which is  argument 1 of the C argument list

\end{funcdesc}

\begin{funcdesc}{transport_3}{...}\index{transport_3}

    Number of Input  Arguments:  1
    Number of Output Arguments:  1
\end{funcdesc}

\begin{funcdesc}{transport_3_e}{...}\index{transport_3_e}

    Number of Input  Arguments:  1
    Number of Output Arguments:  2

The error flag is discarded.
Return Arguments 1 and 2 resemble a gsl_result argument,
	which is  argument 1 of the C argument list

\end{funcdesc}

\begin{funcdesc}{transport_4}{...}\index{transport_4}

    Number of Input  Arguments:  1
    Number of Output Arguments:  1
\end{funcdesc}

\begin{funcdesc}{transport_4_e}{...}\index{transport_4_e}

    Number of Input  Arguments:  1
    Number of Output Arguments:  2

The error flag is discarded.
Return Arguments 1 and 2 resemble a gsl_result argument,
	which is  argument 1 of the C argument list

\end{funcdesc}

\begin{funcdesc}{transport_5}{...}\index{transport_5}

    Number of Input  Arguments:  1
    Number of Output Arguments:  1
\end{funcdesc}

\begin{funcdesc}{transport_5_e}{...}\index{transport_5_e}

    Number of Input  Arguments:  1
    Number of Output Arguments:  2

The error flag is discarded.
Return Arguments 1 and 2 resemble a gsl_result argument,
	which is  argument 1 of the C argument list

\end{funcdesc}

\begin{funcdesc}{zeta}{...}\index{zeta}

    Number of Input  Arguments:  1
    Number of Output Arguments:  1
\end{funcdesc}

\begin{funcdesc}{zeta_e}{...}\index{zeta_e}

    Number of Input  Arguments:  1
    Number of Output Arguments:  2

The error flag is discarded.
Return Arguments 1 and 2 resemble a gsl_result argument,
	which is  argument 1 of the C argument list

\end{funcdesc}

\begin{funcdesc}{zeta_int}{...}\index{zeta_int}

    Number of Input  Arguments:  1
    Number of Output Arguments:  1
\end{funcdesc}

\begin{funcdesc}{zeta_int_e}{...}\index{zeta_int_e}

    Number of Input  Arguments:  1
    Number of Output Arguments:  2

The error flag is discarded.
Return Arguments 1 and 2 resemble a gsl_result argument,
	which is  argument 1 of the C argument list

\end{funcdesc}

\section{Ordinary Functions}

The following array functions have been wrapped. These are supposingly faster
than the equivalent functions from above.
\begin{funcdesc}{bessel_In_array}{...}\index{bessel_In_array}
\end{funcdesc}
\begin{funcdesc}{bessel_Jn_array}{...}\index{bessel_Jn_array}
\end{funcdesc}
\begin{funcdesc}{bessel_Kn_array}{...}\index{bessel_Kn_array}
\end{funcdesc}
\begin{funcdesc}{bessel_Kn_scaled_array}{...}\index{bessel_Kn_scaled_array}
\end{funcdesc}
\begin{funcdesc}{bessel_Yn_array}{...}\index{bessel_Yn_array}
\end{funcdesc}
\begin{funcdesc}{bessel_il_scaled_array}{...}\index{bessel_il_scaled_array}
\end{funcdesc}
\begin{funcdesc}{bessel_jl_array}{...}\index{bessel_jl_array}
\end{funcdesc}
\begin{funcdesc}{bessel_jl_steed_array}{...}\index{bessel_jl_steed_array}
\end{funcdesc}
\begin{funcdesc}{bessel_kl_scaled_array}{...}\index{bessel_kl_scaled_array}
\end{funcdesc}
\begin{funcdesc}{bessel_yl_array}{...}\index{bessel_yl_array}
\end{funcdesc}


%%% Local Variables: 
%%% mode: latex
%%% TeX-master: "ref"
%%% End: 


\appendix

\chapter[\protect\module{pygsl.ieee} --- Floating Point Unit Support]
{\protect\module{pygsl.ieee} \\ Floating Point Unit Support}
\label{cha:ieee-module}
\declaremodule{extension}{pygsl.ieee}
\moduleauthor{Achim G\"adke}{achimgaedke@users.sourceforge.net}

This chapter lists features to configure the ``Floating Point Unit'' of your machine.


\chapter[\protect\module{pygsl.init} --- Library initialisation]
{\protect\module{pygsl.init} \\ Library initialisation}
\label{cha:library-initialisation}
\declaremodule{extension}{pygsl.init}
\moduleauthor{Pierre Schnizer}{schnizer@users.sourceforge.net}
\moduleauthor{Achim G\"adke}{achimgaedke@users.sourceforge.net}

This module is called the first time when loading \module{pygsl}.
All following procedures are called once and before everything other.

\section{Exception handling}
\index{exception handling!initialisation} GSL provides a selectable error
handler, that is called for occuring errors (like domain errors, division by
zero, etc. ).  \module{pygsl.init} installs a handler by calling
\cfunction{gsl_set_error_handler} to set an appropiate exception from
\module{pygsl.errors}.  A \module{pygsl} interface function should return
\code{NULL} in case of an error, so the exception is raised.  If this handler
is called more than once before returning to python, only the first set
exception is raised.

Here is a python level example:
\begin{verbatim}
import pygsl.histogram
import pygsl.errors
hist=pygsl.histogram.histogram2d(100,100)
try:
   hist[-1,-1]=0
except pygsl.errors.gsl_Error,err:
   print err
\end{verbatim}
Will result
\begin{verbatim}
input domain error: index i lies outside valid range of 0 .. nx - 1
\end{verbatim}

\section{IEEE-mode}
\index{ieee-mode!initialisation}
The IEEE mode is set from the environment variable
 \envvar{GSL_IEEE_MODE} via \cfunction{gsl_ieee_env_setup()}.
After the initialisation use \module{pygsl.ieee} for manipulation.

\section{random number generators}
\index{random number generator!initialisation}
Also the random number generator can be initialised from the environment variables
 \envvar{GSL_RNG_TYPE}
and \envvar{GSL_RNG_SEED} using the gsl function \cfunction{gsl_rng_env_setup()}.
Each random number generators are initialised with \envvar{GSL_RNG_SEED}.

The default generator can be created by:\nopagebreak
\begin{verbatim}
import pygsl.rng
my_rng=pygsl.rng.rng()
print my_rng.name()
\end{verbatim}


\chapter[\protect\module{pygsl.errors} --- Error and warning classes]
{\protect\module{pygsl.errors} \\ Error and warning classes} 
\label{cha:error-module}
\declaremodule{standard}{pygsl.errors}
\moduleauthor{Pierre Schnizer}{schnizer@users.sourceforge.net}
\moduleauthor{Original Author: Achim G\"adke}{achimgaedke@users.sourceforge.net}

This chapter provides information about the \exception{gsl_Error} exception class that comes with this module.

\section{Exception Classes}


\begin{excclassdesc} {gsl_Error}{}
derived from \exception{Exception}, can be constructed with any object as parameter.
It is baseclass to all other \gsl{} Exceptions
\end{excclassdesc}
These classes are translations of the \file{<gsl/gsl_errno.h>} to python
exceptions.


\begin{excclassdesc}{gsl_ArithmeticError}{}
derived from \exception{gsl_Error} and \exception{exceptions.ArithmeticError},
base of all common arithmetic exceptions
\end{excclassdesc}

\begin{excclassdesc}{gsl_OverflowError}{}
derived from \exception{gsl_Error} and \exception{exceptions.OverflowError}
\end{excclassdesc}

\begin{excclassdesc}{gsl_ZeroDivisionError}{}
derived from \exception{gsl_Error} and \exception{exceptions.ZeroDivisionError}
\end{excclassdesc}

\begin{excclassdesc}{gsl_FloatingPointError}{}
derived from \exception{gsl_Error} and \exception{exceptions.FloatingPointError}
\end{excclassdesc}

\begin{excclassdesc}{gsl_ArithmeticError}{}
is derived from  \exception{gsl_Error} and from  \exception{ArithmeticError} .
This exception is the    base of all common arithmetic exceptions.
\end{excclassdesc}

\begin{excclassdesc}{gsl_AccuracyLossError}{}
is derived from  \exception{gsl_ArithmeticError} .
This exception is raised if the failed to reach the specified tolerance.
\end{excclassdesc}
\begin{excclassdesc}{gsl_BadFuncError}{}
is derived from  \exception{gsl_Error} .
This exception is raised if problem with a user-supplied function occur.
\end{excclassdesc}
\begin{excclassdesc}{gsl_BadLength}{}
is derived from  \exception{gsl_Error} .
This exception is raised if  matrix or  vector lengths are not conformant.
\end{excclassdesc}
\begin{excclassdesc}{gsl_BadToleranceError}{}
is derived from  \exception{gsl_Error} .
This exception is raised if user specified an tolerance which can not be reached.
\end{excclassdesc}
\begin{excclassdesc}{gsl_CacheLimitError}{}
is derived from  \exception{gsl_Error} .
This exception is raised if the    cache limit is exceeded.
\end{excclassdesc}
\begin{excclassdesc}{gsl_DivergeError}{}
is derived from  \exception{gsl_ArithmeticError} .
This exception is raised if an   integral or series is divergent.
\end{excclassdesc}
\begin{excclassdesc}{gsl_DomainError}{}
is derived from  \exception{gsl_Error} .
This exception is raised if    domain errors occure. e.g. sqrt(-1).
\end{excclassdesc}
\begin{excclassdesc}{gsl_EOFError}{}
is derived from  \exception{gsl_Error} and from  \exception{EOFError} .
This exception is raised if 
    end of file
     .
\end{excclassdesc}
\begin{excclassdesc}{gsl_FactorizationError}{}
is derived from  \exception{gsl_Error} .
This exception is raised if     factorization failed.
\end{excclassdesc}
\begin{excclassdesc}{gsl_FloatingPointError}{}
is derived from  \exception{gsl_Error} and from  \exception{FloatingPointError} .
\end{excclassdesc}
\begin{excclassdesc}{gsl_GenericError}{}
is derived from  \exception{gsl_Error} .
\end{excclassdesc}
\begin{excclassdesc}{gsl_InvalidArgumentError}{}
is derived from  \exception{gsl_Error} .
This exception is raised if an invalid argument is supplied by the user.
\end{excclassdesc}
\begin{excclassdesc}{gsl_JacobianEvaluationError}{}
is derived from  \exception{gsl_ArithmeticError} .
This exception is raised if jacobian evaluations are not improving the solution.
\end{excclassdesc}
\begin{excclassdesc}{gsl_MatrixNotSquare}{}
is derived from  \exception{gsl_Error} .
This exception is raised if the given matrix is not square.
\end{excclassdesc}
\begin{excclassdesc}{gsl_MaximumIterationError}{}
is derived from  \exception{gsl_ArithmeticError} .
This exception is raised if    the maximum number  of iterations is exceeded.
\end{excclassdesc}
\begin{excclassdesc}{gsl_NoHardwareSupportError}{}
is derived from  \exception{gsl_Error} .
This exception is raised if the requested feature is not supported by the hardware.
\end{excclassdesc}
\begin{excclassdesc}{gsl_NoProgressError}{}
is derived from  \exception{gsl_ArithmeticError} .
This exception is raised if the  iteration is not making progress towards solution.
\end{excclassdesc}
\begin{excclassdesc}{gsl_NotImplementedError}{}
is derived from  \exception{gsl_Error} and from  \exception{NotImplementedError} .
This exception is raised if  a requested feature is not (yet) implemented .
\end{excclassdesc}
\begin{excclassdesc}{gsl_OverflowError}{}
is derived from  \exception{gsl_Error} and from  \exception{OverflowError} .
\end{excclassdesc}
\begin{excclassdesc}{gsl_PointerError}{}
is derived from  \exception{gsl_Error} .
This exception is raised if an invalid pointer is found by the C wrapper code
or by the GSL library.
\end{excclassdesc}
\begin{excclassdesc}{gsl_RangeError}{}
is derived from  \exception{gsl_ArithmeticError} .
This exception is raised if     output would be out or range, e.g. exp(1e100)
     .
\end{excclassdesc}
\begin{excclassdesc}{gsl_RoundOffError}{}
is derived from  \exception{gsl_ArithmeticError} .
This exception is raised if  arithmetic failed because of roundoff error.
\end{excclassdesc}
\begin{excclassdesc}{gsl_RunAwayError}{}
is derived from  \exception{gsl_ArithmeticError} .
This exception is raised if   iterative process is out of control.
\end{excclassdesc}
\begin{excclassdesc}{gsl_SanityCheckError}{}
is derived from  \exception{gsl_Error} .
This exception is raised if a sanity check failed - shouldn't happen.
\end{excclassdesc}
\begin{excclassdesc}{gsl_SingularityError}{}
is derived from  \exception{gsl_ArithmeticError} .
This exception is raised if  an   apparent singularity is detected.
\end{excclassdesc}
\begin{excclassdesc}{gsl_TableLimitError}{}
is derived from  \exception{gsl_Error} .
This exception is raised if the table limit is exceeded.
\end{excclassdesc}
\begin{excclassdesc}{gsl_ToleranceError}{}
is derived from  \exception{gsl_ArithmeticError} .
This exception is raised if  the alghorithm failed to reach the specified tolerance.
\end{excclassdesc}
\begin{excclassdesc}{gsl_ToleranceFError}{}
is derived from  \exception{gsl_ArithmeticError} .
This exception is raised if  the alghorithm cannot reach the specified
tolerance in F (typically the variation of the evaluated function).
\end{excclassdesc}
\begin{excclassdesc}{gsl_ToleranceGradientError}{}
is derived from  \exception{gsl_ArithmeticError} .
This exception is raised if  cannot reach the specified tolerance for the gradient.
\end{excclassdesc}
\begin{excclassdesc}{gsl_ToleranceXError}{}
is derived from  \exception{gsl_ArithmeticError} .
This exception is raised if cannot reach the specified tolerance in X
(typically a search result).
\end{excclassdesc}
\begin{excclassdesc}{gsl_UnderflowError}{}
is derived from  \exception{gsl_Error} and from  \exception{OverflowError} .
\end{excclassdesc}
\begin{excclassdesc}{gsl_ZeroDivisionError}{}
is derived from  \exception{gsl_Error} and from  \exception{ZeroDivisionError} .
\end{excclassdesc}

All the above errors are just translations of the errno to python exceptions.

The following two are specific to pygsl:
\begin{excclassdesc}{pygsl.errors.pygsl_NotImplementedError}{}
is derived from  \exception{gsl_Error} and from  \exception{NotImplementedError} .
This exception is raised if a feature is requested but not
implemented. Currently only used if a module requests the debugging enviroment
of the init module, but the init module was not compiled with \code{\#define DEBUG=1}
\end{excclassdesc}
\begin{excclassdesc}{pygsl.errors.pygsl_StrideError}{}
is derived from  \exception{gsl_SanityCheckError} .
GSL uses as strides multiples of the basis type; for a vector or doubles, one
means from one double to the next. Numpy or numarray count the stride in
multiples of the size of a char. Therefore the stride has to be recalculated
before the approbriate \gsl{} function can be called. If that fails this
exception is raised.
\end{excclassdesc}

\section{Warning Classes}

\begin{excclassdesc} {gsl_Warning}{}
The dedicated warning class for \gsl{} has \exception{Warning} as base class.
\end{excclassdesc}

\begin{excclassdesc}{gsl_DomainWarning}{}
derived from \exception{gsl_Warning}, used by some \module{pygsl.histogram} functions
\end{excclassdesc}


\chapter{GNU Free Documentation License}
\label{cha:free-documentation-license}

Version 1.1, March 2000\\

 Copyright \copyright\ 2000  Free Software Foundation, Inc.\\
     59 Temple Place, Suite 330, Boston, MA  02111-1307  USA\\
 Everyone is permitted to copy and distribute verbatim copies
 of this license document, but changing it is not allowed.

\section*{Preamble}

The purpose of this License is to make a manual, textbook, or other
written document ``free'' in the sense of freedom: to assure everyone
the effective freedom to copy and redistribute it, with or without
modifying it, either commercially or noncommercially.  Secondarily,
this License preserves for the author and publisher a way to get
credit for their work, while not being considered responsible for
modifications made by others.

This License is a kind of ``copyleft'', which means that derivative
works of the document must themselves be free in the same sense.  It
complements the GNU General Public License, which is a copyleft
license designed for free software.

We have designed this License in order to use it for manuals for free
software, because free software needs free documentation: a free
program should come with manuals providing the same freedoms that the
software does.  But this License is not limited to software manuals;
it can be used for any textual work, regardless of subject matter or
whether it is published as a printed book.  We recommend this License
principally for works whose purpose is instruction or reference.

\section{Applicability and Definitions}

This License applies to any manual or other work that contains a
notice placed by the copyright holder saying it can be distributed
under the terms of this License.  The ``Document'', below, refers to any
such manual or work.  Any member of the public is a licensee, and is
addressed as ``you''.

A ``Modified Version'' of the Document means any work containing the
Document or a portion of it, either copied verbatim, or with
modifications and/or translated into another language.

A ``Secondary Section'' is a named appendix or a front-matter section of
the Document that deals exclusively with the relationship of the
publishers or authors of the Document to the Document's overall subject
(or to related matters) and contains nothing that could fall directly
within that overall subject.  (For example, if the Document is in part a
textbook of mathematics, a Secondary Section may not explain any
mathematics.)  The relationship could be a matter of historical
connection with the subject or with related matters, or of legal,
commercial, philosophical, ethical or political position regarding
them.

The ``Invariant Sections'' are certain Secondary Sections whose titles
are designated, as being those of Invariant Sections, in the notice
that says that the Document is released under this License.

The ``Cover Texts'' are certain short passages of text that are listed,
as Front-Cover Texts or Back-Cover Texts, in the notice that says that
the Document is released under this License.

A ``Transparent'' copy of the Document means a machine-readable copy,
represented in a format whose specification is available to the
general public, whose contents can be viewed and edited directly and
straightforwardly with generic text editors or (for images composed of
pixels) generic paint programs or (for drawings) some widely available
drawing editor, and that is suitable for input to text formatters or
for automatic translation to a variety of formats suitable for input
to text formatters.  A copy made in an otherwise Transparent file
format whose markup has been designed to thwart or discourage
subsequent modification by readers is not Transparent.  A copy that is
not ``Transparent'' is called ``Opaque''.

Examples of suitable formats for Transparent copies include plain
ASCII without markup, Texinfo input format, \LaTeX~input format, SGML
or XML using a publicly available DTD, and standard-conforming simple
HTML designed for human modification.  Opaque formats include
PostScript, PDF, proprietary formats that can be read and edited only
by proprietary word processors, SGML or XML for which the DTD and/or
processing tools are not generally available, and the
machine-generated HTML produced by some word processors for output
purposes only.

The ``Title Page'' means, for a printed book, the title page itself,
plus such following pages as are needed to hold, legibly, the material
this License requires to appear in the title page.  For works in
formats which do not have any title page as such, ``Title Page'' means
the text near the most prominent appearance of the work's title,
preceding the beginning of the body of the text.


\section{Verbatim Copying}

You may copy and distribute the Document in any medium, either
commercially or noncommercially, provided that this License, the
copyright notices, and the license notice saying this License applies
to the Document are reproduced in all copies, and that you add no other
conditions whatsoever to those of this License.  You may not use
technical measures to obstruct or control the reading or further
copying of the copies you make or distribute.  However, you may accept
compensation in exchange for copies.  If you distribute a large enough
number of copies you must also follow the conditions in section 3.

You may also lend copies, under the same conditions stated above, and
you may publicly display copies.


\section{Copying in Quantity}

If you publish printed copies of the Document numbering more than 100,
and the Document's license notice requires Cover Texts, you must enclose
the copies in covers that carry, clearly and legibly, all these Cover
Texts: Front-Cover Texts on the front cover, and Back-Cover Texts on
the back cover.  Both covers must also clearly and legibly identify
you as the publisher of these copies.  The front cover must present
the full title with all words of the title equally prominent and
visible.  You may add other material on the covers in addition.
Copying with changes limited to the covers, as long as they preserve
the title of the Document and satisfy these conditions, can be treated
as verbatim copying in other respects.

If the required texts for either cover are too voluminous to fit
legibly, you should put the first ones listed (as many as fit
reasonably) on the actual cover, and continue the rest onto adjacent
pages.

If you publish or distribute Opaque copies of the Document numbering
more than 100, you must either include a machine-readable Transparent
copy along with each Opaque copy, or state in or with each Opaque copy
a publicly-accessible computer-network location containing a complete
Transparent copy of the Document, free of added material, which the
general network-using public has access to download anonymously at no
charge using public-standard network protocols.  If you use the latter
option, you must take reasonably prudent steps, when you begin
distribution of Opaque copies in quantity, to ensure that this
Transparent copy will remain thus accessible at the stated location
until at least one year after the last time you distribute an Opaque
copy (directly or through your agents or retailers) of that edition to
the public.

It is requested, but not required, that you contact the authors of the
Document well before redistributing any large number of copies, to give
them a chance to provide you with an updated version of the Document.


\section{Modifications}

You may copy and distribute a Modified Version of the Document under
the conditions of sections 2 and 3 above, provided that you release
the Modified Version under precisely this License, with the Modified
Version filling the role of the Document, thus licensing distribution
and modification of the Modified Version to whoever possesses a copy
of it.  In addition, you must do these things in the Modified Version:

\begin{itemize}

\item Use in the Title Page (and on the covers, if any) a title distinct
   from that of the Document, and from those of previous versions
   (which should, if there were any, be listed in the History section
   of the Document).  You may use the same title as a previous version
   if the original publisher of that version gives permission.
\item List on the Title Page, as authors, one or more persons or entities
   responsible for authorship of the modifications in the Modified
   Version, together with at least five of the principal authors of the
   Document (all of its principal authors, if it has less than five).
\item State on the Title page the name of the publisher of the
   Modified Version, as the publisher.
\item Preserve all the copyright notices of the Document.
\item Add an appropriate copyright notice for your modifications
   adjacent to the other copyright notices.
\item Include, immediately after the copyright notices, a license notice
   giving the public permission to use the Modified Version under the
   terms of this License, in the form shown in the Addendum below.
\item Preserve in that license notice the full lists of Invariant Sections
   and required Cover Texts given in the Document's license notice.
\item Include an unaltered copy of this License.
\item Preserve the section entitled ``History'', and its title, and add to
   it an item stating at least the title, year, new authors, and
   publisher of the Modified Version as given on the Title Page.  If
   there is no section entitled ``History'' in the Document, create one
   stating the title, year, authors, and publisher of the Document as
   given on its Title Page, then add an item describing the Modified
   Version as stated in the previous sentence.
\item Preserve the network location, if any, given in the Document for
   public access to a Transparent copy of the Document, and likewise
   the network locations given in the Document for previous versions
   it was based on.  These may be placed in the ``History'' section.
   You may omit a network location for a work that was published at
   least four years before the Document itself, or if the original
   publisher of the version it refers to gives permission.
\item In any section entitled ``Acknowledgements'' or ``Dedications'',
   preserve the section's title, and preserve in the section all the
   substance and tone of each of the contributor acknowledgements
   and/or dedications given therein.
\item Preserve all the Invariant Sections of the Document,
   unaltered in their text and in their titles.  Section numbers
   or the equivalent are not considered part of the section titles.
\item Delete any section entitled ``Endorsements''.  Such a section
   may not be included in the Modified Version.
\item Do not retitle any existing section as ``Endorsements''
   or to conflict in title with any Invariant Section.

\end{itemize}

If the Modified Version includes new front-matter sections or
appendices that qualify as Secondary Sections and contain no material
copied from the Document, you may at your option designate some or all
of these sections as invariant.  To do this, add their titles to the
list of Invariant Sections in the Modified Version's license notice.
These titles must be distinct from any other section titles.

You may add a section entitled ``Endorsements'', provided it contains
nothing but endorsements of your Modified Version by various
parties -- for example, statements of peer review or that the text has
been approved by an organization as the authoritative definition of a
standard.

You may add a passage of up to five words as a Front-Cover Text, and a
passage of up to 25 words as a Back-Cover Text, to the end of the list
of Cover Texts in the Modified Version.  Only one passage of
Front-Cover Text and one of Back-Cover Text may be added by (or
through arrangements made by) any one entity.  If the Document already
includes a cover text for the same cover, previously added by you or
by arrangement made by the same entity you are acting on behalf of,
you may not add another; but you may replace the old one, on explicit
permission from the previous publisher that added the old one.

The author(s) and publisher(s) of the Document do not by this License
give permission to use their names for publicity for or to assert or
imply endorsement of any Modified Version.


\section{Combining Documents}

You may combine the Document with other documents released under this
License, under the terms defined in section 4 above for modified
versions, provided that you include in the combination all of the
Invariant Sections of all of the original documents, unmodified, and
list them all as Invariant Sections of your combined work in its
license notice.

The combined work need only contain one copy of this License, and
multiple identical Invariant Sections may be replaced with a single
copy.  If there are multiple Invariant Sections with the same name but
different contents, make the title of each such section unique by
adding at the end of it, in parentheses, the name of the original
author or publisher of that section if known, or else a unique number.
Make the same adjustment to the section titles in the list of
Invariant Sections in the license notice of the combined work.

In the combination, you must combine any sections entitled ``History''
in the various original documents, forming one section entitled
``History''; likewise combine any sections entitled ``Acknowledgements'',
and any sections entitled ``Dedications''.  You must delete all sections
entitled ``Endorsements.''


\section{Collections of Documents}

You may make a collection consisting of the Document and other documents
released under this License, and replace the individual copies of this
License in the various documents with a single copy that is included in
the collection, provided that you follow the rules of this License for
verbatim copying of each of the documents in all other respects.

You may extract a single document from such a collection, and distribute
it individually under this License, provided you insert a copy of this
License into the extracted document, and follow this License in all
other respects regarding verbatim copying of that document.



\section{Aggregation With Independent Works}

A compilation of the Document or its derivatives with other separate
and independent documents or works, in or on a volume of a storage or
distribution medium, does not as a whole count as a Modified Version
of the Document, provided no compilation copyright is claimed for the
compilation.  Such a compilation is called an ``aggregate'', and this
License does not apply to the other self-contained works thus compiled
with the Document, on account of their being thus compiled, if they
are not themselves derivative works of the Document.

If the Cover Text requirement of section 3 is applicable to these
copies of the Document, then if the Document is less than one quarter
of the entire aggregate, the Document's Cover Texts may be placed on
covers that surround only the Document within the aggregate.
Otherwise they must appear on covers around the whole aggregate.


\section{Translation}

Translation is considered a kind of modification, so you may
distribute translations of the Document under the terms of section 4.
Replacing Invariant Sections with translations requires special
permission from their copyright holders, but you may include
translations of some or all Invariant Sections in addition to the
original versions of these Invariant Sections.  You may include a
translation of this License provided that you also include the
original English version of this License.  In case of a disagreement
between the translation and the original English version of this
License, the original English version will prevail.


\section{Termination}

You may not copy, modify, sublicense, or distribute the Document except
as expressly provided for under this License.  Any other attempt to
copy, modify, sublicense or distribute the Document is void, and will
automatically terminate your rights under this License.  However,
parties who have received copies, or rights, from you under this
License will not have their licenses terminated so long as such
parties remain in full compliance.


\section{Future Revisions of This License}

The Free Software Foundation may publish new, revised versions
of the GNU Free Documentation License from time to time.  Such new
versions will be similar in spirit to the present version, but may
differ in detail to address new problems or concerns. See
http://www.gnu.org/copyleft/.

Each version of the License is given a distinguishing version number.
If the Document specifies that a particular numbered version of this
License "or any later version" applies to it, you have the option of
following the terms and conditions either of that specified version or
of any later version that has been published (not as a draft) by the
Free Software Foundation.  If the Document does not specify a version
number of this License, you may choose any version ever published (not
as a draft) by the Free Software Foundation.

% Complete documentation on the extended LaTeX markup used for Python
% documentation is available in ``Documenting Python'', which is part
% of the standard documentation for Python.  It may be found online
% at:
%
%     http://www.python.org/doc/current/doc/doc.html

\documentclass[hyperref]{manual}

% latex2html doesn't know [T1]{fontenc}, so we cannot use that:(
\usepackage{amsmath}
\usepackage[latin1]{inputenc}
\usepackage{textcomp}


% this version does not reset module names at section level
%begin{latexonly}
\makeatletter
\let\py@OldOldChapter=\chapter
\renewcommand{\chapter}{\py@reset%
                        \py@OldOldChapter}
\renewcommand{\section}{\@startsection{section}{1}{\z@}%
   {-3.5ex \@plus -1ex \@minus -.2ex}%
   {2.3ex \@plus.2ex}%
   {\reset@font\Large\py@HeaderFamily}}
\makeatother
%end{latexonly}


% some convenience declarations
\newcommand{\gsl}{GSL}
\newcommand{\GSL}{GNU Scientific Library}
\newcommand{\numpy}{NumPy}
\newcommand{\NUMPY}{Numerical Python}
\newcommand{\pygsl}{PyGSL}
\newcommand{\PYGSL}{PyGSL: Python wrapper of the GNU Scientific Library}


\title{PyGSL Reference Manual}

\ifhtml
\author{%
   \ulink{Achim G\"adke}{mailto:achimgaedke@users.sourceforge.net}\\
   Center for Applied Informatics, Cologne \\
   \ulink{Jochen K\"upper}{mailto:jochen@jochen-kuepper.de}\\
   Fritz-Haber-Institut der MPG, Berlin
   \ulink{Sebastien Maret}{mailto:schnizer@users.sourceforge.net}\\
   Gesellschaft f�r Schwerionenforschung Darmstadt.
   \ulink{Pierre Schnizer}{mailto:schnizer@users.sourceforge.net}\\
   Gesellschaft f�r Schwerionenforschung, Darmstadt.
}%
\else
%begin{latexonly}
%% pdfelatex (TeXLive 7) doesn't handle \footnotemark in here...
\author{Achim G\"adke \\ Jochen K\"upper \\ Sebastien Maret \\ Pierre Schnizer}
% Please at least include a long-lived email address!
\authoraddress{
   Center for Applied Informatics, Cologne \\
   \email{achimgaedke@users.sourceforge.net} \\[2mm]
   Fritz-Haber-Institut der MPG, Berlin \\
   \email{jochen@jochen-kuepper.de} \\
      Gesellschaft f�r Schwerionenforschung, Darmstadt\\
   \email{schnizer@users.sourceforge.net}\\
}
%end{latexonly}
\fi

\date{January, 2005}            % update before release!
                                % Use an explicit date so that reformatting
                                % doesn't cause a new date to be used.  Setting
                                % the date to \today can be used during draft
                                % stages to make it easier to handle versions.
\release{0.2}                   % release version; this is used to define the
\setshortversion{0.2}           % \version macro
\makeindex                      % tell \index to actually write the .idx file


\begin{document}

\maketitle

% This makes the contents more accessible from the front page of the HTML.
\ifhtml
\chapter*{Front Matter}
\label{front}
\fi

Copyright \copyright{} 2002 The pygsl Team.

Permission is granted to copy, distribute and/or modify this document under the
terms of the GNU Free Documentation License, Version 1.1 or any later version
published by the Free Software Foundation; with no Invariant Sections, no
Front-Cover Texts, and no Back-Cover Texts.  A copy of the license is included
in section \ref{cha:free-documentation-license} entitled ``GNU Free
Documentation License''.


%% Local Variables:
%% mode: LaTeX
%% mode: auto-fill
%% fill-column: 79
%% indent-tabs-mode: nil
%% ispell-dictionary: "american"
%% reftex-fref-is-default: nil
%% TeX-auto-save: t
%% TeX-command-default: "pdfeLaTeX"
%% TeX-master: "pygsl"
%% TeX-parse-self: t
%% End:


\begin{abstract}
   \noindent
   pygsl grants python users access to the GNU scientific library.  The latest
   version can be found at the project homepage, \url{http://pygsl.sf.net}.

   \textbf{Implemented features:} \\
   \begin{tabular}{ll}
     \module{pygsl.blas}                & basic linear algebra system\\
     \module{pygsl.chebyshev}           & chebyshev approximations\\
     \module{pygsl.combination}         & combinations  \\
     \module{pygsl.const}               & $>200$ often used mathematical and
                                          scientific constants. \\
     \module{pygsl.diff}                & (Deprecated. Use pygsl.deriv instead). \\
     \module{pygsl.deriv}               & Numerical differentiation. \\
     \module{pygsl.eigen}               &\\
     \module{pygsl.fit}                 &\\
     \module{pygsl.histogram}          & 1d and 2d histograms and operations
                                          on histograms. \\
     \module{pygsl.ieee}                & Access to the ieee-arithmetics layer
                                          of gsl. \\ 
     \module{pygsl.integrate}           &\\
     \module{pygsl.interpolation}       &\\ 
     \module{pygsl.linalg}              &\\
     \module{pygsl.math}                &\\
     \module{pygsl.monte}               &\\
     \module{pygsl.minimize}            &\\
     \module{pygsl.multifit}            &\\
     \module{pygsl.multifit_nlin}       &\\    
     \module{pygsl.multimin}            &\\
     \module{pygsl.multiroots}          &\\ 
     \module{pygsl.odeiv}               &\\
     \module{pygsl.permutation}         &\\  
     \module{pygsl.poly}                &\\
     \module{pygsl.qrng}                &\\
     \module{pygsl.rng}                 & random number generators and probability densities. \\
     \module{pygsl.roots}               &\\
     \module{pygsl.siman}               &Simulated anealing\\
     \module{pygsl.sf}                  & $>200$ special functions. \\
     \module{pygsl.statistics}          & Statistical functions. \\
\end{tabular}

\end{abstract}


\tableofcontents


\chapter{System Requirements, Installation}
\label{cha:system-req-installation}
\section{Status}

\paragraph*{Status of GSL-Library}
The gsl-library is since version 1.0 stable and for general use.
More information about it at \url{http://www.gnu.org/software/gsl/}.

\paragraph*{Status of this interface}
Nearly all modules are wrapped. A lot of tests are
covering various functionality. Please report to the mailing list
\url{pygsl-discuss@lists.sourceforge.net} if you find a bug.

The hankel modules have been
wrapped. Please write to the mailing list
\url{pygsl-discuss@lists.sourceforge.net} 
if you require one of the modules
and are willing to help with a simple example. 
If any other function is missing or some other module (e.g. ntuple) or
function, do not hesitate to write to the list.

\paragraph*{Retriving the Interface}
You can download it here: \url{http://sourceforge.net/projects/pygsl}

\section{Requirements}

To build the interface, you will need
\begin{itemize}
\item \ulink{gsl-1.x}{http://sources.redhat.com/gsl},
\item \ulink{python2.6}{http://www.python.org} or better,
\item \ulink{NumPy}{http://numpy.sf.net}, and
\item a c compiler (like \ulink{gcc}{http://gcc.gnu.org}).
\end{itemize}

Supported Platforms are:
\begin{itemize}
\item Linux (Redhat/Debian/SuSE) with python2.* and gsl-1.*
\item Win32
\end{itemize}
It was tested and is tested on an irregular basis on the following platforms
\begin{itemize}
\item SUN
\item Cygwin
\item MacOS X
\end{itemize}
but is supposed to build on any POSIX platforms.

\section{Installing the pygsl interface}

\program{gsl-config} must be on your path:\nopagebreak
\begin{verbatim}
# unpack the source distribution
gzip -d -c pygsl-x.y.z.tar.gz|tar xvf-
cd pygsl-x.y.z
# do this with your prefered python version
# to set the gsl location explicitly use setup.py --gsl-prefix=/path/to/gsl
python setup.py build
# change to an user id, that is allowed to do installation
python setup.py install
\end{verbatim}
Ready....

{\bf Do not test the interface in the distribution root or in the directories
 \file{src} or \file{pygsl}.}

If you find unresolved symbols later on, delete the C source in the
swig_src files. Check that swig can be called from the command line. 
Then start the build process again. 

In this case swig will rebuild the C files. The swig_src files
distributed with pygsl are to an up to date version of GSL (1.16 as of
this writing). Swig parses partly some header header files and builds
the appropriate interface functions. If you have an older GSL version 
locally installed, the sources in the swig_src directory can contain 
links to symbols which are not defined by the locally installed GSL
version.

\subsection{Building on win32}

Windows by default does not allow to run a posix shell. Here a different path
is required. First change into the directory \file{gsl_dist}. Copy the file 
\file{gsl_site_example.py}
and edit it to reflect your installation of GSL and SWIG if you want to run it
yourself. The pygsl windows binaries distributed over 
\url{http://sourceforge.net/projects/pygsl/} are built using the mingw32 
compiler. 

\paragraph*{Uninstall GSL interface}
\code{rm -r }"python install path"\code{/lib/python}"version"\code{/site-packages/pygsl}

\paragraph*{Testing}
the directory \file{tests} contains several testsuites, based on python
\module{unittest}.
The script \file{run_test.py} in this directory will run one after the other.

\paragraph*{Support}
Please send mails to our mailinglist at
\email{pygsl-discuss@lists.sourceforge.net}.

\paragraph*{Developement}
You can browse our cvs tree at
\url{http://cvs.sourceforge.net/cgi-bin/viewcvs.cgi/pygsl/pygsl/}.
\\
Type this to check out the actual version:
\begin{verbatim}
cvs -d:pserver:anonymous@cvs.pygsl.sourceforge.net:/cvsroot/pygsl login
#Hit return for no password.
cvs -z3 -d:pserver:anonymous@cvs.pygsl.sourceforge.net:/cvsroot/pygsl co pygsl
\end{verbatim}
The script \program{tools/extract_tool.py} generates most of the special 
function code.

%\input{install_advanced.tex}
\paragraph*{ToDo}
Implement other parts:


\paragraph*{History}
\begin{itemize}
\item a gsl-interface for python was needed for a project at
\ulink{Center for Applied Informatics Cologne}{http://www.zaik.uni-koeln.de/AFS}.
\item \file{gsl-0.0.3} was released at May 23, 2001
\item \file{gsl-0.0.4} was released at January 8, 2002
\item \file{gsl-0.0.5} is growing since January, 2002
\item \file{gsl-0.2.0} was released at 
\item \file{gsl-0.3.0} was released at 
\item \file{gsl-0.3.1} was released at 
\item \file{gsl-0.3.2} was released at 
\item \file{gsl-0.9.4} was released at 25. October 2008
\end{itemize}

\paragraph*{Thanks}
Jochen K\"upper (\email{jochen@jochen-kuepper.de}) for 
\module{pygsl.statistics} part\\
Fabian Jakobs for \module{pygsl.blas}, \module{pygsl.eigen}
\module{pygsl.linalg}, \module{pygsl.permutation}\\ 
Leonardo Milano for rpm build\\
Eric Gurrola and  Peter Stoltz for testing and supporting the port of pygsl to
the MAC\\
Sebastien Maret for supporting the Fink \url{http://fink.sourceforge.net}
port of pygsl.


\paragraph*{Maintainers}
Achim G\"adke (\email{AchimGaedke@users.sourceforge.net}),\\
Pierre Schnizer (\email{schnizer@users.sourceforge.net})


\paragraph*{Acknowledgment}
\label{sec:acknowledgment}
Parts of this this manual are based on the \GSL{} reference manual.


\chapter[\protect\module{pygsl.const} --- Mathematical and scientific
constants]{\protect\module{pygsl.const} \\ Mathematical and scientific
constants} 
\label{cha:const-module}
\declaremodule{extension}{pygsl.const}
\moduleauthor{Achim G\"adke}{achimgaedke@users.sourceforge.net}

In this module some usefull constants are defined.
There are four groups of constants:

\begin{itemize}
\item mathematical
\item physical in cgs unit system
\item physical in mks unit system
\item physical number constants (e.g. fine structure)
\end{itemize}

The other modules are created during the initialisation of \module{pygsl.const}.
The mathematical, physical mks constants and number constants are available in the namespace of \module{pygsl.const}, e.g.
\begin{verbatim}
import pygsl.const
import pygsl.const.cgs
print pygsl.const.cgs.speed_of_light/pygsl.const.speed_of_light
\end{verbatim}
Of course the result is 100.0.

\section[\protect\module{pygsl.const.math} --- Mathematical constants]
{\protect\module{pygsl.const.math} \\ Mathematical constants} 
\label{cha:const-math-module}

\section[\protect\module{pygsl.const.cgs} --- Scientific constants in cgs units]
{\protect\module{pygsl.const.cgs} \\ Scientific constants in cgs units} 
\label{cha:const-cgs-module}

\section[\protect\module{pygsl.const.mks} --- Scientific constants in mks units]
{\protect\module{pygsl.const.mks} \\ Scientific constants in mks units} 
\label{cha:const-mks-module}

\section[\protect\module{pygsl.const.num} --- Scientific number constants]
{\protect\module{pygsl.const.num} \\ Scientific number constants} 
\label{cha:const-num-module}


\chapter[\protect\module{pygsl.chebyshev}]
{\protect\module{pygsl.chebyshev}}
\label{cha:statistics-module}

\declaremodule{standard}{pygsl.chebyshev}
\moduleauthor{Pierre Schnizer}{schnizer@users.sourceforge.net}

\begin{classdesc}{cheb_series}{}
  This base class can be instantiated by its name
\end{classdesc}
\begin{verbatim}
import pygsl.chebyshev
s=pygsl.chebyshev.cheb_series()
\end{verbatim}

\begin{methoddesc}{__init__}{n}\index{__init__}
            n ... number of coefficients        
\end{methoddesc}
\begin{methoddesc}{init}{f, a, b}\index{init}
        This function computes the Chebyshev approximation for the
        function F over the range (a,b) to the previously specified order.
        The computation of the Chebyshev approximation is an O($n^2$)
        process, and requires n function evaluations.

            f ... a gsl_function
            a ... lower limit
            b ... upper limit
        
\end{methoddesc}
\begin{methoddesc}{eval}{x}\index{eval}
        This function evaluates the Chebyshev series at a given point X.
\end{methoddesc}
\begin{methoddesc}{eval_err}{x}\index{eval_err}
         This function computes the Chebyshev series  at a given point X,
         estimating both the series RESULT and its absolute error ABSERR.
         The error estimate is made from the first neglected term in the
         series.
\end{methoddesc}
\begin{methoddesc}{eval_n}{n, x}\index{eval_n}
         This function evaluates the Chebyshev series at a given point
         x, to (at most) the given order n
\end{methoddesc}
\begin{methoddesc}{eval_n_err}{n, x}\index{eval_n_err}
        This function evaluates a Chebyshev series at a given point X,
        estimating both the series RESULT and its absolute error ABSERR,
        to (at most) the given order ORDER.  The error estimate is made
        from the first neglected term in the series.
\end{methoddesc}

\begin{methoddesc}{calc_deriv}{}\index{calc_deriv}
        This method computes the derivative of the series CS. It returns
        a new instance of the cheb_series class.
\end{methoddesc}
\begin{methoddesc}{calc_integ}{}\index{calc_integ}
        This method computes the integral of the series CS. It returns
        a new instance of the cheb_series class.
\end{methoddesc}
\begin{methoddesc}{get_a}{}\index{get_a}
        Get the lower boundary of the current representation       
\end{methoddesc}
\begin{methoddesc}{get_b}{}\index{get_b}
        Get the upper boundary of the current representation        
\end{methoddesc}
\begin{methoddesc}{get_coefficients}{}\index{get_coefficients}
        Get the chebyshev coefficients.         
\end{methoddesc}
\begin{methoddesc}{get_f}{}\index{get_f}
        Get the value f (what is it ?) The documentation does not tell anything
        about it.        
\end{methoddesc}
\begin{methoddesc}{get_order_sp}{}\index{get_order_sp}
        Get the value f (what is it ?) The documentation does not tell anything
        about it.        
\end{methoddesc}
\begin{methoddesc}{set_a}{}\index{set_a}
        Set the lower boundary of the current representation        
\end{methoddesc}
\begin{methoddesc}{set_b}{}\index{set_b}
        Set the upper boundary of the current         
\end{methoddesc}
\begin{methoddesc}{set_coefficients}{}\index{set_coefficients}
        Sets the chebyshev coefficients. 
\end{methoddesc}
\begin{methoddesc}{set_f}{f}\index{set_f}
        Set the value f (what is it ?)        
\end{methoddesc}
\begin{methoddesc}{set_order_sp}{...}\index{set_order_sp}
        Set the value f (what is it ?)        
\end{methoddesc}


\begin{funcdesc}{gsl_function}{f, params}\index{gsl_function}

    This class defines the callbacks known as gsl_function to
    gsl.

    e.g to supply the function f:
    
    def f(x, params):
        a = params[0]
        b = parmas[1]
        c = params[3]
        return a * x ** 2 + b * x + c

    to some solver, use

    function = gsl_function(f, params)
    
\end{funcdesc}

%%% Local Variables: 
%%% mode: latex
%%% TeX-master: "ref"
%%% End: 

\chapter[\protect\module{pygsl.deriv} --- NumericalDifferentiation]%
{\protect\module{pygsl.deriv} \\ Numerical Differentiation}
\label{cha:diff-module}

\declaremodule{extension}{pygsl.deriv}%
 \moduleauthor{Pierre  Schnizer}{schnizer@users.sourceforge.net}%
 \modulesynopsis{Numerical  Differentiation}%

\begin{quote}
  This chapter describes the available functions for numerical differentiation.
\end{quote}

The functions described in this chapter compute numerical derivatives by finite
differencing.  An adaptive algorithm is used to find the best choice of finite
difference and to estimate the error in the derivative. This module supersedes
the diff module which has been deprecated with the release of GSL-1. XXX


\begin{funcdesc}{central}{func, x, h}
  This function computes the numerical derivative of the function \var{func} at
  the point \var{x} using an adaptive central difference algorithm with a step
  size of h.  A tuple \code{(result, error)} is returned with the derivative
  and its estimated absolute error.
\end{funcdesc}

\begin{funcdesc}{backward}{func, x, h}
  This function computes the numerical derivative of the function \var{func} at
  the point \var{x} using an adaptive backward difference algorithm with a step
  size of h.  The function \var{func} is evaluated only at points smaller than
  \var{x} and at \var{x} itself.  A tuple \code{(result, error)} is returned
  with the derivative and its estimated absolute error.
\end{funcdesc}

\begin{funcdesc}{forward}{func, x, h}
  This function computes the numerical derivative of the function \var{func} at
  the point \var{x} using an adaptive forward difference algorithm with a step
  size of h.  The function \var{func} is evaluated only at points greater than
  \var{x} and at \var{x} itself.  A tuple \code{(result, error)} is returned
  with the derivative and its estimated absolute error.
\end{funcdesc}


\begin{seealso}
  The algorithms used by these functions are described in the following book:
  \seetext{S.D.\ Conte and Carl de Boor, \emph{Elementary Numerical Analysis:
      An Algorithmic Approach}, McGraw-Hill, 1972.}
\end{seealso}



%% Local Variables:
%% mode: LaTeX
%% mode: auto-fill
%% fill-column: 79
%% indent-tabs-mode: nil
%% ispell-dictionary: "british"
%% reftex-fref-is-default: nil
%% TeX-auto-save: t
%% TeX-command-default: "pdfeLaTeX"
%% TeX-master: "pygsl"
%% TeX-parse-self: t
%% End:


\chapter[\protect\module{pygsl.histogram} --- Histogram Types]
{\protect\module{pygsl.histogram} \\ Histogram Types}
\label{cha:histogram-module}



%% Local Variables:
%% mode: LaTeX
%% mode: auto-fill
%% fill-column: 79
%% ispell-dictionary: "american"
%% reftex-fref-is-default: nil
%% TeX-auto-save: t
%% TeX-command-default: "pdfeLaTeX"
%% TeX-master: "pygsl"
%% TeX-parse-self: t
%% End:


\chapter[\protect\module{pygsl.rng} --- Random Number Generators]
{\protect\module{pygsl.rng} \\ Random Number Generators}
\label{cha:rng-module}
\declaremodule{standard}{pygsl.rng}
\moduleauthor{Achim G\"adke}{achimgaedke@users.sourceforge.net}

This chapter introduces the random number generator classes provided by \module{pygsl}.

\section{Random Number Generators}

Each random number generator is a derived sperate class, that returns
a pseudo random number sequence. Methods of the common base class \class{rng}
provide the transformation to different probability distributions and
give access to basic properties of random number generators.
\begin{classdesc}{rng}{\texttt{string} typenamme \code{|} \class{rng} r}
This base class can be instantiated by a name string of the desired generator
\begin{verbatim}
import pygsl.rng
my_ran0=pygsl.rng.rng("ran0")
\end{verbatim}
or a clone of an existing generator can be created by:
\begin{verbatim}
clone_ran0=pygsl.rng.rng(my_ran0)
\end{verbatim}
\end{classdesc}
The type of the allocated generator is given by the method
\begin{methoddesc}{name}{}
which returns its name as string.
\end{methoddesc}
All generators can be seeded with
\begin{methoddesc}{set}{seed}
which sets the internal seed according to the positive integer {\tt seed}. Zero as seed
has a special meaning, please read details in the gsl reference.
\end{methoddesc}
The basic returned number type is integer, these are generated by
\begin{methoddesc}{get}{}
which returns the next number of the pseudo random sequence.
\end{methoddesc}
Basic information about these numbers can be obtained by
\begin{methoddesc}{max}{}
maximum number of this sequence and
\end{methoddesc}
\begin{methoddesc}{min}{}
minimum number of this sequence.
\end{methoddesc}
Implemented uniform probability densities are:
\begin{methoddesc}{uniform}{}
returns a real number between $[0,1)$.
\end{methoddesc}
\begin{methoddesc}{uniform_pos}{}
returns a real number between $(0,1)$ --- this excludes 0.
\end{methoddesc}
\begin{methoddesc}{uniform_int}{upper limit}
returns an integer from 0 to the upper limit (exclusive). If this limit is larger than the
number of return values of the underlying generator, \exception{pygsl.gsl_Error} is raised.
\end{methoddesc}
Furthermore a lot of derived probability densities can be used:
\begin{methoddesc}{gaussian}{}
\end{methoddesc}
\begin{methoddesc}{gaussian\_ratio\_method}{}
\end{methoddesc}
\begin{methoddesc}{ugaussian}{}
\end{methoddesc}
\begin{methoddesc}{ugaussian\_ratio\_method}{}
\end{methoddesc}
\begin{methoddesc}{gaussian\_tail}{}
\end{methoddesc}
\begin{methoddesc}{ugaussian\_tail}{}
\end{methoddesc}
\begin{methoddesc}{bivariate\_gaussian}{}
\end{methoddesc}
\begin{methoddesc}{exponential}{}
\end{methoddesc}
\begin{methoddesc}{laplace}{}
\end{methoddesc}
\begin{methoddesc}{exppow}{}
\end{methoddesc}
\begin{methoddesc}{cauchy}{}
\end{methoddesc}
\begin{methoddesc}{rayleigh}{}
\end{methoddesc}
\begin{methoddesc}{rayleigh\_tail}{}
\end{methoddesc}
\begin{methoddesc}{levy}{}
\end{methoddesc}
\begin{methoddesc}{gamma}{}
\end{methoddesc}
\begin{methoddesc}{gamma\_int}{}
\end{methoddesc}
\begin{methoddesc}{flat}{}
\end{methoddesc}
\begin{methoddesc}{lognormal}{}
\end{methoddesc}
\begin{methoddesc}{chisq}{}
\end{methoddesc}
\begin{methoddesc}{fdist}{}
\end{methoddesc}
\begin{methoddesc}{tdist}{}
\end{methoddesc}
\begin{methoddesc}{beta}{}
\end{methoddesc}
\begin{methoddesc}{logistic}{}
\end{methoddesc}
\begin{methoddesc}{pareto}{}
\end{methoddesc}
\begin{methoddesc}{dir\_2d}{}
\end{methoddesc}
\begin{methoddesc}{dir\_2d\_trig\_method}{}
\end{methoddesc}
\begin{methoddesc}{dir\_3d}{}
\end{methoddesc}
\begin{methoddesc}{dir\_nd}{}
\end{methoddesc}
\begin{methoddesc}{weibull}{}
\end{methoddesc}
\begin{methoddesc}{gumbel1}{}
\end{methoddesc}
\begin{methoddesc}{gumbel2}{}
\end{methoddesc}
\begin{methoddesc}{poisson}{}
\end{methoddesc}
\begin{methoddesc}{bernoulli}{}
\end{methoddesc}
\begin{methoddesc}{binomial}{}
\end{methoddesc}
\begin{methoddesc}{negative\_binomial}{}
\end{methoddesc}
\begin{methoddesc}{pascal}{}
\end{methoddesc}
\begin{methoddesc}{geometric}{}
\end{methoddesc}
\begin{methoddesc}{hypergeometric}{}
\end{methoddesc}
\begin{methoddesc}{logarithmic}{}
\end{methoddesc}
\begin{methoddesc}{landau}{}
\end{methoddesc}
\begin{methoddesc}{erlang}{}
\end{methoddesc}


The different generator classes are created according to the output of \code{gsl_rng_types_setup()}
when the \module{pygsl.rng} is loaded. Here is the list of children from \class{rng} for gsl-1.2:
\newline
\class{rng_borosh13},
\class{rng_coveyou},
\class{rng_cmrg},
\class{rng_fishman18},
\class{rng_fishman20},
\class{rng_fishman2x},
\class{rng_gfsr4},
\class{rng_knuthran},
\class{rng_knuthran2},
\class{rng_lecuyer21},
\class{rng_minstd},
\class{rng_mrg},
\class{rng_mt19937},
\class{rng_mt19937_1999},
\class{rng_mt19937_1998},
\class{rng_r250},
\class{rng_ran0},
\class{rng_ran1},
\class{rng_ran2},
\class{rng_ran3},
\class{rng_rand},
\class{rng_rand48},
\class{rng_random128_bsd},
\class{rng_random128_glibc2},
\class{rng_random128_libc5},
\class{rng_random256_bsd},
\class{rng_random256_glibc2},
\class{rng_random256_libc5},
\class{rng_random32_bsd},
\class{rng_random32_glibc2},
\class{rng_random32_libc5},
\class{rng_random64_bsd},
\class{rng_random64_glibc2},
\class{rng_random64_libc5},
\class{rng_random8_bsd},
\class{rng_random8_glibc2},
\class{rng_random8_libc5},
\class{rng_random_bsd},
\class{rng_random_glibc2},
\class{rng_random_libc5},
\class{rng_randu},
\class{rng_ranf},
\class{rng_ranlux},
\class{rng_ranlux389},
\class{rng_ranlxd1},
\class{rng_ranlxd2},
\class{rng_ranlxs0},
\class{rng_ranlxs1},
\class{rng_ranlxs2},
\class{rng_ranmar},
\class{rng_slatec},
\class{rng_taus},
\class{rng_taus2},
\class{rng_taus113},
\class{rng_transputer},
\class{rng_tt800},
\class{rng_uni},
\class{rng_uni32},
\class{rng_vax},
\class{rng_waterman14}, and
\class{rng_zuf}.
\newline
The default generator of \class{rng} is determined by the environment
variable \envvar{GSL_RNG_TYPE} or defaults to {\tt rng_mt19937}.

\section{Probability Density Functions}


\section{Using probability densities with random number generators}


%% Local Variables:
%% mode: LaTeX
%% mode: auto-fill
%% fill-column: 90
%% indent-tabs-mode: nil
%% ispell-dictionary: "american"
%% reftex-fref-is-default: nil
%% TeX-auto-save: t
%% TeX-command-default: "pdfeLaTeX"
%% TeX-master: "pygsl"
%% TeX-parse-self: t
%% End:


\chapter[\protect\module{pygsl.sf} --- Special Functions]
{\protect\module{pygsl.sf} \\ Special Functions}
\label{cha:sf-module}
\declaremodule{extension}{pygsl.sf}
\moduleauthor{Achim G\"adke}{achimgaedke@users.sourceforge.net}

This chapter shows you the list of implemented special function and explains
details of error handling and return values.

\section{Function list}

\begin{longtableii}{l|l}{texttt}{Function}{Description}
\lineii{}{ToDo}
\end{longtableii}

\section{Return values}

\section{Error handling}

\declaremodule{extension}{pygsl.statistics}
\moduleauthor{Jochen K\"upper}{jochen@jochen-kuepper.de}
\index{mean}
\index{standard deviation}
\index{variance}
\index{estimated standard deviation}
\index{estimated variance}
\index{t-test}
\index{range}
\index{min}
\index{max}

This chapter describes the statistical functions in the library.  The
basic statistical functions include routines to compute the mean,
variance and standard deviation. More advanced functions allow you to
calculate absolute deviations, skewness, and kurtosis as well as the
median and arbitrary percentiles.  The algorithms use recurrence
relations to compute average quantities in a stable way, without large
intermediate values that might overflow. 

All functions work on any Python sequence (of the appropriate
data-type), but see section \ref{sec:stat-speed-considerations} for
advantages and drawbacks of different kinds of input data.


\section{Organization of the module}
\label{sec:stat-organization}

The parts of the GSL functions names, providing artificial name spaces,
are mapped to modules and submodules in pygsl.  That is
\code{gsl_stats_mean} can be found as \code{pygsl.statistics.mean} and
\code{gsl_stats_long_mean} as \code{pygsl.statistics.long.mean}.

The functions in the module are available in versions for datasets in
the standard floating-point and integer types. The generic versions
available in the \code{pygsl.statistics} module are using the generic
GSL \code{double} versions.  The submodules use GSL functions according
to the submodule name, e.g. long for \code{pygsl.statistics.long}.

Currently implemented submodules are \code{pygsl.statistics.double} and
\code{pygsl.statistics.long}.



\section{Speed considerations}
\label{sec:stat-speed-considerations}

All functions work on any Python sequence type but are optimized for use
with NumPy arrays. It is strongly suggested that you install NumPy
(available from \url{http://www.numpy.org})!

If you pass NumPy arrays of the \emph{correct data-type} as input data
to any of the functions they are passed straight to the C functions
along with the stride information of the data.

If you pass generic (non-NumPy) Python sequences or NumPy arrays of the
wrong data-type a suitable copy of the data will be created and passed
to the function.


\section{Further Reading}
\label{sec:stat-further-reading}

See the gsl reference manual for a description of all available
functions and the calculations they perform.


%% Local Variables:
%% mode: LaTeX
%% mode: auto-fill
%% fill-column: 79
%% ispell-dictionary: "american"
%% reftex-fref-is-default: nil
%% TeX-auto-save: t
%% TeX-command-default: "pdfeLaTeX"
%% TeX-master: "pygsl"
%% TeX-parse-self: t
%% End:


\chapter[\protect\module{pygsl.testing} ---  Modules in Testing]
{\protect\module{pygsl.testing} \\ Modules in Testing}
\label{cha:statistics-module}

\declaremodule{standard}{pygsl.testing}

\moduleauthor{Pierre Schnizer}{schnizer@users.sourceforge.net}
Modules in this package are often reimplementations of an original package
with significant change to the original. The current rng implementation, for
example, started its life here. Usage of these modules is encouraged for tests
to see if they work, but use them with caution in your production code!

\section[\protect\module{pygsl.testing.sf} --- Special UFuncs]
{\protect\module{pygsl.testing.sf} \\ Special Functions as UFuncs}

\declaremodule{standard}{pygsl.testing.sf}
\moduleauthor{Pierre Schnizer}{schnizer@users.sourceforge.net}

This chapter provides mainly \numpy{} UFuncs over the special functions. This means
that all input variable can be arrays, and the UFunc will evaluate the gsl
function for all its inputs. It is meant to replace the sf module later;
please use it and find out if it is useful for you. 
Only the python specific part is described here. For a general description of
the function please see the GSL Reference document.  

\section{UFuncs}
These UFuncs allow to evaluate an array of doubles or an array of floats typically.
\begin{funcdesc}{Chi}{...}\index{Chi}

    Number of Input  Arguments:  1
    Number of Output Arguments:  1
\end{funcdesc}

\begin{funcdesc}{Chi_e}{...}\index{Chi_e}

    Number of Input  Arguments:  1
    Number of Output Arguments:  2

The error flag is discarded.
Return Arguments 1 and 2 resemble a gsl_result argument,
	which is  argument 1 of the C argument list

\end{funcdesc}

\begin{funcdesc}{Ci}{...}\index{Ci}

    Number of Input  Arguments:  1
    Number of Output Arguments:  1
\end{funcdesc}

\begin{funcdesc}{Ci_e}{...}\index{Ci_e}

    Number of Input  Arguments:  1
    Number of Output Arguments:  2

The error flag is discarded.
Return Arguments 1 and 2 resemble a gsl_result argument,
	which is  argument 1 of the C argument list

\end{funcdesc}

\begin{funcdesc}{Shi}{...}\index{Shi}

    Number of Input  Arguments:  1
    Number of Output Arguments:  1
\end{funcdesc}

\begin{funcdesc}{Shi_e}{...}\index{Shi_e}

    Number of Input  Arguments:  1
    Number of Output Arguments:  2

The error flag is discarded.
Return Arguments 1 and 2 resemble a gsl_result argument,
	which is  argument 1 of the C argument list

\end{funcdesc}

\begin{funcdesc}{Si}{...}\index{Si}

    Number of Input  Arguments:  1
    Number of Output Arguments:  1
\end{funcdesc}

\begin{funcdesc}{Si_e}{...}\index{Si_e}

    Number of Input  Arguments:  1
    Number of Output Arguments:  2

The error flag is discarded.
Return Arguments 1 and 2 resemble a gsl_result argument,
	which is  argument 1 of the C argument list

\end{funcdesc}

\begin{funcdesc}{airy_Ai}{...}\index{airy_Ai}

    Number of Input  Arguments:  2
    Number of Output Arguments:  1

 Argument 2 is a gsl_mode_t, valid parameters are:
	sf.PREC_DOUBLE or sf.PREC_SINGLE or sf.PREC_APPROX

\end{funcdesc}

\begin{funcdesc}{airy_Ai_deriv}{...}\index{airy_Ai_deriv}

    Number of Input  Arguments:  2
    Number of Output Arguments:  1

 Argument 2 is a gsl_mode_t, valid parameters are:
	sf.PREC_DOUBLE or sf.PREC_SINGLE or sf.PREC_APPROX

\end{funcdesc}

\begin{funcdesc}{airy_Ai_deriv_e}{...}\index{airy_Ai_deriv_e}

    Number of Input  Arguments:  2
    Number of Output Arguments:  2

 Argument 2 is a gsl_mode_t, valid parameters are:
	sf.PREC_DOUBLE or sf.PREC_SINGLE or sf.PREC_APPROX
The error flag is discarded.
Return Arguments 1 and 2 resemble a gsl_result argument,
	which is  argument 2 of the C argument list

\end{funcdesc}

\begin{funcdesc}{airy_Ai_deriv_scaled}{...}\index{airy_Ai_deriv_scaled}

    Number of Input  Arguments:  2
    Number of Output Arguments:  1

 Argument 2 is a gsl_mode_t, valid parameters are:
	sf.PREC_DOUBLE or sf.PREC_SINGLE or sf.PREC_APPROX

\end{funcdesc}

\begin{funcdesc}{airy_Ai_deriv_scaled_e}{...}\index{airy_Ai_deriv_scaled_e}

    Number of Input  Arguments:  2
    Number of Output Arguments:  2

 Argument 2 is a gsl_mode_t, valid parameters are:
	sf.PREC_DOUBLE or sf.PREC_SINGLE or sf.PREC_APPROX
The error flag is discarded.
Return Arguments 1 and 2 resemble a gsl_result argument,
	which is  argument 2 of the C argument list

\end{funcdesc}

\begin{funcdesc}{airy_Ai_e}{...}\index{airy_Ai_e}

    Number of Input  Arguments:  2
    Number of Output Arguments:  2

 Argument 2 is a gsl_mode_t, valid parameters are:
	sf.PREC_DOUBLE or sf.PREC_SINGLE or sf.PREC_APPROX
The error flag is discarded.
Return Arguments 1 and 2 resemble a gsl_result argument,
	which is  argument 2 of the C argument list

\end{funcdesc}

\begin{funcdesc}{airy_Ai_scaled}{...}\index{airy_Ai_scaled}

    Number of Input  Arguments:  2
    Number of Output Arguments:  1

 Argument 2 is a gsl_mode_t, valid parameters are:
	sf.PREC_DOUBLE or sf.PREC_SINGLE or sf.PREC_APPROX

\end{funcdesc}

\begin{funcdesc}{airy_Ai_scaled_e}{...}\index{airy_Ai_scaled_e}

    Number of Input  Arguments:  2
    Number of Output Arguments:  2

 Argument 2 is a gsl_mode_t, valid parameters are:
	sf.PREC_DOUBLE or sf.PREC_SINGLE or sf.PREC_APPROX
The error flag is discarded.
Return Arguments 1 and 2 resemble a gsl_result argument,
	which is  argument 2 of the C argument list

\end{funcdesc}

\begin{funcdesc}{airy_Bi}{...}\index{airy_Bi}

    Number of Input  Arguments:  2
    Number of Output Arguments:  1

 Argument 2 is a gsl_mode_t, valid parameters are:
	sf.PREC_DOUBLE or sf.PREC_SINGLE or sf.PREC_APPROX

\end{funcdesc}

\begin{funcdesc}{airy_Bi_deriv}{...}\index{airy_Bi_deriv}

    Number of Input  Arguments:  2
    Number of Output Arguments:  1

 Argument 2 is a gsl_mode_t, valid parameters are:
	sf.PREC_DOUBLE or sf.PREC_SINGLE or sf.PREC_APPROX

\end{funcdesc}

\begin{funcdesc}{airy_Bi_deriv_e}{...}\index{airy_Bi_deriv_e}

    Number of Input  Arguments:  2
    Number of Output Arguments:  2

 Argument 2 is a gsl_mode_t, valid parameters are:
	sf.PREC_DOUBLE or sf.PREC_SINGLE or sf.PREC_APPROX
The error flag is discarded.
Return Arguments 1 and 2 resemble a gsl_result argument,
	which is  argument 2 of the C argument list

\end{funcdesc}

\begin{funcdesc}{airy_Bi_deriv_scaled}{...}\index{airy_Bi_deriv_scaled}

    Number of Input  Arguments:  2
    Number of Output Arguments:  1

 Argument 2 is a gsl_mode_t, valid parameters are:
	sf.PREC_DOUBLE or sf.PREC_SINGLE or sf.PREC_APPROX

\end{funcdesc}

\begin{funcdesc}{airy_Bi_deriv_scaled_e}{...}\index{airy_Bi_deriv_scaled_e}

    Number of Input  Arguments:  2
    Number of Output Arguments:  2

 Argument 2 is a gsl_mode_t, valid parameters are:
	sf.PREC_DOUBLE or sf.PREC_SINGLE or sf.PREC_APPROX
The error flag is discarded.
Return Arguments 1 and 2 resemble a gsl_result argument,
	which is  argument 2 of the C argument list

\end{funcdesc}

\begin{funcdesc}{airy_Bi_e}{...}\index{airy_Bi_e}

    Number of Input  Arguments:  2
    Number of Output Arguments:  2

 Argument 2 is a gsl_mode_t, valid parameters are:
	sf.PREC_DOUBLE or sf.PREC_SINGLE or sf.PREC_APPROX
The error flag is discarded.
Return Arguments 1 and 2 resemble a gsl_result argument,
	which is  argument 2 of the C argument list

\end{funcdesc}

\begin{funcdesc}{airy_Bi_scaled}{...}\index{airy_Bi_scaled}

    Number of Input  Arguments:  2
    Number of Output Arguments:  1

 Argument 2 is a gsl_mode_t, valid parameters are:
	sf.PREC_DOUBLE or sf.PREC_SINGLE or sf.PREC_APPROX

\end{funcdesc}

\begin{funcdesc}{airy_Bi_scaled_e}{...}\index{airy_Bi_scaled_e}

    Number of Input  Arguments:  2
    Number of Output Arguments:  2

 Argument 2 is a gsl_mode_t, valid parameters are:
	sf.PREC_DOUBLE or sf.PREC_SINGLE or sf.PREC_APPROX
The error flag is discarded.
Return Arguments 1 and 2 resemble a gsl_result argument,
	which is  argument 2 of the C argument list

\end{funcdesc}

\begin{funcdesc}{airy_zero_Ai}{...}\index{airy_zero_Ai}

    Number of Input  Arguments:  1
    Number of Output Arguments:  1
\end{funcdesc}

\begin{funcdesc}{airy_zero_Ai_deriv}{...}\index{airy_zero_Ai_deriv}

    Number of Input  Arguments:  1
    Number of Output Arguments:  1
\end{funcdesc}

\begin{funcdesc}{airy_zero_Ai_deriv_e}{...}\index{airy_zero_Ai_deriv_e}

    Number of Input  Arguments:  1
    Number of Output Arguments:  2

The error flag is discarded.
Return Arguments 1 and 2 resemble a gsl_result argument,
	which is  argument 1 of the C argument list

\end{funcdesc}

\begin{funcdesc}{airy_zero_Ai_e}{...}\index{airy_zero_Ai_e}

    Number of Input  Arguments:  1
    Number of Output Arguments:  2

The error flag is discarded.
Return Arguments 1 and 2 resemble a gsl_result argument,
	which is  argument 1 of the C argument list

\end{funcdesc}

\begin{funcdesc}{airy_zero_Bi}{...}\index{airy_zero_Bi}

    Number of Input  Arguments:  1
    Number of Output Arguments:  1
\end{funcdesc}

\begin{funcdesc}{airy_zero_Bi_deriv}{...}\index{airy_zero_Bi_deriv}

    Number of Input  Arguments:  1
    Number of Output Arguments:  1
\end{funcdesc}

\begin{funcdesc}{airy_zero_Bi_deriv_e}{...}\index{airy_zero_Bi_deriv_e}

    Number of Input  Arguments:  1
    Number of Output Arguments:  2

The error flag is discarded.
Return Arguments 1 and 2 resemble a gsl_result argument,
	which is  argument 1 of the C argument list

\end{funcdesc}

\begin{funcdesc}{airy_zero_Bi_e}{...}\index{airy_zero_Bi_e}

    Number of Input  Arguments:  1
    Number of Output Arguments:  2

The error flag is discarded.
Return Arguments 1 and 2 resemble a gsl_result argument,
	which is  argument 1 of the C argument list

\end{funcdesc}

\begin{funcdesc}{angle_restrict_pos}{...}\index{angle_restrict_pos}

    Number of Input  Arguments:  1
    Number of Output Arguments:  1
\end{funcdesc}

\begin{funcdesc}{angle_restrict_pos_err_e}{...}\index{angle_restrict_pos_err_e}

    Number of Input  Arguments:  1
    Number of Output Arguments:  2

The error flag is discarded.
Return Arguments 1 and 2 resemble a gsl_result argument,
	which is  argument 1 of the C argument list

\end{funcdesc}

\begin{funcdesc}{angle_restrict_symm}{...}\index{angle_restrict_symm}

    Number of Input  Arguments:  1
    Number of Output Arguments:  1
\end{funcdesc}

\begin{funcdesc}{angle_restrict_symm_err_e}{...}\index{angle_restrict_symm_err_e}

    Number of Input  Arguments:  1
    Number of Output Arguments:  2

The error flag is discarded.
Return Arguments 1 and 2 resemble a gsl_result argument,
	which is  argument 1 of the C argument list

\end{funcdesc}

\begin{funcdesc}{atanint}{...}\index{atanint}

    Number of Input  Arguments:  1
    Number of Output Arguments:  1
\end{funcdesc}

\begin{funcdesc}{atanint_e}{...}\index{atanint_e}

    Number of Input  Arguments:  1
    Number of Output Arguments:  2

The error flag is discarded.
Return Arguments 1 and 2 resemble a gsl_result argument,
	which is  argument 1 of the C argument list

\end{funcdesc}

\begin{funcdesc}{bessel_I0}{...}\index{bessel_I0}

    Number of Input  Arguments:  1
    Number of Output Arguments:  1
\end{funcdesc}

\begin{funcdesc}{bessel_I0_e}{...}\index{bessel_I0_e}

    Number of Input  Arguments:  1
    Number of Output Arguments:  2

The error flag is discarded.
Return Arguments 1 and 2 resemble a gsl_result argument,
	which is  argument 1 of the C argument list

\end{funcdesc}

\begin{funcdesc}{bessel_I0_scaled}{...}\index{bessel_I0_scaled}

    Number of Input  Arguments:  1
    Number of Output Arguments:  1
\end{funcdesc}

\begin{funcdesc}{bessel_I0_scaled_e}{...}\index{bessel_I0_scaled_e}

    Number of Input  Arguments:  1
    Number of Output Arguments:  2

The error flag is discarded.
Return Arguments 1 and 2 resemble a gsl_result argument,
	which is  argument 1 of the C argument list

\end{funcdesc}

\begin{funcdesc}{bessel_I1}{...}\index{bessel_I1}

    Number of Input  Arguments:  1
    Number of Output Arguments:  1
\end{funcdesc}

\begin{funcdesc}{bessel_I1_e}{...}\index{bessel_I1_e}

    Number of Input  Arguments:  1
    Number of Output Arguments:  2

The error flag is discarded.
Return Arguments 1 and 2 resemble a gsl_result argument,
	which is  argument 1 of the C argument list

\end{funcdesc}

\begin{funcdesc}{bessel_I1_scaled}{...}\index{bessel_I1_scaled}

    Number of Input  Arguments:  1
    Number of Output Arguments:  1
\end{funcdesc}

\begin{funcdesc}{bessel_I1_scaled_e}{...}\index{bessel_I1_scaled_e}

    Number of Input  Arguments:  1
    Number of Output Arguments:  2

The error flag is discarded.
Return Arguments 1 and 2 resemble a gsl_result argument,
	which is  argument 1 of the C argument list

\end{funcdesc}

\begin{funcdesc}{bessel_In}{...}\index{bessel_In}

    Number of Input  Arguments:  2
    Number of Output Arguments:  1
\end{funcdesc}

\begin{funcdesc}{bessel_In_e}{...}\index{bessel_In_e}

    Number of Input  Arguments:  2
    Number of Output Arguments:  2

The error flag is discarded.
Return Arguments 1 and 2 resemble a gsl_result argument,
	which is  argument 2 of the C argument list

\end{funcdesc}

\begin{funcdesc}{bessel_In_scaled}{...}\index{bessel_In_scaled}

    Number of Input  Arguments:  2
    Number of Output Arguments:  1
\end{funcdesc}

\begin{funcdesc}{bessel_In_scaled_e}{...}\index{bessel_In_scaled_e}

    Number of Input  Arguments:  2
    Number of Output Arguments:  2

The error flag is discarded.
Return Arguments 1 and 2 resemble a gsl_result argument,
	which is  argument 2 of the C argument list

\end{funcdesc}

\begin{funcdesc}{bessel_Inu}{...}\index{bessel_Inu}

    Number of Input  Arguments:  2
    Number of Output Arguments:  1
\end{funcdesc}

\begin{funcdesc}{bessel_Inu_e}{...}\index{bessel_Inu_e}

    Number of Input  Arguments:  2
    Number of Output Arguments:  2

The error flag is discarded.
Return Arguments 1 and 2 resemble a gsl_result argument,
	which is  argument 2 of the C argument list

\end{funcdesc}

\begin{funcdesc}{bessel_Inu_scaled}{...}\index{bessel_Inu_scaled}

    Number of Input  Arguments:  2
    Number of Output Arguments:  1
\end{funcdesc}

\begin{funcdesc}{bessel_Inu_scaled_e}{...}\index{bessel_Inu_scaled_e}

    Number of Input  Arguments:  2
    Number of Output Arguments:  2

The error flag is discarded.
Return Arguments 1 and 2 resemble a gsl_result argument,
	which is  argument 2 of the C argument list

\end{funcdesc}

\begin{funcdesc}{bessel_J0}{...}\index{bessel_J0}

    Number of Input  Arguments:  1
    Number of Output Arguments:  1
\end{funcdesc}

\begin{funcdesc}{bessel_J0_e}{...}\index{bessel_J0_e}

    Number of Input  Arguments:  1
    Number of Output Arguments:  2

The error flag is discarded.
Return Arguments 1 and 2 resemble a gsl_result argument,
	which is  argument 1 of the C argument list

\end{funcdesc}

\begin{funcdesc}{bessel_J1}{...}\index{bessel_J1}

    Number of Input  Arguments:  1
    Number of Output Arguments:  1
\end{funcdesc}

\begin{funcdesc}{bessel_J1_e}{...}\index{bessel_J1_e}

    Number of Input  Arguments:  1
    Number of Output Arguments:  2

The error flag is discarded.
Return Arguments 1 and 2 resemble a gsl_result argument,
	which is  argument 1 of the C argument list

\end{funcdesc}

\begin{funcdesc}{bessel_Jn}{...}\index{bessel_Jn}

    Number of Input  Arguments:  2
    Number of Output Arguments:  1
\end{funcdesc}

\begin{funcdesc}{bessel_Jn_e}{...}\index{bessel_Jn_e}

    Number of Input  Arguments:  2
    Number of Output Arguments:  2

The error flag is discarded.
Return Arguments 1 and 2 resemble a gsl_result argument,
	which is  argument 2 of the C argument list

\end{funcdesc}

\begin{funcdesc}{bessel_Jnu}{...}\index{bessel_Jnu}

    Number of Input  Arguments:  2
    Number of Output Arguments:  1
\end{funcdesc}

\begin{funcdesc}{bessel_Jnu_e}{...}\index{bessel_Jnu_e}

    Number of Input  Arguments:  2
    Number of Output Arguments:  2

The error flag is discarded.
Return Arguments 1 and 2 resemble a gsl_result argument,
	which is  argument 2 of the C argument list

\end{funcdesc}

\begin{funcdesc}{bessel_K0}{...}\index{bessel_K0}

    Number of Input  Arguments:  1
    Number of Output Arguments:  1
\end{funcdesc}

\begin{funcdesc}{bessel_K0_e}{...}\index{bessel_K0_e}

    Number of Input  Arguments:  1
    Number of Output Arguments:  2

The error flag is discarded.
Return Arguments 1 and 2 resemble a gsl_result argument,
	which is  argument 1 of the C argument list

\end{funcdesc}

\begin{funcdesc}{bessel_K0_scaled}{...}\index{bessel_K0_scaled}

    Number of Input  Arguments:  1
    Number of Output Arguments:  1
\end{funcdesc}

\begin{funcdesc}{bessel_K0_scaled_e}{...}\index{bessel_K0_scaled_e}

    Number of Input  Arguments:  1
    Number of Output Arguments:  2

The error flag is discarded.
Return Arguments 1 and 2 resemble a gsl_result argument,
	which is  argument 1 of the C argument list

\end{funcdesc}

\begin{funcdesc}{bessel_K1}{...}\index{bessel_K1}

    Number of Input  Arguments:  1
    Number of Output Arguments:  1
\end{funcdesc}

\begin{funcdesc}{bessel_K1_e}{...}\index{bessel_K1_e}

    Number of Input  Arguments:  1
    Number of Output Arguments:  2

The error flag is discarded.
Return Arguments 1 and 2 resemble a gsl_result argument,
	which is  argument 1 of the C argument list

\end{funcdesc}

\begin{funcdesc}{bessel_K1_scaled}{...}\index{bessel_K1_scaled}

    Number of Input  Arguments:  1
    Number of Output Arguments:  1
\end{funcdesc}

\begin{funcdesc}{bessel_K1_scaled_e}{...}\index{bessel_K1_scaled_e}

    Number of Input  Arguments:  1
    Number of Output Arguments:  2

The error flag is discarded.
Return Arguments 1 and 2 resemble a gsl_result argument,
	which is  argument 1 of the C argument list

\end{funcdesc}

\begin{funcdesc}{bessel_Kn}{...}\index{bessel_Kn}

    Number of Input  Arguments:  2
    Number of Output Arguments:  1
\end{funcdesc}

\begin{funcdesc}{bessel_Kn_e}{...}\index{bessel_Kn_e}

    Number of Input  Arguments:  2
    Number of Output Arguments:  2

The error flag is discarded.
Return Arguments 1 and 2 resemble a gsl_result argument,
	which is  argument 2 of the C argument list

\end{funcdesc}

\begin{funcdesc}{bessel_Kn_scaled}{...}\index{bessel_Kn_scaled}

    Number of Input  Arguments:  2
    Number of Output Arguments:  1
\end{funcdesc}

\begin{funcdesc}{bessel_Kn_scaled_e}{...}\index{bessel_Kn_scaled_e}

    Number of Input  Arguments:  2
    Number of Output Arguments:  2

The error flag is discarded.
Return Arguments 1 and 2 resemble a gsl_result argument,
	which is  argument 2 of the C argument list

\end{funcdesc}

\begin{funcdesc}{bessel_Knu}{...}\index{bessel_Knu}

    Number of Input  Arguments:  2
    Number of Output Arguments:  1
\end{funcdesc}

\begin{funcdesc}{bessel_Knu_e}{...}\index{bessel_Knu_e}

    Number of Input  Arguments:  2
    Number of Output Arguments:  2

The error flag is discarded.
Return Arguments 1 and 2 resemble a gsl_result argument,
	which is  argument 2 of the C argument list

\end{funcdesc}

\begin{funcdesc}{bessel_Knu_scaled}{...}\index{bessel_Knu_scaled}

    Number of Input  Arguments:  2
    Number of Output Arguments:  1
\end{funcdesc}

\begin{funcdesc}{bessel_Knu_scaled_e}{...}\index{bessel_Knu_scaled_e}

    Number of Input  Arguments:  2
    Number of Output Arguments:  2

The error flag is discarded.
Return Arguments 1 and 2 resemble a gsl_result argument,
	which is  argument 2 of the C argument list

\end{funcdesc}

\begin{funcdesc}{bessel_Y0}{...}\index{bessel_Y0}

    Number of Input  Arguments:  1
    Number of Output Arguments:  1
\end{funcdesc}

\begin{funcdesc}{bessel_Y0_e}{...}\index{bessel_Y0_e}

    Number of Input  Arguments:  1
    Number of Output Arguments:  2

The error flag is discarded.
Return Arguments 1 and 2 resemble a gsl_result argument,
	which is  argument 1 of the C argument list

\end{funcdesc}

\begin{funcdesc}{bessel_Y1}{...}\index{bessel_Y1}

    Number of Input  Arguments:  1
    Number of Output Arguments:  1
\end{funcdesc}

\begin{funcdesc}{bessel_Y1_e}{...}\index{bessel_Y1_e}

    Number of Input  Arguments:  1
    Number of Output Arguments:  2

The error flag is discarded.
Return Arguments 1 and 2 resemble a gsl_result argument,
	which is  argument 1 of the C argument list

\end{funcdesc}

\begin{funcdesc}{bessel_Yn}{...}\index{bessel_Yn}

    Number of Input  Arguments:  2
    Number of Output Arguments:  1
\end{funcdesc}

\begin{funcdesc}{bessel_Yn_e}{...}\index{bessel_Yn_e}

    Number of Input  Arguments:  2
    Number of Output Arguments:  2

The error flag is discarded.
Return Arguments 1 and 2 resemble a gsl_result argument,
	which is  argument 2 of the C argument list

\end{funcdesc}

\begin{funcdesc}{bessel_Ynu}{...}\index{bessel_Ynu}

    Number of Input  Arguments:  2
    Number of Output Arguments:  1
\end{funcdesc}

\begin{funcdesc}{bessel_Ynu_e}{...}\index{bessel_Ynu_e}

    Number of Input  Arguments:  2
    Number of Output Arguments:  2

The error flag is discarded.
Return Arguments 1 and 2 resemble a gsl_result argument,
	which is  argument 2 of the C argument list

\end{funcdesc}

\begin{funcdesc}{bessel_i0_scaled}{...}\index{bessel_i0_scaled}

    Number of Input  Arguments:  1
    Number of Output Arguments:  1
\end{funcdesc}

\begin{funcdesc}{bessel_i0_scaled_e}{...}\index{bessel_i0_scaled_e}

    Number of Input  Arguments:  1
    Number of Output Arguments:  2

The error flag is discarded.
Return Arguments 1 and 2 resemble a gsl_result argument,
	which is  argument 1 of the C argument list

\end{funcdesc}

\begin{funcdesc}{bessel_i1_scaled}{...}\index{bessel_i1_scaled}

    Number of Input  Arguments:  1
    Number of Output Arguments:  1
\end{funcdesc}

\begin{funcdesc}{bessel_i1_scaled_e}{...}\index{bessel_i1_scaled_e}

    Number of Input  Arguments:  1
    Number of Output Arguments:  2

The error flag is discarded.
Return Arguments 1 and 2 resemble a gsl_result argument,
	which is  argument 1 of the C argument list

\end{funcdesc}

\begin{funcdesc}{bessel_i2_scaled}{...}\index{bessel_i2_scaled}

    Number of Input  Arguments:  1
    Number of Output Arguments:  1
\end{funcdesc}

\begin{funcdesc}{bessel_i2_scaled_e}{...}\index{bessel_i2_scaled_e}

    Number of Input  Arguments:  1
    Number of Output Arguments:  2

The error flag is discarded.
Return Arguments 1 and 2 resemble a gsl_result argument,
	which is  argument 1 of the C argument list

\end{funcdesc}

\begin{funcdesc}{bessel_il_scaled}{...}\index{bessel_il_scaled}

    Number of Input  Arguments:  2
    Number of Output Arguments:  1
\end{funcdesc}

\begin{funcdesc}{bessel_il_scaled_e}{...}\index{bessel_il_scaled_e}

    Number of Input  Arguments:  2
    Number of Output Arguments:  2

The error flag is discarded.
Return Arguments 1 and 2 resemble a gsl_result argument,
	which is  argument 2 of the C argument list

\end{funcdesc}

\begin{funcdesc}{bessel_j0}{...}\index{bessel_j0}

    Number of Input  Arguments:  1
    Number of Output Arguments:  1
\end{funcdesc}

\begin{funcdesc}{bessel_j0_e}{...}\index{bessel_j0_e}

    Number of Input  Arguments:  1
    Number of Output Arguments:  2

The error flag is discarded.
Return Arguments 1 and 2 resemble a gsl_result argument,
	which is  argument 1 of the C argument list

\end{funcdesc}

\begin{funcdesc}{bessel_j1}{...}\index{bessel_j1}

    Number of Input  Arguments:  1
    Number of Output Arguments:  1
\end{funcdesc}

\begin{funcdesc}{bessel_j1_e}{...}\index{bessel_j1_e}

    Number of Input  Arguments:  1
    Number of Output Arguments:  2

The error flag is discarded.
Return Arguments 1 and 2 resemble a gsl_result argument,
	which is  argument 1 of the C argument list

\end{funcdesc}

\begin{funcdesc}{bessel_j2}{...}\index{bessel_j2}

    Number of Input  Arguments:  1
    Number of Output Arguments:  1
\end{funcdesc}

\begin{funcdesc}{bessel_j2_e}{...}\index{bessel_j2_e}

    Number of Input  Arguments:  1
    Number of Output Arguments:  2

The error flag is discarded.
Return Arguments 1 and 2 resemble a gsl_result argument,
	which is  argument 1 of the C argument list

\end{funcdesc}

\begin{funcdesc}{bessel_jl}{...}\index{bessel_jl}

    Number of Input  Arguments:  2
    Number of Output Arguments:  1
\end{funcdesc}

\begin{funcdesc}{bessel_jl_e}{...}\index{bessel_jl_e}

    Number of Input  Arguments:  2
    Number of Output Arguments:  2

The error flag is discarded.
Return Arguments 1 and 2 resemble a gsl_result argument,
	which is  argument 2 of the C argument list

\end{funcdesc}

\begin{funcdesc}{bessel_k0_scaled}{...}\index{bessel_k0_scaled}

    Number of Input  Arguments:  1
    Number of Output Arguments:  1
\end{funcdesc}

\begin{funcdesc}{bessel_k0_scaled_e}{...}\index{bessel_k0_scaled_e}

    Number of Input  Arguments:  1
    Number of Output Arguments:  2

The error flag is discarded.
Return Arguments 1 and 2 resemble a gsl_result argument,
	which is  argument 1 of the C argument list

\end{funcdesc}

\begin{funcdesc}{bessel_k1_scaled}{...}\index{bessel_k1_scaled}

    Number of Input  Arguments:  1
    Number of Output Arguments:  1
\end{funcdesc}

\begin{funcdesc}{bessel_k1_scaled_e}{...}\index{bessel_k1_scaled_e}

    Number of Input  Arguments:  1
    Number of Output Arguments:  2

The error flag is discarded.
Return Arguments 1 and 2 resemble a gsl_result argument,
	which is  argument 1 of the C argument list

\end{funcdesc}

\begin{funcdesc}{bessel_k2_scaled}{...}\index{bessel_k2_scaled}

    Number of Input  Arguments:  1
    Number of Output Arguments:  1
\end{funcdesc}

\begin{funcdesc}{bessel_k2_scaled_e}{...}\index{bessel_k2_scaled_e}

    Number of Input  Arguments:  1
    Number of Output Arguments:  2

The error flag is discarded.
Return Arguments 1 and 2 resemble a gsl_result argument,
	which is  argument 1 of the C argument list

\end{funcdesc}

\begin{funcdesc}{bessel_kl_scaled}{...}\index{bessel_kl_scaled}

    Number of Input  Arguments:  2
    Number of Output Arguments:  1
\end{funcdesc}

\begin{funcdesc}{bessel_kl_scaled_e}{...}\index{bessel_kl_scaled_e}

    Number of Input  Arguments:  2
    Number of Output Arguments:  2

The error flag is discarded.
Return Arguments 1 and 2 resemble a gsl_result argument,
	which is  argument 2 of the C argument list

\end{funcdesc}

\begin{funcdesc}{bessel_lnKnu}{...}\index{bessel_lnKnu}

    Number of Input  Arguments:  2
    Number of Output Arguments:  1
\end{funcdesc}

\begin{funcdesc}{bessel_lnKnu_e}{...}\index{bessel_lnKnu_e}

    Number of Input  Arguments:  2
    Number of Output Arguments:  2

The error flag is discarded.
Return Arguments 1 and 2 resemble a gsl_result argument,
	which is  argument 2 of the C argument list

\end{funcdesc}

\begin{funcdesc}{bessel_y0}{...}\index{bessel_y0}

    Number of Input  Arguments:  1
    Number of Output Arguments:  1
\end{funcdesc}

\begin{funcdesc}{bessel_y0_e}{...}\index{bessel_y0_e}

    Number of Input  Arguments:  1
    Number of Output Arguments:  2

The error flag is discarded.
Return Arguments 1 and 2 resemble a gsl_result argument,
	which is  argument 1 of the C argument list

\end{funcdesc}

\begin{funcdesc}{bessel_y1}{...}\index{bessel_y1}

    Number of Input  Arguments:  1
    Number of Output Arguments:  1
\end{funcdesc}

\begin{funcdesc}{bessel_y1_e}{...}\index{bessel_y1_e}

    Number of Input  Arguments:  1
    Number of Output Arguments:  2

The error flag is discarded.
Return Arguments 1 and 2 resemble a gsl_result argument,
	which is  argument 1 of the C argument list

\end{funcdesc}

\begin{funcdesc}{bessel_y2}{...}\index{bessel_y2}

    Number of Input  Arguments:  1
    Number of Output Arguments:  1
\end{funcdesc}

\begin{funcdesc}{bessel_y2_e}{...}\index{bessel_y2_e}

    Number of Input  Arguments:  1
    Number of Output Arguments:  2

The error flag is discarded.
Return Arguments 1 and 2 resemble a gsl_result argument,
	which is  argument 1 of the C argument list

\end{funcdesc}

\begin{funcdesc}{bessel_yl}{...}\index{bessel_yl}

    Number of Input  Arguments:  2
    Number of Output Arguments:  1
\end{funcdesc}

\begin{funcdesc}{bessel_yl_e}{...}\index{bessel_yl_e}

    Number of Input  Arguments:  2
    Number of Output Arguments:  2

The error flag is discarded.
Return Arguments 1 and 2 resemble a gsl_result argument,
	which is  argument 2 of the C argument list

\end{funcdesc}

\begin{funcdesc}{bessel_zero_J0}{...}\index{bessel_zero_J0}

    Number of Input  Arguments:  1
    Number of Output Arguments:  1
\end{funcdesc}

\begin{funcdesc}{bessel_zero_J0_e}{...}\index{bessel_zero_J0_e}

    Number of Input  Arguments:  1
    Number of Output Arguments:  2

The error flag is discarded.
Return Arguments 1 and 2 resemble a gsl_result argument,
	which is  argument 1 of the C argument list

\end{funcdesc}

\begin{funcdesc}{bessel_zero_J1}{...}\index{bessel_zero_J1}

    Number of Input  Arguments:  1
    Number of Output Arguments:  1
\end{funcdesc}

\begin{funcdesc}{bessel_zero_J1_e}{...}\index{bessel_zero_J1_e}

    Number of Input  Arguments:  1
    Number of Output Arguments:  2

The error flag is discarded.
Return Arguments 1 and 2 resemble a gsl_result argument,
	which is  argument 1 of the C argument list

\end{funcdesc}

\begin{funcdesc}{bessel_zero_Jnu}{...}\index{bessel_zero_Jnu}

    Number of Input  Arguments:  2
    Number of Output Arguments:  1
\end{funcdesc}

\begin{funcdesc}{bessel_zero_Jnu_e}{...}\index{bessel_zero_Jnu_e}

    Number of Input  Arguments:  2
    Number of Output Arguments:  2

The error flag is discarded.
Return Arguments 1 and 2 resemble a gsl_result argument,
	which is  argument 2 of the C argument list

\end{funcdesc}

\begin{funcdesc}{beta}{...}\index{beta}

    Number of Input  Arguments:  2
    Number of Output Arguments:  1
\end{funcdesc}

\begin{funcdesc}{beta_e}{...}\index{beta_e}

    Number of Input  Arguments:  2
    Number of Output Arguments:  2

The error flag is discarded.
Return Arguments 1 and 2 resemble a gsl_result argument,
	which is  argument 2 of the C argument list

\end{funcdesc}

\begin{funcdesc}{beta_inc}{...}\index{beta_inc}

    Number of Input  Arguments:  3
    Number of Output Arguments:  1
\end{funcdesc}

\begin{funcdesc}{beta_inc_e}{...}\index{beta_inc_e}

    Number of Input  Arguments:  3
    Number of Output Arguments:  2

The error flag is discarded.
Return Arguments 1 and 2 resemble a gsl_result argument,
	which is  argument 3 of the C argument list

\end{funcdesc}

\begin{funcdesc}{choose}{...}\index{choose}

    Number of Input  Arguments:  2
    Number of Output Arguments:  1
\end{funcdesc}

\begin{funcdesc}{choose_e}{...}\index{choose_e}

    Number of Input  Arguments:  2
    Number of Output Arguments:  2

The error flag is discarded.
Return Arguments 1 and 2 resemble a gsl_result argument,
	which is  argument 2 of the C argument list

\end{funcdesc}

\begin{funcdesc}{clausen}{...}\index{clausen}

    Number of Input  Arguments:  1
    Number of Output Arguments:  1
\end{funcdesc}

\begin{funcdesc}{clausen_e}{...}\index{clausen_e}

    Number of Input  Arguments:  1
    Number of Output Arguments:  2

The error flag is discarded.
Return Arguments 1 and 2 resemble a gsl_result argument,
	which is  argument 1 of the C argument list

\end{funcdesc}

\begin{funcdesc}{conicalP_0}{...}\index{conicalP_0}

    Number of Input  Arguments:  2
    Number of Output Arguments:  1
\end{funcdesc}

\begin{funcdesc}{conicalP_0_e}{...}\index{conicalP_0_e}

    Number of Input  Arguments:  2
    Number of Output Arguments:  2

The error flag is discarded.
Return Arguments 1 and 2 resemble a gsl_result argument,
	which is  argument 2 of the C argument list

\end{funcdesc}

\begin{funcdesc}{conicalP_1}{...}\index{conicalP_1}

    Number of Input  Arguments:  2
    Number of Output Arguments:  1
\end{funcdesc}

\begin{funcdesc}{conicalP_1_e}{...}\index{conicalP_1_e}

    Number of Input  Arguments:  2
    Number of Output Arguments:  2

The error flag is discarded.
Return Arguments 1 and 2 resemble a gsl_result argument,
	which is  argument 2 of the C argument list

\end{funcdesc}

\begin{funcdesc}{conicalP_cyl_reg}{...}\index{conicalP_cyl_reg}

    Number of Input  Arguments:  3
    Number of Output Arguments:  1
\end{funcdesc}

\begin{funcdesc}{conicalP_cyl_reg_e}{...}\index{conicalP_cyl_reg_e}

    Number of Input  Arguments:  3
    Number of Output Arguments:  2

The error flag is discarded.
Return Arguments 1 and 2 resemble a gsl_result argument,
	which is  argument 3 of the C argument list

\end{funcdesc}

\begin{funcdesc}{conicalP_half}{...}\index{conicalP_half}

    Number of Input  Arguments:  2
    Number of Output Arguments:  1
\end{funcdesc}

\begin{funcdesc}{conicalP_half_e}{...}\index{conicalP_half_e}

    Number of Input  Arguments:  2
    Number of Output Arguments:  2

The error flag is discarded.
Return Arguments 1 and 2 resemble a gsl_result argument,
	which is  argument 2 of the C argument list

\end{funcdesc}

\begin{funcdesc}{conicalP_mhalf}{...}\index{conicalP_mhalf}

    Number of Input  Arguments:  2
    Number of Output Arguments:  1
\end{funcdesc}

\begin{funcdesc}{conicalP_mhalf_e}{...}\index{conicalP_mhalf_e}

    Number of Input  Arguments:  2
    Number of Output Arguments:  2

The error flag is discarded.
Return Arguments 1 and 2 resemble a gsl_result argument,
	which is  argument 2 of the C argument list

\end{funcdesc}

\begin{funcdesc}{conicalP_sph_reg}{...}\index{conicalP_sph_reg}

    Number of Input  Arguments:  3
    Number of Output Arguments:  1
\end{funcdesc}

\begin{funcdesc}{conicalP_sph_reg_e}{...}\index{conicalP_sph_reg_e}

    Number of Input  Arguments:  3
    Number of Output Arguments:  2

The error flag is discarded.
Return Arguments 1 and 2 resemble a gsl_result argument,
	which is  argument 3 of the C argument list

\end{funcdesc}

\begin{funcdesc}{cos}{...}\index{cos}

    Number of Input  Arguments:  1
    Number of Output Arguments:  1
\end{funcdesc}

\begin{funcdesc}{cos_e}{...}\index{cos_e}

    Number of Input  Arguments:  1
    Number of Output Arguments:  2

The error flag is discarded.
Return Arguments 1 and 2 resemble a gsl_result argument,
	which is  argument 1 of the C argument list

\end{funcdesc}

\begin{funcdesc}{cos_err_e}{...}\index{cos_err_e}

    Number of Input  Arguments:  2
    Number of Output Arguments:  2

The error flag is discarded.
Return Arguments 1 and 2 resemble a gsl_result argument,
	which is  argument 2 of the C argument list

\end{funcdesc}

\begin{funcdesc}{coulomb_CL_e}{...}\index{coulomb_CL_e}

    Number of Input  Arguments:  2
    Number of Output Arguments:  2

The error flag is discarded.
Return Arguments 1 and 2 resemble a gsl_result argument,
	which is  argument 2 of the C argument list

\end{funcdesc}

\begin{funcdesc}{coulomb_wave_FG_e}{...}\index{coulomb_wave_FG_e}

    Number of Input  Arguments:  4
    Number of Output Arguments: 10

The error flag is discarded.
Return Arguments 1 and 2 resemble a gsl_result argument,
	which is  argument 4 of the C argument list
Return Arguments 3 and 4 resemble a gsl_result argument,
	which is  argument 5 of the C argument list
Return Arguments 5 and 6 resemble a gsl_result argument,
	which is  argument 6 of the C argument list
Return Arguments 7 and 8 resemble a gsl_result argument,
	which is  argument 7 of the C argument list

\end{funcdesc}

\begin{funcdesc}{coupling_3j}{...}\index{coupling_3j}

    Number of Input  Arguments:  6
    Number of Output Arguments:  1
\end{funcdesc}

\begin{funcdesc}{coupling_3j_e}{...}\index{coupling_3j_e}

    Number of Input  Arguments:  6
    Number of Output Arguments:  2

The error flag is discarded.
Return Arguments 1 and 2 resemble a gsl_result argument,
	which is  argument 6 of the C argument list

\end{funcdesc}

\begin{funcdesc}{coupling_6j}{...}\index{coupling_6j}

    Number of Input  Arguments:  6
    Number of Output Arguments:  1
\end{funcdesc}

\begin{funcdesc}{coupling_6j_e}{...}\index{coupling_6j_e}

    Number of Input  Arguments:  6
    Number of Output Arguments:  2

The error flag is discarded.
Return Arguments 1 and 2 resemble a gsl_result argument,
	which is  argument 6 of the C argument list

\end{funcdesc}

\begin{funcdesc}{coupling_9j}{...}\index{coupling_9j}

    Number of Input  Arguments:  9
    Number of Output Arguments:  1
\end{funcdesc}

\begin{funcdesc}{coupling_9j_e}{...}\index{coupling_9j_e}

    Number of Input  Arguments:  9
    Number of Output Arguments:  2

The error flag is discarded.
Return Arguments 1 and 2 resemble a gsl_result argument,
	which is  argument 9 of the C argument list

\end{funcdesc}

\begin{funcdesc}{coupling_RacahW}{...}\index{coupling_RacahW}

    Number of Input  Arguments:  6
    Number of Output Arguments:  1
\end{funcdesc}

\begin{funcdesc}{coupling_RacahW_e}{...}\index{coupling_RacahW_e}

    Number of Input  Arguments:  6
    Number of Output Arguments:  2

The error flag is discarded.
Return Arguments 1 and 2 resemble a gsl_result argument,
	which is  argument 6 of the C argument list

\end{funcdesc}

\begin{funcdesc}{dawson}{...}\index{dawson}

    Number of Input  Arguments:  1
    Number of Output Arguments:  1
\end{funcdesc}

\begin{funcdesc}{dawson_e}{...}\index{dawson_e}

    Number of Input  Arguments:  1
    Number of Output Arguments:  2

The error flag is discarded.
Return Arguments 1 and 2 resemble a gsl_result argument,
	which is  argument 1 of the C argument list

\end{funcdesc}

\begin{funcdesc}{debye_1}{...}\index{debye_1}

    Number of Input  Arguments:  1
    Number of Output Arguments:  1
\end{funcdesc}

\begin{funcdesc}{debye_1_e}{...}\index{debye_1_e}

    Number of Input  Arguments:  1
    Number of Output Arguments:  2

The error flag is discarded.
Return Arguments 1 and 2 resemble a gsl_result argument,
	which is  argument 1 of the C argument list

\end{funcdesc}

\begin{funcdesc}{debye_2}{...}\index{debye_2}

    Number of Input  Arguments:  1
    Number of Output Arguments:  1
\end{funcdesc}

\begin{funcdesc}{debye_2_e}{...}\index{debye_2_e}

    Number of Input  Arguments:  1
    Number of Output Arguments:  2

The error flag is discarded.
Return Arguments 1 and 2 resemble a gsl_result argument,
	which is  argument 1 of the C argument list

\end{funcdesc}

\begin{funcdesc}{debye_3}{...}\index{debye_3}

    Number of Input  Arguments:  1
    Number of Output Arguments:  1
\end{funcdesc}

\begin{funcdesc}{debye_3_e}{...}\index{debye_3_e}

    Number of Input  Arguments:  1
    Number of Output Arguments:  2

The error flag is discarded.
Return Arguments 1 and 2 resemble a gsl_result argument,
	which is  argument 1 of the C argument list

\end{funcdesc}

\begin{funcdesc}{debye_4}{...}\index{debye_4}

    Number of Input  Arguments:  1
    Number of Output Arguments:  1
\end{funcdesc}

\begin{funcdesc}{debye_4_e}{...}\index{debye_4_e}

    Number of Input  Arguments:  1
    Number of Output Arguments:  2

The error flag is discarded.
Return Arguments 1 and 2 resemble a gsl_result argument,
	which is  argument 1 of the C argument list

\end{funcdesc}

\begin{funcdesc}{dilog}{...}\index{dilog}

    Number of Input  Arguments:  1
    Number of Output Arguments:  1
\end{funcdesc}

\begin{funcdesc}{dilog_e}{...}\index{dilog_e}

    Number of Input  Arguments:  1
    Number of Output Arguments:  2

The error flag is discarded.
Return Arguments 1 and 2 resemble a gsl_result argument,
	which is  argument 1 of the C argument list

\end{funcdesc}

\begin{funcdesc}{doublefact}{...}\index{doublefact}

    Number of Input  Arguments:  1
    Number of Output Arguments:  1
\end{funcdesc}

\begin{funcdesc}{doublefact_e}{...}\index{doublefact_e}

    Number of Input  Arguments:  1
    Number of Output Arguments:  2

The error flag is discarded.
Return Arguments 1 and 2 resemble a gsl_result argument,
	which is  argument 1 of the C argument list

\end{funcdesc}

\begin{funcdesc}{ellint_D}{...}\index{ellint_D}

    Number of Input  Arguments:  4
    Number of Output Arguments:  1

 Argument 4 is a gsl_mode_t, valid parameters are:
	sf.PREC_DOUBLE or sf.PREC_SINGLE or sf.PREC_APPROX

\end{funcdesc}

\begin{funcdesc}{ellint_D_e}{...}\index{ellint_D_e}

    Number of Input  Arguments:  4
    Number of Output Arguments:  2

 Argument 4 is a gsl_mode_t, valid parameters are:
	sf.PREC_DOUBLE or sf.PREC_SINGLE or sf.PREC_APPROX
The error flag is discarded.
Return Arguments 1 and 2 resemble a gsl_result argument,
	which is  argument 4 of the C argument list

\end{funcdesc}

\begin{funcdesc}{ellint_E}{...}\index{ellint_E}

    Number of Input  Arguments:  3
    Number of Output Arguments:  1

 Argument 3 is a gsl_mode_t, valid parameters are:
	sf.PREC_DOUBLE or sf.PREC_SINGLE or sf.PREC_APPROX

\end{funcdesc}

\begin{funcdesc}{ellint_E_e}{...}\index{ellint_E_e}

    Number of Input  Arguments:  3
    Number of Output Arguments:  2

 Argument 3 is a gsl_mode_t, valid parameters are:
	sf.PREC_DOUBLE or sf.PREC_SINGLE or sf.PREC_APPROX
The error flag is discarded.
Return Arguments 1 and 2 resemble a gsl_result argument,
	which is  argument 3 of the C argument list

\end{funcdesc}

\begin{funcdesc}{ellint_Ecomp}{...}\index{ellint_Ecomp}

    Number of Input  Arguments:  2
    Number of Output Arguments:  1

 Argument 2 is a gsl_mode_t, valid parameters are:
	sf.PREC_DOUBLE or sf.PREC_SINGLE or sf.PREC_APPROX

\end{funcdesc}

\begin{funcdesc}{ellint_Ecomp_e}{...}\index{ellint_Ecomp_e}

    Number of Input  Arguments:  2
    Number of Output Arguments:  2

 Argument 2 is a gsl_mode_t, valid parameters are:
	sf.PREC_DOUBLE or sf.PREC_SINGLE or sf.PREC_APPROX
The error flag is discarded.
Return Arguments 1 and 2 resemble a gsl_result argument,
	which is  argument 2 of the C argument list

\end{funcdesc}

\begin{funcdesc}{ellint_F}{...}\index{ellint_F}

    Number of Input  Arguments:  3
    Number of Output Arguments:  1

 Argument 3 is a gsl_mode_t, valid parameters are:
	sf.PREC_DOUBLE or sf.PREC_SINGLE or sf.PREC_APPROX

\end{funcdesc}

\begin{funcdesc}{ellint_F_e}{...}\index{ellint_F_e}

    Number of Input  Arguments:  3
    Number of Output Arguments:  2

 Argument 3 is a gsl_mode_t, valid parameters are:
	sf.PREC_DOUBLE or sf.PREC_SINGLE or sf.PREC_APPROX
The error flag is discarded.
Return Arguments 1 and 2 resemble a gsl_result argument,
	which is  argument 3 of the C argument list

\end{funcdesc}

\begin{funcdesc}{ellint_Kcomp}{...}\index{ellint_Kcomp}

    Number of Input  Arguments:  2
    Number of Output Arguments:  1

 Argument 2 is a gsl_mode_t, valid parameters are:
	sf.PREC_DOUBLE or sf.PREC_SINGLE or sf.PREC_APPROX

\end{funcdesc}

\begin{funcdesc}{ellint_Kcomp_e}{...}\index{ellint_Kcomp_e}

    Number of Input  Arguments:  2
    Number of Output Arguments:  2

 Argument 2 is a gsl_mode_t, valid parameters are:
	sf.PREC_DOUBLE or sf.PREC_SINGLE or sf.PREC_APPROX
The error flag is discarded.
Return Arguments 1 and 2 resemble a gsl_result argument,
	which is  argument 2 of the C argument list

\end{funcdesc}

\begin{funcdesc}{ellint_P}{...}\index{ellint_P}

    Number of Input  Arguments:  4
    Number of Output Arguments:  1

 Argument 4 is a gsl_mode_t, valid parameters are:
	sf.PREC_DOUBLE or sf.PREC_SINGLE or sf.PREC_APPROX

\end{funcdesc}

\begin{funcdesc}{ellint_P_e}{...}\index{ellint_P_e}

    Number of Input  Arguments:  4
    Number of Output Arguments:  2

 Argument 4 is a gsl_mode_t, valid parameters are:
	sf.PREC_DOUBLE or sf.PREC_SINGLE or sf.PREC_APPROX
The error flag is discarded.
Return Arguments 1 and 2 resemble a gsl_result argument,
	which is  argument 4 of the C argument list

\end{funcdesc}

\begin{funcdesc}{ellint_RC}{...}\index{ellint_RC}

    Number of Input  Arguments:  3
    Number of Output Arguments:  1

 Argument 3 is a gsl_mode_t, valid parameters are:
	sf.PREC_DOUBLE or sf.PREC_SINGLE or sf.PREC_APPROX

\end{funcdesc}

\begin{funcdesc}{ellint_RC_e}{...}\index{ellint_RC_e}

    Number of Input  Arguments:  3
    Number of Output Arguments:  2

 Argument 3 is a gsl_mode_t, valid parameters are:
	sf.PREC_DOUBLE or sf.PREC_SINGLE or sf.PREC_APPROX
The error flag is discarded.
Return Arguments 1 and 2 resemble a gsl_result argument,
	which is  argument 3 of the C argument list

\end{funcdesc}

\begin{funcdesc}{ellint_RD}{...}\index{ellint_RD}

    Number of Input  Arguments:  4
    Number of Output Arguments:  1

 Argument 4 is a gsl_mode_t, valid parameters are:
	sf.PREC_DOUBLE or sf.PREC_SINGLE or sf.PREC_APPROX

\end{funcdesc}

\begin{funcdesc}{ellint_RD_e}{...}\index{ellint_RD_e}

    Number of Input  Arguments:  4
    Number of Output Arguments:  2

 Argument 4 is a gsl_mode_t, valid parameters are:
	sf.PREC_DOUBLE or sf.PREC_SINGLE or sf.PREC_APPROX
The error flag is discarded.
Return Arguments 1 and 2 resemble a gsl_result argument,
	which is  argument 4 of the C argument list

\end{funcdesc}

\begin{funcdesc}{ellint_RF}{...}\index{ellint_RF}

    Number of Input  Arguments:  4
    Number of Output Arguments:  1

 Argument 4 is a gsl_mode_t, valid parameters are:
	sf.PREC_DOUBLE or sf.PREC_SINGLE or sf.PREC_APPROX

\end{funcdesc}

\begin{funcdesc}{ellint_RF_e}{...}\index{ellint_RF_e}

    Number of Input  Arguments:  4
    Number of Output Arguments:  2

 Argument 4 is a gsl_mode_t, valid parameters are:
	sf.PREC_DOUBLE or sf.PREC_SINGLE or sf.PREC_APPROX
The error flag is discarded.
Return Arguments 1 and 2 resemble a gsl_result argument,
	which is  argument 4 of the C argument list

\end{funcdesc}

\begin{funcdesc}{ellint_RJ}{...}\index{ellint_RJ}

    Number of Input  Arguments:  5
    Number of Output Arguments:  1

 Argument 5 is a gsl_mode_t, valid parameters are:
	sf.PREC_DOUBLE or sf.PREC_SINGLE or sf.PREC_APPROX

\end{funcdesc}

\begin{funcdesc}{ellint_RJ_e}{...}\index{ellint_RJ_e}

    Number of Input  Arguments:  5
    Number of Output Arguments:  2

 Argument 5 is a gsl_mode_t, valid parameters are:
	sf.PREC_DOUBLE or sf.PREC_SINGLE or sf.PREC_APPROX
The error flag is discarded.
Return Arguments 1 and 2 resemble a gsl_result argument,
	which is  argument 5 of the C argument list

\end{funcdesc}

\begin{funcdesc}{elljac_e}{...}\index{elljac_e}

    Number of Input  Arguments:  2
    Number of Output Arguments:  3

The error flag is discarded.

\end{funcdesc}

\begin{funcdesc}{erf}{...}\index{erf}

    Number of Input  Arguments:  1
    Number of Output Arguments:  1
\end{funcdesc}

\begin{funcdesc}{erf_Q}{...}\index{erf_Q}

    Number of Input  Arguments:  1
    Number of Output Arguments:  1
\end{funcdesc}

\begin{funcdesc}{erf_Q_e}{...}\index{erf_Q_e}

    Number of Input  Arguments:  1
    Number of Output Arguments:  2

The error flag is discarded.
Return Arguments 1 and 2 resemble a gsl_result argument,
	which is  argument 1 of the C argument list

\end{funcdesc}

\begin{funcdesc}{erf_Z}{...}\index{erf_Z}

    Number of Input  Arguments:  1
    Number of Output Arguments:  1
\end{funcdesc}

\begin{funcdesc}{erf_Z_e}{...}\index{erf_Z_e}

    Number of Input  Arguments:  1
    Number of Output Arguments:  2

The error flag is discarded.
Return Arguments 1 and 2 resemble a gsl_result argument,
	which is  argument 1 of the C argument list

\end{funcdesc}

\begin{funcdesc}{erf_e}{...}\index{erf_e}

    Number of Input  Arguments:  1
    Number of Output Arguments:  2

The error flag is discarded.
Return Arguments 1 and 2 resemble a gsl_result argument,
	which is  argument 1 of the C argument list

\end{funcdesc}

\begin{funcdesc}{erfc}{...}\index{erfc}

    Number of Input  Arguments:  1
    Number of Output Arguments:  1
\end{funcdesc}

\begin{funcdesc}{erfc_e}{...}\index{erfc_e}

    Number of Input  Arguments:  1
    Number of Output Arguments:  2

The error flag is discarded.
Return Arguments 1 and 2 resemble a gsl_result argument,
	which is  argument 1 of the C argument list

\end{funcdesc}

\begin{funcdesc}{eta}{...}\index{eta}

    Number of Input  Arguments:  1
    Number of Output Arguments:  1
\end{funcdesc}

\begin{funcdesc}{eta_e}{...}\index{eta_e}

    Number of Input  Arguments:  1
    Number of Output Arguments:  2

The error flag is discarded.
Return Arguments 1 and 2 resemble a gsl_result argument,
	which is  argument 1 of the C argument list

\end{funcdesc}

\begin{funcdesc}{eta_int}{...}\index{eta_int}

    Number of Input  Arguments:  1
    Number of Output Arguments:  1
\end{funcdesc}

\begin{funcdesc}{eta_int_e}{...}\index{eta_int_e}

    Number of Input  Arguments:  1
    Number of Output Arguments:  2

The error flag is discarded.
Return Arguments 1 and 2 resemble a gsl_result argument,
	which is  argument 1 of the C argument list

\end{funcdesc}

\begin{funcdesc}{expint_3}{...}\index{expint_3}

    Number of Input  Arguments:  1
    Number of Output Arguments:  1
\end{funcdesc}

\begin{funcdesc}{expint_3_e}{...}\index{expint_3_e}

    Number of Input  Arguments:  1
    Number of Output Arguments:  2

The error flag is discarded.
Return Arguments 1 and 2 resemble a gsl_result argument,
	which is  argument 1 of the C argument list

\end{funcdesc}

\begin{funcdesc}{expint_E1}{...}\index{expint_E1}

    Number of Input  Arguments:  1
    Number of Output Arguments:  1
\end{funcdesc}

\begin{funcdesc}{expint_E1_e}{...}\index{expint_E1_e}

    Number of Input  Arguments:  1
    Number of Output Arguments:  2

The error flag is discarded.
Return Arguments 1 and 2 resemble a gsl_result argument,
	which is  argument 1 of the C argument list

\end{funcdesc}

\begin{funcdesc}{expint_E1_scaled}{...}\index{expint_E1_scaled}

    Number of Input  Arguments:  1
    Number of Output Arguments:  1
\end{funcdesc}

\begin{funcdesc}{expint_E1_scaled_e}{...}\index{expint_E1_scaled_e}

    Number of Input  Arguments:  1
    Number of Output Arguments:  2

The error flag is discarded.
Return Arguments 1 and 2 resemble a gsl_result argument,
	which is  argument 1 of the C argument list

\end{funcdesc}

\begin{funcdesc}{expint_E2}{...}\index{expint_E2}

    Number of Input  Arguments:  1
    Number of Output Arguments:  1
\end{funcdesc}

\begin{funcdesc}{expint_E2_e}{...}\index{expint_E2_e}

    Number of Input  Arguments:  1
    Number of Output Arguments:  2

The error flag is discarded.
Return Arguments 1 and 2 resemble a gsl_result argument,
	which is  argument 1 of the C argument list

\end{funcdesc}

\begin{funcdesc}{expint_E2_scaled}{...}\index{expint_E2_scaled}

    Number of Input  Arguments:  1
    Number of Output Arguments:  1
\end{funcdesc}

\begin{funcdesc}{expint_E2_scaled_e}{...}\index{expint_E2_scaled_e}

    Number of Input  Arguments:  1
    Number of Output Arguments:  2

The error flag is discarded.
Return Arguments 1 and 2 resemble a gsl_result argument,
	which is  argument 1 of the C argument list

\end{funcdesc}

\begin{funcdesc}{expint_Ei}{...}\index{expint_Ei}

    Number of Input  Arguments:  1
    Number of Output Arguments:  1
\end{funcdesc}

\begin{funcdesc}{expint_Ei_e}{...}\index{expint_Ei_e}

    Number of Input  Arguments:  1
    Number of Output Arguments:  2

The error flag is discarded.
Return Arguments 1 and 2 resemble a gsl_result argument,
	which is  argument 1 of the C argument list

\end{funcdesc}

\begin{funcdesc}{expint_Ei_scaled}{...}\index{expint_Ei_scaled}

    Number of Input  Arguments:  1
    Number of Output Arguments:  1
\end{funcdesc}

\begin{funcdesc}{expint_Ei_scaled_e}{...}\index{expint_Ei_scaled_e}

    Number of Input  Arguments:  1
    Number of Output Arguments:  2

The error flag is discarded.
Return Arguments 1 and 2 resemble a gsl_result argument,
	which is  argument 1 of the C argument list

\end{funcdesc}

\begin{funcdesc}{fact}{...}\index{fact}

    Number of Input  Arguments:  1
    Number of Output Arguments:  1
\end{funcdesc}

\begin{funcdesc}{fact_e}{...}\index{fact_e}

    Number of Input  Arguments:  1
    Number of Output Arguments:  2

The error flag is discarded.
Return Arguments 1 and 2 resemble a gsl_result argument,
	which is  argument 1 of the C argument list

\end{funcdesc}

\begin{funcdesc}{fermi_dirac_0}{...}\index{fermi_dirac_0}

    Number of Input  Arguments:  1
    Number of Output Arguments:  1
\end{funcdesc}

\begin{funcdesc}{fermi_dirac_0_e}{...}\index{fermi_dirac_0_e}

    Number of Input  Arguments:  1
    Number of Output Arguments:  2

The error flag is discarded.
Return Arguments 1 and 2 resemble a gsl_result argument,
	which is  argument 1 of the C argument list

\end{funcdesc}

\begin{funcdesc}{fermi_dirac_1}{...}\index{fermi_dirac_1}

    Number of Input  Arguments:  1
    Number of Output Arguments:  1
\end{funcdesc}

\begin{funcdesc}{fermi_dirac_1_e}{...}\index{fermi_dirac_1_e}

    Number of Input  Arguments:  1
    Number of Output Arguments:  2

The error flag is discarded.
Return Arguments 1 and 2 resemble a gsl_result argument,
	which is  argument 1 of the C argument list

\end{funcdesc}

\begin{funcdesc}{fermi_dirac_2}{...}\index{fermi_dirac_2}

    Number of Input  Arguments:  1
    Number of Output Arguments:  1
\end{funcdesc}

\begin{funcdesc}{fermi_dirac_2_e}{...}\index{fermi_dirac_2_e}

    Number of Input  Arguments:  1
    Number of Output Arguments:  2

The error flag is discarded.
Return Arguments 1 and 2 resemble a gsl_result argument,
	which is  argument 1 of the C argument list

\end{funcdesc}

\begin{funcdesc}{fermi_dirac_3half}{...}\index{fermi_dirac_3half}

    Number of Input  Arguments:  1
    Number of Output Arguments:  1
\end{funcdesc}

\begin{funcdesc}{fermi_dirac_3half_e}{...}\index{fermi_dirac_3half_e}

    Number of Input  Arguments:  1
    Number of Output Arguments:  2

The error flag is discarded.
Return Arguments 1 and 2 resemble a gsl_result argument,
	which is  argument 1 of the C argument list

\end{funcdesc}

\begin{funcdesc}{fermi_dirac_half}{...}\index{fermi_dirac_half}

    Number of Input  Arguments:  1
    Number of Output Arguments:  1
\end{funcdesc}

\begin{funcdesc}{fermi_dirac_half_e}{...}\index{fermi_dirac_half_e}

    Number of Input  Arguments:  1
    Number of Output Arguments:  2

The error flag is discarded.
Return Arguments 1 and 2 resemble a gsl_result argument,
	which is  argument 1 of the C argument list

\end{funcdesc}

\begin{funcdesc}{fermi_dirac_inc_0}{...}\index{fermi_dirac_inc_0}

    Number of Input  Arguments:  2
    Number of Output Arguments:  1
\end{funcdesc}

\begin{funcdesc}{fermi_dirac_inc_0_e}{...}\index{fermi_dirac_inc_0_e}

    Number of Input  Arguments:  2
    Number of Output Arguments:  2

The error flag is discarded.
Return Arguments 1 and 2 resemble a gsl_result argument,
	which is  argument 2 of the C argument list

\end{funcdesc}

\begin{funcdesc}{fermi_dirac_int}{...}\index{fermi_dirac_int}

    Number of Input  Arguments:  2
    Number of Output Arguments:  1
\end{funcdesc}

\begin{funcdesc}{fermi_dirac_int_e}{...}\index{fermi_dirac_int_e}

    Number of Input  Arguments:  2
    Number of Output Arguments:  2

The error flag is discarded.
Return Arguments 1 and 2 resemble a gsl_result argument,
	which is  argument 2 of the C argument list

\end{funcdesc}

\begin{funcdesc}{fermi_dirac_m1}{...}\index{fermi_dirac_m1}

    Number of Input  Arguments:  1
    Number of Output Arguments:  1
\end{funcdesc}

\begin{funcdesc}{fermi_dirac_m1_e}{...}\index{fermi_dirac_m1_e}

    Number of Input  Arguments:  1
    Number of Output Arguments:  2

The error flag is discarded.
Return Arguments 1 and 2 resemble a gsl_result argument,
	which is  argument 1 of the C argument list

\end{funcdesc}

\begin{funcdesc}{fermi_dirac_mhalf}{...}\index{fermi_dirac_mhalf}

    Number of Input  Arguments:  1
    Number of Output Arguments:  1
\end{funcdesc}

\begin{funcdesc}{fermi_dirac_mhalf_e}{...}\index{fermi_dirac_mhalf_e}

    Number of Input  Arguments:  1
    Number of Output Arguments:  2

The error flag is discarded.
Return Arguments 1 and 2 resemble a gsl_result argument,
	which is  argument 1 of the C argument list

\end{funcdesc}

\begin{funcdesc}{gamma}{...}\index{gamma}

    Number of Input  Arguments:  1
    Number of Output Arguments:  1
\end{funcdesc}

\begin{funcdesc}{gamma_e}{...}\index{gamma_e}

    Number of Input  Arguments:  1
    Number of Output Arguments:  2

The error flag is discarded.
Return Arguments 1 and 2 resemble a gsl_result argument,
	which is  argument 1 of the C argument list

\end{funcdesc}

\begin{funcdesc}{gamma_inc_P}{...}\index{gamma_inc_P}

    Number of Input  Arguments:  2
    Number of Output Arguments:  1
\end{funcdesc}

\begin{funcdesc}{gamma_inc_P_e}{...}\index{gamma_inc_P_e}

    Number of Input  Arguments:  2
    Number of Output Arguments:  2

The error flag is discarded.
Return Arguments 1 and 2 resemble a gsl_result argument,
	which is  argument 2 of the C argument list

\end{funcdesc}

\begin{funcdesc}{gamma_inc_Q}{...}\index{gamma_inc_Q}

    Number of Input  Arguments:  2
    Number of Output Arguments:  1
\end{funcdesc}

\begin{funcdesc}{gamma_inc_Q_e}{...}\index{gamma_inc_Q_e}

    Number of Input  Arguments:  2
    Number of Output Arguments:  2

The error flag is discarded.
Return Arguments 1 and 2 resemble a gsl_result argument,
	which is  argument 2 of the C argument list

\end{funcdesc}

\begin{funcdesc}{gammainv}{...}\index{gammainv}

    Number of Input  Arguments:  1
    Number of Output Arguments:  1
\end{funcdesc}

\begin{funcdesc}{gammainv_e}{...}\index{gammainv_e}

    Number of Input  Arguments:  1
    Number of Output Arguments:  2

The error flag is discarded.
Return Arguments 1 and 2 resemble a gsl_result argument,
	which is  argument 1 of the C argument list

\end{funcdesc}

\begin{funcdesc}{gammastar}{...}\index{gammastar}

    Number of Input  Arguments:  1
    Number of Output Arguments:  1
\end{funcdesc}

\begin{funcdesc}{gammastar_e}{...}\index{gammastar_e}

    Number of Input  Arguments:  1
    Number of Output Arguments:  2

The error flag is discarded.
Return Arguments 1 and 2 resemble a gsl_result argument,
	which is  argument 1 of the C argument list

\end{funcdesc}

\begin{funcdesc}{gegenpoly_1}{...}\index{gegenpoly_1}

    Number of Input  Arguments:  2
    Number of Output Arguments:  1
\end{funcdesc}

\begin{funcdesc}{gegenpoly_1_e}{...}\index{gegenpoly_1_e}

    Number of Input  Arguments:  2
    Number of Output Arguments:  2

The error flag is discarded.
Return Arguments 1 and 2 resemble a gsl_result argument,
	which is  argument 2 of the C argument list

\end{funcdesc}

\begin{funcdesc}{gegenpoly_2}{...}\index{gegenpoly_2}

    Number of Input  Arguments:  2
    Number of Output Arguments:  1
\end{funcdesc}

\begin{funcdesc}{gegenpoly_2_e}{...}\index{gegenpoly_2_e}

    Number of Input  Arguments:  2
    Number of Output Arguments:  2

The error flag is discarded.
Return Arguments 1 and 2 resemble a gsl_result argument,
	which is  argument 2 of the C argument list

\end{funcdesc}

\begin{funcdesc}{gegenpoly_3}{...}\index{gegenpoly_3}

    Number of Input  Arguments:  2
    Number of Output Arguments:  1
\end{funcdesc}

\begin{funcdesc}{gegenpoly_3_e}{...}\index{gegenpoly_3_e}

    Number of Input  Arguments:  2
    Number of Output Arguments:  2

The error flag is discarded.
Return Arguments 1 and 2 resemble a gsl_result argument,
	which is  argument 2 of the C argument list

\end{funcdesc}

\begin{funcdesc}{gegenpoly_n}{...}\index{gegenpoly_n}

    Number of Input  Arguments:  3
    Number of Output Arguments:  1
\end{funcdesc}

\begin{funcdesc}{gegenpoly_n_e}{...}\index{gegenpoly_n_e}

    Number of Input  Arguments:  3
    Number of Output Arguments:  2

The error flag is discarded.
Return Arguments 1 and 2 resemble a gsl_result argument,
	which is  argument 3 of the C argument list

\end{funcdesc}

\begin{funcdesc}{hydrogenicR}{...}\index{hydrogenicR}

    Number of Input  Arguments:  4
    Number of Output Arguments:  1
\end{funcdesc}

\begin{funcdesc}{hydrogenicR_1}{...}\index{hydrogenicR_1}

    Number of Input  Arguments:  2
    Number of Output Arguments:  1
\end{funcdesc}

\begin{funcdesc}{hydrogenicR_1_e}{...}\index{hydrogenicR_1_e}

    Number of Input  Arguments:  2
    Number of Output Arguments:  2

The error flag is discarded.
Return Arguments 1 and 2 resemble a gsl_result argument,
	which is  argument 2 of the C argument list

\end{funcdesc}

\begin{funcdesc}{hydrogenicR_e}{...}\index{hydrogenicR_e}

    Number of Input  Arguments:  4
    Number of Output Arguments:  2

The error flag is discarded.
Return Arguments 1 and 2 resemble a gsl_result argument,
	which is  argument 4 of the C argument list

\end{funcdesc}

\begin{funcdesc}{hyperg_0F1}{...}\index{hyperg_0F1}

    Number of Input  Arguments:  2
    Number of Output Arguments:  1
\end{funcdesc}

\begin{funcdesc}{hyperg_0F1_e}{...}\index{hyperg_0F1_e}

    Number of Input  Arguments:  2
    Number of Output Arguments:  2

The error flag is discarded.
Return Arguments 1 and 2 resemble a gsl_result argument,
	which is  argument 2 of the C argument list

\end{funcdesc}

\begin{funcdesc}{hyperg_1F1}{...}\index{hyperg_1F1}

    Number of Input  Arguments:  3
    Number of Output Arguments:  1
\end{funcdesc}

\begin{funcdesc}{hyperg_1F1_e}{...}\index{hyperg_1F1_e}

    Number of Input  Arguments:  3
    Number of Output Arguments:  2

The error flag is discarded.
Return Arguments 1 and 2 resemble a gsl_result argument,
	which is  argument 3 of the C argument list

\end{funcdesc}

\begin{funcdesc}{hyperg_1F1_int}{...}\index{hyperg_1F1_int}

    Number of Input  Arguments:  3
    Number of Output Arguments:  1
\end{funcdesc}

\begin{funcdesc}{hyperg_1F1_int_e}{...}\index{hyperg_1F1_int_e}

    Number of Input  Arguments:  3
    Number of Output Arguments:  2

The error flag is discarded.
Return Arguments 1 and 2 resemble a gsl_result argument,
	which is  argument 3 of the C argument list

\end{funcdesc}

\begin{funcdesc}{hyperg_2F0}{...}\index{hyperg_2F0}

    Number of Input  Arguments:  3
    Number of Output Arguments:  1
\end{funcdesc}

\begin{funcdesc}{hyperg_2F0_e}{...}\index{hyperg_2F0_e}

    Number of Input  Arguments:  3
    Number of Output Arguments:  2

The error flag is discarded.
Return Arguments 1 and 2 resemble a gsl_result argument,
	which is  argument 3 of the C argument list

\end{funcdesc}

\begin{funcdesc}{hyperg_2F1}{...}\index{hyperg_2F1}

    Number of Input  Arguments:  4
    Number of Output Arguments:  1
\end{funcdesc}

\begin{funcdesc}{hyperg_2F1_conj}{...}\index{hyperg_2F1_conj}

    Number of Input  Arguments:  4
    Number of Output Arguments:  1
\end{funcdesc}

\begin{funcdesc}{hyperg_2F1_conj_e}{...}\index{hyperg_2F1_conj_e}

    Number of Input  Arguments:  4
    Number of Output Arguments:  2

The error flag is discarded.
Return Arguments 1 and 2 resemble a gsl_result argument,
	which is  argument 4 of the C argument list

\end{funcdesc}

\begin{funcdesc}{hyperg_2F1_conj_renorm}{...}\index{hyperg_2F1_conj_renorm}

    Number of Input  Arguments:  4
    Number of Output Arguments:  1
\end{funcdesc}

\begin{funcdesc}{hyperg_2F1_conj_renorm_e}{...}\index{hyperg_2F1_conj_renorm_e}

    Number of Input  Arguments:  4
    Number of Output Arguments:  2

The error flag is discarded.
Return Arguments 1 and 2 resemble a gsl_result argument,
	which is  argument 4 of the C argument list

\end{funcdesc}

\begin{funcdesc}{hyperg_2F1_e}{...}\index{hyperg_2F1_e}

    Number of Input  Arguments:  4
    Number of Output Arguments:  2

The error flag is discarded.
Return Arguments 1 and 2 resemble a gsl_result argument,
	which is  argument 4 of the C argument list

\end{funcdesc}

\begin{funcdesc}{hyperg_2F1_renorm}{...}\index{hyperg_2F1_renorm}

    Number of Input  Arguments:  4
    Number of Output Arguments:  1
\end{funcdesc}

\begin{funcdesc}{hyperg_2F1_renorm_e}{...}\index{hyperg_2F1_renorm_e}

    Number of Input  Arguments:  4
    Number of Output Arguments:  2

The error flag is discarded.
Return Arguments 1 and 2 resemble a gsl_result argument,
	which is  argument 4 of the C argument list

\end{funcdesc}

\begin{funcdesc}{hyperg_U}{...}\index{hyperg_U}

    Number of Input  Arguments:  3
    Number of Output Arguments:  1
\end{funcdesc}

\begin{funcdesc}{hyperg_U_e}{...}\index{hyperg_U_e}

    Number of Input  Arguments:  3
    Number of Output Arguments:  2

The error flag is discarded.
Return Arguments 1 and 2 resemble a gsl_result argument,
	which is  argument 3 of the C argument list

\end{funcdesc}

\begin{funcdesc}{hyperg_U_e10_e}{...}\index{hyperg_U_e10_e}

    Number of Input  Arguments:  3
    Number of Output Arguments:  3

The error flag is discarded.
Return Arguments 1 - 3 resemble a gsl_result_e10 argument,
	which is argument 3 of the C argument list

\end{funcdesc}

\begin{funcdesc}{hyperg_U_int}{...}\index{hyperg_U_int}

    Number of Input  Arguments:  3
    Number of Output Arguments:  1
\end{funcdesc}

\begin{funcdesc}{hyperg_U_int_e}{...}\index{hyperg_U_int_e}

    Number of Input  Arguments:  3
    Number of Output Arguments:  2

The error flag is discarded.
Return Arguments 1 and 2 resemble a gsl_result argument,
	which is  argument 3 of the C argument list

\end{funcdesc}

\begin{funcdesc}{hyperg_U_int_e10_e}{...}\index{hyperg_U_int_e10_e}

    Number of Input  Arguments:  3
    Number of Output Arguments:  3

The error flag is discarded.
Return Arguments 1 - 3 resemble a gsl_result_e10 argument,
	which is argument 3 of the C argument list

\end{funcdesc}

\begin{funcdesc}{hypot}{...}\index{hypot}

    Number of Input  Arguments:  2
    Number of Output Arguments:  1
\end{funcdesc}

\begin{funcdesc}{hypot_e}{...}\index{hypot_e}

    Number of Input  Arguments:  2
    Number of Output Arguments:  2

The error flag is discarded.
Return Arguments 1 and 2 resemble a gsl_result argument,
	which is  argument 2 of the C argument list

\end{funcdesc}

\begin{funcdesc}{hzeta}{...}\index{hzeta}

    Number of Input  Arguments:  2
    Number of Output Arguments:  1
\end{funcdesc}

\begin{funcdesc}{hzeta_e}{...}\index{hzeta_e}

    Number of Input  Arguments:  2
    Number of Output Arguments:  2

The error flag is discarded.
Return Arguments 1 and 2 resemble a gsl_result argument,
	which is  argument 2 of the C argument list

\end{funcdesc}

\begin{funcdesc}{laguerre_1}{...}\index{laguerre_1}

    Number of Input  Arguments:  2
    Number of Output Arguments:  1
\end{funcdesc}

\begin{funcdesc}{laguerre_1_e}{...}\index{laguerre_1_e}

    Number of Input  Arguments:  2
    Number of Output Arguments:  2

The error flag is discarded.
Return Arguments 1 and 2 resemble a gsl_result argument,
	which is  argument 2 of the C argument list

\end{funcdesc}

\begin{funcdesc}{laguerre_2}{...}\index{laguerre_2}

    Number of Input  Arguments:  2
    Number of Output Arguments:  1
\end{funcdesc}

\begin{funcdesc}{laguerre_2_e}{...}\index{laguerre_2_e}

    Number of Input  Arguments:  2
    Number of Output Arguments:  2

The error flag is discarded.
Return Arguments 1 and 2 resemble a gsl_result argument,
	which is  argument 2 of the C argument list

\end{funcdesc}

\begin{funcdesc}{laguerre_3}{...}\index{laguerre_3}

    Number of Input  Arguments:  2
    Number of Output Arguments:  1
\end{funcdesc}

\begin{funcdesc}{laguerre_3_e}{...}\index{laguerre_3_e}

    Number of Input  Arguments:  2
    Number of Output Arguments:  2

The error flag is discarded.
Return Arguments 1 and 2 resemble a gsl_result argument,
	which is  argument 2 of the C argument list

\end{funcdesc}

\begin{funcdesc}{laguerre_n}{...}\index{laguerre_n}

    Number of Input  Arguments:  3
    Number of Output Arguments:  1
\end{funcdesc}

\begin{funcdesc}{laguerre_n_e}{...}\index{laguerre_n_e}

    Number of Input  Arguments:  3
    Number of Output Arguments:  2

The error flag is discarded.
Return Arguments 1 and 2 resemble a gsl_result argument,
	which is  argument 3 of the C argument list

\end{funcdesc}

\begin{funcdesc}{lambert_W0}{...}\index{lambert_W0}

    Number of Input  Arguments:  1
    Number of Output Arguments:  1
\end{funcdesc}

\begin{funcdesc}{lambert_W0_e}{...}\index{lambert_W0_e}

    Number of Input  Arguments:  1
    Number of Output Arguments:  2

The error flag is discarded.
Return Arguments 1 and 2 resemble a gsl_result argument,
	which is  argument 1 of the C argument list

\end{funcdesc}

\begin{funcdesc}{lambert_Wm1}{...}\index{lambert_Wm1}

    Number of Input  Arguments:  1
    Number of Output Arguments:  1
\end{funcdesc}

\begin{funcdesc}{lambert_Wm1_e}{...}\index{lambert_Wm1_e}

    Number of Input  Arguments:  1
    Number of Output Arguments:  2

The error flag is discarded.
Return Arguments 1 and 2 resemble a gsl_result argument,
	which is  argument 1 of the C argument list

\end{funcdesc}

\begin{funcdesc}{legendre_H3d}{...}\index{legendre_H3d}

    Number of Input  Arguments:  3
    Number of Output Arguments:  1
\end{funcdesc}

\begin{funcdesc}{legendre_H3d_0}{...}\index{legendre_H3d_0}

    Number of Input  Arguments:  2
    Number of Output Arguments:  1
\end{funcdesc}

\begin{funcdesc}{legendre_H3d_0_e}{...}\index{legendre_H3d_0_e}

    Number of Input  Arguments:  2
    Number of Output Arguments:  2

The error flag is discarded.
Return Arguments 1 and 2 resemble a gsl_result argument,
	which is  argument 2 of the C argument list

\end{funcdesc}

\begin{funcdesc}{legendre_H3d_1}{...}\index{legendre_H3d_1}

    Number of Input  Arguments:  2
    Number of Output Arguments:  1
\end{funcdesc}

\begin{funcdesc}{legendre_H3d_1_e}{...}\index{legendre_H3d_1_e}

    Number of Input  Arguments:  2
    Number of Output Arguments:  2

The error flag is discarded.
Return Arguments 1 and 2 resemble a gsl_result argument,
	which is  argument 2 of the C argument list

\end{funcdesc}

\begin{funcdesc}{legendre_H3d_e}{...}\index{legendre_H3d_e}

    Number of Input  Arguments:  3
    Number of Output Arguments:  2

The error flag is discarded.
Return Arguments 1 and 2 resemble a gsl_result argument,
	which is  argument 3 of the C argument list

\end{funcdesc}

\begin{funcdesc}{legendre_P1}{...}\index{legendre_P1}

    Number of Input  Arguments:  1
    Number of Output Arguments:  1
\end{funcdesc}

\begin{funcdesc}{legendre_P1_e}{...}\index{legendre_P1_e}

    Number of Input  Arguments:  1
    Number of Output Arguments:  2

The error flag is discarded.
Return Arguments 1 and 2 resemble a gsl_result argument,
	which is  argument 1 of the C argument list

\end{funcdesc}

\begin{funcdesc}{legendre_P2}{...}\index{legendre_P2}

    Number of Input  Arguments:  1
    Number of Output Arguments:  1
\end{funcdesc}

\begin{funcdesc}{legendre_P2_e}{...}\index{legendre_P2_e}

    Number of Input  Arguments:  1
    Number of Output Arguments:  2

The error flag is discarded.
Return Arguments 1 and 2 resemble a gsl_result argument,
	which is  argument 1 of the C argument list

\end{funcdesc}

\begin{funcdesc}{legendre_P3}{...}\index{legendre_P3}

    Number of Input  Arguments:  1
    Number of Output Arguments:  1
\end{funcdesc}

\begin{funcdesc}{legendre_P3_e}{...}\index{legendre_P3_e}

    Number of Input  Arguments:  1
    Number of Output Arguments:  2

The error flag is discarded.
Return Arguments 1 and 2 resemble a gsl_result argument,
	which is  argument 1 of the C argument list

\end{funcdesc}

\begin{funcdesc}{legendre_Pl}{...}\index{legendre_Pl}

    Number of Input  Arguments:  2
    Number of Output Arguments:  1
\end{funcdesc}

\begin{funcdesc}{legendre_Pl_e}{...}\index{legendre_Pl_e}

    Number of Input  Arguments:  2
    Number of Output Arguments:  2

The error flag is discarded.
Return Arguments 1 and 2 resemble a gsl_result argument,
	which is  argument 2 of the C argument list

\end{funcdesc}

\begin{funcdesc}{legendre_Plm}{...}\index{legendre_Plm}

    Number of Input  Arguments:  3
    Number of Output Arguments:  1
\end{funcdesc}

\begin{funcdesc}{legendre_Plm_e}{...}\index{legendre_Plm_e}

    Number of Input  Arguments:  3
    Number of Output Arguments:  2

The error flag is discarded.
Return Arguments 1 and 2 resemble a gsl_result argument,
	which is  argument 3 of the C argument list

\end{funcdesc}

\begin{funcdesc}{legendre_Q0}{...}\index{legendre_Q0}

    Number of Input  Arguments:  1
    Number of Output Arguments:  1
\end{funcdesc}

\begin{funcdesc}{legendre_Q0_e}{...}\index{legendre_Q0_e}

    Number of Input  Arguments:  1
    Number of Output Arguments:  2

The error flag is discarded.
Return Arguments 1 and 2 resemble a gsl_result argument,
	which is  argument 1 of the C argument list

\end{funcdesc}

\begin{funcdesc}{legendre_Q1}{...}\index{legendre_Q1}

    Number of Input  Arguments:  1
    Number of Output Arguments:  1
\end{funcdesc}

\begin{funcdesc}{legendre_Q1_e}{...}\index{legendre_Q1_e}

    Number of Input  Arguments:  1
    Number of Output Arguments:  2

The error flag is discarded.
Return Arguments 1 and 2 resemble a gsl_result argument,
	which is  argument 1 of the C argument list

\end{funcdesc}

\begin{funcdesc}{legendre_Ql}{...}\index{legendre_Ql}

    Number of Input  Arguments:  2
    Number of Output Arguments:  1
\end{funcdesc}

\begin{funcdesc}{legendre_Ql_e}{...}\index{legendre_Ql_e}

    Number of Input  Arguments:  2
    Number of Output Arguments:  2

The error flag is discarded.
Return Arguments 1 and 2 resemble a gsl_result argument,
	which is  argument 2 of the C argument list

\end{funcdesc}

\begin{funcdesc}{legendre_sphPlm}{...}\index{legendre_sphPlm}

    Number of Input  Arguments:  3
    Number of Output Arguments:  1
\end{funcdesc}

\begin{funcdesc}{legendre_sphPlm_e}{...}\index{legendre_sphPlm_e}

    Number of Input  Arguments:  3
    Number of Output Arguments:  2

The error flag is discarded.
Return Arguments 1 and 2 resemble a gsl_result argument,
	which is  argument 3 of the C argument list

\end{funcdesc}

\begin{funcdesc}{lnbeta}{...}\index{lnbeta}

    Number of Input  Arguments:  2
    Number of Output Arguments:  1
\end{funcdesc}

\begin{funcdesc}{lnbeta_e}{...}\index{lnbeta_e}

    Number of Input  Arguments:  2
    Number of Output Arguments:  2

The error flag is discarded.
Return Arguments 1 and 2 resemble a gsl_result argument,
	which is  argument 2 of the C argument list

\end{funcdesc}

\begin{funcdesc}{lnchoose}{...}\index{lnchoose}

    Number of Input  Arguments:  2
    Number of Output Arguments:  1
\end{funcdesc}

\begin{funcdesc}{lnchoose_e}{...}\index{lnchoose_e}

    Number of Input  Arguments:  2
    Number of Output Arguments:  2

The error flag is discarded.
Return Arguments 1 and 2 resemble a gsl_result argument,
	which is  argument 2 of the C argument list

\end{funcdesc}

\begin{funcdesc}{lncosh}{...}\index{lncosh}

    Number of Input  Arguments:  1
    Number of Output Arguments:  1
\end{funcdesc}

\begin{funcdesc}{lncosh_e}{...}\index{lncosh_e}

    Number of Input  Arguments:  1
    Number of Output Arguments:  2

The error flag is discarded.
Return Arguments 1 and 2 resemble a gsl_result argument,
	which is  argument 1 of the C argument list

\end{funcdesc}

\begin{funcdesc}{lndoublefact}{...}\index{lndoublefact}

    Number of Input  Arguments:  1
    Number of Output Arguments:  1
\end{funcdesc}

\begin{funcdesc}{lndoublefact_e}{...}\index{lndoublefact_e}

    Number of Input  Arguments:  1
    Number of Output Arguments:  2

The error flag is discarded.
Return Arguments 1 and 2 resemble a gsl_result argument,
	which is  argument 1 of the C argument list

\end{funcdesc}

\begin{funcdesc}{lnfact}{...}\index{lnfact}

    Number of Input  Arguments:  1
    Number of Output Arguments:  1
\end{funcdesc}

\begin{funcdesc}{lnfact_e}{...}\index{lnfact_e}

    Number of Input  Arguments:  1
    Number of Output Arguments:  2

The error flag is discarded.
Return Arguments 1 and 2 resemble a gsl_result argument,
	which is  argument 1 of the C argument list

\end{funcdesc}

\begin{funcdesc}{lngamma}{...}\index{lngamma}

    Number of Input  Arguments:  1
    Number of Output Arguments:  1
\end{funcdesc}

\begin{funcdesc}{lngamma_e}{...}\index{lngamma_e}

    Number of Input  Arguments:  1
    Number of Output Arguments:  2

The error flag is discarded.
Return Arguments 1 and 2 resemble a gsl_result argument,
	which is  argument 1 of the C argument list

\end{funcdesc}

\begin{funcdesc}{lngamma_sgn_e}{...}\index{lngamma_sgn_e}

    Number of Input  Arguments:  1
    Number of Output Arguments:  3

The error flag is discarded.
Return Arguments 1 and 2 resemble a gsl_result argument,
	which is  argument 1 of the C argument list

\end{funcdesc}

\begin{funcdesc}{lnpoch}{...}\index{lnpoch}

    Number of Input  Arguments:  2
    Number of Output Arguments:  1
\end{funcdesc}

\begin{funcdesc}{lnpoch_e}{...}\index{lnpoch_e}

    Number of Input  Arguments:  2
    Number of Output Arguments:  2

The error flag is discarded.
Return Arguments 1 and 2 resemble a gsl_result argument,
	which is  argument 2 of the C argument list

\end{funcdesc}

\begin{funcdesc}{lnpoch_sgn_e}{...}\index{lnpoch_sgn_e}

    Number of Input  Arguments:  2
    Number of Output Arguments:  3

The error flag is discarded.
Return Arguments 1 and 2 resemble a gsl_result argument,
	which is  argument 2 of the C argument list

\end{funcdesc}

\begin{funcdesc}{lnsinh}{...}\index{lnsinh}

    Number of Input  Arguments:  1
    Number of Output Arguments:  1
\end{funcdesc}

\begin{funcdesc}{lnsinh_e}{...}\index{lnsinh_e}

    Number of Input  Arguments:  1
    Number of Output Arguments:  2

The error flag is discarded.
Return Arguments 1 and 2 resemble a gsl_result argument,
	which is  argument 1 of the C argument list

\end{funcdesc}

\begin{funcdesc}{log}{...}\index{log}

    Number of Input  Arguments:  1
    Number of Output Arguments:  1
\end{funcdesc}

\begin{funcdesc}{log_1plusx}{...}\index{log_1plusx}

    Number of Input  Arguments:  1
    Number of Output Arguments:  1
\end{funcdesc}

\begin{funcdesc}{log_1plusx_e}{...}\index{log_1plusx_e}

    Number of Input  Arguments:  1
    Number of Output Arguments:  2

The error flag is discarded.
Return Arguments 1 and 2 resemble a gsl_result argument,
	which is  argument 1 of the C argument list

\end{funcdesc}

\begin{funcdesc}{log_1plusx_mx}{...}\index{log_1plusx_mx}

    Number of Input  Arguments:  1
    Number of Output Arguments:  1
\end{funcdesc}

\begin{funcdesc}{log_1plusx_mx_e}{...}\index{log_1plusx_mx_e}

    Number of Input  Arguments:  1
    Number of Output Arguments:  2

The error flag is discarded.
Return Arguments 1 and 2 resemble a gsl_result argument,
	which is  argument 1 of the C argument list

\end{funcdesc}

\begin{funcdesc}{log_abs}{...}\index{log_abs}

    Number of Input  Arguments:  1
    Number of Output Arguments:  1
\end{funcdesc}

\begin{funcdesc}{log_abs_e}{...}\index{log_abs_e}

    Number of Input  Arguments:  1
    Number of Output Arguments:  2

The error flag is discarded.
Return Arguments 1 and 2 resemble a gsl_result argument,
	which is  argument 1 of the C argument list

\end{funcdesc}

\begin{funcdesc}{log_e}{...}\index{log_e}

    Number of Input  Arguments:  1
    Number of Output Arguments:  2

The error flag is discarded.
Return Arguments 1 and 2 resemble a gsl_result argument,
	which is  argument 1 of the C argument list

\end{funcdesc}

\begin{funcdesc}{log_erfc}{...}\index{log_erfc}

    Number of Input  Arguments:  1
    Number of Output Arguments:  1
\end{funcdesc}

\begin{funcdesc}{log_erfc_e}{...}\index{log_erfc_e}

    Number of Input  Arguments:  1
    Number of Output Arguments:  2

The error flag is discarded.
Return Arguments 1 and 2 resemble a gsl_result argument,
	which is  argument 1 of the C argument list

\end{funcdesc}

\begin{funcdesc}{multiply}{...}\index{multiply}

    Number of Input  Arguments:  2
    Number of Output Arguments:  1
\end{funcdesc}

\begin{funcdesc}{multiply_e}{...}\index{multiply_e}

    Number of Input  Arguments:  2
    Number of Output Arguments:  2

The error flag is discarded.
Return Arguments 1 and 2 resemble a gsl_result argument,
	which is  argument 2 of the C argument list

\end{funcdesc}

\begin{funcdesc}{multiply_err_e}{...}\index{multiply_err_e}

    Number of Input  Arguments:  4
    Number of Output Arguments:  2

The error flag is discarded.
Return Arguments 1 and 2 resemble a gsl_result argument,
	which is  argument 4 of the C argument list

\end{funcdesc}

\begin{funcdesc}{poch}{...}\index{poch}

    Number of Input  Arguments:  2
    Number of Output Arguments:  1
\end{funcdesc}

\begin{funcdesc}{poch_e}{...}\index{poch_e}

    Number of Input  Arguments:  2
    Number of Output Arguments:  2

The error flag is discarded.
Return Arguments 1 and 2 resemble a gsl_result argument,
	which is  argument 2 of the C argument list

\end{funcdesc}

\begin{funcdesc}{pochrel}{...}\index{pochrel}

    Number of Input  Arguments:  2
    Number of Output Arguments:  1
\end{funcdesc}

\begin{funcdesc}{pochrel_e}{...}\index{pochrel_e}

    Number of Input  Arguments:  2
    Number of Output Arguments:  2

The error flag is discarded.
Return Arguments 1 and 2 resemble a gsl_result argument,
	which is  argument 2 of the C argument list

\end{funcdesc}

\begin{funcdesc}{polar_to_rect}{...}\index{polar_to_rect}

\end{funcdesc}

\begin{funcdesc}{pow_int}{...}\index{pow_int}

    Number of Input  Arguments:  2
    Number of Output Arguments:  1
\end{funcdesc}

\begin{funcdesc}{pow_int_e}{...}\index{pow_int_e}

    Number of Input  Arguments:  2
    Number of Output Arguments:  2

The error flag is discarded.
Return Arguments 1 and 2 resemble a gsl_result argument,
	which is  argument 2 of the C argument list

\end{funcdesc}

\begin{funcdesc}{psi}{...}\index{psi}

    Number of Input  Arguments:  1
    Number of Output Arguments:  1
\end{funcdesc}

\begin{funcdesc}{psi_1_int}{...}\index{psi_1_int}

    Number of Input  Arguments:  1
    Number of Output Arguments:  1
\end{funcdesc}

\begin{funcdesc}{psi_1_int_e}{...}\index{psi_1_int_e}

    Number of Input  Arguments:  1
    Number of Output Arguments:  2

The error flag is discarded.
Return Arguments 1 and 2 resemble a gsl_result argument,
	which is  argument 1 of the C argument list

\end{funcdesc}

\begin{funcdesc}{psi_1piy}{...}\index{psi_1piy}

    Number of Input  Arguments:  1
    Number of Output Arguments:  1
\end{funcdesc}

\begin{funcdesc}{psi_1piy_e}{...}\index{psi_1piy_e}

    Number of Input  Arguments:  1
    Number of Output Arguments:  2

The error flag is discarded.
Return Arguments 1 and 2 resemble a gsl_result argument,
	which is  argument 1 of the C argument list

\end{funcdesc}

\begin{funcdesc}{psi_e}{...}\index{psi_e}

    Number of Input  Arguments:  1
    Number of Output Arguments:  2

The error flag is discarded.
Return Arguments 1 and 2 resemble a gsl_result argument,
	which is  argument 1 of the C argument list

\end{funcdesc}

\begin{funcdesc}{psi_int}{...}\index{psi_int}

    Number of Input  Arguments:  1
    Number of Output Arguments:  1
\end{funcdesc}

\begin{funcdesc}{psi_int_e}{...}\index{psi_int_e}

    Number of Input  Arguments:  1
    Number of Output Arguments:  2

The error flag is discarded.
Return Arguments 1 and 2 resemble a gsl_result argument,
	which is  argument 1 of the C argument list

\end{funcdesc}

\begin{funcdesc}{psi_n}{...}\index{psi_n}

    Number of Input  Arguments:  2
    Number of Output Arguments:  1
\end{funcdesc}

\begin{funcdesc}{psi_n_e}{...}\index{psi_n_e}

    Number of Input  Arguments:  2
    Number of Output Arguments:  2

The error flag is discarded.
Return Arguments 1 and 2 resemble a gsl_result argument,
	which is  argument 2 of the C argument list

\end{funcdesc}

\begin{funcdesc}{rect_to_polar}{...}\index{rect_to_polar}

\end{funcdesc}

\begin{funcdesc}{sin}{...}\index{sin}

    Number of Input  Arguments:  1
    Number of Output Arguments:  1
\end{funcdesc}

\begin{funcdesc}{sin_e}{...}\index{sin_e}

    Number of Input  Arguments:  1
    Number of Output Arguments:  2

The error flag is discarded.
Return Arguments 1 and 2 resemble a gsl_result argument,
	which is  argument 1 of the C argument list

\end{funcdesc}

\begin{funcdesc}{sin_err_e}{...}\index{sin_err_e}

    Number of Input  Arguments:  2
    Number of Output Arguments:  2

The error flag is discarded.
Return Arguments 1 and 2 resemble a gsl_result argument,
	which is  argument 2 of the C argument list

\end{funcdesc}

\begin{funcdesc}{sinc}{...}\index{sinc}

    Number of Input  Arguments:  1
    Number of Output Arguments:  1
\end{funcdesc}

\begin{funcdesc}{sinc_e}{...}\index{sinc_e}

    Number of Input  Arguments:  1
    Number of Output Arguments:  2

The error flag is discarded.
Return Arguments 1 and 2 resemble a gsl_result argument,
	which is  argument 1 of the C argument list

\end{funcdesc}

\begin{funcdesc}{synchrotron_1}{...}\index{synchrotron_1}

    Number of Input  Arguments:  1
    Number of Output Arguments:  1
\end{funcdesc}

\begin{funcdesc}{synchrotron_1_e}{...}\index{synchrotron_1_e}

    Number of Input  Arguments:  1
    Number of Output Arguments:  2

The error flag is discarded.
Return Arguments 1 and 2 resemble a gsl_result argument,
	which is  argument 1 of the C argument list

\end{funcdesc}

\begin{funcdesc}{synchrotron_2}{...}\index{synchrotron_2}

    Number of Input  Arguments:  1
    Number of Output Arguments:  1
\end{funcdesc}

\begin{funcdesc}{synchrotron_2_e}{...}\index{synchrotron_2_e}

    Number of Input  Arguments:  1
    Number of Output Arguments:  2

The error flag is discarded.
Return Arguments 1 and 2 resemble a gsl_result argument,
	which is  argument 1 of the C argument list

\end{funcdesc}

\begin{funcdesc}{taylorcoeff}{...}\index{taylorcoeff}

    Number of Input  Arguments:  2
    Number of Output Arguments:  1
\end{funcdesc}

\begin{funcdesc}{taylorcoeff_e}{...}\index{taylorcoeff_e}

    Number of Input  Arguments:  2
    Number of Output Arguments:  2

The error flag is discarded.
Return Arguments 1 and 2 resemble a gsl_result argument,
	which is  argument 2 of the C argument list

\end{funcdesc}

\begin{funcdesc}{transport_2}{...}\index{transport_2}

    Number of Input  Arguments:  1
    Number of Output Arguments:  1
\end{funcdesc}

\begin{funcdesc}{transport_2_e}{...}\index{transport_2_e}

    Number of Input  Arguments:  1
    Number of Output Arguments:  2

The error flag is discarded.
Return Arguments 1 and 2 resemble a gsl_result argument,
	which is  argument 1 of the C argument list

\end{funcdesc}

\begin{funcdesc}{transport_3}{...}\index{transport_3}

    Number of Input  Arguments:  1
    Number of Output Arguments:  1
\end{funcdesc}

\begin{funcdesc}{transport_3_e}{...}\index{transport_3_e}

    Number of Input  Arguments:  1
    Number of Output Arguments:  2

The error flag is discarded.
Return Arguments 1 and 2 resemble a gsl_result argument,
	which is  argument 1 of the C argument list

\end{funcdesc}

\begin{funcdesc}{transport_4}{...}\index{transport_4}

    Number of Input  Arguments:  1
    Number of Output Arguments:  1
\end{funcdesc}

\begin{funcdesc}{transport_4_e}{...}\index{transport_4_e}

    Number of Input  Arguments:  1
    Number of Output Arguments:  2

The error flag is discarded.
Return Arguments 1 and 2 resemble a gsl_result argument,
	which is  argument 1 of the C argument list

\end{funcdesc}

\begin{funcdesc}{transport_5}{...}\index{transport_5}

    Number of Input  Arguments:  1
    Number of Output Arguments:  1
\end{funcdesc}

\begin{funcdesc}{transport_5_e}{...}\index{transport_5_e}

    Number of Input  Arguments:  1
    Number of Output Arguments:  2

The error flag is discarded.
Return Arguments 1 and 2 resemble a gsl_result argument,
	which is  argument 1 of the C argument list

\end{funcdesc}

\begin{funcdesc}{zeta}{...}\index{zeta}

    Number of Input  Arguments:  1
    Number of Output Arguments:  1
\end{funcdesc}

\begin{funcdesc}{zeta_e}{...}\index{zeta_e}

    Number of Input  Arguments:  1
    Number of Output Arguments:  2

The error flag is discarded.
Return Arguments 1 and 2 resemble a gsl_result argument,
	which is  argument 1 of the C argument list

\end{funcdesc}

\begin{funcdesc}{zeta_int}{...}\index{zeta_int}

    Number of Input  Arguments:  1
    Number of Output Arguments:  1
\end{funcdesc}

\begin{funcdesc}{zeta_int_e}{...}\index{zeta_int_e}

    Number of Input  Arguments:  1
    Number of Output Arguments:  2

The error flag is discarded.
Return Arguments 1 and 2 resemble a gsl_result argument,
	which is  argument 1 of the C argument list

\end{funcdesc}

\section{Ordinary Functions}

The following array functions have been wrapped. These are supposingly faster
than the equivalent functions from above.
\begin{funcdesc}{bessel_In_array}{...}\index{bessel_In_array}
\end{funcdesc}
\begin{funcdesc}{bessel_Jn_array}{...}\index{bessel_Jn_array}
\end{funcdesc}
\begin{funcdesc}{bessel_Kn_array}{...}\index{bessel_Kn_array}
\end{funcdesc}
\begin{funcdesc}{bessel_Kn_scaled_array}{...}\index{bessel_Kn_scaled_array}
\end{funcdesc}
\begin{funcdesc}{bessel_Yn_array}{...}\index{bessel_Yn_array}
\end{funcdesc}
\begin{funcdesc}{bessel_il_scaled_array}{...}\index{bessel_il_scaled_array}
\end{funcdesc}
\begin{funcdesc}{bessel_jl_array}{...}\index{bessel_jl_array}
\end{funcdesc}
\begin{funcdesc}{bessel_jl_steed_array}{...}\index{bessel_jl_steed_array}
\end{funcdesc}
\begin{funcdesc}{bessel_kl_scaled_array}{...}\index{bessel_kl_scaled_array}
\end{funcdesc}
\begin{funcdesc}{bessel_yl_array}{...}\index{bessel_yl_array}
\end{funcdesc}


%%% Local Variables: 
%%% mode: latex
%%% TeX-master: "ref"
%%% End: 


\appendix

\chapter[\protect\module{pygsl.ieee} --- Floating Point Unit Support]
{\protect\module{pygsl.ieee} \\ Floating Point Unit Support}
\label{cha:ieee-module}
\declaremodule{extension}{pygsl.ieee}
\moduleauthor{Achim G\"adke}{achimgaedke@users.sourceforge.net}

This chapter lists features to configure the ``Floating Point Unit'' of your machine.


\chapter[\protect\module{pygsl.init} --- Library initialisation]
{\protect\module{pygsl.init} \\ Library initialisation}
\label{cha:library-initialisation}
\declaremodule{extension}{pygsl.init}
\moduleauthor{Pierre Schnizer}{schnizer@users.sourceforge.net}
\moduleauthor{Achim G\"adke}{achimgaedke@users.sourceforge.net}

This module is called the first time when loading \module{pygsl}.
All following procedures are called once and before everything other.

\section{Exception handling}
\index{exception handling!initialisation} GSL provides a selectable error
handler, that is called for occuring errors (like domain errors, division by
zero, etc. ).  \module{pygsl.init} installs a handler by calling
\cfunction{gsl_set_error_handler} to set an appropiate exception from
\module{pygsl.errors}.  A \module{pygsl} interface function should return
\code{NULL} in case of an error, so the exception is raised.  If this handler
is called more than once before returning to python, only the first set
exception is raised.

Here is a python level example:
\begin{verbatim}
import pygsl.histogram
import pygsl.errors
hist=pygsl.histogram.histogram2d(100,100)
try:
   hist[-1,-1]=0
except pygsl.errors.gsl_Error,err:
   print err
\end{verbatim}
Will result
\begin{verbatim}
input domain error: index i lies outside valid range of 0 .. nx - 1
\end{verbatim}

\section{IEEE-mode}
\index{ieee-mode!initialisation}
The IEEE mode is set from the environment variable
 \envvar{GSL_IEEE_MODE} via \cfunction{gsl_ieee_env_setup()}.
After the initialisation use \module{pygsl.ieee} for manipulation.

\section{random number generators}
\index{random number generator!initialisation}
Also the random number generator can be initialised from the environment variables
 \envvar{GSL_RNG_TYPE}
and \envvar{GSL_RNG_SEED} using the gsl function \cfunction{gsl_rng_env_setup()}.
Each random number generators are initialised with \envvar{GSL_RNG_SEED}.

The default generator can be created by:\nopagebreak
\begin{verbatim}
import pygsl.rng
my_rng=pygsl.rng.rng()
print my_rng.name()
\end{verbatim}


\chapter[\protect\module{pygsl.errors} --- Error and warning classes]
{\protect\module{pygsl.errors} \\ Error and warning classes} 
\label{cha:error-module}
\declaremodule{standard}{pygsl.errors}
\moduleauthor{Pierre Schnizer}{schnizer@users.sourceforge.net}
\moduleauthor{Original Author: Achim G\"adke}{achimgaedke@users.sourceforge.net}

This chapter provides information about the \exception{gsl_Error} exception class that comes with this module.

\section{Exception Classes}


\begin{excclassdesc} {gsl_Error}{}
derived from \exception{Exception}, can be constructed with any object as parameter.
It is baseclass to all other \gsl{} Exceptions
\end{excclassdesc}
These classes are translations of the \file{<gsl/gsl_errno.h>} to python
exceptions.


\begin{excclassdesc}{gsl_ArithmeticError}{}
derived from \exception{gsl_Error} and \exception{exceptions.ArithmeticError},
base of all common arithmetic exceptions
\end{excclassdesc}

\begin{excclassdesc}{gsl_OverflowError}{}
derived from \exception{gsl_Error} and \exception{exceptions.OverflowError}
\end{excclassdesc}

\begin{excclassdesc}{gsl_ZeroDivisionError}{}
derived from \exception{gsl_Error} and \exception{exceptions.ZeroDivisionError}
\end{excclassdesc}

\begin{excclassdesc}{gsl_FloatingPointError}{}
derived from \exception{gsl_Error} and \exception{exceptions.FloatingPointError}
\end{excclassdesc}

\begin{excclassdesc}{gsl_ArithmeticError}{}
is derived from  \exception{gsl_Error} and from  \exception{ArithmeticError} .
This exception is the    base of all common arithmetic exceptions.
\end{excclassdesc}

\begin{excclassdesc}{gsl_AccuracyLossError}{}
is derived from  \exception{gsl_ArithmeticError} .
This exception is raised if the failed to reach the specified tolerance.
\end{excclassdesc}
\begin{excclassdesc}{gsl_BadFuncError}{}
is derived from  \exception{gsl_Error} .
This exception is raised if problem with a user-supplied function occur.
\end{excclassdesc}
\begin{excclassdesc}{gsl_BadLength}{}
is derived from  \exception{gsl_Error} .
This exception is raised if  matrix or  vector lengths are not conformant.
\end{excclassdesc}
\begin{excclassdesc}{gsl_BadToleranceError}{}
is derived from  \exception{gsl_Error} .
This exception is raised if user specified an tolerance which can not be reached.
\end{excclassdesc}
\begin{excclassdesc}{gsl_CacheLimitError}{}
is derived from  \exception{gsl_Error} .
This exception is raised if the    cache limit is exceeded.
\end{excclassdesc}
\begin{excclassdesc}{gsl_DivergeError}{}
is derived from  \exception{gsl_ArithmeticError} .
This exception is raised if an   integral or series is divergent.
\end{excclassdesc}
\begin{excclassdesc}{gsl_DomainError}{}
is derived from  \exception{gsl_Error} .
This exception is raised if    domain errors occure. e.g. sqrt(-1).
\end{excclassdesc}
\begin{excclassdesc}{gsl_EOFError}{}
is derived from  \exception{gsl_Error} and from  \exception{EOFError} .
This exception is raised if 
    end of file
     .
\end{excclassdesc}
\begin{excclassdesc}{gsl_FactorizationError}{}
is derived from  \exception{gsl_Error} .
This exception is raised if     factorization failed.
\end{excclassdesc}
\begin{excclassdesc}{gsl_FloatingPointError}{}
is derived from  \exception{gsl_Error} and from  \exception{FloatingPointError} .
\end{excclassdesc}
\begin{excclassdesc}{gsl_GenericError}{}
is derived from  \exception{gsl_Error} .
\end{excclassdesc}
\begin{excclassdesc}{gsl_InvalidArgumentError}{}
is derived from  \exception{gsl_Error} .
This exception is raised if an invalid argument is supplied by the user.
\end{excclassdesc}
\begin{excclassdesc}{gsl_JacobianEvaluationError}{}
is derived from  \exception{gsl_ArithmeticError} .
This exception is raised if jacobian evaluations are not improving the solution.
\end{excclassdesc}
\begin{excclassdesc}{gsl_MatrixNotSquare}{}
is derived from  \exception{gsl_Error} .
This exception is raised if the given matrix is not square.
\end{excclassdesc}
\begin{excclassdesc}{gsl_MaximumIterationError}{}
is derived from  \exception{gsl_ArithmeticError} .
This exception is raised if    the maximum number  of iterations is exceeded.
\end{excclassdesc}
\begin{excclassdesc}{gsl_NoHardwareSupportError}{}
is derived from  \exception{gsl_Error} .
This exception is raised if the requested feature is not supported by the hardware.
\end{excclassdesc}
\begin{excclassdesc}{gsl_NoProgressError}{}
is derived from  \exception{gsl_ArithmeticError} .
This exception is raised if the  iteration is not making progress towards solution.
\end{excclassdesc}
\begin{excclassdesc}{gsl_NotImplementedError}{}
is derived from  \exception{gsl_Error} and from  \exception{NotImplementedError} .
This exception is raised if  a requested feature is not (yet) implemented .
\end{excclassdesc}
\begin{excclassdesc}{gsl_OverflowError}{}
is derived from  \exception{gsl_Error} and from  \exception{OverflowError} .
\end{excclassdesc}
\begin{excclassdesc}{gsl_PointerError}{}
is derived from  \exception{gsl_Error} .
This exception is raised if an invalid pointer is found by the C wrapper code
or by the GSL library.
\end{excclassdesc}
\begin{excclassdesc}{gsl_RangeError}{}
is derived from  \exception{gsl_ArithmeticError} .
This exception is raised if     output would be out or range, e.g. exp(1e100)
     .
\end{excclassdesc}
\begin{excclassdesc}{gsl_RoundOffError}{}
is derived from  \exception{gsl_ArithmeticError} .
This exception is raised if  arithmetic failed because of roundoff error.
\end{excclassdesc}
\begin{excclassdesc}{gsl_RunAwayError}{}
is derived from  \exception{gsl_ArithmeticError} .
This exception is raised if   iterative process is out of control.
\end{excclassdesc}
\begin{excclassdesc}{gsl_SanityCheckError}{}
is derived from  \exception{gsl_Error} .
This exception is raised if a sanity check failed - shouldn't happen.
\end{excclassdesc}
\begin{excclassdesc}{gsl_SingularityError}{}
is derived from  \exception{gsl_ArithmeticError} .
This exception is raised if  an   apparent singularity is detected.
\end{excclassdesc}
\begin{excclassdesc}{gsl_TableLimitError}{}
is derived from  \exception{gsl_Error} .
This exception is raised if the table limit is exceeded.
\end{excclassdesc}
\begin{excclassdesc}{gsl_ToleranceError}{}
is derived from  \exception{gsl_ArithmeticError} .
This exception is raised if  the alghorithm failed to reach the specified tolerance.
\end{excclassdesc}
\begin{excclassdesc}{gsl_ToleranceFError}{}
is derived from  \exception{gsl_ArithmeticError} .
This exception is raised if  the alghorithm cannot reach the specified
tolerance in F (typically the variation of the evaluated function).
\end{excclassdesc}
\begin{excclassdesc}{gsl_ToleranceGradientError}{}
is derived from  \exception{gsl_ArithmeticError} .
This exception is raised if  cannot reach the specified tolerance for the gradient.
\end{excclassdesc}
\begin{excclassdesc}{gsl_ToleranceXError}{}
is derived from  \exception{gsl_ArithmeticError} .
This exception is raised if cannot reach the specified tolerance in X
(typically a search result).
\end{excclassdesc}
\begin{excclassdesc}{gsl_UnderflowError}{}
is derived from  \exception{gsl_Error} and from  \exception{OverflowError} .
\end{excclassdesc}
\begin{excclassdesc}{gsl_ZeroDivisionError}{}
is derived from  \exception{gsl_Error} and from  \exception{ZeroDivisionError} .
\end{excclassdesc}

All the above errors are just translations of the errno to python exceptions.

The following two are specific to pygsl:
\begin{excclassdesc}{pygsl.errors.pygsl_NotImplementedError}{}
is derived from  \exception{gsl_Error} and from  \exception{NotImplementedError} .
This exception is raised if a feature is requested but not
implemented. Currently only used if a module requests the debugging enviroment
of the init module, but the init module was not compiled with \code{\#define DEBUG=1}
\end{excclassdesc}
\begin{excclassdesc}{pygsl.errors.pygsl_StrideError}{}
is derived from  \exception{gsl_SanityCheckError} .
GSL uses as strides multiples of the basis type; for a vector or doubles, one
means from one double to the next. Numpy or numarray count the stride in
multiples of the size of a char. Therefore the stride has to be recalculated
before the approbriate \gsl{} function can be called. If that fails this
exception is raised.
\end{excclassdesc}

\section{Warning Classes}

\begin{excclassdesc} {gsl_Warning}{}
The dedicated warning class for \gsl{} has \exception{Warning} as base class.
\end{excclassdesc}

\begin{excclassdesc}{gsl_DomainWarning}{}
derived from \exception{gsl_Warning}, used by some \module{pygsl.histogram} functions
\end{excclassdesc}


\chapter{GNU Free Documentation License}
\label{cha:free-documentation-license}

Version 1.1, March 2000\\

 Copyright \copyright\ 2000  Free Software Foundation, Inc.\\
     59 Temple Place, Suite 330, Boston, MA  02111-1307  USA\\
 Everyone is permitted to copy and distribute verbatim copies
 of this license document, but changing it is not allowed.

\section*{Preamble}

The purpose of this License is to make a manual, textbook, or other
written document ``free'' in the sense of freedom: to assure everyone
the effective freedom to copy and redistribute it, with or without
modifying it, either commercially or noncommercially.  Secondarily,
this License preserves for the author and publisher a way to get
credit for their work, while not being considered responsible for
modifications made by others.

This License is a kind of ``copyleft'', which means that derivative
works of the document must themselves be free in the same sense.  It
complements the GNU General Public License, which is a copyleft
license designed for free software.

We have designed this License in order to use it for manuals for free
software, because free software needs free documentation: a free
program should come with manuals providing the same freedoms that the
software does.  But this License is not limited to software manuals;
it can be used for any textual work, regardless of subject matter or
whether it is published as a printed book.  We recommend this License
principally for works whose purpose is instruction or reference.

\section{Applicability and Definitions}

This License applies to any manual or other work that contains a
notice placed by the copyright holder saying it can be distributed
under the terms of this License.  The ``Document'', below, refers to any
such manual or work.  Any member of the public is a licensee, and is
addressed as ``you''.

A ``Modified Version'' of the Document means any work containing the
Document or a portion of it, either copied verbatim, or with
modifications and/or translated into another language.

A ``Secondary Section'' is a named appendix or a front-matter section of
the Document that deals exclusively with the relationship of the
publishers or authors of the Document to the Document's overall subject
(or to related matters) and contains nothing that could fall directly
within that overall subject.  (For example, if the Document is in part a
textbook of mathematics, a Secondary Section may not explain any
mathematics.)  The relationship could be a matter of historical
connection with the subject or with related matters, or of legal,
commercial, philosophical, ethical or political position regarding
them.

The ``Invariant Sections'' are certain Secondary Sections whose titles
are designated, as being those of Invariant Sections, in the notice
that says that the Document is released under this License.

The ``Cover Texts'' are certain short passages of text that are listed,
as Front-Cover Texts or Back-Cover Texts, in the notice that says that
the Document is released under this License.

A ``Transparent'' copy of the Document means a machine-readable copy,
represented in a format whose specification is available to the
general public, whose contents can be viewed and edited directly and
straightforwardly with generic text editors or (for images composed of
pixels) generic paint programs or (for drawings) some widely available
drawing editor, and that is suitable for input to text formatters or
for automatic translation to a variety of formats suitable for input
to text formatters.  A copy made in an otherwise Transparent file
format whose markup has been designed to thwart or discourage
subsequent modification by readers is not Transparent.  A copy that is
not ``Transparent'' is called ``Opaque''.

Examples of suitable formats for Transparent copies include plain
ASCII without markup, Texinfo input format, \LaTeX~input format, SGML
or XML using a publicly available DTD, and standard-conforming simple
HTML designed for human modification.  Opaque formats include
PostScript, PDF, proprietary formats that can be read and edited only
by proprietary word processors, SGML or XML for which the DTD and/or
processing tools are not generally available, and the
machine-generated HTML produced by some word processors for output
purposes only.

The ``Title Page'' means, for a printed book, the title page itself,
plus such following pages as are needed to hold, legibly, the material
this License requires to appear in the title page.  For works in
formats which do not have any title page as such, ``Title Page'' means
the text near the most prominent appearance of the work's title,
preceding the beginning of the body of the text.


\section{Verbatim Copying}

You may copy and distribute the Document in any medium, either
commercially or noncommercially, provided that this License, the
copyright notices, and the license notice saying this License applies
to the Document are reproduced in all copies, and that you add no other
conditions whatsoever to those of this License.  You may not use
technical measures to obstruct or control the reading or further
copying of the copies you make or distribute.  However, you may accept
compensation in exchange for copies.  If you distribute a large enough
number of copies you must also follow the conditions in section 3.

You may also lend copies, under the same conditions stated above, and
you may publicly display copies.


\section{Copying in Quantity}

If you publish printed copies of the Document numbering more than 100,
and the Document's license notice requires Cover Texts, you must enclose
the copies in covers that carry, clearly and legibly, all these Cover
Texts: Front-Cover Texts on the front cover, and Back-Cover Texts on
the back cover.  Both covers must also clearly and legibly identify
you as the publisher of these copies.  The front cover must present
the full title with all words of the title equally prominent and
visible.  You may add other material on the covers in addition.
Copying with changes limited to the covers, as long as they preserve
the title of the Document and satisfy these conditions, can be treated
as verbatim copying in other respects.

If the required texts for either cover are too voluminous to fit
legibly, you should put the first ones listed (as many as fit
reasonably) on the actual cover, and continue the rest onto adjacent
pages.

If you publish or distribute Opaque copies of the Document numbering
more than 100, you must either include a machine-readable Transparent
copy along with each Opaque copy, or state in or with each Opaque copy
a publicly-accessible computer-network location containing a complete
Transparent copy of the Document, free of added material, which the
general network-using public has access to download anonymously at no
charge using public-standard network protocols.  If you use the latter
option, you must take reasonably prudent steps, when you begin
distribution of Opaque copies in quantity, to ensure that this
Transparent copy will remain thus accessible at the stated location
until at least one year after the last time you distribute an Opaque
copy (directly or through your agents or retailers) of that edition to
the public.

It is requested, but not required, that you contact the authors of the
Document well before redistributing any large number of copies, to give
them a chance to provide you with an updated version of the Document.


\section{Modifications}

You may copy and distribute a Modified Version of the Document under
the conditions of sections 2 and 3 above, provided that you release
the Modified Version under precisely this License, with the Modified
Version filling the role of the Document, thus licensing distribution
and modification of the Modified Version to whoever possesses a copy
of it.  In addition, you must do these things in the Modified Version:

\begin{itemize}

\item Use in the Title Page (and on the covers, if any) a title distinct
   from that of the Document, and from those of previous versions
   (which should, if there were any, be listed in the History section
   of the Document).  You may use the same title as a previous version
   if the original publisher of that version gives permission.
\item List on the Title Page, as authors, one or more persons or entities
   responsible for authorship of the modifications in the Modified
   Version, together with at least five of the principal authors of the
   Document (all of its principal authors, if it has less than five).
\item State on the Title page the name of the publisher of the
   Modified Version, as the publisher.
\item Preserve all the copyright notices of the Document.
\item Add an appropriate copyright notice for your modifications
   adjacent to the other copyright notices.
\item Include, immediately after the copyright notices, a license notice
   giving the public permission to use the Modified Version under the
   terms of this License, in the form shown in the Addendum below.
\item Preserve in that license notice the full lists of Invariant Sections
   and required Cover Texts given in the Document's license notice.
\item Include an unaltered copy of this License.
\item Preserve the section entitled ``History'', and its title, and add to
   it an item stating at least the title, year, new authors, and
   publisher of the Modified Version as given on the Title Page.  If
   there is no section entitled ``History'' in the Document, create one
   stating the title, year, authors, and publisher of the Document as
   given on its Title Page, then add an item describing the Modified
   Version as stated in the previous sentence.
\item Preserve the network location, if any, given in the Document for
   public access to a Transparent copy of the Document, and likewise
   the network locations given in the Document for previous versions
   it was based on.  These may be placed in the ``History'' section.
   You may omit a network location for a work that was published at
   least four years before the Document itself, or if the original
   publisher of the version it refers to gives permission.
\item In any section entitled ``Acknowledgements'' or ``Dedications'',
   preserve the section's title, and preserve in the section all the
   substance and tone of each of the contributor acknowledgements
   and/or dedications given therein.
\item Preserve all the Invariant Sections of the Document,
   unaltered in their text and in their titles.  Section numbers
   or the equivalent are not considered part of the section titles.
\item Delete any section entitled ``Endorsements''.  Such a section
   may not be included in the Modified Version.
\item Do not retitle any existing section as ``Endorsements''
   or to conflict in title with any Invariant Section.

\end{itemize}

If the Modified Version includes new front-matter sections or
appendices that qualify as Secondary Sections and contain no material
copied from the Document, you may at your option designate some or all
of these sections as invariant.  To do this, add their titles to the
list of Invariant Sections in the Modified Version's license notice.
These titles must be distinct from any other section titles.

You may add a section entitled ``Endorsements'', provided it contains
nothing but endorsements of your Modified Version by various
parties -- for example, statements of peer review or that the text has
been approved by an organization as the authoritative definition of a
standard.

You may add a passage of up to five words as a Front-Cover Text, and a
passage of up to 25 words as a Back-Cover Text, to the end of the list
of Cover Texts in the Modified Version.  Only one passage of
Front-Cover Text and one of Back-Cover Text may be added by (or
through arrangements made by) any one entity.  If the Document already
includes a cover text for the same cover, previously added by you or
by arrangement made by the same entity you are acting on behalf of,
you may not add another; but you may replace the old one, on explicit
permission from the previous publisher that added the old one.

The author(s) and publisher(s) of the Document do not by this License
give permission to use their names for publicity for or to assert or
imply endorsement of any Modified Version.


\section{Combining Documents}

You may combine the Document with other documents released under this
License, under the terms defined in section 4 above for modified
versions, provided that you include in the combination all of the
Invariant Sections of all of the original documents, unmodified, and
list them all as Invariant Sections of your combined work in its
license notice.

The combined work need only contain one copy of this License, and
multiple identical Invariant Sections may be replaced with a single
copy.  If there are multiple Invariant Sections with the same name but
different contents, make the title of each such section unique by
adding at the end of it, in parentheses, the name of the original
author or publisher of that section if known, or else a unique number.
Make the same adjustment to the section titles in the list of
Invariant Sections in the license notice of the combined work.

In the combination, you must combine any sections entitled ``History''
in the various original documents, forming one section entitled
``History''; likewise combine any sections entitled ``Acknowledgements'',
and any sections entitled ``Dedications''.  You must delete all sections
entitled ``Endorsements.''


\section{Collections of Documents}

You may make a collection consisting of the Document and other documents
released under this License, and replace the individual copies of this
License in the various documents with a single copy that is included in
the collection, provided that you follow the rules of this License for
verbatim copying of each of the documents in all other respects.

You may extract a single document from such a collection, and distribute
it individually under this License, provided you insert a copy of this
License into the extracted document, and follow this License in all
other respects regarding verbatim copying of that document.



\section{Aggregation With Independent Works}

A compilation of the Document or its derivatives with other separate
and independent documents or works, in or on a volume of a storage or
distribution medium, does not as a whole count as a Modified Version
of the Document, provided no compilation copyright is claimed for the
compilation.  Such a compilation is called an ``aggregate'', and this
License does not apply to the other self-contained works thus compiled
with the Document, on account of their being thus compiled, if they
are not themselves derivative works of the Document.

If the Cover Text requirement of section 3 is applicable to these
copies of the Document, then if the Document is less than one quarter
of the entire aggregate, the Document's Cover Texts may be placed on
covers that surround only the Document within the aggregate.
Otherwise they must appear on covers around the whole aggregate.


\section{Translation}

Translation is considered a kind of modification, so you may
distribute translations of the Document under the terms of section 4.
Replacing Invariant Sections with translations requires special
permission from their copyright holders, but you may include
translations of some or all Invariant Sections in addition to the
original versions of these Invariant Sections.  You may include a
translation of this License provided that you also include the
original English version of this License.  In case of a disagreement
between the translation and the original English version of this
License, the original English version will prevail.


\section{Termination}

You may not copy, modify, sublicense, or distribute the Document except
as expressly provided for under this License.  Any other attempt to
copy, modify, sublicense or distribute the Document is void, and will
automatically terminate your rights under this License.  However,
parties who have received copies, or rights, from you under this
License will not have their licenses terminated so long as such
parties remain in full compliance.


\section{Future Revisions of This License}

The Free Software Foundation may publish new, revised versions
of the GNU Free Documentation License from time to time.  Such new
versions will be similar in spirit to the present version, but may
differ in detail to address new problems or concerns. See
http://www.gnu.org/copyleft/.

Each version of the License is given a distinguishing version number.
If the Document specifies that a particular numbered version of this
License "or any later version" applies to it, you have the option of
following the terms and conditions either of that specified version or
of any later version that has been published (not as a draft) by the
Free Software Foundation.  If the Document does not specify a version
number of this License, you may choose any version ever published (not
as a draft) by the Free Software Foundation.

% Complete documentation on the extended LaTeX markup used for Python
% documentation is available in ``Documenting Python'', which is part
% of the standard documentation for Python.  It may be found online
% at:
%
%     http://www.python.org/doc/current/doc/doc.html

\documentclass[hyperref]{manual}

% latex2html doesn't know [T1]{fontenc}, so we cannot use that:(
\usepackage{amsmath}
\usepackage[latin1]{inputenc}
\usepackage{textcomp}


% this version does not reset module names at section level
%begin{latexonly}
\makeatletter
\let\py@OldOldChapter=\chapter
\renewcommand{\chapter}{\py@reset%
                        \py@OldOldChapter}
\renewcommand{\section}{\@startsection{section}{1}{\z@}%
   {-3.5ex \@plus -1ex \@minus -.2ex}%
   {2.3ex \@plus.2ex}%
   {\reset@font\Large\py@HeaderFamily}}
\makeatother
%end{latexonly}


% some convenience declarations
\newcommand{\gsl}{GSL}
\newcommand{\GSL}{GNU Scientific Library}
\newcommand{\numpy}{NumPy}
\newcommand{\NUMPY}{Numerical Python}
\newcommand{\pygsl}{PyGSL}
\newcommand{\PYGSL}{PyGSL: Python wrapper of the GNU Scientific Library}


\title{PyGSL Reference Manual}

\ifhtml
\author{%
   \ulink{Achim G\"adke}{mailto:achimgaedke@users.sourceforge.net}\\
   Center for Applied Informatics, Cologne \\
   \ulink{Jochen K\"upper}{mailto:jochen@jochen-kuepper.de}\\
   Fritz-Haber-Institut der MPG, Berlin
   \ulink{Sebastien Maret}{mailto:schnizer@users.sourceforge.net}\\
   Gesellschaft f�r Schwerionenforschung Darmstadt.
   \ulink{Pierre Schnizer}{mailto:schnizer@users.sourceforge.net}\\
   Gesellschaft f�r Schwerionenforschung, Darmstadt.
}%
\else
%begin{latexonly}
%% pdfelatex (TeXLive 7) doesn't handle \footnotemark in here...
\author{Achim G\"adke \\ Jochen K\"upper \\ Sebastien Maret \\ Pierre Schnizer}
% Please at least include a long-lived email address!
\authoraddress{
   Center for Applied Informatics, Cologne \\
   \email{achimgaedke@users.sourceforge.net} \\[2mm]
   Fritz-Haber-Institut der MPG, Berlin \\
   \email{jochen@jochen-kuepper.de} \\
      Gesellschaft f�r Schwerionenforschung, Darmstadt\\
   \email{schnizer@users.sourceforge.net}\\
}
%end{latexonly}
\fi

\date{January, 2005}            % update before release!
                                % Use an explicit date so that reformatting
                                % doesn't cause a new date to be used.  Setting
                                % the date to \today can be used during draft
                                % stages to make it easier to handle versions.
\release{0.2}                   % release version; this is used to define the
\setshortversion{0.2}           % \version macro
\makeindex                      % tell \index to actually write the .idx file


\begin{document}

\maketitle

% This makes the contents more accessible from the front page of the HTML.
\ifhtml
\chapter*{Front Matter}
\label{front}
\fi

Copyright \copyright{} 2002 The pygsl Team.

Permission is granted to copy, distribute and/or modify this document under the
terms of the GNU Free Documentation License, Version 1.1 or any later version
published by the Free Software Foundation; with no Invariant Sections, no
Front-Cover Texts, and no Back-Cover Texts.  A copy of the license is included
in section \ref{cha:free-documentation-license} entitled ``GNU Free
Documentation License''.


%% Local Variables:
%% mode: LaTeX
%% mode: auto-fill
%% fill-column: 79
%% indent-tabs-mode: nil
%% ispell-dictionary: "american"
%% reftex-fref-is-default: nil
%% TeX-auto-save: t
%% TeX-command-default: "pdfeLaTeX"
%% TeX-master: "pygsl"
%% TeX-parse-self: t
%% End:


\begin{abstract}
   \noindent
   pygsl grants python users access to the GNU scientific library.  The latest
   version can be found at the project homepage, \url{http://pygsl.sf.net}.

   \textbf{Implemented features:} \\
   \begin{tabular}{ll}
     \module{pygsl.blas}                & basic linear algebra system\\
     \module{pygsl.chebyshev}           & chebyshev approximations\\
     \module{pygsl.combination}         & combinations  \\
     \module{pygsl.const}               & $>200$ often used mathematical and
                                          scientific constants. \\
     \module{pygsl.diff}                & (Deprecated. Use pygsl.deriv instead). \\
     \module{pygsl.deriv}               & Numerical differentiation. \\
     \module{pygsl.eigen}               &\\
     \module{pygsl.fit}                 &\\
     \module{pygsl.histogram}          & 1d and 2d histograms and operations
                                          on histograms. \\
     \module{pygsl.ieee}                & Access to the ieee-arithmetics layer
                                          of gsl. \\ 
     \module{pygsl.integrate}           &\\
     \module{pygsl.interpolation}       &\\ 
     \module{pygsl.linalg}              &\\
     \module{pygsl.math}                &\\
     \module{pygsl.monte}               &\\
     \module{pygsl.minimize}            &\\
     \module{pygsl.multifit}            &\\
     \module{pygsl.multifit_nlin}       &\\    
     \module{pygsl.multimin}            &\\
     \module{pygsl.multiroots}          &\\ 
     \module{pygsl.odeiv}               &\\
     \module{pygsl.permutation}         &\\  
     \module{pygsl.poly}                &\\
     \module{pygsl.qrng}                &\\
     \module{pygsl.rng}                 & random number generators and probability densities. \\
     \module{pygsl.roots}               &\\
     \module{pygsl.siman}               &Simulated anealing\\
     \module{pygsl.sf}                  & $>200$ special functions. \\
     \module{pygsl.statistics}          & Statistical functions. \\
\end{tabular}

\end{abstract}


\tableofcontents


\chapter{System Requirements, Installation}
\label{cha:system-req-installation}
\section{Status}

\paragraph*{Status of GSL-Library}
The gsl-library is since version 1.0 stable and for general use.
More information about it at \url{http://www.gnu.org/software/gsl/}.

\paragraph*{Status of this interface}
Nearly all modules are wrapped. A lot of tests are
covering various functionality. Please report to the mailing list
\url{pygsl-discuss@lists.sourceforge.net} if you find a bug.

The hankel modules have been
wrapped. Please write to the mailing list
\url{pygsl-discuss@lists.sourceforge.net} 
if you require one of the modules
and are willing to help with a simple example. 
If any other function is missing or some other module (e.g. ntuple) or
function, do not hesitate to write to the list.

\paragraph*{Retriving the Interface}
You can download it here: \url{http://sourceforge.net/projects/pygsl}

\section{Requirements}

To build the interface, you will need
\begin{itemize}
\item \ulink{gsl-1.x}{http://sources.redhat.com/gsl},
\item \ulink{python2.6}{http://www.python.org} or better,
\item \ulink{NumPy}{http://numpy.sf.net}, and
\item a c compiler (like \ulink{gcc}{http://gcc.gnu.org}).
\end{itemize}

Supported Platforms are:
\begin{itemize}
\item Linux (Redhat/Debian/SuSE) with python2.* and gsl-1.*
\item Win32
\end{itemize}
It was tested and is tested on an irregular basis on the following platforms
\begin{itemize}
\item SUN
\item Cygwin
\item MacOS X
\end{itemize}
but is supposed to build on any POSIX platforms.

\section{Installing the pygsl interface}

\program{gsl-config} must be on your path:\nopagebreak
\begin{verbatim}
# unpack the source distribution
gzip -d -c pygsl-x.y.z.tar.gz|tar xvf-
cd pygsl-x.y.z
# do this with your prefered python version
# to set the gsl location explicitly use setup.py --gsl-prefix=/path/to/gsl
python setup.py build
# change to an user id, that is allowed to do installation
python setup.py install
\end{verbatim}
Ready....

{\bf Do not test the interface in the distribution root or in the directories
 \file{src} or \file{pygsl}.}

If you find unresolved symbols later on, delete the C source in the
swig_src files. Check that swig can be called from the command line. 
Then start the build process again. 

In this case swig will rebuild the C files. The swig_src files
distributed with pygsl are to an up to date version of GSL (1.16 as of
this writing). Swig parses partly some header header files and builds
the appropriate interface functions. If you have an older GSL version 
locally installed, the sources in the swig_src directory can contain 
links to symbols which are not defined by the locally installed GSL
version.

\subsection{Building on win32}

Windows by default does not allow to run a posix shell. Here a different path
is required. First change into the directory \file{gsl_dist}. Copy the file 
\file{gsl_site_example.py}
and edit it to reflect your installation of GSL and SWIG if you want to run it
yourself. The pygsl windows binaries distributed over 
\url{http://sourceforge.net/projects/pygsl/} are built using the mingw32 
compiler. 

\paragraph*{Uninstall GSL interface}
\code{rm -r }"python install path"\code{/lib/python}"version"\code{/site-packages/pygsl}

\paragraph*{Testing}
the directory \file{tests} contains several testsuites, based on python
\module{unittest}.
The script \file{run_test.py} in this directory will run one after the other.

\paragraph*{Support}
Please send mails to our mailinglist at
\email{pygsl-discuss@lists.sourceforge.net}.

\paragraph*{Developement}
You can browse our cvs tree at
\url{http://cvs.sourceforge.net/cgi-bin/viewcvs.cgi/pygsl/pygsl/}.
\\
Type this to check out the actual version:
\begin{verbatim}
cvs -d:pserver:anonymous@cvs.pygsl.sourceforge.net:/cvsroot/pygsl login
#Hit return for no password.
cvs -z3 -d:pserver:anonymous@cvs.pygsl.sourceforge.net:/cvsroot/pygsl co pygsl
\end{verbatim}
The script \program{tools/extract_tool.py} generates most of the special 
function code.

%\input{install_advanced.tex}
\paragraph*{ToDo}
Implement other parts:


\paragraph*{History}
\begin{itemize}
\item a gsl-interface for python was needed for a project at
\ulink{Center for Applied Informatics Cologne}{http://www.zaik.uni-koeln.de/AFS}.
\item \file{gsl-0.0.3} was released at May 23, 2001
\item \file{gsl-0.0.4} was released at January 8, 2002
\item \file{gsl-0.0.5} is growing since January, 2002
\item \file{gsl-0.2.0} was released at 
\item \file{gsl-0.3.0} was released at 
\item \file{gsl-0.3.1} was released at 
\item \file{gsl-0.3.2} was released at 
\item \file{gsl-0.9.4} was released at 25. October 2008
\end{itemize}

\paragraph*{Thanks}
Jochen K\"upper (\email{jochen@jochen-kuepper.de}) for 
\module{pygsl.statistics} part\\
Fabian Jakobs for \module{pygsl.blas}, \module{pygsl.eigen}
\module{pygsl.linalg}, \module{pygsl.permutation}\\ 
Leonardo Milano for rpm build\\
Eric Gurrola and  Peter Stoltz for testing and supporting the port of pygsl to
the MAC\\
Sebastien Maret for supporting the Fink \url{http://fink.sourceforge.net}
port of pygsl.


\paragraph*{Maintainers}
Achim G\"adke (\email{AchimGaedke@users.sourceforge.net}),\\
Pierre Schnizer (\email{schnizer@users.sourceforge.net})


\paragraph*{Acknowledgment}
\label{sec:acknowledgment}
Parts of this this manual are based on the \GSL{} reference manual.


\chapter[\protect\module{pygsl.const} --- Mathematical and scientific
constants]{\protect\module{pygsl.const} \\ Mathematical and scientific
constants} 
\label{cha:const-module}
\declaremodule{extension}{pygsl.const}
\moduleauthor{Achim G\"adke}{achimgaedke@users.sourceforge.net}

In this module some usefull constants are defined.
There are four groups of constants:

\begin{itemize}
\item mathematical
\item physical in cgs unit system
\item physical in mks unit system
\item physical number constants (e.g. fine structure)
\end{itemize}

The other modules are created during the initialisation of \module{pygsl.const}.
The mathematical, physical mks constants and number constants are available in the namespace of \module{pygsl.const}, e.g.
\begin{verbatim}
import pygsl.const
import pygsl.const.cgs
print pygsl.const.cgs.speed_of_light/pygsl.const.speed_of_light
\end{verbatim}
Of course the result is 100.0.

\section[\protect\module{pygsl.const.math} --- Mathematical constants]
{\protect\module{pygsl.const.math} \\ Mathematical constants} 
\label{cha:const-math-module}

\section[\protect\module{pygsl.const.cgs} --- Scientific constants in cgs units]
{\protect\module{pygsl.const.cgs} \\ Scientific constants in cgs units} 
\label{cha:const-cgs-module}

\section[\protect\module{pygsl.const.mks} --- Scientific constants in mks units]
{\protect\module{pygsl.const.mks} \\ Scientific constants in mks units} 
\label{cha:const-mks-module}

\section[\protect\module{pygsl.const.num} --- Scientific number constants]
{\protect\module{pygsl.const.num} \\ Scientific number constants} 
\label{cha:const-num-module}


\chapter[\protect\module{pygsl.chebyshev}]
{\protect\module{pygsl.chebyshev}}
\label{cha:statistics-module}

\declaremodule{standard}{pygsl.chebyshev}
\moduleauthor{Pierre Schnizer}{schnizer@users.sourceforge.net}

\begin{classdesc}{cheb_series}{}
  This base class can be instantiated by its name
\end{classdesc}
\begin{verbatim}
import pygsl.chebyshev
s=pygsl.chebyshev.cheb_series()
\end{verbatim}

\begin{methoddesc}{__init__}{n}\index{__init__}
            n ... number of coefficients        
\end{methoddesc}
\begin{methoddesc}{init}{f, a, b}\index{init}
        This function computes the Chebyshev approximation for the
        function F over the range (a,b) to the previously specified order.
        The computation of the Chebyshev approximation is an O($n^2$)
        process, and requires n function evaluations.

            f ... a gsl_function
            a ... lower limit
            b ... upper limit
        
\end{methoddesc}
\begin{methoddesc}{eval}{x}\index{eval}
        This function evaluates the Chebyshev series at a given point X.
\end{methoddesc}
\begin{methoddesc}{eval_err}{x}\index{eval_err}
         This function computes the Chebyshev series  at a given point X,
         estimating both the series RESULT and its absolute error ABSERR.
         The error estimate is made from the first neglected term in the
         series.
\end{methoddesc}
\begin{methoddesc}{eval_n}{n, x}\index{eval_n}
         This function evaluates the Chebyshev series at a given point
         x, to (at most) the given order n
\end{methoddesc}
\begin{methoddesc}{eval_n_err}{n, x}\index{eval_n_err}
        This function evaluates a Chebyshev series at a given point X,
        estimating both the series RESULT and its absolute error ABSERR,
        to (at most) the given order ORDER.  The error estimate is made
        from the first neglected term in the series.
\end{methoddesc}

\begin{methoddesc}{calc_deriv}{}\index{calc_deriv}
        This method computes the derivative of the series CS. It returns
        a new instance of the cheb_series class.
\end{methoddesc}
\begin{methoddesc}{calc_integ}{}\index{calc_integ}
        This method computes the integral of the series CS. It returns
        a new instance of the cheb_series class.
\end{methoddesc}
\begin{methoddesc}{get_a}{}\index{get_a}
        Get the lower boundary of the current representation       
\end{methoddesc}
\begin{methoddesc}{get_b}{}\index{get_b}
        Get the upper boundary of the current representation        
\end{methoddesc}
\begin{methoddesc}{get_coefficients}{}\index{get_coefficients}
        Get the chebyshev coefficients.         
\end{methoddesc}
\begin{methoddesc}{get_f}{}\index{get_f}
        Get the value f (what is it ?) The documentation does not tell anything
        about it.        
\end{methoddesc}
\begin{methoddesc}{get_order_sp}{}\index{get_order_sp}
        Get the value f (what is it ?) The documentation does not tell anything
        about it.        
\end{methoddesc}
\begin{methoddesc}{set_a}{}\index{set_a}
        Set the lower boundary of the current representation        
\end{methoddesc}
\begin{methoddesc}{set_b}{}\index{set_b}
        Set the upper boundary of the current         
\end{methoddesc}
\begin{methoddesc}{set_coefficients}{}\index{set_coefficients}
        Sets the chebyshev coefficients. 
\end{methoddesc}
\begin{methoddesc}{set_f}{f}\index{set_f}
        Set the value f (what is it ?)        
\end{methoddesc}
\begin{methoddesc}{set_order_sp}{...}\index{set_order_sp}
        Set the value f (what is it ?)        
\end{methoddesc}


\begin{funcdesc}{gsl_function}{f, params}\index{gsl_function}

    This class defines the callbacks known as gsl_function to
    gsl.

    e.g to supply the function f:
    
    def f(x, params):
        a = params[0]
        b = parmas[1]
        c = params[3]
        return a * x ** 2 + b * x + c

    to some solver, use

    function = gsl_function(f, params)
    
\end{funcdesc}

%%% Local Variables: 
%%% mode: latex
%%% TeX-master: "ref"
%%% End: 

\chapter[\protect\module{pygsl.deriv} --- NumericalDifferentiation]%
{\protect\module{pygsl.deriv} \\ Numerical Differentiation}
\label{cha:diff-module}

\declaremodule{extension}{pygsl.deriv}%
 \moduleauthor{Pierre  Schnizer}{schnizer@users.sourceforge.net}%
 \modulesynopsis{Numerical  Differentiation}%

\begin{quote}
  This chapter describes the available functions for numerical differentiation.
\end{quote}

The functions described in this chapter compute numerical derivatives by finite
differencing.  An adaptive algorithm is used to find the best choice of finite
difference and to estimate the error in the derivative. This module supersedes
the diff module which has been deprecated with the release of GSL-1. XXX


\begin{funcdesc}{central}{func, x, h}
  This function computes the numerical derivative of the function \var{func} at
  the point \var{x} using an adaptive central difference algorithm with a step
  size of h.  A tuple \code{(result, error)} is returned with the derivative
  and its estimated absolute error.
\end{funcdesc}

\begin{funcdesc}{backward}{func, x, h}
  This function computes the numerical derivative of the function \var{func} at
  the point \var{x} using an adaptive backward difference algorithm with a step
  size of h.  The function \var{func} is evaluated only at points smaller than
  \var{x} and at \var{x} itself.  A tuple \code{(result, error)} is returned
  with the derivative and its estimated absolute error.
\end{funcdesc}

\begin{funcdesc}{forward}{func, x, h}
  This function computes the numerical derivative of the function \var{func} at
  the point \var{x} using an adaptive forward difference algorithm with a step
  size of h.  The function \var{func} is evaluated only at points greater than
  \var{x} and at \var{x} itself.  A tuple \code{(result, error)} is returned
  with the derivative and its estimated absolute error.
\end{funcdesc}


\begin{seealso}
  The algorithms used by these functions are described in the following book:
  \seetext{S.D.\ Conte and Carl de Boor, \emph{Elementary Numerical Analysis:
      An Algorithmic Approach}, McGraw-Hill, 1972.}
\end{seealso}



%% Local Variables:
%% mode: LaTeX
%% mode: auto-fill
%% fill-column: 79
%% indent-tabs-mode: nil
%% ispell-dictionary: "british"
%% reftex-fref-is-default: nil
%% TeX-auto-save: t
%% TeX-command-default: "pdfeLaTeX"
%% TeX-master: "pygsl"
%% TeX-parse-self: t
%% End:


\chapter[\protect\module{pygsl.histogram} --- Histogram Types]
{\protect\module{pygsl.histogram} \\ Histogram Types}
\label{cha:histogram-module}



%% Local Variables:
%% mode: LaTeX
%% mode: auto-fill
%% fill-column: 79
%% ispell-dictionary: "american"
%% reftex-fref-is-default: nil
%% TeX-auto-save: t
%% TeX-command-default: "pdfeLaTeX"
%% TeX-master: "pygsl"
%% TeX-parse-self: t
%% End:


\chapter[\protect\module{pygsl.rng} --- Random Number Generators]
{\protect\module{pygsl.rng} \\ Random Number Generators}
\label{cha:rng-module}
\declaremodule{standard}{pygsl.rng}
\moduleauthor{Achim G\"adke}{achimgaedke@users.sourceforge.net}

This chapter introduces the random number generator classes provided by \module{pygsl}.

\section{Random Number Generators}

Each random number generator is a derived sperate class, that returns
a pseudo random number sequence. Methods of the common base class \class{rng}
provide the transformation to different probability distributions and
give access to basic properties of random number generators.
\begin{classdesc}{rng}{\texttt{string} typenamme \code{|} \class{rng} r}
This base class can be instantiated by a name string of the desired generator
\begin{verbatim}
import pygsl.rng
my_ran0=pygsl.rng.rng("ran0")
\end{verbatim}
or a clone of an existing generator can be created by:
\begin{verbatim}
clone_ran0=pygsl.rng.rng(my_ran0)
\end{verbatim}
\end{classdesc}
The type of the allocated generator is given by the method
\begin{methoddesc}{name}{}
which returns its name as string.
\end{methoddesc}
All generators can be seeded with
\begin{methoddesc}{set}{seed}
which sets the internal seed according to the positive integer {\tt seed}. Zero as seed
has a special meaning, please read details in the gsl reference.
\end{methoddesc}
The basic returned number type is integer, these are generated by
\begin{methoddesc}{get}{}
which returns the next number of the pseudo random sequence.
\end{methoddesc}
Basic information about these numbers can be obtained by
\begin{methoddesc}{max}{}
maximum number of this sequence and
\end{methoddesc}
\begin{methoddesc}{min}{}
minimum number of this sequence.
\end{methoddesc}
Implemented uniform probability densities are:
\begin{methoddesc}{uniform}{}
returns a real number between $[0,1)$.
\end{methoddesc}
\begin{methoddesc}{uniform_pos}{}
returns a real number between $(0,1)$ --- this excludes 0.
\end{methoddesc}
\begin{methoddesc}{uniform_int}{upper limit}
returns an integer from 0 to the upper limit (exclusive). If this limit is larger than the
number of return values of the underlying generator, \exception{pygsl.gsl_Error} is raised.
\end{methoddesc}
Furthermore a lot of derived probability densities can be used:
\begin{methoddesc}{gaussian}{}
\end{methoddesc}
\begin{methoddesc}{gaussian\_ratio\_method}{}
\end{methoddesc}
\begin{methoddesc}{ugaussian}{}
\end{methoddesc}
\begin{methoddesc}{ugaussian\_ratio\_method}{}
\end{methoddesc}
\begin{methoddesc}{gaussian\_tail}{}
\end{methoddesc}
\begin{methoddesc}{ugaussian\_tail}{}
\end{methoddesc}
\begin{methoddesc}{bivariate\_gaussian}{}
\end{methoddesc}
\begin{methoddesc}{exponential}{}
\end{methoddesc}
\begin{methoddesc}{laplace}{}
\end{methoddesc}
\begin{methoddesc}{exppow}{}
\end{methoddesc}
\begin{methoddesc}{cauchy}{}
\end{methoddesc}
\begin{methoddesc}{rayleigh}{}
\end{methoddesc}
\begin{methoddesc}{rayleigh\_tail}{}
\end{methoddesc}
\begin{methoddesc}{levy}{}
\end{methoddesc}
\begin{methoddesc}{gamma}{}
\end{methoddesc}
\begin{methoddesc}{gamma\_int}{}
\end{methoddesc}
\begin{methoddesc}{flat}{}
\end{methoddesc}
\begin{methoddesc}{lognormal}{}
\end{methoddesc}
\begin{methoddesc}{chisq}{}
\end{methoddesc}
\begin{methoddesc}{fdist}{}
\end{methoddesc}
\begin{methoddesc}{tdist}{}
\end{methoddesc}
\begin{methoddesc}{beta}{}
\end{methoddesc}
\begin{methoddesc}{logistic}{}
\end{methoddesc}
\begin{methoddesc}{pareto}{}
\end{methoddesc}
\begin{methoddesc}{dir\_2d}{}
\end{methoddesc}
\begin{methoddesc}{dir\_2d\_trig\_method}{}
\end{methoddesc}
\begin{methoddesc}{dir\_3d}{}
\end{methoddesc}
\begin{methoddesc}{dir\_nd}{}
\end{methoddesc}
\begin{methoddesc}{weibull}{}
\end{methoddesc}
\begin{methoddesc}{gumbel1}{}
\end{methoddesc}
\begin{methoddesc}{gumbel2}{}
\end{methoddesc}
\begin{methoddesc}{poisson}{}
\end{methoddesc}
\begin{methoddesc}{bernoulli}{}
\end{methoddesc}
\begin{methoddesc}{binomial}{}
\end{methoddesc}
\begin{methoddesc}{negative\_binomial}{}
\end{methoddesc}
\begin{methoddesc}{pascal}{}
\end{methoddesc}
\begin{methoddesc}{geometric}{}
\end{methoddesc}
\begin{methoddesc}{hypergeometric}{}
\end{methoddesc}
\begin{methoddesc}{logarithmic}{}
\end{methoddesc}
\begin{methoddesc}{landau}{}
\end{methoddesc}
\begin{methoddesc}{erlang}{}
\end{methoddesc}


The different generator classes are created according to the output of \code{gsl_rng_types_setup()}
when the \module{pygsl.rng} is loaded. Here is the list of children from \class{rng} for gsl-1.2:
\newline
\class{rng_borosh13},
\class{rng_coveyou},
\class{rng_cmrg},
\class{rng_fishman18},
\class{rng_fishman20},
\class{rng_fishman2x},
\class{rng_gfsr4},
\class{rng_knuthran},
\class{rng_knuthran2},
\class{rng_lecuyer21},
\class{rng_minstd},
\class{rng_mrg},
\class{rng_mt19937},
\class{rng_mt19937_1999},
\class{rng_mt19937_1998},
\class{rng_r250},
\class{rng_ran0},
\class{rng_ran1},
\class{rng_ran2},
\class{rng_ran3},
\class{rng_rand},
\class{rng_rand48},
\class{rng_random128_bsd},
\class{rng_random128_glibc2},
\class{rng_random128_libc5},
\class{rng_random256_bsd},
\class{rng_random256_glibc2},
\class{rng_random256_libc5},
\class{rng_random32_bsd},
\class{rng_random32_glibc2},
\class{rng_random32_libc5},
\class{rng_random64_bsd},
\class{rng_random64_glibc2},
\class{rng_random64_libc5},
\class{rng_random8_bsd},
\class{rng_random8_glibc2},
\class{rng_random8_libc5},
\class{rng_random_bsd},
\class{rng_random_glibc2},
\class{rng_random_libc5},
\class{rng_randu},
\class{rng_ranf},
\class{rng_ranlux},
\class{rng_ranlux389},
\class{rng_ranlxd1},
\class{rng_ranlxd2},
\class{rng_ranlxs0},
\class{rng_ranlxs1},
\class{rng_ranlxs2},
\class{rng_ranmar},
\class{rng_slatec},
\class{rng_taus},
\class{rng_taus2},
\class{rng_taus113},
\class{rng_transputer},
\class{rng_tt800},
\class{rng_uni},
\class{rng_uni32},
\class{rng_vax},
\class{rng_waterman14}, and
\class{rng_zuf}.
\newline
The default generator of \class{rng} is determined by the environment
variable \envvar{GSL_RNG_TYPE} or defaults to {\tt rng_mt19937}.

\section{Probability Density Functions}


\section{Using probability densities with random number generators}


%% Local Variables:
%% mode: LaTeX
%% mode: auto-fill
%% fill-column: 90
%% indent-tabs-mode: nil
%% ispell-dictionary: "american"
%% reftex-fref-is-default: nil
%% TeX-auto-save: t
%% TeX-command-default: "pdfeLaTeX"
%% TeX-master: "pygsl"
%% TeX-parse-self: t
%% End:


\chapter[\protect\module{pygsl.sf} --- Special Functions]
{\protect\module{pygsl.sf} \\ Special Functions}
\label{cha:sf-module}
\declaremodule{extension}{pygsl.sf}
\moduleauthor{Achim G\"adke}{achimgaedke@users.sourceforge.net}

This chapter shows you the list of implemented special function and explains
details of error handling and return values.

\section{Function list}

\begin{longtableii}{l|l}{texttt}{Function}{Description}
\lineii{}{ToDo}
\end{longtableii}

\section{Return values}

\section{Error handling}

\declaremodule{extension}{pygsl.statistics}
\moduleauthor{Jochen K\"upper}{jochen@jochen-kuepper.de}
\index{mean}
\index{standard deviation}
\index{variance}
\index{estimated standard deviation}
\index{estimated variance}
\index{t-test}
\index{range}
\index{min}
\index{max}

This chapter describes the statistical functions in the library.  The
basic statistical functions include routines to compute the mean,
variance and standard deviation. More advanced functions allow you to
calculate absolute deviations, skewness, and kurtosis as well as the
median and arbitrary percentiles.  The algorithms use recurrence
relations to compute average quantities in a stable way, without large
intermediate values that might overflow. 

All functions work on any Python sequence (of the appropriate
data-type), but see section \ref{sec:stat-speed-considerations} for
advantages and drawbacks of different kinds of input data.


\section{Organization of the module}
\label{sec:stat-organization}

The parts of the GSL functions names, providing artificial name spaces,
are mapped to modules and submodules in pygsl.  That is
\code{gsl_stats_mean} can be found as \code{pygsl.statistics.mean} and
\code{gsl_stats_long_mean} as \code{pygsl.statistics.long.mean}.

The functions in the module are available in versions for datasets in
the standard floating-point and integer types. The generic versions
available in the \code{pygsl.statistics} module are using the generic
GSL \code{double} versions.  The submodules use GSL functions according
to the submodule name, e.g. long for \code{pygsl.statistics.long}.

Currently implemented submodules are \code{pygsl.statistics.double} and
\code{pygsl.statistics.long}.



\section{Speed considerations}
\label{sec:stat-speed-considerations}

All functions work on any Python sequence type but are optimized for use
with NumPy arrays. It is strongly suggested that you install NumPy
(available from \url{http://www.numpy.org})!

If you pass NumPy arrays of the \emph{correct data-type} as input data
to any of the functions they are passed straight to the C functions
along with the stride information of the data.

If you pass generic (non-NumPy) Python sequences or NumPy arrays of the
wrong data-type a suitable copy of the data will be created and passed
to the function.


\section{Further Reading}
\label{sec:stat-further-reading}

See the gsl reference manual for a description of all available
functions and the calculations they perform.


%% Local Variables:
%% mode: LaTeX
%% mode: auto-fill
%% fill-column: 79
%% ispell-dictionary: "american"
%% reftex-fref-is-default: nil
%% TeX-auto-save: t
%% TeX-command-default: "pdfeLaTeX"
%% TeX-master: "pygsl"
%% TeX-parse-self: t
%% End:


\chapter[\protect\module{pygsl.testing} ---  Modules in Testing]
{\protect\module{pygsl.testing} \\ Modules in Testing}
\label{cha:statistics-module}

\declaremodule{standard}{pygsl.testing}

\moduleauthor{Pierre Schnizer}{schnizer@users.sourceforge.net}
Modules in this package are often reimplementations of an original package
with significant change to the original. The current rng implementation, for
example, started its life here. Usage of these modules is encouraged for tests
to see if they work, but use them with caution in your production code!

\section[\protect\module{pygsl.testing.sf} --- Special UFuncs]
{\protect\module{pygsl.testing.sf} \\ Special Functions as UFuncs}

\declaremodule{standard}{pygsl.testing.sf}
\moduleauthor{Pierre Schnizer}{schnizer@users.sourceforge.net}

This chapter provides mainly \numpy{} UFuncs over the special functions. This means
that all input variable can be arrays, and the UFunc will evaluate the gsl
function for all its inputs. It is meant to replace the sf module later;
please use it and find out if it is useful for you. 
Only the python specific part is described here. For a general description of
the function please see the GSL Reference document.  

\section{UFuncs}
These UFuncs allow to evaluate an array of doubles or an array of floats typically.
\begin{funcdesc}{Chi}{...}\index{Chi}

    Number of Input  Arguments:  1
    Number of Output Arguments:  1
\end{funcdesc}

\begin{funcdesc}{Chi_e}{...}\index{Chi_e}

    Number of Input  Arguments:  1
    Number of Output Arguments:  2

The error flag is discarded.
Return Arguments 1 and 2 resemble a gsl_result argument,
	which is  argument 1 of the C argument list

\end{funcdesc}

\begin{funcdesc}{Ci}{...}\index{Ci}

    Number of Input  Arguments:  1
    Number of Output Arguments:  1
\end{funcdesc}

\begin{funcdesc}{Ci_e}{...}\index{Ci_e}

    Number of Input  Arguments:  1
    Number of Output Arguments:  2

The error flag is discarded.
Return Arguments 1 and 2 resemble a gsl_result argument,
	which is  argument 1 of the C argument list

\end{funcdesc}

\begin{funcdesc}{Shi}{...}\index{Shi}

    Number of Input  Arguments:  1
    Number of Output Arguments:  1
\end{funcdesc}

\begin{funcdesc}{Shi_e}{...}\index{Shi_e}

    Number of Input  Arguments:  1
    Number of Output Arguments:  2

The error flag is discarded.
Return Arguments 1 and 2 resemble a gsl_result argument,
	which is  argument 1 of the C argument list

\end{funcdesc}

\begin{funcdesc}{Si}{...}\index{Si}

    Number of Input  Arguments:  1
    Number of Output Arguments:  1
\end{funcdesc}

\begin{funcdesc}{Si_e}{...}\index{Si_e}

    Number of Input  Arguments:  1
    Number of Output Arguments:  2

The error flag is discarded.
Return Arguments 1 and 2 resemble a gsl_result argument,
	which is  argument 1 of the C argument list

\end{funcdesc}

\begin{funcdesc}{airy_Ai}{...}\index{airy_Ai}

    Number of Input  Arguments:  2
    Number of Output Arguments:  1

 Argument 2 is a gsl_mode_t, valid parameters are:
	sf.PREC_DOUBLE or sf.PREC_SINGLE or sf.PREC_APPROX

\end{funcdesc}

\begin{funcdesc}{airy_Ai_deriv}{...}\index{airy_Ai_deriv}

    Number of Input  Arguments:  2
    Number of Output Arguments:  1

 Argument 2 is a gsl_mode_t, valid parameters are:
	sf.PREC_DOUBLE or sf.PREC_SINGLE or sf.PREC_APPROX

\end{funcdesc}

\begin{funcdesc}{airy_Ai_deriv_e}{...}\index{airy_Ai_deriv_e}

    Number of Input  Arguments:  2
    Number of Output Arguments:  2

 Argument 2 is a gsl_mode_t, valid parameters are:
	sf.PREC_DOUBLE or sf.PREC_SINGLE or sf.PREC_APPROX
The error flag is discarded.
Return Arguments 1 and 2 resemble a gsl_result argument,
	which is  argument 2 of the C argument list

\end{funcdesc}

\begin{funcdesc}{airy_Ai_deriv_scaled}{...}\index{airy_Ai_deriv_scaled}

    Number of Input  Arguments:  2
    Number of Output Arguments:  1

 Argument 2 is a gsl_mode_t, valid parameters are:
	sf.PREC_DOUBLE or sf.PREC_SINGLE or sf.PREC_APPROX

\end{funcdesc}

\begin{funcdesc}{airy_Ai_deriv_scaled_e}{...}\index{airy_Ai_deriv_scaled_e}

    Number of Input  Arguments:  2
    Number of Output Arguments:  2

 Argument 2 is a gsl_mode_t, valid parameters are:
	sf.PREC_DOUBLE or sf.PREC_SINGLE or sf.PREC_APPROX
The error flag is discarded.
Return Arguments 1 and 2 resemble a gsl_result argument,
	which is  argument 2 of the C argument list

\end{funcdesc}

\begin{funcdesc}{airy_Ai_e}{...}\index{airy_Ai_e}

    Number of Input  Arguments:  2
    Number of Output Arguments:  2

 Argument 2 is a gsl_mode_t, valid parameters are:
	sf.PREC_DOUBLE or sf.PREC_SINGLE or sf.PREC_APPROX
The error flag is discarded.
Return Arguments 1 and 2 resemble a gsl_result argument,
	which is  argument 2 of the C argument list

\end{funcdesc}

\begin{funcdesc}{airy_Ai_scaled}{...}\index{airy_Ai_scaled}

    Number of Input  Arguments:  2
    Number of Output Arguments:  1

 Argument 2 is a gsl_mode_t, valid parameters are:
	sf.PREC_DOUBLE or sf.PREC_SINGLE or sf.PREC_APPROX

\end{funcdesc}

\begin{funcdesc}{airy_Ai_scaled_e}{...}\index{airy_Ai_scaled_e}

    Number of Input  Arguments:  2
    Number of Output Arguments:  2

 Argument 2 is a gsl_mode_t, valid parameters are:
	sf.PREC_DOUBLE or sf.PREC_SINGLE or sf.PREC_APPROX
The error flag is discarded.
Return Arguments 1 and 2 resemble a gsl_result argument,
	which is  argument 2 of the C argument list

\end{funcdesc}

\begin{funcdesc}{airy_Bi}{...}\index{airy_Bi}

    Number of Input  Arguments:  2
    Number of Output Arguments:  1

 Argument 2 is a gsl_mode_t, valid parameters are:
	sf.PREC_DOUBLE or sf.PREC_SINGLE or sf.PREC_APPROX

\end{funcdesc}

\begin{funcdesc}{airy_Bi_deriv}{...}\index{airy_Bi_deriv}

    Number of Input  Arguments:  2
    Number of Output Arguments:  1

 Argument 2 is a gsl_mode_t, valid parameters are:
	sf.PREC_DOUBLE or sf.PREC_SINGLE or sf.PREC_APPROX

\end{funcdesc}

\begin{funcdesc}{airy_Bi_deriv_e}{...}\index{airy_Bi_deriv_e}

    Number of Input  Arguments:  2
    Number of Output Arguments:  2

 Argument 2 is a gsl_mode_t, valid parameters are:
	sf.PREC_DOUBLE or sf.PREC_SINGLE or sf.PREC_APPROX
The error flag is discarded.
Return Arguments 1 and 2 resemble a gsl_result argument,
	which is  argument 2 of the C argument list

\end{funcdesc}

\begin{funcdesc}{airy_Bi_deriv_scaled}{...}\index{airy_Bi_deriv_scaled}

    Number of Input  Arguments:  2
    Number of Output Arguments:  1

 Argument 2 is a gsl_mode_t, valid parameters are:
	sf.PREC_DOUBLE or sf.PREC_SINGLE or sf.PREC_APPROX

\end{funcdesc}

\begin{funcdesc}{airy_Bi_deriv_scaled_e}{...}\index{airy_Bi_deriv_scaled_e}

    Number of Input  Arguments:  2
    Number of Output Arguments:  2

 Argument 2 is a gsl_mode_t, valid parameters are:
	sf.PREC_DOUBLE or sf.PREC_SINGLE or sf.PREC_APPROX
The error flag is discarded.
Return Arguments 1 and 2 resemble a gsl_result argument,
	which is  argument 2 of the C argument list

\end{funcdesc}

\begin{funcdesc}{airy_Bi_e}{...}\index{airy_Bi_e}

    Number of Input  Arguments:  2
    Number of Output Arguments:  2

 Argument 2 is a gsl_mode_t, valid parameters are:
	sf.PREC_DOUBLE or sf.PREC_SINGLE or sf.PREC_APPROX
The error flag is discarded.
Return Arguments 1 and 2 resemble a gsl_result argument,
	which is  argument 2 of the C argument list

\end{funcdesc}

\begin{funcdesc}{airy_Bi_scaled}{...}\index{airy_Bi_scaled}

    Number of Input  Arguments:  2
    Number of Output Arguments:  1

 Argument 2 is a gsl_mode_t, valid parameters are:
	sf.PREC_DOUBLE or sf.PREC_SINGLE or sf.PREC_APPROX

\end{funcdesc}

\begin{funcdesc}{airy_Bi_scaled_e}{...}\index{airy_Bi_scaled_e}

    Number of Input  Arguments:  2
    Number of Output Arguments:  2

 Argument 2 is a gsl_mode_t, valid parameters are:
	sf.PREC_DOUBLE or sf.PREC_SINGLE or sf.PREC_APPROX
The error flag is discarded.
Return Arguments 1 and 2 resemble a gsl_result argument,
	which is  argument 2 of the C argument list

\end{funcdesc}

\begin{funcdesc}{airy_zero_Ai}{...}\index{airy_zero_Ai}

    Number of Input  Arguments:  1
    Number of Output Arguments:  1
\end{funcdesc}

\begin{funcdesc}{airy_zero_Ai_deriv}{...}\index{airy_zero_Ai_deriv}

    Number of Input  Arguments:  1
    Number of Output Arguments:  1
\end{funcdesc}

\begin{funcdesc}{airy_zero_Ai_deriv_e}{...}\index{airy_zero_Ai_deriv_e}

    Number of Input  Arguments:  1
    Number of Output Arguments:  2

The error flag is discarded.
Return Arguments 1 and 2 resemble a gsl_result argument,
	which is  argument 1 of the C argument list

\end{funcdesc}

\begin{funcdesc}{airy_zero_Ai_e}{...}\index{airy_zero_Ai_e}

    Number of Input  Arguments:  1
    Number of Output Arguments:  2

The error flag is discarded.
Return Arguments 1 and 2 resemble a gsl_result argument,
	which is  argument 1 of the C argument list

\end{funcdesc}

\begin{funcdesc}{airy_zero_Bi}{...}\index{airy_zero_Bi}

    Number of Input  Arguments:  1
    Number of Output Arguments:  1
\end{funcdesc}

\begin{funcdesc}{airy_zero_Bi_deriv}{...}\index{airy_zero_Bi_deriv}

    Number of Input  Arguments:  1
    Number of Output Arguments:  1
\end{funcdesc}

\begin{funcdesc}{airy_zero_Bi_deriv_e}{...}\index{airy_zero_Bi_deriv_e}

    Number of Input  Arguments:  1
    Number of Output Arguments:  2

The error flag is discarded.
Return Arguments 1 and 2 resemble a gsl_result argument,
	which is  argument 1 of the C argument list

\end{funcdesc}

\begin{funcdesc}{airy_zero_Bi_e}{...}\index{airy_zero_Bi_e}

    Number of Input  Arguments:  1
    Number of Output Arguments:  2

The error flag is discarded.
Return Arguments 1 and 2 resemble a gsl_result argument,
	which is  argument 1 of the C argument list

\end{funcdesc}

\begin{funcdesc}{angle_restrict_pos}{...}\index{angle_restrict_pos}

    Number of Input  Arguments:  1
    Number of Output Arguments:  1
\end{funcdesc}

\begin{funcdesc}{angle_restrict_pos_err_e}{...}\index{angle_restrict_pos_err_e}

    Number of Input  Arguments:  1
    Number of Output Arguments:  2

The error flag is discarded.
Return Arguments 1 and 2 resemble a gsl_result argument,
	which is  argument 1 of the C argument list

\end{funcdesc}

\begin{funcdesc}{angle_restrict_symm}{...}\index{angle_restrict_symm}

    Number of Input  Arguments:  1
    Number of Output Arguments:  1
\end{funcdesc}

\begin{funcdesc}{angle_restrict_symm_err_e}{...}\index{angle_restrict_symm_err_e}

    Number of Input  Arguments:  1
    Number of Output Arguments:  2

The error flag is discarded.
Return Arguments 1 and 2 resemble a gsl_result argument,
	which is  argument 1 of the C argument list

\end{funcdesc}

\begin{funcdesc}{atanint}{...}\index{atanint}

    Number of Input  Arguments:  1
    Number of Output Arguments:  1
\end{funcdesc}

\begin{funcdesc}{atanint_e}{...}\index{atanint_e}

    Number of Input  Arguments:  1
    Number of Output Arguments:  2

The error flag is discarded.
Return Arguments 1 and 2 resemble a gsl_result argument,
	which is  argument 1 of the C argument list

\end{funcdesc}

\begin{funcdesc}{bessel_I0}{...}\index{bessel_I0}

    Number of Input  Arguments:  1
    Number of Output Arguments:  1
\end{funcdesc}

\begin{funcdesc}{bessel_I0_e}{...}\index{bessel_I0_e}

    Number of Input  Arguments:  1
    Number of Output Arguments:  2

The error flag is discarded.
Return Arguments 1 and 2 resemble a gsl_result argument,
	which is  argument 1 of the C argument list

\end{funcdesc}

\begin{funcdesc}{bessel_I0_scaled}{...}\index{bessel_I0_scaled}

    Number of Input  Arguments:  1
    Number of Output Arguments:  1
\end{funcdesc}

\begin{funcdesc}{bessel_I0_scaled_e}{...}\index{bessel_I0_scaled_e}

    Number of Input  Arguments:  1
    Number of Output Arguments:  2

The error flag is discarded.
Return Arguments 1 and 2 resemble a gsl_result argument,
	which is  argument 1 of the C argument list

\end{funcdesc}

\begin{funcdesc}{bessel_I1}{...}\index{bessel_I1}

    Number of Input  Arguments:  1
    Number of Output Arguments:  1
\end{funcdesc}

\begin{funcdesc}{bessel_I1_e}{...}\index{bessel_I1_e}

    Number of Input  Arguments:  1
    Number of Output Arguments:  2

The error flag is discarded.
Return Arguments 1 and 2 resemble a gsl_result argument,
	which is  argument 1 of the C argument list

\end{funcdesc}

\begin{funcdesc}{bessel_I1_scaled}{...}\index{bessel_I1_scaled}

    Number of Input  Arguments:  1
    Number of Output Arguments:  1
\end{funcdesc}

\begin{funcdesc}{bessel_I1_scaled_e}{...}\index{bessel_I1_scaled_e}

    Number of Input  Arguments:  1
    Number of Output Arguments:  2

The error flag is discarded.
Return Arguments 1 and 2 resemble a gsl_result argument,
	which is  argument 1 of the C argument list

\end{funcdesc}

\begin{funcdesc}{bessel_In}{...}\index{bessel_In}

    Number of Input  Arguments:  2
    Number of Output Arguments:  1
\end{funcdesc}

\begin{funcdesc}{bessel_In_e}{...}\index{bessel_In_e}

    Number of Input  Arguments:  2
    Number of Output Arguments:  2

The error flag is discarded.
Return Arguments 1 and 2 resemble a gsl_result argument,
	which is  argument 2 of the C argument list

\end{funcdesc}

\begin{funcdesc}{bessel_In_scaled}{...}\index{bessel_In_scaled}

    Number of Input  Arguments:  2
    Number of Output Arguments:  1
\end{funcdesc}

\begin{funcdesc}{bessel_In_scaled_e}{...}\index{bessel_In_scaled_e}

    Number of Input  Arguments:  2
    Number of Output Arguments:  2

The error flag is discarded.
Return Arguments 1 and 2 resemble a gsl_result argument,
	which is  argument 2 of the C argument list

\end{funcdesc}

\begin{funcdesc}{bessel_Inu}{...}\index{bessel_Inu}

    Number of Input  Arguments:  2
    Number of Output Arguments:  1
\end{funcdesc}

\begin{funcdesc}{bessel_Inu_e}{...}\index{bessel_Inu_e}

    Number of Input  Arguments:  2
    Number of Output Arguments:  2

The error flag is discarded.
Return Arguments 1 and 2 resemble a gsl_result argument,
	which is  argument 2 of the C argument list

\end{funcdesc}

\begin{funcdesc}{bessel_Inu_scaled}{...}\index{bessel_Inu_scaled}

    Number of Input  Arguments:  2
    Number of Output Arguments:  1
\end{funcdesc}

\begin{funcdesc}{bessel_Inu_scaled_e}{...}\index{bessel_Inu_scaled_e}

    Number of Input  Arguments:  2
    Number of Output Arguments:  2

The error flag is discarded.
Return Arguments 1 and 2 resemble a gsl_result argument,
	which is  argument 2 of the C argument list

\end{funcdesc}

\begin{funcdesc}{bessel_J0}{...}\index{bessel_J0}

    Number of Input  Arguments:  1
    Number of Output Arguments:  1
\end{funcdesc}

\begin{funcdesc}{bessel_J0_e}{...}\index{bessel_J0_e}

    Number of Input  Arguments:  1
    Number of Output Arguments:  2

The error flag is discarded.
Return Arguments 1 and 2 resemble a gsl_result argument,
	which is  argument 1 of the C argument list

\end{funcdesc}

\begin{funcdesc}{bessel_J1}{...}\index{bessel_J1}

    Number of Input  Arguments:  1
    Number of Output Arguments:  1
\end{funcdesc}

\begin{funcdesc}{bessel_J1_e}{...}\index{bessel_J1_e}

    Number of Input  Arguments:  1
    Number of Output Arguments:  2

The error flag is discarded.
Return Arguments 1 and 2 resemble a gsl_result argument,
	which is  argument 1 of the C argument list

\end{funcdesc}

\begin{funcdesc}{bessel_Jn}{...}\index{bessel_Jn}

    Number of Input  Arguments:  2
    Number of Output Arguments:  1
\end{funcdesc}

\begin{funcdesc}{bessel_Jn_e}{...}\index{bessel_Jn_e}

    Number of Input  Arguments:  2
    Number of Output Arguments:  2

The error flag is discarded.
Return Arguments 1 and 2 resemble a gsl_result argument,
	which is  argument 2 of the C argument list

\end{funcdesc}

\begin{funcdesc}{bessel_Jnu}{...}\index{bessel_Jnu}

    Number of Input  Arguments:  2
    Number of Output Arguments:  1
\end{funcdesc}

\begin{funcdesc}{bessel_Jnu_e}{...}\index{bessel_Jnu_e}

    Number of Input  Arguments:  2
    Number of Output Arguments:  2

The error flag is discarded.
Return Arguments 1 and 2 resemble a gsl_result argument,
	which is  argument 2 of the C argument list

\end{funcdesc}

\begin{funcdesc}{bessel_K0}{...}\index{bessel_K0}

    Number of Input  Arguments:  1
    Number of Output Arguments:  1
\end{funcdesc}

\begin{funcdesc}{bessel_K0_e}{...}\index{bessel_K0_e}

    Number of Input  Arguments:  1
    Number of Output Arguments:  2

The error flag is discarded.
Return Arguments 1 and 2 resemble a gsl_result argument,
	which is  argument 1 of the C argument list

\end{funcdesc}

\begin{funcdesc}{bessel_K0_scaled}{...}\index{bessel_K0_scaled}

    Number of Input  Arguments:  1
    Number of Output Arguments:  1
\end{funcdesc}

\begin{funcdesc}{bessel_K0_scaled_e}{...}\index{bessel_K0_scaled_e}

    Number of Input  Arguments:  1
    Number of Output Arguments:  2

The error flag is discarded.
Return Arguments 1 and 2 resemble a gsl_result argument,
	which is  argument 1 of the C argument list

\end{funcdesc}

\begin{funcdesc}{bessel_K1}{...}\index{bessel_K1}

    Number of Input  Arguments:  1
    Number of Output Arguments:  1
\end{funcdesc}

\begin{funcdesc}{bessel_K1_e}{...}\index{bessel_K1_e}

    Number of Input  Arguments:  1
    Number of Output Arguments:  2

The error flag is discarded.
Return Arguments 1 and 2 resemble a gsl_result argument,
	which is  argument 1 of the C argument list

\end{funcdesc}

\begin{funcdesc}{bessel_K1_scaled}{...}\index{bessel_K1_scaled}

    Number of Input  Arguments:  1
    Number of Output Arguments:  1
\end{funcdesc}

\begin{funcdesc}{bessel_K1_scaled_e}{...}\index{bessel_K1_scaled_e}

    Number of Input  Arguments:  1
    Number of Output Arguments:  2

The error flag is discarded.
Return Arguments 1 and 2 resemble a gsl_result argument,
	which is  argument 1 of the C argument list

\end{funcdesc}

\begin{funcdesc}{bessel_Kn}{...}\index{bessel_Kn}

    Number of Input  Arguments:  2
    Number of Output Arguments:  1
\end{funcdesc}

\begin{funcdesc}{bessel_Kn_e}{...}\index{bessel_Kn_e}

    Number of Input  Arguments:  2
    Number of Output Arguments:  2

The error flag is discarded.
Return Arguments 1 and 2 resemble a gsl_result argument,
	which is  argument 2 of the C argument list

\end{funcdesc}

\begin{funcdesc}{bessel_Kn_scaled}{...}\index{bessel_Kn_scaled}

    Number of Input  Arguments:  2
    Number of Output Arguments:  1
\end{funcdesc}

\begin{funcdesc}{bessel_Kn_scaled_e}{...}\index{bessel_Kn_scaled_e}

    Number of Input  Arguments:  2
    Number of Output Arguments:  2

The error flag is discarded.
Return Arguments 1 and 2 resemble a gsl_result argument,
	which is  argument 2 of the C argument list

\end{funcdesc}

\begin{funcdesc}{bessel_Knu}{...}\index{bessel_Knu}

    Number of Input  Arguments:  2
    Number of Output Arguments:  1
\end{funcdesc}

\begin{funcdesc}{bessel_Knu_e}{...}\index{bessel_Knu_e}

    Number of Input  Arguments:  2
    Number of Output Arguments:  2

The error flag is discarded.
Return Arguments 1 and 2 resemble a gsl_result argument,
	which is  argument 2 of the C argument list

\end{funcdesc}

\begin{funcdesc}{bessel_Knu_scaled}{...}\index{bessel_Knu_scaled}

    Number of Input  Arguments:  2
    Number of Output Arguments:  1
\end{funcdesc}

\begin{funcdesc}{bessel_Knu_scaled_e}{...}\index{bessel_Knu_scaled_e}

    Number of Input  Arguments:  2
    Number of Output Arguments:  2

The error flag is discarded.
Return Arguments 1 and 2 resemble a gsl_result argument,
	which is  argument 2 of the C argument list

\end{funcdesc}

\begin{funcdesc}{bessel_Y0}{...}\index{bessel_Y0}

    Number of Input  Arguments:  1
    Number of Output Arguments:  1
\end{funcdesc}

\begin{funcdesc}{bessel_Y0_e}{...}\index{bessel_Y0_e}

    Number of Input  Arguments:  1
    Number of Output Arguments:  2

The error flag is discarded.
Return Arguments 1 and 2 resemble a gsl_result argument,
	which is  argument 1 of the C argument list

\end{funcdesc}

\begin{funcdesc}{bessel_Y1}{...}\index{bessel_Y1}

    Number of Input  Arguments:  1
    Number of Output Arguments:  1
\end{funcdesc}

\begin{funcdesc}{bessel_Y1_e}{...}\index{bessel_Y1_e}

    Number of Input  Arguments:  1
    Number of Output Arguments:  2

The error flag is discarded.
Return Arguments 1 and 2 resemble a gsl_result argument,
	which is  argument 1 of the C argument list

\end{funcdesc}

\begin{funcdesc}{bessel_Yn}{...}\index{bessel_Yn}

    Number of Input  Arguments:  2
    Number of Output Arguments:  1
\end{funcdesc}

\begin{funcdesc}{bessel_Yn_e}{...}\index{bessel_Yn_e}

    Number of Input  Arguments:  2
    Number of Output Arguments:  2

The error flag is discarded.
Return Arguments 1 and 2 resemble a gsl_result argument,
	which is  argument 2 of the C argument list

\end{funcdesc}

\begin{funcdesc}{bessel_Ynu}{...}\index{bessel_Ynu}

    Number of Input  Arguments:  2
    Number of Output Arguments:  1
\end{funcdesc}

\begin{funcdesc}{bessel_Ynu_e}{...}\index{bessel_Ynu_e}

    Number of Input  Arguments:  2
    Number of Output Arguments:  2

The error flag is discarded.
Return Arguments 1 and 2 resemble a gsl_result argument,
	which is  argument 2 of the C argument list

\end{funcdesc}

\begin{funcdesc}{bessel_i0_scaled}{...}\index{bessel_i0_scaled}

    Number of Input  Arguments:  1
    Number of Output Arguments:  1
\end{funcdesc}

\begin{funcdesc}{bessel_i0_scaled_e}{...}\index{bessel_i0_scaled_e}

    Number of Input  Arguments:  1
    Number of Output Arguments:  2

The error flag is discarded.
Return Arguments 1 and 2 resemble a gsl_result argument,
	which is  argument 1 of the C argument list

\end{funcdesc}

\begin{funcdesc}{bessel_i1_scaled}{...}\index{bessel_i1_scaled}

    Number of Input  Arguments:  1
    Number of Output Arguments:  1
\end{funcdesc}

\begin{funcdesc}{bessel_i1_scaled_e}{...}\index{bessel_i1_scaled_e}

    Number of Input  Arguments:  1
    Number of Output Arguments:  2

The error flag is discarded.
Return Arguments 1 and 2 resemble a gsl_result argument,
	which is  argument 1 of the C argument list

\end{funcdesc}

\begin{funcdesc}{bessel_i2_scaled}{...}\index{bessel_i2_scaled}

    Number of Input  Arguments:  1
    Number of Output Arguments:  1
\end{funcdesc}

\begin{funcdesc}{bessel_i2_scaled_e}{...}\index{bessel_i2_scaled_e}

    Number of Input  Arguments:  1
    Number of Output Arguments:  2

The error flag is discarded.
Return Arguments 1 and 2 resemble a gsl_result argument,
	which is  argument 1 of the C argument list

\end{funcdesc}

\begin{funcdesc}{bessel_il_scaled}{...}\index{bessel_il_scaled}

    Number of Input  Arguments:  2
    Number of Output Arguments:  1
\end{funcdesc}

\begin{funcdesc}{bessel_il_scaled_e}{...}\index{bessel_il_scaled_e}

    Number of Input  Arguments:  2
    Number of Output Arguments:  2

The error flag is discarded.
Return Arguments 1 and 2 resemble a gsl_result argument,
	which is  argument 2 of the C argument list

\end{funcdesc}

\begin{funcdesc}{bessel_j0}{...}\index{bessel_j0}

    Number of Input  Arguments:  1
    Number of Output Arguments:  1
\end{funcdesc}

\begin{funcdesc}{bessel_j0_e}{...}\index{bessel_j0_e}

    Number of Input  Arguments:  1
    Number of Output Arguments:  2

The error flag is discarded.
Return Arguments 1 and 2 resemble a gsl_result argument,
	which is  argument 1 of the C argument list

\end{funcdesc}

\begin{funcdesc}{bessel_j1}{...}\index{bessel_j1}

    Number of Input  Arguments:  1
    Number of Output Arguments:  1
\end{funcdesc}

\begin{funcdesc}{bessel_j1_e}{...}\index{bessel_j1_e}

    Number of Input  Arguments:  1
    Number of Output Arguments:  2

The error flag is discarded.
Return Arguments 1 and 2 resemble a gsl_result argument,
	which is  argument 1 of the C argument list

\end{funcdesc}

\begin{funcdesc}{bessel_j2}{...}\index{bessel_j2}

    Number of Input  Arguments:  1
    Number of Output Arguments:  1
\end{funcdesc}

\begin{funcdesc}{bessel_j2_e}{...}\index{bessel_j2_e}

    Number of Input  Arguments:  1
    Number of Output Arguments:  2

The error flag is discarded.
Return Arguments 1 and 2 resemble a gsl_result argument,
	which is  argument 1 of the C argument list

\end{funcdesc}

\begin{funcdesc}{bessel_jl}{...}\index{bessel_jl}

    Number of Input  Arguments:  2
    Number of Output Arguments:  1
\end{funcdesc}

\begin{funcdesc}{bessel_jl_e}{...}\index{bessel_jl_e}

    Number of Input  Arguments:  2
    Number of Output Arguments:  2

The error flag is discarded.
Return Arguments 1 and 2 resemble a gsl_result argument,
	which is  argument 2 of the C argument list

\end{funcdesc}

\begin{funcdesc}{bessel_k0_scaled}{...}\index{bessel_k0_scaled}

    Number of Input  Arguments:  1
    Number of Output Arguments:  1
\end{funcdesc}

\begin{funcdesc}{bessel_k0_scaled_e}{...}\index{bessel_k0_scaled_e}

    Number of Input  Arguments:  1
    Number of Output Arguments:  2

The error flag is discarded.
Return Arguments 1 and 2 resemble a gsl_result argument,
	which is  argument 1 of the C argument list

\end{funcdesc}

\begin{funcdesc}{bessel_k1_scaled}{...}\index{bessel_k1_scaled}

    Number of Input  Arguments:  1
    Number of Output Arguments:  1
\end{funcdesc}

\begin{funcdesc}{bessel_k1_scaled_e}{...}\index{bessel_k1_scaled_e}

    Number of Input  Arguments:  1
    Number of Output Arguments:  2

The error flag is discarded.
Return Arguments 1 and 2 resemble a gsl_result argument,
	which is  argument 1 of the C argument list

\end{funcdesc}

\begin{funcdesc}{bessel_k2_scaled}{...}\index{bessel_k2_scaled}

    Number of Input  Arguments:  1
    Number of Output Arguments:  1
\end{funcdesc}

\begin{funcdesc}{bessel_k2_scaled_e}{...}\index{bessel_k2_scaled_e}

    Number of Input  Arguments:  1
    Number of Output Arguments:  2

The error flag is discarded.
Return Arguments 1 and 2 resemble a gsl_result argument,
	which is  argument 1 of the C argument list

\end{funcdesc}

\begin{funcdesc}{bessel_kl_scaled}{...}\index{bessel_kl_scaled}

    Number of Input  Arguments:  2
    Number of Output Arguments:  1
\end{funcdesc}

\begin{funcdesc}{bessel_kl_scaled_e}{...}\index{bessel_kl_scaled_e}

    Number of Input  Arguments:  2
    Number of Output Arguments:  2

The error flag is discarded.
Return Arguments 1 and 2 resemble a gsl_result argument,
	which is  argument 2 of the C argument list

\end{funcdesc}

\begin{funcdesc}{bessel_lnKnu}{...}\index{bessel_lnKnu}

    Number of Input  Arguments:  2
    Number of Output Arguments:  1
\end{funcdesc}

\begin{funcdesc}{bessel_lnKnu_e}{...}\index{bessel_lnKnu_e}

    Number of Input  Arguments:  2
    Number of Output Arguments:  2

The error flag is discarded.
Return Arguments 1 and 2 resemble a gsl_result argument,
	which is  argument 2 of the C argument list

\end{funcdesc}

\begin{funcdesc}{bessel_y0}{...}\index{bessel_y0}

    Number of Input  Arguments:  1
    Number of Output Arguments:  1
\end{funcdesc}

\begin{funcdesc}{bessel_y0_e}{...}\index{bessel_y0_e}

    Number of Input  Arguments:  1
    Number of Output Arguments:  2

The error flag is discarded.
Return Arguments 1 and 2 resemble a gsl_result argument,
	which is  argument 1 of the C argument list

\end{funcdesc}

\begin{funcdesc}{bessel_y1}{...}\index{bessel_y1}

    Number of Input  Arguments:  1
    Number of Output Arguments:  1
\end{funcdesc}

\begin{funcdesc}{bessel_y1_e}{...}\index{bessel_y1_e}

    Number of Input  Arguments:  1
    Number of Output Arguments:  2

The error flag is discarded.
Return Arguments 1 and 2 resemble a gsl_result argument,
	which is  argument 1 of the C argument list

\end{funcdesc}

\begin{funcdesc}{bessel_y2}{...}\index{bessel_y2}

    Number of Input  Arguments:  1
    Number of Output Arguments:  1
\end{funcdesc}

\begin{funcdesc}{bessel_y2_e}{...}\index{bessel_y2_e}

    Number of Input  Arguments:  1
    Number of Output Arguments:  2

The error flag is discarded.
Return Arguments 1 and 2 resemble a gsl_result argument,
	which is  argument 1 of the C argument list

\end{funcdesc}

\begin{funcdesc}{bessel_yl}{...}\index{bessel_yl}

    Number of Input  Arguments:  2
    Number of Output Arguments:  1
\end{funcdesc}

\begin{funcdesc}{bessel_yl_e}{...}\index{bessel_yl_e}

    Number of Input  Arguments:  2
    Number of Output Arguments:  2

The error flag is discarded.
Return Arguments 1 and 2 resemble a gsl_result argument,
	which is  argument 2 of the C argument list

\end{funcdesc}

\begin{funcdesc}{bessel_zero_J0}{...}\index{bessel_zero_J0}

    Number of Input  Arguments:  1
    Number of Output Arguments:  1
\end{funcdesc}

\begin{funcdesc}{bessel_zero_J0_e}{...}\index{bessel_zero_J0_e}

    Number of Input  Arguments:  1
    Number of Output Arguments:  2

The error flag is discarded.
Return Arguments 1 and 2 resemble a gsl_result argument,
	which is  argument 1 of the C argument list

\end{funcdesc}

\begin{funcdesc}{bessel_zero_J1}{...}\index{bessel_zero_J1}

    Number of Input  Arguments:  1
    Number of Output Arguments:  1
\end{funcdesc}

\begin{funcdesc}{bessel_zero_J1_e}{...}\index{bessel_zero_J1_e}

    Number of Input  Arguments:  1
    Number of Output Arguments:  2

The error flag is discarded.
Return Arguments 1 and 2 resemble a gsl_result argument,
	which is  argument 1 of the C argument list

\end{funcdesc}

\begin{funcdesc}{bessel_zero_Jnu}{...}\index{bessel_zero_Jnu}

    Number of Input  Arguments:  2
    Number of Output Arguments:  1
\end{funcdesc}

\begin{funcdesc}{bessel_zero_Jnu_e}{...}\index{bessel_zero_Jnu_e}

    Number of Input  Arguments:  2
    Number of Output Arguments:  2

The error flag is discarded.
Return Arguments 1 and 2 resemble a gsl_result argument,
	which is  argument 2 of the C argument list

\end{funcdesc}

\begin{funcdesc}{beta}{...}\index{beta}

    Number of Input  Arguments:  2
    Number of Output Arguments:  1
\end{funcdesc}

\begin{funcdesc}{beta_e}{...}\index{beta_e}

    Number of Input  Arguments:  2
    Number of Output Arguments:  2

The error flag is discarded.
Return Arguments 1 and 2 resemble a gsl_result argument,
	which is  argument 2 of the C argument list

\end{funcdesc}

\begin{funcdesc}{beta_inc}{...}\index{beta_inc}

    Number of Input  Arguments:  3
    Number of Output Arguments:  1
\end{funcdesc}

\begin{funcdesc}{beta_inc_e}{...}\index{beta_inc_e}

    Number of Input  Arguments:  3
    Number of Output Arguments:  2

The error flag is discarded.
Return Arguments 1 and 2 resemble a gsl_result argument,
	which is  argument 3 of the C argument list

\end{funcdesc}

\begin{funcdesc}{choose}{...}\index{choose}

    Number of Input  Arguments:  2
    Number of Output Arguments:  1
\end{funcdesc}

\begin{funcdesc}{choose_e}{...}\index{choose_e}

    Number of Input  Arguments:  2
    Number of Output Arguments:  2

The error flag is discarded.
Return Arguments 1 and 2 resemble a gsl_result argument,
	which is  argument 2 of the C argument list

\end{funcdesc}

\begin{funcdesc}{clausen}{...}\index{clausen}

    Number of Input  Arguments:  1
    Number of Output Arguments:  1
\end{funcdesc}

\begin{funcdesc}{clausen_e}{...}\index{clausen_e}

    Number of Input  Arguments:  1
    Number of Output Arguments:  2

The error flag is discarded.
Return Arguments 1 and 2 resemble a gsl_result argument,
	which is  argument 1 of the C argument list

\end{funcdesc}

\begin{funcdesc}{conicalP_0}{...}\index{conicalP_0}

    Number of Input  Arguments:  2
    Number of Output Arguments:  1
\end{funcdesc}

\begin{funcdesc}{conicalP_0_e}{...}\index{conicalP_0_e}

    Number of Input  Arguments:  2
    Number of Output Arguments:  2

The error flag is discarded.
Return Arguments 1 and 2 resemble a gsl_result argument,
	which is  argument 2 of the C argument list

\end{funcdesc}

\begin{funcdesc}{conicalP_1}{...}\index{conicalP_1}

    Number of Input  Arguments:  2
    Number of Output Arguments:  1
\end{funcdesc}

\begin{funcdesc}{conicalP_1_e}{...}\index{conicalP_1_e}

    Number of Input  Arguments:  2
    Number of Output Arguments:  2

The error flag is discarded.
Return Arguments 1 and 2 resemble a gsl_result argument,
	which is  argument 2 of the C argument list

\end{funcdesc}

\begin{funcdesc}{conicalP_cyl_reg}{...}\index{conicalP_cyl_reg}

    Number of Input  Arguments:  3
    Number of Output Arguments:  1
\end{funcdesc}

\begin{funcdesc}{conicalP_cyl_reg_e}{...}\index{conicalP_cyl_reg_e}

    Number of Input  Arguments:  3
    Number of Output Arguments:  2

The error flag is discarded.
Return Arguments 1 and 2 resemble a gsl_result argument,
	which is  argument 3 of the C argument list

\end{funcdesc}

\begin{funcdesc}{conicalP_half}{...}\index{conicalP_half}

    Number of Input  Arguments:  2
    Number of Output Arguments:  1
\end{funcdesc}

\begin{funcdesc}{conicalP_half_e}{...}\index{conicalP_half_e}

    Number of Input  Arguments:  2
    Number of Output Arguments:  2

The error flag is discarded.
Return Arguments 1 and 2 resemble a gsl_result argument,
	which is  argument 2 of the C argument list

\end{funcdesc}

\begin{funcdesc}{conicalP_mhalf}{...}\index{conicalP_mhalf}

    Number of Input  Arguments:  2
    Number of Output Arguments:  1
\end{funcdesc}

\begin{funcdesc}{conicalP_mhalf_e}{...}\index{conicalP_mhalf_e}

    Number of Input  Arguments:  2
    Number of Output Arguments:  2

The error flag is discarded.
Return Arguments 1 and 2 resemble a gsl_result argument,
	which is  argument 2 of the C argument list

\end{funcdesc}

\begin{funcdesc}{conicalP_sph_reg}{...}\index{conicalP_sph_reg}

    Number of Input  Arguments:  3
    Number of Output Arguments:  1
\end{funcdesc}

\begin{funcdesc}{conicalP_sph_reg_e}{...}\index{conicalP_sph_reg_e}

    Number of Input  Arguments:  3
    Number of Output Arguments:  2

The error flag is discarded.
Return Arguments 1 and 2 resemble a gsl_result argument,
	which is  argument 3 of the C argument list

\end{funcdesc}

\begin{funcdesc}{cos}{...}\index{cos}

    Number of Input  Arguments:  1
    Number of Output Arguments:  1
\end{funcdesc}

\begin{funcdesc}{cos_e}{...}\index{cos_e}

    Number of Input  Arguments:  1
    Number of Output Arguments:  2

The error flag is discarded.
Return Arguments 1 and 2 resemble a gsl_result argument,
	which is  argument 1 of the C argument list

\end{funcdesc}

\begin{funcdesc}{cos_err_e}{...}\index{cos_err_e}

    Number of Input  Arguments:  2
    Number of Output Arguments:  2

The error flag is discarded.
Return Arguments 1 and 2 resemble a gsl_result argument,
	which is  argument 2 of the C argument list

\end{funcdesc}

\begin{funcdesc}{coulomb_CL_e}{...}\index{coulomb_CL_e}

    Number of Input  Arguments:  2
    Number of Output Arguments:  2

The error flag is discarded.
Return Arguments 1 and 2 resemble a gsl_result argument,
	which is  argument 2 of the C argument list

\end{funcdesc}

\begin{funcdesc}{coulomb_wave_FG_e}{...}\index{coulomb_wave_FG_e}

    Number of Input  Arguments:  4
    Number of Output Arguments: 10

The error flag is discarded.
Return Arguments 1 and 2 resemble a gsl_result argument,
	which is  argument 4 of the C argument list
Return Arguments 3 and 4 resemble a gsl_result argument,
	which is  argument 5 of the C argument list
Return Arguments 5 and 6 resemble a gsl_result argument,
	which is  argument 6 of the C argument list
Return Arguments 7 and 8 resemble a gsl_result argument,
	which is  argument 7 of the C argument list

\end{funcdesc}

\begin{funcdesc}{coupling_3j}{...}\index{coupling_3j}

    Number of Input  Arguments:  6
    Number of Output Arguments:  1
\end{funcdesc}

\begin{funcdesc}{coupling_3j_e}{...}\index{coupling_3j_e}

    Number of Input  Arguments:  6
    Number of Output Arguments:  2

The error flag is discarded.
Return Arguments 1 and 2 resemble a gsl_result argument,
	which is  argument 6 of the C argument list

\end{funcdesc}

\begin{funcdesc}{coupling_6j}{...}\index{coupling_6j}

    Number of Input  Arguments:  6
    Number of Output Arguments:  1
\end{funcdesc}

\begin{funcdesc}{coupling_6j_e}{...}\index{coupling_6j_e}

    Number of Input  Arguments:  6
    Number of Output Arguments:  2

The error flag is discarded.
Return Arguments 1 and 2 resemble a gsl_result argument,
	which is  argument 6 of the C argument list

\end{funcdesc}

\begin{funcdesc}{coupling_9j}{...}\index{coupling_9j}

    Number of Input  Arguments:  9
    Number of Output Arguments:  1
\end{funcdesc}

\begin{funcdesc}{coupling_9j_e}{...}\index{coupling_9j_e}

    Number of Input  Arguments:  9
    Number of Output Arguments:  2

The error flag is discarded.
Return Arguments 1 and 2 resemble a gsl_result argument,
	which is  argument 9 of the C argument list

\end{funcdesc}

\begin{funcdesc}{coupling_RacahW}{...}\index{coupling_RacahW}

    Number of Input  Arguments:  6
    Number of Output Arguments:  1
\end{funcdesc}

\begin{funcdesc}{coupling_RacahW_e}{...}\index{coupling_RacahW_e}

    Number of Input  Arguments:  6
    Number of Output Arguments:  2

The error flag is discarded.
Return Arguments 1 and 2 resemble a gsl_result argument,
	which is  argument 6 of the C argument list

\end{funcdesc}

\begin{funcdesc}{dawson}{...}\index{dawson}

    Number of Input  Arguments:  1
    Number of Output Arguments:  1
\end{funcdesc}

\begin{funcdesc}{dawson_e}{...}\index{dawson_e}

    Number of Input  Arguments:  1
    Number of Output Arguments:  2

The error flag is discarded.
Return Arguments 1 and 2 resemble a gsl_result argument,
	which is  argument 1 of the C argument list

\end{funcdesc}

\begin{funcdesc}{debye_1}{...}\index{debye_1}

    Number of Input  Arguments:  1
    Number of Output Arguments:  1
\end{funcdesc}

\begin{funcdesc}{debye_1_e}{...}\index{debye_1_e}

    Number of Input  Arguments:  1
    Number of Output Arguments:  2

The error flag is discarded.
Return Arguments 1 and 2 resemble a gsl_result argument,
	which is  argument 1 of the C argument list

\end{funcdesc}

\begin{funcdesc}{debye_2}{...}\index{debye_2}

    Number of Input  Arguments:  1
    Number of Output Arguments:  1
\end{funcdesc}

\begin{funcdesc}{debye_2_e}{...}\index{debye_2_e}

    Number of Input  Arguments:  1
    Number of Output Arguments:  2

The error flag is discarded.
Return Arguments 1 and 2 resemble a gsl_result argument,
	which is  argument 1 of the C argument list

\end{funcdesc}

\begin{funcdesc}{debye_3}{...}\index{debye_3}

    Number of Input  Arguments:  1
    Number of Output Arguments:  1
\end{funcdesc}

\begin{funcdesc}{debye_3_e}{...}\index{debye_3_e}

    Number of Input  Arguments:  1
    Number of Output Arguments:  2

The error flag is discarded.
Return Arguments 1 and 2 resemble a gsl_result argument,
	which is  argument 1 of the C argument list

\end{funcdesc}

\begin{funcdesc}{debye_4}{...}\index{debye_4}

    Number of Input  Arguments:  1
    Number of Output Arguments:  1
\end{funcdesc}

\begin{funcdesc}{debye_4_e}{...}\index{debye_4_e}

    Number of Input  Arguments:  1
    Number of Output Arguments:  2

The error flag is discarded.
Return Arguments 1 and 2 resemble a gsl_result argument,
	which is  argument 1 of the C argument list

\end{funcdesc}

\begin{funcdesc}{dilog}{...}\index{dilog}

    Number of Input  Arguments:  1
    Number of Output Arguments:  1
\end{funcdesc}

\begin{funcdesc}{dilog_e}{...}\index{dilog_e}

    Number of Input  Arguments:  1
    Number of Output Arguments:  2

The error flag is discarded.
Return Arguments 1 and 2 resemble a gsl_result argument,
	which is  argument 1 of the C argument list

\end{funcdesc}

\begin{funcdesc}{doublefact}{...}\index{doublefact}

    Number of Input  Arguments:  1
    Number of Output Arguments:  1
\end{funcdesc}

\begin{funcdesc}{doublefact_e}{...}\index{doublefact_e}

    Number of Input  Arguments:  1
    Number of Output Arguments:  2

The error flag is discarded.
Return Arguments 1 and 2 resemble a gsl_result argument,
	which is  argument 1 of the C argument list

\end{funcdesc}

\begin{funcdesc}{ellint_D}{...}\index{ellint_D}

    Number of Input  Arguments:  4
    Number of Output Arguments:  1

 Argument 4 is a gsl_mode_t, valid parameters are:
	sf.PREC_DOUBLE or sf.PREC_SINGLE or sf.PREC_APPROX

\end{funcdesc}

\begin{funcdesc}{ellint_D_e}{...}\index{ellint_D_e}

    Number of Input  Arguments:  4
    Number of Output Arguments:  2

 Argument 4 is a gsl_mode_t, valid parameters are:
	sf.PREC_DOUBLE or sf.PREC_SINGLE or sf.PREC_APPROX
The error flag is discarded.
Return Arguments 1 and 2 resemble a gsl_result argument,
	which is  argument 4 of the C argument list

\end{funcdesc}

\begin{funcdesc}{ellint_E}{...}\index{ellint_E}

    Number of Input  Arguments:  3
    Number of Output Arguments:  1

 Argument 3 is a gsl_mode_t, valid parameters are:
	sf.PREC_DOUBLE or sf.PREC_SINGLE or sf.PREC_APPROX

\end{funcdesc}

\begin{funcdesc}{ellint_E_e}{...}\index{ellint_E_e}

    Number of Input  Arguments:  3
    Number of Output Arguments:  2

 Argument 3 is a gsl_mode_t, valid parameters are:
	sf.PREC_DOUBLE or sf.PREC_SINGLE or sf.PREC_APPROX
The error flag is discarded.
Return Arguments 1 and 2 resemble a gsl_result argument,
	which is  argument 3 of the C argument list

\end{funcdesc}

\begin{funcdesc}{ellint_Ecomp}{...}\index{ellint_Ecomp}

    Number of Input  Arguments:  2
    Number of Output Arguments:  1

 Argument 2 is a gsl_mode_t, valid parameters are:
	sf.PREC_DOUBLE or sf.PREC_SINGLE or sf.PREC_APPROX

\end{funcdesc}

\begin{funcdesc}{ellint_Ecomp_e}{...}\index{ellint_Ecomp_e}

    Number of Input  Arguments:  2
    Number of Output Arguments:  2

 Argument 2 is a gsl_mode_t, valid parameters are:
	sf.PREC_DOUBLE or sf.PREC_SINGLE or sf.PREC_APPROX
The error flag is discarded.
Return Arguments 1 and 2 resemble a gsl_result argument,
	which is  argument 2 of the C argument list

\end{funcdesc}

\begin{funcdesc}{ellint_F}{...}\index{ellint_F}

    Number of Input  Arguments:  3
    Number of Output Arguments:  1

 Argument 3 is a gsl_mode_t, valid parameters are:
	sf.PREC_DOUBLE or sf.PREC_SINGLE or sf.PREC_APPROX

\end{funcdesc}

\begin{funcdesc}{ellint_F_e}{...}\index{ellint_F_e}

    Number of Input  Arguments:  3
    Number of Output Arguments:  2

 Argument 3 is a gsl_mode_t, valid parameters are:
	sf.PREC_DOUBLE or sf.PREC_SINGLE or sf.PREC_APPROX
The error flag is discarded.
Return Arguments 1 and 2 resemble a gsl_result argument,
	which is  argument 3 of the C argument list

\end{funcdesc}

\begin{funcdesc}{ellint_Kcomp}{...}\index{ellint_Kcomp}

    Number of Input  Arguments:  2
    Number of Output Arguments:  1

 Argument 2 is a gsl_mode_t, valid parameters are:
	sf.PREC_DOUBLE or sf.PREC_SINGLE or sf.PREC_APPROX

\end{funcdesc}

\begin{funcdesc}{ellint_Kcomp_e}{...}\index{ellint_Kcomp_e}

    Number of Input  Arguments:  2
    Number of Output Arguments:  2

 Argument 2 is a gsl_mode_t, valid parameters are:
	sf.PREC_DOUBLE or sf.PREC_SINGLE or sf.PREC_APPROX
The error flag is discarded.
Return Arguments 1 and 2 resemble a gsl_result argument,
	which is  argument 2 of the C argument list

\end{funcdesc}

\begin{funcdesc}{ellint_P}{...}\index{ellint_P}

    Number of Input  Arguments:  4
    Number of Output Arguments:  1

 Argument 4 is a gsl_mode_t, valid parameters are:
	sf.PREC_DOUBLE or sf.PREC_SINGLE or sf.PREC_APPROX

\end{funcdesc}

\begin{funcdesc}{ellint_P_e}{...}\index{ellint_P_e}

    Number of Input  Arguments:  4
    Number of Output Arguments:  2

 Argument 4 is a gsl_mode_t, valid parameters are:
	sf.PREC_DOUBLE or sf.PREC_SINGLE or sf.PREC_APPROX
The error flag is discarded.
Return Arguments 1 and 2 resemble a gsl_result argument,
	which is  argument 4 of the C argument list

\end{funcdesc}

\begin{funcdesc}{ellint_RC}{...}\index{ellint_RC}

    Number of Input  Arguments:  3
    Number of Output Arguments:  1

 Argument 3 is a gsl_mode_t, valid parameters are:
	sf.PREC_DOUBLE or sf.PREC_SINGLE or sf.PREC_APPROX

\end{funcdesc}

\begin{funcdesc}{ellint_RC_e}{...}\index{ellint_RC_e}

    Number of Input  Arguments:  3
    Number of Output Arguments:  2

 Argument 3 is a gsl_mode_t, valid parameters are:
	sf.PREC_DOUBLE or sf.PREC_SINGLE or sf.PREC_APPROX
The error flag is discarded.
Return Arguments 1 and 2 resemble a gsl_result argument,
	which is  argument 3 of the C argument list

\end{funcdesc}

\begin{funcdesc}{ellint_RD}{...}\index{ellint_RD}

    Number of Input  Arguments:  4
    Number of Output Arguments:  1

 Argument 4 is a gsl_mode_t, valid parameters are:
	sf.PREC_DOUBLE or sf.PREC_SINGLE or sf.PREC_APPROX

\end{funcdesc}

\begin{funcdesc}{ellint_RD_e}{...}\index{ellint_RD_e}

    Number of Input  Arguments:  4
    Number of Output Arguments:  2

 Argument 4 is a gsl_mode_t, valid parameters are:
	sf.PREC_DOUBLE or sf.PREC_SINGLE or sf.PREC_APPROX
The error flag is discarded.
Return Arguments 1 and 2 resemble a gsl_result argument,
	which is  argument 4 of the C argument list

\end{funcdesc}

\begin{funcdesc}{ellint_RF}{...}\index{ellint_RF}

    Number of Input  Arguments:  4
    Number of Output Arguments:  1

 Argument 4 is a gsl_mode_t, valid parameters are:
	sf.PREC_DOUBLE or sf.PREC_SINGLE or sf.PREC_APPROX

\end{funcdesc}

\begin{funcdesc}{ellint_RF_e}{...}\index{ellint_RF_e}

    Number of Input  Arguments:  4
    Number of Output Arguments:  2

 Argument 4 is a gsl_mode_t, valid parameters are:
	sf.PREC_DOUBLE or sf.PREC_SINGLE or sf.PREC_APPROX
The error flag is discarded.
Return Arguments 1 and 2 resemble a gsl_result argument,
	which is  argument 4 of the C argument list

\end{funcdesc}

\begin{funcdesc}{ellint_RJ}{...}\index{ellint_RJ}

    Number of Input  Arguments:  5
    Number of Output Arguments:  1

 Argument 5 is a gsl_mode_t, valid parameters are:
	sf.PREC_DOUBLE or sf.PREC_SINGLE or sf.PREC_APPROX

\end{funcdesc}

\begin{funcdesc}{ellint_RJ_e}{...}\index{ellint_RJ_e}

    Number of Input  Arguments:  5
    Number of Output Arguments:  2

 Argument 5 is a gsl_mode_t, valid parameters are:
	sf.PREC_DOUBLE or sf.PREC_SINGLE or sf.PREC_APPROX
The error flag is discarded.
Return Arguments 1 and 2 resemble a gsl_result argument,
	which is  argument 5 of the C argument list

\end{funcdesc}

\begin{funcdesc}{elljac_e}{...}\index{elljac_e}

    Number of Input  Arguments:  2
    Number of Output Arguments:  3

The error flag is discarded.

\end{funcdesc}

\begin{funcdesc}{erf}{...}\index{erf}

    Number of Input  Arguments:  1
    Number of Output Arguments:  1
\end{funcdesc}

\begin{funcdesc}{erf_Q}{...}\index{erf_Q}

    Number of Input  Arguments:  1
    Number of Output Arguments:  1
\end{funcdesc}

\begin{funcdesc}{erf_Q_e}{...}\index{erf_Q_e}

    Number of Input  Arguments:  1
    Number of Output Arguments:  2

The error flag is discarded.
Return Arguments 1 and 2 resemble a gsl_result argument,
	which is  argument 1 of the C argument list

\end{funcdesc}

\begin{funcdesc}{erf_Z}{...}\index{erf_Z}

    Number of Input  Arguments:  1
    Number of Output Arguments:  1
\end{funcdesc}

\begin{funcdesc}{erf_Z_e}{...}\index{erf_Z_e}

    Number of Input  Arguments:  1
    Number of Output Arguments:  2

The error flag is discarded.
Return Arguments 1 and 2 resemble a gsl_result argument,
	which is  argument 1 of the C argument list

\end{funcdesc}

\begin{funcdesc}{erf_e}{...}\index{erf_e}

    Number of Input  Arguments:  1
    Number of Output Arguments:  2

The error flag is discarded.
Return Arguments 1 and 2 resemble a gsl_result argument,
	which is  argument 1 of the C argument list

\end{funcdesc}

\begin{funcdesc}{erfc}{...}\index{erfc}

    Number of Input  Arguments:  1
    Number of Output Arguments:  1
\end{funcdesc}

\begin{funcdesc}{erfc_e}{...}\index{erfc_e}

    Number of Input  Arguments:  1
    Number of Output Arguments:  2

The error flag is discarded.
Return Arguments 1 and 2 resemble a gsl_result argument,
	which is  argument 1 of the C argument list

\end{funcdesc}

\begin{funcdesc}{eta}{...}\index{eta}

    Number of Input  Arguments:  1
    Number of Output Arguments:  1
\end{funcdesc}

\begin{funcdesc}{eta_e}{...}\index{eta_e}

    Number of Input  Arguments:  1
    Number of Output Arguments:  2

The error flag is discarded.
Return Arguments 1 and 2 resemble a gsl_result argument,
	which is  argument 1 of the C argument list

\end{funcdesc}

\begin{funcdesc}{eta_int}{...}\index{eta_int}

    Number of Input  Arguments:  1
    Number of Output Arguments:  1
\end{funcdesc}

\begin{funcdesc}{eta_int_e}{...}\index{eta_int_e}

    Number of Input  Arguments:  1
    Number of Output Arguments:  2

The error flag is discarded.
Return Arguments 1 and 2 resemble a gsl_result argument,
	which is  argument 1 of the C argument list

\end{funcdesc}

\begin{funcdesc}{expint_3}{...}\index{expint_3}

    Number of Input  Arguments:  1
    Number of Output Arguments:  1
\end{funcdesc}

\begin{funcdesc}{expint_3_e}{...}\index{expint_3_e}

    Number of Input  Arguments:  1
    Number of Output Arguments:  2

The error flag is discarded.
Return Arguments 1 and 2 resemble a gsl_result argument,
	which is  argument 1 of the C argument list

\end{funcdesc}

\begin{funcdesc}{expint_E1}{...}\index{expint_E1}

    Number of Input  Arguments:  1
    Number of Output Arguments:  1
\end{funcdesc}

\begin{funcdesc}{expint_E1_e}{...}\index{expint_E1_e}

    Number of Input  Arguments:  1
    Number of Output Arguments:  2

The error flag is discarded.
Return Arguments 1 and 2 resemble a gsl_result argument,
	which is  argument 1 of the C argument list

\end{funcdesc}

\begin{funcdesc}{expint_E1_scaled}{...}\index{expint_E1_scaled}

    Number of Input  Arguments:  1
    Number of Output Arguments:  1
\end{funcdesc}

\begin{funcdesc}{expint_E1_scaled_e}{...}\index{expint_E1_scaled_e}

    Number of Input  Arguments:  1
    Number of Output Arguments:  2

The error flag is discarded.
Return Arguments 1 and 2 resemble a gsl_result argument,
	which is  argument 1 of the C argument list

\end{funcdesc}

\begin{funcdesc}{expint_E2}{...}\index{expint_E2}

    Number of Input  Arguments:  1
    Number of Output Arguments:  1
\end{funcdesc}

\begin{funcdesc}{expint_E2_e}{...}\index{expint_E2_e}

    Number of Input  Arguments:  1
    Number of Output Arguments:  2

The error flag is discarded.
Return Arguments 1 and 2 resemble a gsl_result argument,
	which is  argument 1 of the C argument list

\end{funcdesc}

\begin{funcdesc}{expint_E2_scaled}{...}\index{expint_E2_scaled}

    Number of Input  Arguments:  1
    Number of Output Arguments:  1
\end{funcdesc}

\begin{funcdesc}{expint_E2_scaled_e}{...}\index{expint_E2_scaled_e}

    Number of Input  Arguments:  1
    Number of Output Arguments:  2

The error flag is discarded.
Return Arguments 1 and 2 resemble a gsl_result argument,
	which is  argument 1 of the C argument list

\end{funcdesc}

\begin{funcdesc}{expint_Ei}{...}\index{expint_Ei}

    Number of Input  Arguments:  1
    Number of Output Arguments:  1
\end{funcdesc}

\begin{funcdesc}{expint_Ei_e}{...}\index{expint_Ei_e}

    Number of Input  Arguments:  1
    Number of Output Arguments:  2

The error flag is discarded.
Return Arguments 1 and 2 resemble a gsl_result argument,
	which is  argument 1 of the C argument list

\end{funcdesc}

\begin{funcdesc}{expint_Ei_scaled}{...}\index{expint_Ei_scaled}

    Number of Input  Arguments:  1
    Number of Output Arguments:  1
\end{funcdesc}

\begin{funcdesc}{expint_Ei_scaled_e}{...}\index{expint_Ei_scaled_e}

    Number of Input  Arguments:  1
    Number of Output Arguments:  2

The error flag is discarded.
Return Arguments 1 and 2 resemble a gsl_result argument,
	which is  argument 1 of the C argument list

\end{funcdesc}

\begin{funcdesc}{fact}{...}\index{fact}

    Number of Input  Arguments:  1
    Number of Output Arguments:  1
\end{funcdesc}

\begin{funcdesc}{fact_e}{...}\index{fact_e}

    Number of Input  Arguments:  1
    Number of Output Arguments:  2

The error flag is discarded.
Return Arguments 1 and 2 resemble a gsl_result argument,
	which is  argument 1 of the C argument list

\end{funcdesc}

\begin{funcdesc}{fermi_dirac_0}{...}\index{fermi_dirac_0}

    Number of Input  Arguments:  1
    Number of Output Arguments:  1
\end{funcdesc}

\begin{funcdesc}{fermi_dirac_0_e}{...}\index{fermi_dirac_0_e}

    Number of Input  Arguments:  1
    Number of Output Arguments:  2

The error flag is discarded.
Return Arguments 1 and 2 resemble a gsl_result argument,
	which is  argument 1 of the C argument list

\end{funcdesc}

\begin{funcdesc}{fermi_dirac_1}{...}\index{fermi_dirac_1}

    Number of Input  Arguments:  1
    Number of Output Arguments:  1
\end{funcdesc}

\begin{funcdesc}{fermi_dirac_1_e}{...}\index{fermi_dirac_1_e}

    Number of Input  Arguments:  1
    Number of Output Arguments:  2

The error flag is discarded.
Return Arguments 1 and 2 resemble a gsl_result argument,
	which is  argument 1 of the C argument list

\end{funcdesc}

\begin{funcdesc}{fermi_dirac_2}{...}\index{fermi_dirac_2}

    Number of Input  Arguments:  1
    Number of Output Arguments:  1
\end{funcdesc}

\begin{funcdesc}{fermi_dirac_2_e}{...}\index{fermi_dirac_2_e}

    Number of Input  Arguments:  1
    Number of Output Arguments:  2

The error flag is discarded.
Return Arguments 1 and 2 resemble a gsl_result argument,
	which is  argument 1 of the C argument list

\end{funcdesc}

\begin{funcdesc}{fermi_dirac_3half}{...}\index{fermi_dirac_3half}

    Number of Input  Arguments:  1
    Number of Output Arguments:  1
\end{funcdesc}

\begin{funcdesc}{fermi_dirac_3half_e}{...}\index{fermi_dirac_3half_e}

    Number of Input  Arguments:  1
    Number of Output Arguments:  2

The error flag is discarded.
Return Arguments 1 and 2 resemble a gsl_result argument,
	which is  argument 1 of the C argument list

\end{funcdesc}

\begin{funcdesc}{fermi_dirac_half}{...}\index{fermi_dirac_half}

    Number of Input  Arguments:  1
    Number of Output Arguments:  1
\end{funcdesc}

\begin{funcdesc}{fermi_dirac_half_e}{...}\index{fermi_dirac_half_e}

    Number of Input  Arguments:  1
    Number of Output Arguments:  2

The error flag is discarded.
Return Arguments 1 and 2 resemble a gsl_result argument,
	which is  argument 1 of the C argument list

\end{funcdesc}

\begin{funcdesc}{fermi_dirac_inc_0}{...}\index{fermi_dirac_inc_0}

    Number of Input  Arguments:  2
    Number of Output Arguments:  1
\end{funcdesc}

\begin{funcdesc}{fermi_dirac_inc_0_e}{...}\index{fermi_dirac_inc_0_e}

    Number of Input  Arguments:  2
    Number of Output Arguments:  2

The error flag is discarded.
Return Arguments 1 and 2 resemble a gsl_result argument,
	which is  argument 2 of the C argument list

\end{funcdesc}

\begin{funcdesc}{fermi_dirac_int}{...}\index{fermi_dirac_int}

    Number of Input  Arguments:  2
    Number of Output Arguments:  1
\end{funcdesc}

\begin{funcdesc}{fermi_dirac_int_e}{...}\index{fermi_dirac_int_e}

    Number of Input  Arguments:  2
    Number of Output Arguments:  2

The error flag is discarded.
Return Arguments 1 and 2 resemble a gsl_result argument,
	which is  argument 2 of the C argument list

\end{funcdesc}

\begin{funcdesc}{fermi_dirac_m1}{...}\index{fermi_dirac_m1}

    Number of Input  Arguments:  1
    Number of Output Arguments:  1
\end{funcdesc}

\begin{funcdesc}{fermi_dirac_m1_e}{...}\index{fermi_dirac_m1_e}

    Number of Input  Arguments:  1
    Number of Output Arguments:  2

The error flag is discarded.
Return Arguments 1 and 2 resemble a gsl_result argument,
	which is  argument 1 of the C argument list

\end{funcdesc}

\begin{funcdesc}{fermi_dirac_mhalf}{...}\index{fermi_dirac_mhalf}

    Number of Input  Arguments:  1
    Number of Output Arguments:  1
\end{funcdesc}

\begin{funcdesc}{fermi_dirac_mhalf_e}{...}\index{fermi_dirac_mhalf_e}

    Number of Input  Arguments:  1
    Number of Output Arguments:  2

The error flag is discarded.
Return Arguments 1 and 2 resemble a gsl_result argument,
	which is  argument 1 of the C argument list

\end{funcdesc}

\begin{funcdesc}{gamma}{...}\index{gamma}

    Number of Input  Arguments:  1
    Number of Output Arguments:  1
\end{funcdesc}

\begin{funcdesc}{gamma_e}{...}\index{gamma_e}

    Number of Input  Arguments:  1
    Number of Output Arguments:  2

The error flag is discarded.
Return Arguments 1 and 2 resemble a gsl_result argument,
	which is  argument 1 of the C argument list

\end{funcdesc}

\begin{funcdesc}{gamma_inc_P}{...}\index{gamma_inc_P}

    Number of Input  Arguments:  2
    Number of Output Arguments:  1
\end{funcdesc}

\begin{funcdesc}{gamma_inc_P_e}{...}\index{gamma_inc_P_e}

    Number of Input  Arguments:  2
    Number of Output Arguments:  2

The error flag is discarded.
Return Arguments 1 and 2 resemble a gsl_result argument,
	which is  argument 2 of the C argument list

\end{funcdesc}

\begin{funcdesc}{gamma_inc_Q}{...}\index{gamma_inc_Q}

    Number of Input  Arguments:  2
    Number of Output Arguments:  1
\end{funcdesc}

\begin{funcdesc}{gamma_inc_Q_e}{...}\index{gamma_inc_Q_e}

    Number of Input  Arguments:  2
    Number of Output Arguments:  2

The error flag is discarded.
Return Arguments 1 and 2 resemble a gsl_result argument,
	which is  argument 2 of the C argument list

\end{funcdesc}

\begin{funcdesc}{gammainv}{...}\index{gammainv}

    Number of Input  Arguments:  1
    Number of Output Arguments:  1
\end{funcdesc}

\begin{funcdesc}{gammainv_e}{...}\index{gammainv_e}

    Number of Input  Arguments:  1
    Number of Output Arguments:  2

The error flag is discarded.
Return Arguments 1 and 2 resemble a gsl_result argument,
	which is  argument 1 of the C argument list

\end{funcdesc}

\begin{funcdesc}{gammastar}{...}\index{gammastar}

    Number of Input  Arguments:  1
    Number of Output Arguments:  1
\end{funcdesc}

\begin{funcdesc}{gammastar_e}{...}\index{gammastar_e}

    Number of Input  Arguments:  1
    Number of Output Arguments:  2

The error flag is discarded.
Return Arguments 1 and 2 resemble a gsl_result argument,
	which is  argument 1 of the C argument list

\end{funcdesc}

\begin{funcdesc}{gegenpoly_1}{...}\index{gegenpoly_1}

    Number of Input  Arguments:  2
    Number of Output Arguments:  1
\end{funcdesc}

\begin{funcdesc}{gegenpoly_1_e}{...}\index{gegenpoly_1_e}

    Number of Input  Arguments:  2
    Number of Output Arguments:  2

The error flag is discarded.
Return Arguments 1 and 2 resemble a gsl_result argument,
	which is  argument 2 of the C argument list

\end{funcdesc}

\begin{funcdesc}{gegenpoly_2}{...}\index{gegenpoly_2}

    Number of Input  Arguments:  2
    Number of Output Arguments:  1
\end{funcdesc}

\begin{funcdesc}{gegenpoly_2_e}{...}\index{gegenpoly_2_e}

    Number of Input  Arguments:  2
    Number of Output Arguments:  2

The error flag is discarded.
Return Arguments 1 and 2 resemble a gsl_result argument,
	which is  argument 2 of the C argument list

\end{funcdesc}

\begin{funcdesc}{gegenpoly_3}{...}\index{gegenpoly_3}

    Number of Input  Arguments:  2
    Number of Output Arguments:  1
\end{funcdesc}

\begin{funcdesc}{gegenpoly_3_e}{...}\index{gegenpoly_3_e}

    Number of Input  Arguments:  2
    Number of Output Arguments:  2

The error flag is discarded.
Return Arguments 1 and 2 resemble a gsl_result argument,
	which is  argument 2 of the C argument list

\end{funcdesc}

\begin{funcdesc}{gegenpoly_n}{...}\index{gegenpoly_n}

    Number of Input  Arguments:  3
    Number of Output Arguments:  1
\end{funcdesc}

\begin{funcdesc}{gegenpoly_n_e}{...}\index{gegenpoly_n_e}

    Number of Input  Arguments:  3
    Number of Output Arguments:  2

The error flag is discarded.
Return Arguments 1 and 2 resemble a gsl_result argument,
	which is  argument 3 of the C argument list

\end{funcdesc}

\begin{funcdesc}{hydrogenicR}{...}\index{hydrogenicR}

    Number of Input  Arguments:  4
    Number of Output Arguments:  1
\end{funcdesc}

\begin{funcdesc}{hydrogenicR_1}{...}\index{hydrogenicR_1}

    Number of Input  Arguments:  2
    Number of Output Arguments:  1
\end{funcdesc}

\begin{funcdesc}{hydrogenicR_1_e}{...}\index{hydrogenicR_1_e}

    Number of Input  Arguments:  2
    Number of Output Arguments:  2

The error flag is discarded.
Return Arguments 1 and 2 resemble a gsl_result argument,
	which is  argument 2 of the C argument list

\end{funcdesc}

\begin{funcdesc}{hydrogenicR_e}{...}\index{hydrogenicR_e}

    Number of Input  Arguments:  4
    Number of Output Arguments:  2

The error flag is discarded.
Return Arguments 1 and 2 resemble a gsl_result argument,
	which is  argument 4 of the C argument list

\end{funcdesc}

\begin{funcdesc}{hyperg_0F1}{...}\index{hyperg_0F1}

    Number of Input  Arguments:  2
    Number of Output Arguments:  1
\end{funcdesc}

\begin{funcdesc}{hyperg_0F1_e}{...}\index{hyperg_0F1_e}

    Number of Input  Arguments:  2
    Number of Output Arguments:  2

The error flag is discarded.
Return Arguments 1 and 2 resemble a gsl_result argument,
	which is  argument 2 of the C argument list

\end{funcdesc}

\begin{funcdesc}{hyperg_1F1}{...}\index{hyperg_1F1}

    Number of Input  Arguments:  3
    Number of Output Arguments:  1
\end{funcdesc}

\begin{funcdesc}{hyperg_1F1_e}{...}\index{hyperg_1F1_e}

    Number of Input  Arguments:  3
    Number of Output Arguments:  2

The error flag is discarded.
Return Arguments 1 and 2 resemble a gsl_result argument,
	which is  argument 3 of the C argument list

\end{funcdesc}

\begin{funcdesc}{hyperg_1F1_int}{...}\index{hyperg_1F1_int}

    Number of Input  Arguments:  3
    Number of Output Arguments:  1
\end{funcdesc}

\begin{funcdesc}{hyperg_1F1_int_e}{...}\index{hyperg_1F1_int_e}

    Number of Input  Arguments:  3
    Number of Output Arguments:  2

The error flag is discarded.
Return Arguments 1 and 2 resemble a gsl_result argument,
	which is  argument 3 of the C argument list

\end{funcdesc}

\begin{funcdesc}{hyperg_2F0}{...}\index{hyperg_2F0}

    Number of Input  Arguments:  3
    Number of Output Arguments:  1
\end{funcdesc}

\begin{funcdesc}{hyperg_2F0_e}{...}\index{hyperg_2F0_e}

    Number of Input  Arguments:  3
    Number of Output Arguments:  2

The error flag is discarded.
Return Arguments 1 and 2 resemble a gsl_result argument,
	which is  argument 3 of the C argument list

\end{funcdesc}

\begin{funcdesc}{hyperg_2F1}{...}\index{hyperg_2F1}

    Number of Input  Arguments:  4
    Number of Output Arguments:  1
\end{funcdesc}

\begin{funcdesc}{hyperg_2F1_conj}{...}\index{hyperg_2F1_conj}

    Number of Input  Arguments:  4
    Number of Output Arguments:  1
\end{funcdesc}

\begin{funcdesc}{hyperg_2F1_conj_e}{...}\index{hyperg_2F1_conj_e}

    Number of Input  Arguments:  4
    Number of Output Arguments:  2

The error flag is discarded.
Return Arguments 1 and 2 resemble a gsl_result argument,
	which is  argument 4 of the C argument list

\end{funcdesc}

\begin{funcdesc}{hyperg_2F1_conj_renorm}{...}\index{hyperg_2F1_conj_renorm}

    Number of Input  Arguments:  4
    Number of Output Arguments:  1
\end{funcdesc}

\begin{funcdesc}{hyperg_2F1_conj_renorm_e}{...}\index{hyperg_2F1_conj_renorm_e}

    Number of Input  Arguments:  4
    Number of Output Arguments:  2

The error flag is discarded.
Return Arguments 1 and 2 resemble a gsl_result argument,
	which is  argument 4 of the C argument list

\end{funcdesc}

\begin{funcdesc}{hyperg_2F1_e}{...}\index{hyperg_2F1_e}

    Number of Input  Arguments:  4
    Number of Output Arguments:  2

The error flag is discarded.
Return Arguments 1 and 2 resemble a gsl_result argument,
	which is  argument 4 of the C argument list

\end{funcdesc}

\begin{funcdesc}{hyperg_2F1_renorm}{...}\index{hyperg_2F1_renorm}

    Number of Input  Arguments:  4
    Number of Output Arguments:  1
\end{funcdesc}

\begin{funcdesc}{hyperg_2F1_renorm_e}{...}\index{hyperg_2F1_renorm_e}

    Number of Input  Arguments:  4
    Number of Output Arguments:  2

The error flag is discarded.
Return Arguments 1 and 2 resemble a gsl_result argument,
	which is  argument 4 of the C argument list

\end{funcdesc}

\begin{funcdesc}{hyperg_U}{...}\index{hyperg_U}

    Number of Input  Arguments:  3
    Number of Output Arguments:  1
\end{funcdesc}

\begin{funcdesc}{hyperg_U_e}{...}\index{hyperg_U_e}

    Number of Input  Arguments:  3
    Number of Output Arguments:  2

The error flag is discarded.
Return Arguments 1 and 2 resemble a gsl_result argument,
	which is  argument 3 of the C argument list

\end{funcdesc}

\begin{funcdesc}{hyperg_U_e10_e}{...}\index{hyperg_U_e10_e}

    Number of Input  Arguments:  3
    Number of Output Arguments:  3

The error flag is discarded.
Return Arguments 1 - 3 resemble a gsl_result_e10 argument,
	which is argument 3 of the C argument list

\end{funcdesc}

\begin{funcdesc}{hyperg_U_int}{...}\index{hyperg_U_int}

    Number of Input  Arguments:  3
    Number of Output Arguments:  1
\end{funcdesc}

\begin{funcdesc}{hyperg_U_int_e}{...}\index{hyperg_U_int_e}

    Number of Input  Arguments:  3
    Number of Output Arguments:  2

The error flag is discarded.
Return Arguments 1 and 2 resemble a gsl_result argument,
	which is  argument 3 of the C argument list

\end{funcdesc}

\begin{funcdesc}{hyperg_U_int_e10_e}{...}\index{hyperg_U_int_e10_e}

    Number of Input  Arguments:  3
    Number of Output Arguments:  3

The error flag is discarded.
Return Arguments 1 - 3 resemble a gsl_result_e10 argument,
	which is argument 3 of the C argument list

\end{funcdesc}

\begin{funcdesc}{hypot}{...}\index{hypot}

    Number of Input  Arguments:  2
    Number of Output Arguments:  1
\end{funcdesc}

\begin{funcdesc}{hypot_e}{...}\index{hypot_e}

    Number of Input  Arguments:  2
    Number of Output Arguments:  2

The error flag is discarded.
Return Arguments 1 and 2 resemble a gsl_result argument,
	which is  argument 2 of the C argument list

\end{funcdesc}

\begin{funcdesc}{hzeta}{...}\index{hzeta}

    Number of Input  Arguments:  2
    Number of Output Arguments:  1
\end{funcdesc}

\begin{funcdesc}{hzeta_e}{...}\index{hzeta_e}

    Number of Input  Arguments:  2
    Number of Output Arguments:  2

The error flag is discarded.
Return Arguments 1 and 2 resemble a gsl_result argument,
	which is  argument 2 of the C argument list

\end{funcdesc}

\begin{funcdesc}{laguerre_1}{...}\index{laguerre_1}

    Number of Input  Arguments:  2
    Number of Output Arguments:  1
\end{funcdesc}

\begin{funcdesc}{laguerre_1_e}{...}\index{laguerre_1_e}

    Number of Input  Arguments:  2
    Number of Output Arguments:  2

The error flag is discarded.
Return Arguments 1 and 2 resemble a gsl_result argument,
	which is  argument 2 of the C argument list

\end{funcdesc}

\begin{funcdesc}{laguerre_2}{...}\index{laguerre_2}

    Number of Input  Arguments:  2
    Number of Output Arguments:  1
\end{funcdesc}

\begin{funcdesc}{laguerre_2_e}{...}\index{laguerre_2_e}

    Number of Input  Arguments:  2
    Number of Output Arguments:  2

The error flag is discarded.
Return Arguments 1 and 2 resemble a gsl_result argument,
	which is  argument 2 of the C argument list

\end{funcdesc}

\begin{funcdesc}{laguerre_3}{...}\index{laguerre_3}

    Number of Input  Arguments:  2
    Number of Output Arguments:  1
\end{funcdesc}

\begin{funcdesc}{laguerre_3_e}{...}\index{laguerre_3_e}

    Number of Input  Arguments:  2
    Number of Output Arguments:  2

The error flag is discarded.
Return Arguments 1 and 2 resemble a gsl_result argument,
	which is  argument 2 of the C argument list

\end{funcdesc}

\begin{funcdesc}{laguerre_n}{...}\index{laguerre_n}

    Number of Input  Arguments:  3
    Number of Output Arguments:  1
\end{funcdesc}

\begin{funcdesc}{laguerre_n_e}{...}\index{laguerre_n_e}

    Number of Input  Arguments:  3
    Number of Output Arguments:  2

The error flag is discarded.
Return Arguments 1 and 2 resemble a gsl_result argument,
	which is  argument 3 of the C argument list

\end{funcdesc}

\begin{funcdesc}{lambert_W0}{...}\index{lambert_W0}

    Number of Input  Arguments:  1
    Number of Output Arguments:  1
\end{funcdesc}

\begin{funcdesc}{lambert_W0_e}{...}\index{lambert_W0_e}

    Number of Input  Arguments:  1
    Number of Output Arguments:  2

The error flag is discarded.
Return Arguments 1 and 2 resemble a gsl_result argument,
	which is  argument 1 of the C argument list

\end{funcdesc}

\begin{funcdesc}{lambert_Wm1}{...}\index{lambert_Wm1}

    Number of Input  Arguments:  1
    Number of Output Arguments:  1
\end{funcdesc}

\begin{funcdesc}{lambert_Wm1_e}{...}\index{lambert_Wm1_e}

    Number of Input  Arguments:  1
    Number of Output Arguments:  2

The error flag is discarded.
Return Arguments 1 and 2 resemble a gsl_result argument,
	which is  argument 1 of the C argument list

\end{funcdesc}

\begin{funcdesc}{legendre_H3d}{...}\index{legendre_H3d}

    Number of Input  Arguments:  3
    Number of Output Arguments:  1
\end{funcdesc}

\begin{funcdesc}{legendre_H3d_0}{...}\index{legendre_H3d_0}

    Number of Input  Arguments:  2
    Number of Output Arguments:  1
\end{funcdesc}

\begin{funcdesc}{legendre_H3d_0_e}{...}\index{legendre_H3d_0_e}

    Number of Input  Arguments:  2
    Number of Output Arguments:  2

The error flag is discarded.
Return Arguments 1 and 2 resemble a gsl_result argument,
	which is  argument 2 of the C argument list

\end{funcdesc}

\begin{funcdesc}{legendre_H3d_1}{...}\index{legendre_H3d_1}

    Number of Input  Arguments:  2
    Number of Output Arguments:  1
\end{funcdesc}

\begin{funcdesc}{legendre_H3d_1_e}{...}\index{legendre_H3d_1_e}

    Number of Input  Arguments:  2
    Number of Output Arguments:  2

The error flag is discarded.
Return Arguments 1 and 2 resemble a gsl_result argument,
	which is  argument 2 of the C argument list

\end{funcdesc}

\begin{funcdesc}{legendre_H3d_e}{...}\index{legendre_H3d_e}

    Number of Input  Arguments:  3
    Number of Output Arguments:  2

The error flag is discarded.
Return Arguments 1 and 2 resemble a gsl_result argument,
	which is  argument 3 of the C argument list

\end{funcdesc}

\begin{funcdesc}{legendre_P1}{...}\index{legendre_P1}

    Number of Input  Arguments:  1
    Number of Output Arguments:  1
\end{funcdesc}

\begin{funcdesc}{legendre_P1_e}{...}\index{legendre_P1_e}

    Number of Input  Arguments:  1
    Number of Output Arguments:  2

The error flag is discarded.
Return Arguments 1 and 2 resemble a gsl_result argument,
	which is  argument 1 of the C argument list

\end{funcdesc}

\begin{funcdesc}{legendre_P2}{...}\index{legendre_P2}

    Number of Input  Arguments:  1
    Number of Output Arguments:  1
\end{funcdesc}

\begin{funcdesc}{legendre_P2_e}{...}\index{legendre_P2_e}

    Number of Input  Arguments:  1
    Number of Output Arguments:  2

The error flag is discarded.
Return Arguments 1 and 2 resemble a gsl_result argument,
	which is  argument 1 of the C argument list

\end{funcdesc}

\begin{funcdesc}{legendre_P3}{...}\index{legendre_P3}

    Number of Input  Arguments:  1
    Number of Output Arguments:  1
\end{funcdesc}

\begin{funcdesc}{legendre_P3_e}{...}\index{legendre_P3_e}

    Number of Input  Arguments:  1
    Number of Output Arguments:  2

The error flag is discarded.
Return Arguments 1 and 2 resemble a gsl_result argument,
	which is  argument 1 of the C argument list

\end{funcdesc}

\begin{funcdesc}{legendre_Pl}{...}\index{legendre_Pl}

    Number of Input  Arguments:  2
    Number of Output Arguments:  1
\end{funcdesc}

\begin{funcdesc}{legendre_Pl_e}{...}\index{legendre_Pl_e}

    Number of Input  Arguments:  2
    Number of Output Arguments:  2

The error flag is discarded.
Return Arguments 1 and 2 resemble a gsl_result argument,
	which is  argument 2 of the C argument list

\end{funcdesc}

\begin{funcdesc}{legendre_Plm}{...}\index{legendre_Plm}

    Number of Input  Arguments:  3
    Number of Output Arguments:  1
\end{funcdesc}

\begin{funcdesc}{legendre_Plm_e}{...}\index{legendre_Plm_e}

    Number of Input  Arguments:  3
    Number of Output Arguments:  2

The error flag is discarded.
Return Arguments 1 and 2 resemble a gsl_result argument,
	which is  argument 3 of the C argument list

\end{funcdesc}

\begin{funcdesc}{legendre_Q0}{...}\index{legendre_Q0}

    Number of Input  Arguments:  1
    Number of Output Arguments:  1
\end{funcdesc}

\begin{funcdesc}{legendre_Q0_e}{...}\index{legendre_Q0_e}

    Number of Input  Arguments:  1
    Number of Output Arguments:  2

The error flag is discarded.
Return Arguments 1 and 2 resemble a gsl_result argument,
	which is  argument 1 of the C argument list

\end{funcdesc}

\begin{funcdesc}{legendre_Q1}{...}\index{legendre_Q1}

    Number of Input  Arguments:  1
    Number of Output Arguments:  1
\end{funcdesc}

\begin{funcdesc}{legendre_Q1_e}{...}\index{legendre_Q1_e}

    Number of Input  Arguments:  1
    Number of Output Arguments:  2

The error flag is discarded.
Return Arguments 1 and 2 resemble a gsl_result argument,
	which is  argument 1 of the C argument list

\end{funcdesc}

\begin{funcdesc}{legendre_Ql}{...}\index{legendre_Ql}

    Number of Input  Arguments:  2
    Number of Output Arguments:  1
\end{funcdesc}

\begin{funcdesc}{legendre_Ql_e}{...}\index{legendre_Ql_e}

    Number of Input  Arguments:  2
    Number of Output Arguments:  2

The error flag is discarded.
Return Arguments 1 and 2 resemble a gsl_result argument,
	which is  argument 2 of the C argument list

\end{funcdesc}

\begin{funcdesc}{legendre_sphPlm}{...}\index{legendre_sphPlm}

    Number of Input  Arguments:  3
    Number of Output Arguments:  1
\end{funcdesc}

\begin{funcdesc}{legendre_sphPlm_e}{...}\index{legendre_sphPlm_e}

    Number of Input  Arguments:  3
    Number of Output Arguments:  2

The error flag is discarded.
Return Arguments 1 and 2 resemble a gsl_result argument,
	which is  argument 3 of the C argument list

\end{funcdesc}

\begin{funcdesc}{lnbeta}{...}\index{lnbeta}

    Number of Input  Arguments:  2
    Number of Output Arguments:  1
\end{funcdesc}

\begin{funcdesc}{lnbeta_e}{...}\index{lnbeta_e}

    Number of Input  Arguments:  2
    Number of Output Arguments:  2

The error flag is discarded.
Return Arguments 1 and 2 resemble a gsl_result argument,
	which is  argument 2 of the C argument list

\end{funcdesc}

\begin{funcdesc}{lnchoose}{...}\index{lnchoose}

    Number of Input  Arguments:  2
    Number of Output Arguments:  1
\end{funcdesc}

\begin{funcdesc}{lnchoose_e}{...}\index{lnchoose_e}

    Number of Input  Arguments:  2
    Number of Output Arguments:  2

The error flag is discarded.
Return Arguments 1 and 2 resemble a gsl_result argument,
	which is  argument 2 of the C argument list

\end{funcdesc}

\begin{funcdesc}{lncosh}{...}\index{lncosh}

    Number of Input  Arguments:  1
    Number of Output Arguments:  1
\end{funcdesc}

\begin{funcdesc}{lncosh_e}{...}\index{lncosh_e}

    Number of Input  Arguments:  1
    Number of Output Arguments:  2

The error flag is discarded.
Return Arguments 1 and 2 resemble a gsl_result argument,
	which is  argument 1 of the C argument list

\end{funcdesc}

\begin{funcdesc}{lndoublefact}{...}\index{lndoublefact}

    Number of Input  Arguments:  1
    Number of Output Arguments:  1
\end{funcdesc}

\begin{funcdesc}{lndoublefact_e}{...}\index{lndoublefact_e}

    Number of Input  Arguments:  1
    Number of Output Arguments:  2

The error flag is discarded.
Return Arguments 1 and 2 resemble a gsl_result argument,
	which is  argument 1 of the C argument list

\end{funcdesc}

\begin{funcdesc}{lnfact}{...}\index{lnfact}

    Number of Input  Arguments:  1
    Number of Output Arguments:  1
\end{funcdesc}

\begin{funcdesc}{lnfact_e}{...}\index{lnfact_e}

    Number of Input  Arguments:  1
    Number of Output Arguments:  2

The error flag is discarded.
Return Arguments 1 and 2 resemble a gsl_result argument,
	which is  argument 1 of the C argument list

\end{funcdesc}

\begin{funcdesc}{lngamma}{...}\index{lngamma}

    Number of Input  Arguments:  1
    Number of Output Arguments:  1
\end{funcdesc}

\begin{funcdesc}{lngamma_e}{...}\index{lngamma_e}

    Number of Input  Arguments:  1
    Number of Output Arguments:  2

The error flag is discarded.
Return Arguments 1 and 2 resemble a gsl_result argument,
	which is  argument 1 of the C argument list

\end{funcdesc}

\begin{funcdesc}{lngamma_sgn_e}{...}\index{lngamma_sgn_e}

    Number of Input  Arguments:  1
    Number of Output Arguments:  3

The error flag is discarded.
Return Arguments 1 and 2 resemble a gsl_result argument,
	which is  argument 1 of the C argument list

\end{funcdesc}

\begin{funcdesc}{lnpoch}{...}\index{lnpoch}

    Number of Input  Arguments:  2
    Number of Output Arguments:  1
\end{funcdesc}

\begin{funcdesc}{lnpoch_e}{...}\index{lnpoch_e}

    Number of Input  Arguments:  2
    Number of Output Arguments:  2

The error flag is discarded.
Return Arguments 1 and 2 resemble a gsl_result argument,
	which is  argument 2 of the C argument list

\end{funcdesc}

\begin{funcdesc}{lnpoch_sgn_e}{...}\index{lnpoch_sgn_e}

    Number of Input  Arguments:  2
    Number of Output Arguments:  3

The error flag is discarded.
Return Arguments 1 and 2 resemble a gsl_result argument,
	which is  argument 2 of the C argument list

\end{funcdesc}

\begin{funcdesc}{lnsinh}{...}\index{lnsinh}

    Number of Input  Arguments:  1
    Number of Output Arguments:  1
\end{funcdesc}

\begin{funcdesc}{lnsinh_e}{...}\index{lnsinh_e}

    Number of Input  Arguments:  1
    Number of Output Arguments:  2

The error flag is discarded.
Return Arguments 1 and 2 resemble a gsl_result argument,
	which is  argument 1 of the C argument list

\end{funcdesc}

\begin{funcdesc}{log}{...}\index{log}

    Number of Input  Arguments:  1
    Number of Output Arguments:  1
\end{funcdesc}

\begin{funcdesc}{log_1plusx}{...}\index{log_1plusx}

    Number of Input  Arguments:  1
    Number of Output Arguments:  1
\end{funcdesc}

\begin{funcdesc}{log_1plusx_e}{...}\index{log_1plusx_e}

    Number of Input  Arguments:  1
    Number of Output Arguments:  2

The error flag is discarded.
Return Arguments 1 and 2 resemble a gsl_result argument,
	which is  argument 1 of the C argument list

\end{funcdesc}

\begin{funcdesc}{log_1plusx_mx}{...}\index{log_1plusx_mx}

    Number of Input  Arguments:  1
    Number of Output Arguments:  1
\end{funcdesc}

\begin{funcdesc}{log_1plusx_mx_e}{...}\index{log_1plusx_mx_e}

    Number of Input  Arguments:  1
    Number of Output Arguments:  2

The error flag is discarded.
Return Arguments 1 and 2 resemble a gsl_result argument,
	which is  argument 1 of the C argument list

\end{funcdesc}

\begin{funcdesc}{log_abs}{...}\index{log_abs}

    Number of Input  Arguments:  1
    Number of Output Arguments:  1
\end{funcdesc}

\begin{funcdesc}{log_abs_e}{...}\index{log_abs_e}

    Number of Input  Arguments:  1
    Number of Output Arguments:  2

The error flag is discarded.
Return Arguments 1 and 2 resemble a gsl_result argument,
	which is  argument 1 of the C argument list

\end{funcdesc}

\begin{funcdesc}{log_e}{...}\index{log_e}

    Number of Input  Arguments:  1
    Number of Output Arguments:  2

The error flag is discarded.
Return Arguments 1 and 2 resemble a gsl_result argument,
	which is  argument 1 of the C argument list

\end{funcdesc}

\begin{funcdesc}{log_erfc}{...}\index{log_erfc}

    Number of Input  Arguments:  1
    Number of Output Arguments:  1
\end{funcdesc}

\begin{funcdesc}{log_erfc_e}{...}\index{log_erfc_e}

    Number of Input  Arguments:  1
    Number of Output Arguments:  2

The error flag is discarded.
Return Arguments 1 and 2 resemble a gsl_result argument,
	which is  argument 1 of the C argument list

\end{funcdesc}

\begin{funcdesc}{multiply}{...}\index{multiply}

    Number of Input  Arguments:  2
    Number of Output Arguments:  1
\end{funcdesc}

\begin{funcdesc}{multiply_e}{...}\index{multiply_e}

    Number of Input  Arguments:  2
    Number of Output Arguments:  2

The error flag is discarded.
Return Arguments 1 and 2 resemble a gsl_result argument,
	which is  argument 2 of the C argument list

\end{funcdesc}

\begin{funcdesc}{multiply_err_e}{...}\index{multiply_err_e}

    Number of Input  Arguments:  4
    Number of Output Arguments:  2

The error flag is discarded.
Return Arguments 1 and 2 resemble a gsl_result argument,
	which is  argument 4 of the C argument list

\end{funcdesc}

\begin{funcdesc}{poch}{...}\index{poch}

    Number of Input  Arguments:  2
    Number of Output Arguments:  1
\end{funcdesc}

\begin{funcdesc}{poch_e}{...}\index{poch_e}

    Number of Input  Arguments:  2
    Number of Output Arguments:  2

The error flag is discarded.
Return Arguments 1 and 2 resemble a gsl_result argument,
	which is  argument 2 of the C argument list

\end{funcdesc}

\begin{funcdesc}{pochrel}{...}\index{pochrel}

    Number of Input  Arguments:  2
    Number of Output Arguments:  1
\end{funcdesc}

\begin{funcdesc}{pochrel_e}{...}\index{pochrel_e}

    Number of Input  Arguments:  2
    Number of Output Arguments:  2

The error flag is discarded.
Return Arguments 1 and 2 resemble a gsl_result argument,
	which is  argument 2 of the C argument list

\end{funcdesc}

\begin{funcdesc}{polar_to_rect}{...}\index{polar_to_rect}

\end{funcdesc}

\begin{funcdesc}{pow_int}{...}\index{pow_int}

    Number of Input  Arguments:  2
    Number of Output Arguments:  1
\end{funcdesc}

\begin{funcdesc}{pow_int_e}{...}\index{pow_int_e}

    Number of Input  Arguments:  2
    Number of Output Arguments:  2

The error flag is discarded.
Return Arguments 1 and 2 resemble a gsl_result argument,
	which is  argument 2 of the C argument list

\end{funcdesc}

\begin{funcdesc}{psi}{...}\index{psi}

    Number of Input  Arguments:  1
    Number of Output Arguments:  1
\end{funcdesc}

\begin{funcdesc}{psi_1_int}{...}\index{psi_1_int}

    Number of Input  Arguments:  1
    Number of Output Arguments:  1
\end{funcdesc}

\begin{funcdesc}{psi_1_int_e}{...}\index{psi_1_int_e}

    Number of Input  Arguments:  1
    Number of Output Arguments:  2

The error flag is discarded.
Return Arguments 1 and 2 resemble a gsl_result argument,
	which is  argument 1 of the C argument list

\end{funcdesc}

\begin{funcdesc}{psi_1piy}{...}\index{psi_1piy}

    Number of Input  Arguments:  1
    Number of Output Arguments:  1
\end{funcdesc}

\begin{funcdesc}{psi_1piy_e}{...}\index{psi_1piy_e}

    Number of Input  Arguments:  1
    Number of Output Arguments:  2

The error flag is discarded.
Return Arguments 1 and 2 resemble a gsl_result argument,
	which is  argument 1 of the C argument list

\end{funcdesc}

\begin{funcdesc}{psi_e}{...}\index{psi_e}

    Number of Input  Arguments:  1
    Number of Output Arguments:  2

The error flag is discarded.
Return Arguments 1 and 2 resemble a gsl_result argument,
	which is  argument 1 of the C argument list

\end{funcdesc}

\begin{funcdesc}{psi_int}{...}\index{psi_int}

    Number of Input  Arguments:  1
    Number of Output Arguments:  1
\end{funcdesc}

\begin{funcdesc}{psi_int_e}{...}\index{psi_int_e}

    Number of Input  Arguments:  1
    Number of Output Arguments:  2

The error flag is discarded.
Return Arguments 1 and 2 resemble a gsl_result argument,
	which is  argument 1 of the C argument list

\end{funcdesc}

\begin{funcdesc}{psi_n}{...}\index{psi_n}

    Number of Input  Arguments:  2
    Number of Output Arguments:  1
\end{funcdesc}

\begin{funcdesc}{psi_n_e}{...}\index{psi_n_e}

    Number of Input  Arguments:  2
    Number of Output Arguments:  2

The error flag is discarded.
Return Arguments 1 and 2 resemble a gsl_result argument,
	which is  argument 2 of the C argument list

\end{funcdesc}

\begin{funcdesc}{rect_to_polar}{...}\index{rect_to_polar}

\end{funcdesc}

\begin{funcdesc}{sin}{...}\index{sin}

    Number of Input  Arguments:  1
    Number of Output Arguments:  1
\end{funcdesc}

\begin{funcdesc}{sin_e}{...}\index{sin_e}

    Number of Input  Arguments:  1
    Number of Output Arguments:  2

The error flag is discarded.
Return Arguments 1 and 2 resemble a gsl_result argument,
	which is  argument 1 of the C argument list

\end{funcdesc}

\begin{funcdesc}{sin_err_e}{...}\index{sin_err_e}

    Number of Input  Arguments:  2
    Number of Output Arguments:  2

The error flag is discarded.
Return Arguments 1 and 2 resemble a gsl_result argument,
	which is  argument 2 of the C argument list

\end{funcdesc}

\begin{funcdesc}{sinc}{...}\index{sinc}

    Number of Input  Arguments:  1
    Number of Output Arguments:  1
\end{funcdesc}

\begin{funcdesc}{sinc_e}{...}\index{sinc_e}

    Number of Input  Arguments:  1
    Number of Output Arguments:  2

The error flag is discarded.
Return Arguments 1 and 2 resemble a gsl_result argument,
	which is  argument 1 of the C argument list

\end{funcdesc}

\begin{funcdesc}{synchrotron_1}{...}\index{synchrotron_1}

    Number of Input  Arguments:  1
    Number of Output Arguments:  1
\end{funcdesc}

\begin{funcdesc}{synchrotron_1_e}{...}\index{synchrotron_1_e}

    Number of Input  Arguments:  1
    Number of Output Arguments:  2

The error flag is discarded.
Return Arguments 1 and 2 resemble a gsl_result argument,
	which is  argument 1 of the C argument list

\end{funcdesc}

\begin{funcdesc}{synchrotron_2}{...}\index{synchrotron_2}

    Number of Input  Arguments:  1
    Number of Output Arguments:  1
\end{funcdesc}

\begin{funcdesc}{synchrotron_2_e}{...}\index{synchrotron_2_e}

    Number of Input  Arguments:  1
    Number of Output Arguments:  2

The error flag is discarded.
Return Arguments 1 and 2 resemble a gsl_result argument,
	which is  argument 1 of the C argument list

\end{funcdesc}

\begin{funcdesc}{taylorcoeff}{...}\index{taylorcoeff}

    Number of Input  Arguments:  2
    Number of Output Arguments:  1
\end{funcdesc}

\begin{funcdesc}{taylorcoeff_e}{...}\index{taylorcoeff_e}

    Number of Input  Arguments:  2
    Number of Output Arguments:  2

The error flag is discarded.
Return Arguments 1 and 2 resemble a gsl_result argument,
	which is  argument 2 of the C argument list

\end{funcdesc}

\begin{funcdesc}{transport_2}{...}\index{transport_2}

    Number of Input  Arguments:  1
    Number of Output Arguments:  1
\end{funcdesc}

\begin{funcdesc}{transport_2_e}{...}\index{transport_2_e}

    Number of Input  Arguments:  1
    Number of Output Arguments:  2

The error flag is discarded.
Return Arguments 1 and 2 resemble a gsl_result argument,
	which is  argument 1 of the C argument list

\end{funcdesc}

\begin{funcdesc}{transport_3}{...}\index{transport_3}

    Number of Input  Arguments:  1
    Number of Output Arguments:  1
\end{funcdesc}

\begin{funcdesc}{transport_3_e}{...}\index{transport_3_e}

    Number of Input  Arguments:  1
    Number of Output Arguments:  2

The error flag is discarded.
Return Arguments 1 and 2 resemble a gsl_result argument,
	which is  argument 1 of the C argument list

\end{funcdesc}

\begin{funcdesc}{transport_4}{...}\index{transport_4}

    Number of Input  Arguments:  1
    Number of Output Arguments:  1
\end{funcdesc}

\begin{funcdesc}{transport_4_e}{...}\index{transport_4_e}

    Number of Input  Arguments:  1
    Number of Output Arguments:  2

The error flag is discarded.
Return Arguments 1 and 2 resemble a gsl_result argument,
	which is  argument 1 of the C argument list

\end{funcdesc}

\begin{funcdesc}{transport_5}{...}\index{transport_5}

    Number of Input  Arguments:  1
    Number of Output Arguments:  1
\end{funcdesc}

\begin{funcdesc}{transport_5_e}{...}\index{transport_5_e}

    Number of Input  Arguments:  1
    Number of Output Arguments:  2

The error flag is discarded.
Return Arguments 1 and 2 resemble a gsl_result argument,
	which is  argument 1 of the C argument list

\end{funcdesc}

\begin{funcdesc}{zeta}{...}\index{zeta}

    Number of Input  Arguments:  1
    Number of Output Arguments:  1
\end{funcdesc}

\begin{funcdesc}{zeta_e}{...}\index{zeta_e}

    Number of Input  Arguments:  1
    Number of Output Arguments:  2

The error flag is discarded.
Return Arguments 1 and 2 resemble a gsl_result argument,
	which is  argument 1 of the C argument list

\end{funcdesc}

\begin{funcdesc}{zeta_int}{...}\index{zeta_int}

    Number of Input  Arguments:  1
    Number of Output Arguments:  1
\end{funcdesc}

\begin{funcdesc}{zeta_int_e}{...}\index{zeta_int_e}

    Number of Input  Arguments:  1
    Number of Output Arguments:  2

The error flag is discarded.
Return Arguments 1 and 2 resemble a gsl_result argument,
	which is  argument 1 of the C argument list

\end{funcdesc}

\section{Ordinary Functions}

The following array functions have been wrapped. These are supposingly faster
than the equivalent functions from above.
\begin{funcdesc}{bessel_In_array}{...}\index{bessel_In_array}
\end{funcdesc}
\begin{funcdesc}{bessel_Jn_array}{...}\index{bessel_Jn_array}
\end{funcdesc}
\begin{funcdesc}{bessel_Kn_array}{...}\index{bessel_Kn_array}
\end{funcdesc}
\begin{funcdesc}{bessel_Kn_scaled_array}{...}\index{bessel_Kn_scaled_array}
\end{funcdesc}
\begin{funcdesc}{bessel_Yn_array}{...}\index{bessel_Yn_array}
\end{funcdesc}
\begin{funcdesc}{bessel_il_scaled_array}{...}\index{bessel_il_scaled_array}
\end{funcdesc}
\begin{funcdesc}{bessel_jl_array}{...}\index{bessel_jl_array}
\end{funcdesc}
\begin{funcdesc}{bessel_jl_steed_array}{...}\index{bessel_jl_steed_array}
\end{funcdesc}
\begin{funcdesc}{bessel_kl_scaled_array}{...}\index{bessel_kl_scaled_array}
\end{funcdesc}
\begin{funcdesc}{bessel_yl_array}{...}\index{bessel_yl_array}
\end{funcdesc}


%%% Local Variables: 
%%% mode: latex
%%% TeX-master: "ref"
%%% End: 


\appendix

\chapter[\protect\module{pygsl.ieee} --- Floating Point Unit Support]
{\protect\module{pygsl.ieee} \\ Floating Point Unit Support}
\label{cha:ieee-module}
\declaremodule{extension}{pygsl.ieee}
\moduleauthor{Achim G\"adke}{achimgaedke@users.sourceforge.net}

This chapter lists features to configure the ``Floating Point Unit'' of your machine.


\chapter[\protect\module{pygsl.init} --- Library initialisation]
{\protect\module{pygsl.init} \\ Library initialisation}
\label{cha:library-initialisation}
\declaremodule{extension}{pygsl.init}
\moduleauthor{Pierre Schnizer}{schnizer@users.sourceforge.net}
\moduleauthor{Achim G\"adke}{achimgaedke@users.sourceforge.net}

This module is called the first time when loading \module{pygsl}.
All following procedures are called once and before everything other.

\section{Exception handling}
\index{exception handling!initialisation} GSL provides a selectable error
handler, that is called for occuring errors (like domain errors, division by
zero, etc. ).  \module{pygsl.init} installs a handler by calling
\cfunction{gsl_set_error_handler} to set an appropiate exception from
\module{pygsl.errors}.  A \module{pygsl} interface function should return
\code{NULL} in case of an error, so the exception is raised.  If this handler
is called more than once before returning to python, only the first set
exception is raised.

Here is a python level example:
\begin{verbatim}
import pygsl.histogram
import pygsl.errors
hist=pygsl.histogram.histogram2d(100,100)
try:
   hist[-1,-1]=0
except pygsl.errors.gsl_Error,err:
   print err
\end{verbatim}
Will result
\begin{verbatim}
input domain error: index i lies outside valid range of 0 .. nx - 1
\end{verbatim}

\section{IEEE-mode}
\index{ieee-mode!initialisation}
The IEEE mode is set from the environment variable
 \envvar{GSL_IEEE_MODE} via \cfunction{gsl_ieee_env_setup()}.
After the initialisation use \module{pygsl.ieee} for manipulation.

\section{random number generators}
\index{random number generator!initialisation}
Also the random number generator can be initialised from the environment variables
 \envvar{GSL_RNG_TYPE}
and \envvar{GSL_RNG_SEED} using the gsl function \cfunction{gsl_rng_env_setup()}.
Each random number generators are initialised with \envvar{GSL_RNG_SEED}.

The default generator can be created by:\nopagebreak
\begin{verbatim}
import pygsl.rng
my_rng=pygsl.rng.rng()
print my_rng.name()
\end{verbatim}


\chapter[\protect\module{pygsl.errors} --- Error and warning classes]
{\protect\module{pygsl.errors} \\ Error and warning classes} 
\label{cha:error-module}
\declaremodule{standard}{pygsl.errors}
\moduleauthor{Pierre Schnizer}{schnizer@users.sourceforge.net}
\moduleauthor{Original Author: Achim G\"adke}{achimgaedke@users.sourceforge.net}

This chapter provides information about the \exception{gsl_Error} exception class that comes with this module.

\section{Exception Classes}


\begin{excclassdesc} {gsl_Error}{}
derived from \exception{Exception}, can be constructed with any object as parameter.
It is baseclass to all other \gsl{} Exceptions
\end{excclassdesc}
These classes are translations of the \file{<gsl/gsl_errno.h>} to python
exceptions.


\begin{excclassdesc}{gsl_ArithmeticError}{}
derived from \exception{gsl_Error} and \exception{exceptions.ArithmeticError},
base of all common arithmetic exceptions
\end{excclassdesc}

\begin{excclassdesc}{gsl_OverflowError}{}
derived from \exception{gsl_Error} and \exception{exceptions.OverflowError}
\end{excclassdesc}

\begin{excclassdesc}{gsl_ZeroDivisionError}{}
derived from \exception{gsl_Error} and \exception{exceptions.ZeroDivisionError}
\end{excclassdesc}

\begin{excclassdesc}{gsl_FloatingPointError}{}
derived from \exception{gsl_Error} and \exception{exceptions.FloatingPointError}
\end{excclassdesc}

\begin{excclassdesc}{gsl_ArithmeticError}{}
is derived from  \exception{gsl_Error} and from  \exception{ArithmeticError} .
This exception is the    base of all common arithmetic exceptions.
\end{excclassdesc}

\begin{excclassdesc}{gsl_AccuracyLossError}{}
is derived from  \exception{gsl_ArithmeticError} .
This exception is raised if the failed to reach the specified tolerance.
\end{excclassdesc}
\begin{excclassdesc}{gsl_BadFuncError}{}
is derived from  \exception{gsl_Error} .
This exception is raised if problem with a user-supplied function occur.
\end{excclassdesc}
\begin{excclassdesc}{gsl_BadLength}{}
is derived from  \exception{gsl_Error} .
This exception is raised if  matrix or  vector lengths are not conformant.
\end{excclassdesc}
\begin{excclassdesc}{gsl_BadToleranceError}{}
is derived from  \exception{gsl_Error} .
This exception is raised if user specified an tolerance which can not be reached.
\end{excclassdesc}
\begin{excclassdesc}{gsl_CacheLimitError}{}
is derived from  \exception{gsl_Error} .
This exception is raised if the    cache limit is exceeded.
\end{excclassdesc}
\begin{excclassdesc}{gsl_DivergeError}{}
is derived from  \exception{gsl_ArithmeticError} .
This exception is raised if an   integral or series is divergent.
\end{excclassdesc}
\begin{excclassdesc}{gsl_DomainError}{}
is derived from  \exception{gsl_Error} .
This exception is raised if    domain errors occure. e.g. sqrt(-1).
\end{excclassdesc}
\begin{excclassdesc}{gsl_EOFError}{}
is derived from  \exception{gsl_Error} and from  \exception{EOFError} .
This exception is raised if 
    end of file
     .
\end{excclassdesc}
\begin{excclassdesc}{gsl_FactorizationError}{}
is derived from  \exception{gsl_Error} .
This exception is raised if     factorization failed.
\end{excclassdesc}
\begin{excclassdesc}{gsl_FloatingPointError}{}
is derived from  \exception{gsl_Error} and from  \exception{FloatingPointError} .
\end{excclassdesc}
\begin{excclassdesc}{gsl_GenericError}{}
is derived from  \exception{gsl_Error} .
\end{excclassdesc}
\begin{excclassdesc}{gsl_InvalidArgumentError}{}
is derived from  \exception{gsl_Error} .
This exception is raised if an invalid argument is supplied by the user.
\end{excclassdesc}
\begin{excclassdesc}{gsl_JacobianEvaluationError}{}
is derived from  \exception{gsl_ArithmeticError} .
This exception is raised if jacobian evaluations are not improving the solution.
\end{excclassdesc}
\begin{excclassdesc}{gsl_MatrixNotSquare}{}
is derived from  \exception{gsl_Error} .
This exception is raised if the given matrix is not square.
\end{excclassdesc}
\begin{excclassdesc}{gsl_MaximumIterationError}{}
is derived from  \exception{gsl_ArithmeticError} .
This exception is raised if    the maximum number  of iterations is exceeded.
\end{excclassdesc}
\begin{excclassdesc}{gsl_NoHardwareSupportError}{}
is derived from  \exception{gsl_Error} .
This exception is raised if the requested feature is not supported by the hardware.
\end{excclassdesc}
\begin{excclassdesc}{gsl_NoProgressError}{}
is derived from  \exception{gsl_ArithmeticError} .
This exception is raised if the  iteration is not making progress towards solution.
\end{excclassdesc}
\begin{excclassdesc}{gsl_NotImplementedError}{}
is derived from  \exception{gsl_Error} and from  \exception{NotImplementedError} .
This exception is raised if  a requested feature is not (yet) implemented .
\end{excclassdesc}
\begin{excclassdesc}{gsl_OverflowError}{}
is derived from  \exception{gsl_Error} and from  \exception{OverflowError} .
\end{excclassdesc}
\begin{excclassdesc}{gsl_PointerError}{}
is derived from  \exception{gsl_Error} .
This exception is raised if an invalid pointer is found by the C wrapper code
or by the GSL library.
\end{excclassdesc}
\begin{excclassdesc}{gsl_RangeError}{}
is derived from  \exception{gsl_ArithmeticError} .
This exception is raised if     output would be out or range, e.g. exp(1e100)
     .
\end{excclassdesc}
\begin{excclassdesc}{gsl_RoundOffError}{}
is derived from  \exception{gsl_ArithmeticError} .
This exception is raised if  arithmetic failed because of roundoff error.
\end{excclassdesc}
\begin{excclassdesc}{gsl_RunAwayError}{}
is derived from  \exception{gsl_ArithmeticError} .
This exception is raised if   iterative process is out of control.
\end{excclassdesc}
\begin{excclassdesc}{gsl_SanityCheckError}{}
is derived from  \exception{gsl_Error} .
This exception is raised if a sanity check failed - shouldn't happen.
\end{excclassdesc}
\begin{excclassdesc}{gsl_SingularityError}{}
is derived from  \exception{gsl_ArithmeticError} .
This exception is raised if  an   apparent singularity is detected.
\end{excclassdesc}
\begin{excclassdesc}{gsl_TableLimitError}{}
is derived from  \exception{gsl_Error} .
This exception is raised if the table limit is exceeded.
\end{excclassdesc}
\begin{excclassdesc}{gsl_ToleranceError}{}
is derived from  \exception{gsl_ArithmeticError} .
This exception is raised if  the alghorithm failed to reach the specified tolerance.
\end{excclassdesc}
\begin{excclassdesc}{gsl_ToleranceFError}{}
is derived from  \exception{gsl_ArithmeticError} .
This exception is raised if  the alghorithm cannot reach the specified
tolerance in F (typically the variation of the evaluated function).
\end{excclassdesc}
\begin{excclassdesc}{gsl_ToleranceGradientError}{}
is derived from  \exception{gsl_ArithmeticError} .
This exception is raised if  cannot reach the specified tolerance for the gradient.
\end{excclassdesc}
\begin{excclassdesc}{gsl_ToleranceXError}{}
is derived from  \exception{gsl_ArithmeticError} .
This exception is raised if cannot reach the specified tolerance in X
(typically a search result).
\end{excclassdesc}
\begin{excclassdesc}{gsl_UnderflowError}{}
is derived from  \exception{gsl_Error} and from  \exception{OverflowError} .
\end{excclassdesc}
\begin{excclassdesc}{gsl_ZeroDivisionError}{}
is derived from  \exception{gsl_Error} and from  \exception{ZeroDivisionError} .
\end{excclassdesc}

All the above errors are just translations of the errno to python exceptions.

The following two are specific to pygsl:
\begin{excclassdesc}{pygsl.errors.pygsl_NotImplementedError}{}
is derived from  \exception{gsl_Error} and from  \exception{NotImplementedError} .
This exception is raised if a feature is requested but not
implemented. Currently only used if a module requests the debugging enviroment
of the init module, but the init module was not compiled with \code{\#define DEBUG=1}
\end{excclassdesc}
\begin{excclassdesc}{pygsl.errors.pygsl_StrideError}{}
is derived from  \exception{gsl_SanityCheckError} .
GSL uses as strides multiples of the basis type; for a vector or doubles, one
means from one double to the next. Numpy or numarray count the stride in
multiples of the size of a char. Therefore the stride has to be recalculated
before the approbriate \gsl{} function can be called. If that fails this
exception is raised.
\end{excclassdesc}

\section{Warning Classes}

\begin{excclassdesc} {gsl_Warning}{}
The dedicated warning class for \gsl{} has \exception{Warning} as base class.
\end{excclassdesc}

\begin{excclassdesc}{gsl_DomainWarning}{}
derived from \exception{gsl_Warning}, used by some \module{pygsl.histogram} functions
\end{excclassdesc}


\chapter{GNU Free Documentation License}
\label{cha:free-documentation-license}

Version 1.1, March 2000\\

 Copyright \copyright\ 2000  Free Software Foundation, Inc.\\
     59 Temple Place, Suite 330, Boston, MA  02111-1307  USA\\
 Everyone is permitted to copy and distribute verbatim copies
 of this license document, but changing it is not allowed.

\section*{Preamble}

The purpose of this License is to make a manual, textbook, or other
written document ``free'' in the sense of freedom: to assure everyone
the effective freedom to copy and redistribute it, with or without
modifying it, either commercially or noncommercially.  Secondarily,
this License preserves for the author and publisher a way to get
credit for their work, while not being considered responsible for
modifications made by others.

This License is a kind of ``copyleft'', which means that derivative
works of the document must themselves be free in the same sense.  It
complements the GNU General Public License, which is a copyleft
license designed for free software.

We have designed this License in order to use it for manuals for free
software, because free software needs free documentation: a free
program should come with manuals providing the same freedoms that the
software does.  But this License is not limited to software manuals;
it can be used for any textual work, regardless of subject matter or
whether it is published as a printed book.  We recommend this License
principally for works whose purpose is instruction or reference.

\section{Applicability and Definitions}

This License applies to any manual or other work that contains a
notice placed by the copyright holder saying it can be distributed
under the terms of this License.  The ``Document'', below, refers to any
such manual or work.  Any member of the public is a licensee, and is
addressed as ``you''.

A ``Modified Version'' of the Document means any work containing the
Document or a portion of it, either copied verbatim, or with
modifications and/or translated into another language.

A ``Secondary Section'' is a named appendix or a front-matter section of
the Document that deals exclusively with the relationship of the
publishers or authors of the Document to the Document's overall subject
(or to related matters) and contains nothing that could fall directly
within that overall subject.  (For example, if the Document is in part a
textbook of mathematics, a Secondary Section may not explain any
mathematics.)  The relationship could be a matter of historical
connection with the subject or with related matters, or of legal,
commercial, philosophical, ethical or political position regarding
them.

The ``Invariant Sections'' are certain Secondary Sections whose titles
are designated, as being those of Invariant Sections, in the notice
that says that the Document is released under this License.

The ``Cover Texts'' are certain short passages of text that are listed,
as Front-Cover Texts or Back-Cover Texts, in the notice that says that
the Document is released under this License.

A ``Transparent'' copy of the Document means a machine-readable copy,
represented in a format whose specification is available to the
general public, whose contents can be viewed and edited directly and
straightforwardly with generic text editors or (for images composed of
pixels) generic paint programs or (for drawings) some widely available
drawing editor, and that is suitable for input to text formatters or
for automatic translation to a variety of formats suitable for input
to text formatters.  A copy made in an otherwise Transparent file
format whose markup has been designed to thwart or discourage
subsequent modification by readers is not Transparent.  A copy that is
not ``Transparent'' is called ``Opaque''.

Examples of suitable formats for Transparent copies include plain
ASCII without markup, Texinfo input format, \LaTeX~input format, SGML
or XML using a publicly available DTD, and standard-conforming simple
HTML designed for human modification.  Opaque formats include
PostScript, PDF, proprietary formats that can be read and edited only
by proprietary word processors, SGML or XML for which the DTD and/or
processing tools are not generally available, and the
machine-generated HTML produced by some word processors for output
purposes only.

The ``Title Page'' means, for a printed book, the title page itself,
plus such following pages as are needed to hold, legibly, the material
this License requires to appear in the title page.  For works in
formats which do not have any title page as such, ``Title Page'' means
the text near the most prominent appearance of the work's title,
preceding the beginning of the body of the text.


\section{Verbatim Copying}

You may copy and distribute the Document in any medium, either
commercially or noncommercially, provided that this License, the
copyright notices, and the license notice saying this License applies
to the Document are reproduced in all copies, and that you add no other
conditions whatsoever to those of this License.  You may not use
technical measures to obstruct or control the reading or further
copying of the copies you make or distribute.  However, you may accept
compensation in exchange for copies.  If you distribute a large enough
number of copies you must also follow the conditions in section 3.

You may also lend copies, under the same conditions stated above, and
you may publicly display copies.


\section{Copying in Quantity}

If you publish printed copies of the Document numbering more than 100,
and the Document's license notice requires Cover Texts, you must enclose
the copies in covers that carry, clearly and legibly, all these Cover
Texts: Front-Cover Texts on the front cover, and Back-Cover Texts on
the back cover.  Both covers must also clearly and legibly identify
you as the publisher of these copies.  The front cover must present
the full title with all words of the title equally prominent and
visible.  You may add other material on the covers in addition.
Copying with changes limited to the covers, as long as they preserve
the title of the Document and satisfy these conditions, can be treated
as verbatim copying in other respects.

If the required texts for either cover are too voluminous to fit
legibly, you should put the first ones listed (as many as fit
reasonably) on the actual cover, and continue the rest onto adjacent
pages.

If you publish or distribute Opaque copies of the Document numbering
more than 100, you must either include a machine-readable Transparent
copy along with each Opaque copy, or state in or with each Opaque copy
a publicly-accessible computer-network location containing a complete
Transparent copy of the Document, free of added material, which the
general network-using public has access to download anonymously at no
charge using public-standard network protocols.  If you use the latter
option, you must take reasonably prudent steps, when you begin
distribution of Opaque copies in quantity, to ensure that this
Transparent copy will remain thus accessible at the stated location
until at least one year after the last time you distribute an Opaque
copy (directly or through your agents or retailers) of that edition to
the public.

It is requested, but not required, that you contact the authors of the
Document well before redistributing any large number of copies, to give
them a chance to provide you with an updated version of the Document.


\section{Modifications}

You may copy and distribute a Modified Version of the Document under
the conditions of sections 2 and 3 above, provided that you release
the Modified Version under precisely this License, with the Modified
Version filling the role of the Document, thus licensing distribution
and modification of the Modified Version to whoever possesses a copy
of it.  In addition, you must do these things in the Modified Version:

\begin{itemize}

\item Use in the Title Page (and on the covers, if any) a title distinct
   from that of the Document, and from those of previous versions
   (which should, if there were any, be listed in the History section
   of the Document).  You may use the same title as a previous version
   if the original publisher of that version gives permission.
\item List on the Title Page, as authors, one or more persons or entities
   responsible for authorship of the modifications in the Modified
   Version, together with at least five of the principal authors of the
   Document (all of its principal authors, if it has less than five).
\item State on the Title page the name of the publisher of the
   Modified Version, as the publisher.
\item Preserve all the copyright notices of the Document.
\item Add an appropriate copyright notice for your modifications
   adjacent to the other copyright notices.
\item Include, immediately after the copyright notices, a license notice
   giving the public permission to use the Modified Version under the
   terms of this License, in the form shown in the Addendum below.
\item Preserve in that license notice the full lists of Invariant Sections
   and required Cover Texts given in the Document's license notice.
\item Include an unaltered copy of this License.
\item Preserve the section entitled ``History'', and its title, and add to
   it an item stating at least the title, year, new authors, and
   publisher of the Modified Version as given on the Title Page.  If
   there is no section entitled ``History'' in the Document, create one
   stating the title, year, authors, and publisher of the Document as
   given on its Title Page, then add an item describing the Modified
   Version as stated in the previous sentence.
\item Preserve the network location, if any, given in the Document for
   public access to a Transparent copy of the Document, and likewise
   the network locations given in the Document for previous versions
   it was based on.  These may be placed in the ``History'' section.
   You may omit a network location for a work that was published at
   least four years before the Document itself, or if the original
   publisher of the version it refers to gives permission.
\item In any section entitled ``Acknowledgements'' or ``Dedications'',
   preserve the section's title, and preserve in the section all the
   substance and tone of each of the contributor acknowledgements
   and/or dedications given therein.
\item Preserve all the Invariant Sections of the Document,
   unaltered in their text and in their titles.  Section numbers
   or the equivalent are not considered part of the section titles.
\item Delete any section entitled ``Endorsements''.  Such a section
   may not be included in the Modified Version.
\item Do not retitle any existing section as ``Endorsements''
   or to conflict in title with any Invariant Section.

\end{itemize}

If the Modified Version includes new front-matter sections or
appendices that qualify as Secondary Sections and contain no material
copied from the Document, you may at your option designate some or all
of these sections as invariant.  To do this, add their titles to the
list of Invariant Sections in the Modified Version's license notice.
These titles must be distinct from any other section titles.

You may add a section entitled ``Endorsements'', provided it contains
nothing but endorsements of your Modified Version by various
parties -- for example, statements of peer review or that the text has
been approved by an organization as the authoritative definition of a
standard.

You may add a passage of up to five words as a Front-Cover Text, and a
passage of up to 25 words as a Back-Cover Text, to the end of the list
of Cover Texts in the Modified Version.  Only one passage of
Front-Cover Text and one of Back-Cover Text may be added by (or
through arrangements made by) any one entity.  If the Document already
includes a cover text for the same cover, previously added by you or
by arrangement made by the same entity you are acting on behalf of,
you may not add another; but you may replace the old one, on explicit
permission from the previous publisher that added the old one.

The author(s) and publisher(s) of the Document do not by this License
give permission to use their names for publicity for or to assert or
imply endorsement of any Modified Version.


\section{Combining Documents}

You may combine the Document with other documents released under this
License, under the terms defined in section 4 above for modified
versions, provided that you include in the combination all of the
Invariant Sections of all of the original documents, unmodified, and
list them all as Invariant Sections of your combined work in its
license notice.

The combined work need only contain one copy of this License, and
multiple identical Invariant Sections may be replaced with a single
copy.  If there are multiple Invariant Sections with the same name but
different contents, make the title of each such section unique by
adding at the end of it, in parentheses, the name of the original
author or publisher of that section if known, or else a unique number.
Make the same adjustment to the section titles in the list of
Invariant Sections in the license notice of the combined work.

In the combination, you must combine any sections entitled ``History''
in the various original documents, forming one section entitled
``History''; likewise combine any sections entitled ``Acknowledgements'',
and any sections entitled ``Dedications''.  You must delete all sections
entitled ``Endorsements.''


\section{Collections of Documents}

You may make a collection consisting of the Document and other documents
released under this License, and replace the individual copies of this
License in the various documents with a single copy that is included in
the collection, provided that you follow the rules of this License for
verbatim copying of each of the documents in all other respects.

You may extract a single document from such a collection, and distribute
it individually under this License, provided you insert a copy of this
License into the extracted document, and follow this License in all
other respects regarding verbatim copying of that document.



\section{Aggregation With Independent Works}

A compilation of the Document or its derivatives with other separate
and independent documents or works, in or on a volume of a storage or
distribution medium, does not as a whole count as a Modified Version
of the Document, provided no compilation copyright is claimed for the
compilation.  Such a compilation is called an ``aggregate'', and this
License does not apply to the other self-contained works thus compiled
with the Document, on account of their being thus compiled, if they
are not themselves derivative works of the Document.

If the Cover Text requirement of section 3 is applicable to these
copies of the Document, then if the Document is less than one quarter
of the entire aggregate, the Document's Cover Texts may be placed on
covers that surround only the Document within the aggregate.
Otherwise they must appear on covers around the whole aggregate.


\section{Translation}

Translation is considered a kind of modification, so you may
distribute translations of the Document under the terms of section 4.
Replacing Invariant Sections with translations requires special
permission from their copyright holders, but you may include
translations of some or all Invariant Sections in addition to the
original versions of these Invariant Sections.  You may include a
translation of this License provided that you also include the
original English version of this License.  In case of a disagreement
between the translation and the original English version of this
License, the original English version will prevail.


\section{Termination}

You may not copy, modify, sublicense, or distribute the Document except
as expressly provided for under this License.  Any other attempt to
copy, modify, sublicense or distribute the Document is void, and will
automatically terminate your rights under this License.  However,
parties who have received copies, or rights, from you under this
License will not have their licenses terminated so long as such
parties remain in full compliance.


\section{Future Revisions of This License}

The Free Software Foundation may publish new, revised versions
of the GNU Free Documentation License from time to time.  Such new
versions will be similar in spirit to the present version, but may
differ in detail to address new problems or concerns. See
http://www.gnu.org/copyleft/.

Each version of the License is given a distinguishing version number.
If the Document specifies that a particular numbered version of this
License "or any later version" applies to it, you have the option of
following the terms and conditions either of that specified version or
of any later version that has been published (not as a draft) by the
Free Software Foundation.  If the Document does not specify a version
number of this License, you may choose any version ever published (not
as a draft) by the Free Software Foundation.

% Complete documentation on the extended LaTeX markup used for Python
% documentation is available in ``Documenting Python'', which is part
% of the standard documentation for Python.  It may be found online
% at:
%
%     http://www.python.org/doc/current/doc/doc.html

\documentclass[hyperref]{manual}

% latex2html doesn't know [T1]{fontenc}, so we cannot use that:(
\usepackage{amsmath}
\usepackage[latin1]{inputenc}
\usepackage{textcomp}


% this version does not reset module names at section level
%begin{latexonly}
\makeatletter
\let\py@OldOldChapter=\chapter
\renewcommand{\chapter}{\py@reset%
                        \py@OldOldChapter}
\renewcommand{\section}{\@startsection{section}{1}{\z@}%
   {-3.5ex \@plus -1ex \@minus -.2ex}%
   {2.3ex \@plus.2ex}%
   {\reset@font\Large\py@HeaderFamily}}
\makeatother
%end{latexonly}


% some convenience declarations
\newcommand{\gsl}{GSL}
\newcommand{\GSL}{GNU Scientific Library}
\newcommand{\numpy}{NumPy}
\newcommand{\NUMPY}{Numerical Python}
\newcommand{\pygsl}{PyGSL}
\newcommand{\PYGSL}{PyGSL: Python wrapper of the GNU Scientific Library}


\title{PyGSL Reference Manual}

\ifhtml
\author{%
   \ulink{Achim G\"adke}{mailto:achimgaedke@users.sourceforge.net}\\
   Center for Applied Informatics, Cologne \\
   \ulink{Jochen K\"upper}{mailto:jochen@jochen-kuepper.de}\\
   Fritz-Haber-Institut der MPG, Berlin
   \ulink{Sebastien Maret}{mailto:schnizer@users.sourceforge.net}\\
   Gesellschaft f�r Schwerionenforschung Darmstadt.
   \ulink{Pierre Schnizer}{mailto:schnizer@users.sourceforge.net}\\
   Gesellschaft f�r Schwerionenforschung, Darmstadt.
}%
\else
%begin{latexonly}
%% pdfelatex (TeXLive 7) doesn't handle \footnotemark in here...
\author{Achim G\"adke \\ Jochen K\"upper \\ Sebastien Maret \\ Pierre Schnizer}
% Please at least include a long-lived email address!
\authoraddress{
   Center for Applied Informatics, Cologne \\
   \email{achimgaedke@users.sourceforge.net} \\[2mm]
   Fritz-Haber-Institut der MPG, Berlin \\
   \email{jochen@jochen-kuepper.de} \\
      Gesellschaft f�r Schwerionenforschung, Darmstadt\\
   \email{schnizer@users.sourceforge.net}\\
}
%end{latexonly}
\fi

\date{January, 2005}            % update before release!
                                % Use an explicit date so that reformatting
                                % doesn't cause a new date to be used.  Setting
                                % the date to \today can be used during draft
                                % stages to make it easier to handle versions.
\release{0.2}                   % release version; this is used to define the
\setshortversion{0.2}           % \version macro
\makeindex                      % tell \index to actually write the .idx file


\begin{document}

\maketitle

% This makes the contents more accessible from the front page of the HTML.
\ifhtml
\chapter*{Front Matter}
\label{front}
\fi

\input{copyright}

\begin{abstract}
   \noindent
   pygsl grants python users access to the GNU scientific library.  The latest
   version can be found at the project homepage, \url{http://pygsl.sf.net}.

   \textbf{Implemented features:} \\
   \begin{tabular}{ll}
     \module{pygsl.blas}                & basic linear algebra system\\
     \module{pygsl.chebyshev}           & chebyshev approximations\\
     \module{pygsl.combination}         & combinations  \\
     \module{pygsl.const}               & $>200$ often used mathematical and
                                          scientific constants. \\
     \module{pygsl.diff}                & (Deprecated. Use pygsl.deriv instead). \\
     \module{pygsl.deriv}               & Numerical differentiation. \\
     \module{pygsl.eigen}               &\\
     \module{pygsl.fit}                 &\\
     \module{pygsl.histogram}          & 1d and 2d histograms and operations
                                          on histograms. \\
     \module{pygsl.ieee}                & Access to the ieee-arithmetics layer
                                          of gsl. \\ 
     \module{pygsl.integrate}           &\\
     \module{pygsl.interpolation}       &\\ 
     \module{pygsl.linalg}              &\\
     \module{pygsl.math}                &\\
     \module{pygsl.monte}               &\\
     \module{pygsl.minimize}            &\\
     \module{pygsl.multifit}            &\\
     \module{pygsl.multifit_nlin}       &\\    
     \module{pygsl.multimin}            &\\
     \module{pygsl.multiroots}          &\\ 
     \module{pygsl.odeiv}               &\\
     \module{pygsl.permutation}         &\\  
     \module{pygsl.poly}                &\\
     \module{pygsl.qrng}                &\\
     \module{pygsl.rng}                 & random number generators and probability densities. \\
     \module{pygsl.roots}               &\\
     \module{pygsl.siman}               &Simulated anealing\\
     \module{pygsl.sf}                  & $>200$ special functions. \\
     \module{pygsl.statistics}          & Statistical functions. \\
\end{tabular}

\end{abstract}


\tableofcontents


\chapter{System Requirements, Installation}
\label{cha:system-req-installation}
\input{install}

\paragraph*{Acknowledgment}
\label{sec:acknowledgment}
Parts of this this manual are based on the \GSL{} reference manual.


\chapter[\protect\module{pygsl.const} --- Mathematical and scientific
constants]{\protect\module{pygsl.const} \\ Mathematical and scientific
constants} 
\label{cha:const-module}
\input{const}
\input{chebyshev}
\input{differentiation}

\chapter[\protect\module{pygsl.histogram} --- Histogram Types]
{\protect\module{pygsl.histogram} \\ Histogram Types}
\label{cha:histogram-module}
\input{histogram}

\chapter[\protect\module{pygsl.rng} --- Random Number Generators]
{\protect\module{pygsl.rng} \\ Random Number Generators}
\label{cha:rng-module}
\input{rng}

\chapter[\protect\module{pygsl.sf} --- Special Functions]
{\protect\module{pygsl.sf} \\ Special Functions}
\label{cha:sf-module}
\input{sf}
\input{statistics}

\input{testing}

\appendix

\chapter[\protect\module{pygsl.ieee} --- Floating Point Unit Support]
{\protect\module{pygsl.ieee} \\ Floating Point Unit Support}
\label{cha:ieee-module}
\input{ieee}

\chapter[\protect\module{pygsl.init} --- Library initialisation]
{\protect\module{pygsl.init} \\ Library initialisation}
\label{cha:library-initialisation}
\input{init}

\chapter[\protect\module{pygsl.errors} --- Error and warning classes]
{\protect\module{pygsl.errors} \\ Error and warning classes} 
\label{cha:error-module}
\input{errors}

\input{freedoc}
\input{ref.ind}                    % Index

\end{document}


%% Local Variables:
%% mode: LaTeX
%% mode: auto-fill
%% fill-column: 79
%% indent-tabs-mode: nil
%% ispell-dictionary: "american"
%% reftex-fref-is-default: nil
%% TeX-auto-save: t
%% TeX-command-default: "pdfeLaTeX"
%% TeX-parse-self: t
%% End:
                    % Index

\end{document}


%% Local Variables:
%% mode: LaTeX
%% mode: auto-fill
%% fill-column: 79
%% indent-tabs-mode: nil
%% ispell-dictionary: "american"
%% reftex-fref-is-default: nil
%% TeX-auto-save: t
%% TeX-command-default: "pdfeLaTeX"
%% TeX-parse-self: t
%% End:
                    % Index

\end{document}


%% Local Variables:
%% mode: LaTeX
%% mode: auto-fill
%% fill-column: 79
%% indent-tabs-mode: nil
%% ispell-dictionary: "american"
%% reftex-fref-is-default: nil
%% TeX-auto-save: t
%% TeX-command-default: "pdfeLaTeX"
%% TeX-parse-self: t
%% End:
                    % Index

\end{document}


%% Local Variables:
%% mode: LaTeX
%% mode: auto-fill
%% fill-column: 79
%% indent-tabs-mode: nil
%% ispell-dictionary: "american"
%% reftex-fref-is-default: nil
%% TeX-auto-save: t
%% TeX-command-default: "pdfeLaTeX"
%% TeX-parse-self: t
%% End:
