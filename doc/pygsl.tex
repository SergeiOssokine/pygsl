% Complete documentation on the extended LaTeX markup used for Python
% documentation is available in ``Documenting Python'', which is part
% of the standard documentation for Python.  It may be found online
% at:
%
%     http://www.python.org/doc/current/doc/doc.html

\documentclass{manual}

\title{pygsl Manual}

\author{Achim G\"adke, Jochen K\"upper}

% Please at least include a long-lived email address;
% the rest is at your discretion.
\authoraddress{
	Center for applied informatics, Cologne\\
	Email: \email{achimgaedke@users.sourceforge.net}
	University of North Carolina\\
	Email: \email{jochen@jochen-kuepper.de}

}

\date{January, 2002}		% update before release!
				% Use an explicit date so that reformatting
				% doesn't cause a new date to be used.  Setting
				% the date to \today can be used during draft
				% stages to make it easier to handle versions.

\release{0.0.5}			% release version; this is used to define the
				% \version macro

\makeindex			% tell \index to actually write the .idx file

\begin{document}

\maketitle

% This makes the contents more accessible from the front page of the HTML.
\ifhtml
\chapter*{Front Matter\label{front}}
\fi

Copyright \copyright{} 2002 The pygsl Team.

Permission is granted to copy, distribute and/or modify this document under the
terms of the GNU Free Documentation License, Version 1.1 or any later version
published by the Free Software Foundation; with no Invariant Sections, no
Front-Cover Texts, and no Back-Cover Texts.  A copy of the license is included
in section \ref{cha:free-documentation-license} entitled ``GNU Free
Documentation License''.


%% Local Variables:
%% mode: LaTeX
%% mode: auto-fill
%% fill-column: 79
%% indent-tabs-mode: nil
%% ispell-dictionary: "american"
%% reftex-fref-is-default: nil
%% TeX-auto-save: t
%% TeX-command-default: "pdfeLaTeX"
%% TeX-master: "pygsl"
%% TeX-parse-self: t
%% End:


\begin{abstract}

\noindent
pygsl grants python users access to the GNU scientific library.
More information about pygsl can be found at the project homepage,\url{http://pygsl.sf.net}.
\end{abstract}

\tableofcontents


\chapter{Statistics Module}
\declaremodule{extension}{pygsl.statistics}
\moduleauthor{Jochen K\"upper}{jochen@jochen-kuepper.de}
\index{mean}
\index{standard deviation}
\index{variance}
\index{estimated standard deviation}
\index{estimated variance}
\index{t-test}
\index{range}
\index{min}
\index{max}

This chapter describes the statistical functions in the library.  The
basic statistical functions include routines to compute the mean,
variance and standard deviation. More advanced functions allow you to
calculate absolute deviations, skewness, and kurtosis as well as the
median and arbitrary percentiles.  The algorithms use recurrence
relations to compute average quantities in a stable way, without large
intermediate values that might overflow. 

All functions work on any Python sequence (of the appropriate
data-type), but see section \ref{sec:stat-speed-considerations} for
advantages and drawbacks of different kinds of input data.


\section{Organization of the module}
\label{sec:stat-organization}

The parts of the GSL functions names, providing artificial name spaces,
are mapped to modules and submodules in pygsl.  That is
\code{gsl_stats_mean} can be found as \code{pygsl.statistics.mean} and
\code{gsl_stats_long_mean} as \code{pygsl.statistics.long.mean}.

The functions in the module are available in versions for datasets in
the standard floating-point and integer types. The generic versions
available in the \code{pygsl.statistics} module are using the generic
GSL \code{double} versions.  The submodules use GSL functions according
to the submodule name, e.g. long for \code{pygsl.statistics.long}.

Currently implemented submodules are \code{pygsl.statistics.double} and
\code{pygsl.statistics.long}.



\section{Speed considerations}
\label{sec:stat-speed-considerations}

All functions work on any Python sequence type but are optimized for use
with NumPy arrays. It is strongly suggested that you install NumPy
(available from \url{http://www.numpy.org})!

If you pass NumPy arrays of the \emph{correct data-type} as input data
to any of the functions they are passed straight to the C functions
along with the stride information of the data.

If you pass generic (non-NumPy) Python sequences or NumPy arrays of the
wrong data-type a suitable copy of the data will be created and passed
to the function.


\section{Further Reading}
\label{sec:stat-further-reading}

See the gsl reference manual for a description of all available
functions and the calculations they perform.


%% Local Variables:
%% mode: LaTeX
%% mode: auto-fill
%% fill-column: 79
%% ispell-dictionary: "american"
%% reftex-fref-is-default: nil
%% TeX-auto-save: t
%% TeX-command-default: "pdfeLaTeX"
%% TeX-master: "pygsl"
%% TeX-parse-self: t
%% End:


\chapter{Histogram Module}



%% Local Variables:
%% mode: LaTeX
%% mode: auto-fill
%% fill-column: 79
%% ispell-dictionary: "american"
%% reftex-fref-is-default: nil
%% TeX-auto-save: t
%% TeX-command-default: "pdfeLaTeX"
%% TeX-master: "pygsl"
%% TeX-parse-self: t
%% End:


% \appendix
% \chapter{...}
% My appendix.
% The \code{\e appendix} markup need not be repeated for additional
% appendices.

\input{\jobname.ind}			% Index

\end{document}
