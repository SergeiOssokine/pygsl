% Complete documentation on the extended LaTeX markup used for Python
% documentation is available in ``Documenting Python'', which is part
% of the standard documentation for Python.  It may be found online
% at:
%
%     http://www.python.org/doc/current/doc/doc.html

\documentclass{manual}

% some convenience declarations
\newcommand{\gsl}{GSL}
\newcommand{\GSL}{GNU Scientific Library}
\newcommand{\numpy}{NumPy}
\newcommand{\NUMPY}{Numerical Python}
\newcommand{\pygsl}{Pygsl}
\newcommand{\PYGSL}{Pygsl: Python wrapper of the GNU Scientific Library}

\title{pygsl Manual}

\author{Achim G\"adke\footnotemark[1] \\
Jochen K\"upper}

% Please at least include a long-lived email address;
% the rest is at your discretion.
\authoraddress{
        \footnotemark[1]Center for Applied Informatics, Cologne\\
        Email: \email{achimgaedke@users.sourceforge.net}\\
        Email: \email{jochen@jochen-kuepper.de}
}

\date{January, 2002}            % update before release!
                                % Use an explicit date so that reformatting
                                % doesn't cause a new date to be used.  Setting
                                % the date to \today can be used during draft
                                % stages to make it easier to handle versions.

\release{0.0.5}                 % release version; this is used to define the
                                % \version macro

\makeindex                      % tell \index to actually write the .idx file

\begin{document}

\maketitle

% This makes the contents more accessible from the front page of the HTML.
\ifhtml
\chapter*{Front Matter}
\label{front}
\fi

Copyright \copyright{} 2002 The pygsl Team.

Permission is granted to copy, distribute and/or modify this document under the
terms of the GNU Free Documentation License, Version 1.1 or any later version
published by the Free Software Foundation; with no Invariant Sections, no
Front-Cover Texts, and no Back-Cover Texts.  A copy of the license is included
in section \ref{cha:free-documentation-license} entitled ``GNU Free
Documentation License''.


%% Local Variables:
%% mode: LaTeX
%% mode: auto-fill
%% fill-column: 79
%% indent-tabs-mode: nil
%% ispell-dictionary: "american"
%% reftex-fref-is-default: nil
%% TeX-auto-save: t
%% TeX-command-default: "pdfeLaTeX"
%% TeX-master: "pygsl"
%% TeX-parse-self: t
%% End:


\begin{abstract}
   \noindent
   pygsl grants python users access to the GNU scientific library.
   More information about pygsl can be found at the project
   homepage,\url{http://pygsl.sf.net}.
\end{abstract}

\tableofcontents


\chapter[\protect\module{pygsl.histogram} --- Histogram Types]
{\protect\module{pygsl.histogram} \\ Histogram Types}
\label{cha:histogram-module}



%% Local Variables:
%% mode: LaTeX
%% mode: auto-fill
%% fill-column: 79
%% ispell-dictionary: "american"
%% reftex-fref-is-default: nil
%% TeX-auto-save: t
%% TeX-command-default: "pdfeLaTeX"
%% TeX-master: "pygsl"
%% TeX-parse-self: t
%% End:


\chapter[\protect\module{pygsl.sf} --- Special Functions]
{\protect\module{pygsl.sf} \\ Special Functions}
\label{cha:sf-module}
\declaremodule{extension}{pygsl.sf}
\moduleauthor{Achim G\"adke}{achimgaedke@users.sourceforge.net}

This chapter shows you the list of implemented special function and explains
details of error handling and return values.

\section{Function list}

\begin{longtableii}{l|l}{texttt}{Function}{Description}
\lineii{}{ToDo}
\end{longtableii}

\section{Return values}

\section{Error handling}


\chapter[\protect\module{pygsl.const} --- Mathematical and scientific
constants]{\protect\module{pygsl.const} \\ Mathematical and scientific
constants} 
 \label{cha:const-module}
 \declaremodule{extension}{pygsl.const}
\moduleauthor{Achim G\"adke}{achimgaedke@users.sourceforge.net}

In this module some usefull constants are defined.
There are four groups of constants:

\begin{itemize}
\item mathematical
\item physical in cgs unit system
\item physical in mks unit system
\item physical number constants (e.g. fine structure)
\end{itemize}

The other modules are created during the initialisation of \module{pygsl.const}.
The mathematical, physical mks constants and number constants are available in the namespace of \module{pygsl.const}, e.g.
\begin{verbatim}
import pygsl.const
import pygsl.const.cgs
print pygsl.const.cgs.speed_of_light/pygsl.const.speed_of_light
\end{verbatim}
Of course the result is 100.0.

\section[\protect\module{pygsl.const.math} --- Mathematical constants]
{\protect\module{pygsl.const.math} \\ Mathematical constants} 
\label{cha:const-math-module}

\section[\protect\module{pygsl.const.cgs} --- Scientific constants in cgs units]
{\protect\module{pygsl.const.cgs} \\ Scientific constants in cgs units} 
\label{cha:const-cgs-module}

\section[\protect\module{pygsl.const.mks} --- Scientific constants in mks units]
{\protect\module{pygsl.const.mks} \\ Scientific constants in mks units} 
\label{cha:const-mks-module}

\section[\protect\module{pygsl.const.num} --- Scientific number constants]
{\protect\module{pygsl.const.num} \\ Scientific number constants} 
\label{cha:const-num-module}




\chapter[\protect\module{pygsl.statistics} --- Statistics
functions]{\protect\module{pygsl.statistics} \\ Statistics functions}
\label{cha:statistics-module}
\declaremodule{extension}{pygsl.statistics}
\moduleauthor{Jochen K\"upper}{jochen@jochen-kuepper.de}
\index{mean}
\index{standard deviation}
\index{variance}
\index{estimated standard deviation}
\index{estimated variance}
\index{t-test}
\index{range}
\index{min}
\index{max}

This chapter describes the statistical functions in the library.  The
basic statistical functions include routines to compute the mean,
variance and standard deviation. More advanced functions allow you to
calculate absolute deviations, skewness, and kurtosis as well as the
median and arbitrary percentiles.  The algorithms use recurrence
relations to compute average quantities in a stable way, without large
intermediate values that might overflow. 

All functions work on any Python sequence (of the appropriate
data-type), but see section \ref{sec:stat-speed-considerations} for
advantages and drawbacks of different kinds of input data.


\section{Organization of the module}
\label{sec:stat-organization}

The parts of the GSL functions names, providing artificial name spaces,
are mapped to modules and submodules in pygsl.  That is
\code{gsl_stats_mean} can be found as \code{pygsl.statistics.mean} and
\code{gsl_stats_long_mean} as \code{pygsl.statistics.long.mean}.

The functions in the module are available in versions for datasets in
the standard floating-point and integer types. The generic versions
available in the \code{pygsl.statistics} module are using the generic
GSL \code{double} versions.  The submodules use GSL functions according
to the submodule name, e.g. long for \code{pygsl.statistics.long}.

Currently implemented submodules are \code{pygsl.statistics.double} and
\code{pygsl.statistics.long}.



\section{Speed considerations}
\label{sec:stat-speed-considerations}

All functions work on any Python sequence type but are optimized for use
with NumPy arrays. It is strongly suggested that you install NumPy
(available from \url{http://www.numpy.org})!

If you pass NumPy arrays of the \emph{correct data-type} as input data
to any of the functions they are passed straight to the C functions
along with the stride information of the data.

If you pass generic (non-NumPy) Python sequences or NumPy arrays of the
wrong data-type a suitable copy of the data will be created and passed
to the function.


\section{Further Reading}
\label{sec:stat-further-reading}

See the gsl reference manual for a description of all available
functions and the calculations they perform.


%% Local Variables:
%% mode: LaTeX
%% mode: auto-fill
%% fill-column: 79
%% ispell-dictionary: "american"
%% reftex-fref-is-default: nil
%% TeX-auto-save: t
%% TeX-command-default: "pdfeLaTeX"
%% TeX-master: "pygsl"
%% TeX-parse-self: t
%% End:


\chapter[\protect\module{pygsl.rng} --- Random Number generator
Module]{\protect\module{pygsl.rng} \\ Random Number generator Module}
\label{cha:rng-module}
\declaremodule{standard}{pygsl.rng}
\moduleauthor{Achim G\"adke}{achimgaedke@users.sourceforge.net}

This chapter introduces the random number generator classes provided by \module{pygsl}.

\section{Random Number Generators}

Each random number generator is a derived sperate class, that returns
a pseudo random number sequence. Methods of the common base class \class{rng}
provide the transformation to different probability distributions and
give access to basic properties of random number generators.
\begin{classdesc}{rng}{\texttt{string} typenamme \code{|} \class{rng} r}
This base class can be instantiated by a name string of the desired generator
\begin{verbatim}
import pygsl.rng
my_ran0=pygsl.rng.rng("ran0")
\end{verbatim}
or a clone of an existing generator can be created by:
\begin{verbatim}
clone_ran0=pygsl.rng.rng(my_ran0)
\end{verbatim}
\end{classdesc}
The type of the allocated generator is given by the method
\begin{methoddesc}{name}{}
which returns its name as string.
\end{methoddesc}
All generators can be seeded with
\begin{methoddesc}{set}{seed}
which sets the internal seed according to the positive integer {\tt seed}. Zero as seed
has a special meaning, please read details in the gsl reference.
\end{methoddesc}
The basic returned number type is integer, these are generated by
\begin{methoddesc}{get}{}
which returns the next number of the pseudo random sequence.
\end{methoddesc}
Basic information about these numbers can be obtained by
\begin{methoddesc}{max}{}
maximum number of this sequence and
\end{methoddesc}
\begin{methoddesc}{min}{}
minimum number of this sequence.
\end{methoddesc}
Implemented uniform probability densities are:
\begin{methoddesc}{uniform}{}
returns a real number between $[0,1)$.
\end{methoddesc}
\begin{methoddesc}{uniform_pos}{}
returns a real number between $(0,1)$ --- this excludes 0.
\end{methoddesc}
\begin{methoddesc}{uniform_int}{upper limit}
returns an integer from 0 to the upper limit (exclusive). If this limit is larger than the
number of return values of the underlying generator, \exception{pygsl.gsl_Error} is raised.
\end{methoddesc}
Furthermore a lot of derived probability densities can be used:
\begin{methoddesc}{gaussian}{}
\end{methoddesc}
\begin{methoddesc}{gaussian\_ratio\_method}{}
\end{methoddesc}
\begin{methoddesc}{ugaussian}{}
\end{methoddesc}
\begin{methoddesc}{ugaussian\_ratio\_method}{}
\end{methoddesc}
\begin{methoddesc}{gaussian\_tail}{}
\end{methoddesc}
\begin{methoddesc}{ugaussian\_tail}{}
\end{methoddesc}
\begin{methoddesc}{bivariate\_gaussian}{}
\end{methoddesc}
\begin{methoddesc}{exponential}{}
\end{methoddesc}
\begin{methoddesc}{laplace}{}
\end{methoddesc}
\begin{methoddesc}{exppow}{}
\end{methoddesc}
\begin{methoddesc}{cauchy}{}
\end{methoddesc}
\begin{methoddesc}{rayleigh}{}
\end{methoddesc}
\begin{methoddesc}{rayleigh\_tail}{}
\end{methoddesc}
\begin{methoddesc}{levy}{}
\end{methoddesc}
\begin{methoddesc}{gamma}{}
\end{methoddesc}
\begin{methoddesc}{gamma\_int}{}
\end{methoddesc}
\begin{methoddesc}{flat}{}
\end{methoddesc}
\begin{methoddesc}{lognormal}{}
\end{methoddesc}
\begin{methoddesc}{chisq}{}
\end{methoddesc}
\begin{methoddesc}{fdist}{}
\end{methoddesc}
\begin{methoddesc}{tdist}{}
\end{methoddesc}
\begin{methoddesc}{beta}{}
\end{methoddesc}
\begin{methoddesc}{logistic}{}
\end{methoddesc}
\begin{methoddesc}{pareto}{}
\end{methoddesc}
\begin{methoddesc}{dir\_2d}{}
\end{methoddesc}
\begin{methoddesc}{dir\_2d\_trig\_method}{}
\end{methoddesc}
\begin{methoddesc}{dir\_3d}{}
\end{methoddesc}
\begin{methoddesc}{dir\_nd}{}
\end{methoddesc}
\begin{methoddesc}{weibull}{}
\end{methoddesc}
\begin{methoddesc}{gumbel1}{}
\end{methoddesc}
\begin{methoddesc}{gumbel2}{}
\end{methoddesc}
\begin{methoddesc}{poisson}{}
\end{methoddesc}
\begin{methoddesc}{bernoulli}{}
\end{methoddesc}
\begin{methoddesc}{binomial}{}
\end{methoddesc}
\begin{methoddesc}{negative\_binomial}{}
\end{methoddesc}
\begin{methoddesc}{pascal}{}
\end{methoddesc}
\begin{methoddesc}{geometric}{}
\end{methoddesc}
\begin{methoddesc}{hypergeometric}{}
\end{methoddesc}
\begin{methoddesc}{logarithmic}{}
\end{methoddesc}
\begin{methoddesc}{landau}{}
\end{methoddesc}
\begin{methoddesc}{erlang}{}
\end{methoddesc}


The different generator classes are created according to the output of \code{gsl_rng_types_setup()}
when the \module{pygsl.rng} is loaded. Here is the list of children from \class{rng} for gsl-1.2:
\newline
\class{rng_borosh13},
\class{rng_coveyou},
\class{rng_cmrg},
\class{rng_fishman18},
\class{rng_fishman20},
\class{rng_fishman2x},
\class{rng_gfsr4},
\class{rng_knuthran},
\class{rng_knuthran2},
\class{rng_lecuyer21},
\class{rng_minstd},
\class{rng_mrg},
\class{rng_mt19937},
\class{rng_mt19937_1999},
\class{rng_mt19937_1998},
\class{rng_r250},
\class{rng_ran0},
\class{rng_ran1},
\class{rng_ran2},
\class{rng_ran3},
\class{rng_rand},
\class{rng_rand48},
\class{rng_random128_bsd},
\class{rng_random128_glibc2},
\class{rng_random128_libc5},
\class{rng_random256_bsd},
\class{rng_random256_glibc2},
\class{rng_random256_libc5},
\class{rng_random32_bsd},
\class{rng_random32_glibc2},
\class{rng_random32_libc5},
\class{rng_random64_bsd},
\class{rng_random64_glibc2},
\class{rng_random64_libc5},
\class{rng_random8_bsd},
\class{rng_random8_glibc2},
\class{rng_random8_libc5},
\class{rng_random_bsd},
\class{rng_random_glibc2},
\class{rng_random_libc5},
\class{rng_randu},
\class{rng_ranf},
\class{rng_ranlux},
\class{rng_ranlux389},
\class{rng_ranlxd1},
\class{rng_ranlxd2},
\class{rng_ranlxs0},
\class{rng_ranlxs1},
\class{rng_ranlxs2},
\class{rng_ranmar},
\class{rng_slatec},
\class{rng_taus},
\class{rng_taus2},
\class{rng_taus113},
\class{rng_transputer},
\class{rng_tt800},
\class{rng_uni},
\class{rng_uni32},
\class{rng_vax},
\class{rng_waterman14}, and
\class{rng_zuf}.
\newline
The default generator of \class{rng} is determined by the environment
variable \envvar{GSL_RNG_TYPE} or defaults to {\tt rng_mt19937}.

\section{Probability Density Functions}


\section{Using probability densities with random number generators}


%% Local Variables:
%% mode: LaTeX
%% mode: auto-fill
%% fill-column: 90
%% indent-tabs-mode: nil
%% ispell-dictionary: "american"
%% reftex-fref-is-default: nil
%% TeX-auto-save: t
%% TeX-command-default: "pdfeLaTeX"
%% TeX-master: "pygsl"
%% TeX-parse-self: t
%% End:


\chapter[\protect\module{pygsl.init} --- Library
initialisation]{\protect\module{pygsl.init} \\ Library initialisation}
\label{cha:library-initialisation}
\declaremodule{extension}{pygsl.init}
\moduleauthor{Pierre Schnizer}{schnizer@users.sourceforge.net}
\moduleauthor{Achim G\"adke}{achimgaedke@users.sourceforge.net}

This module is called the first time when loading \module{pygsl}.
All following procedures are called once and before everything other.

\section{Exception handling}
\index{exception handling!initialisation} GSL provides a selectable error
handler, that is called for occuring errors (like domain errors, division by
zero, etc. ).  \module{pygsl.init} installs a handler by calling
\cfunction{gsl_set_error_handler} to set an appropiate exception from
\module{pygsl.errors}.  A \module{pygsl} interface function should return
\code{NULL} in case of an error, so the exception is raised.  If this handler
is called more than once before returning to python, only the first set
exception is raised.

Here is a python level example:
\begin{verbatim}
import pygsl.histogram
import pygsl.errors
hist=pygsl.histogram.histogram2d(100,100)
try:
   hist[-1,-1]=0
except pygsl.errors.gsl_Error,err:
   print err
\end{verbatim}
Will result
\begin{verbatim}
input domain error: index i lies outside valid range of 0 .. nx - 1
\end{verbatim}

\section{IEEE-mode}
\index{ieee-mode!initialisation}
The IEEE mode is set from the environment variable
 \envvar{GSL_IEEE_MODE} via \cfunction{gsl_ieee_env_setup()}.
After the initialisation use \module{pygsl.ieee} for manipulation.

\section{random number generators}
\index{random number generator!initialisation}
Also the random number generator can be initialised from the environment variables
 \envvar{GSL_RNG_TYPE}
and \envvar{GSL_RNG_SEED} using the gsl function \cfunction{gsl_rng_env_setup()}.
Each random number generators are initialised with \envvar{GSL_RNG_SEED}.

The default generator can be created by:\nopagebreak
\begin{verbatim}
import pygsl.rng
my_rng=pygsl.rng.rng()
print my_rng.name()
\end{verbatim}


\chapter[\protect\module{pygsl.errors} --- Error and warning
classes]{\protect\module{pygsl.errors} \\ Error and warning classes} 
\label{cha:error-module}
\declaremodule{standard}{pygsl.errors}
\moduleauthor{Pierre Schnizer}{schnizer@users.sourceforge.net}
\moduleauthor{Original Author: Achim G\"adke}{achimgaedke@users.sourceforge.net}

This chapter provides information about the \exception{gsl_Error} exception class that comes with this module.

\section{Exception Classes}


\begin{excclassdesc} {gsl_Error}{}
derived from \exception{Exception}, can be constructed with any object as parameter.
It is baseclass to all other \gsl{} Exceptions
\end{excclassdesc}
These classes are translations of the \file{<gsl/gsl_errno.h>} to python
exceptions.


\begin{excclassdesc}{gsl_ArithmeticError}{}
derived from \exception{gsl_Error} and \exception{exceptions.ArithmeticError},
base of all common arithmetic exceptions
\end{excclassdesc}

\begin{excclassdesc}{gsl_OverflowError}{}
derived from \exception{gsl_Error} and \exception{exceptions.OverflowError}
\end{excclassdesc}

\begin{excclassdesc}{gsl_ZeroDivisionError}{}
derived from \exception{gsl_Error} and \exception{exceptions.ZeroDivisionError}
\end{excclassdesc}

\begin{excclassdesc}{gsl_FloatingPointError}{}
derived from \exception{gsl_Error} and \exception{exceptions.FloatingPointError}
\end{excclassdesc}

\begin{excclassdesc}{gsl_ArithmeticError}{}
is derived from  \exception{gsl_Error} and from  \exception{ArithmeticError} .
This exception is the    base of all common arithmetic exceptions.
\end{excclassdesc}

\begin{excclassdesc}{gsl_AccuracyLossError}{}
is derived from  \exception{gsl_ArithmeticError} .
This exception is raised if the failed to reach the specified tolerance.
\end{excclassdesc}
\begin{excclassdesc}{gsl_BadFuncError}{}
is derived from  \exception{gsl_Error} .
This exception is raised if problem with a user-supplied function occur.
\end{excclassdesc}
\begin{excclassdesc}{gsl_BadLength}{}
is derived from  \exception{gsl_Error} .
This exception is raised if  matrix or  vector lengths are not conformant.
\end{excclassdesc}
\begin{excclassdesc}{gsl_BadToleranceError}{}
is derived from  \exception{gsl_Error} .
This exception is raised if user specified an tolerance which can not be reached.
\end{excclassdesc}
\begin{excclassdesc}{gsl_CacheLimitError}{}
is derived from  \exception{gsl_Error} .
This exception is raised if the    cache limit is exceeded.
\end{excclassdesc}
\begin{excclassdesc}{gsl_DivergeError}{}
is derived from  \exception{gsl_ArithmeticError} .
This exception is raised if an   integral or series is divergent.
\end{excclassdesc}
\begin{excclassdesc}{gsl_DomainError}{}
is derived from  \exception{gsl_Error} .
This exception is raised if    domain errors occure. e.g. sqrt(-1).
\end{excclassdesc}
\begin{excclassdesc}{gsl_EOFError}{}
is derived from  \exception{gsl_Error} and from  \exception{EOFError} .
This exception is raised if 
    end of file
     .
\end{excclassdesc}
\begin{excclassdesc}{gsl_FactorizationError}{}
is derived from  \exception{gsl_Error} .
This exception is raised if     factorization failed.
\end{excclassdesc}
\begin{excclassdesc}{gsl_FloatingPointError}{}
is derived from  \exception{gsl_Error} and from  \exception{FloatingPointError} .
\end{excclassdesc}
\begin{excclassdesc}{gsl_GenericError}{}
is derived from  \exception{gsl_Error} .
\end{excclassdesc}
\begin{excclassdesc}{gsl_InvalidArgumentError}{}
is derived from  \exception{gsl_Error} .
This exception is raised if an invalid argument is supplied by the user.
\end{excclassdesc}
\begin{excclassdesc}{gsl_JacobianEvaluationError}{}
is derived from  \exception{gsl_ArithmeticError} .
This exception is raised if jacobian evaluations are not improving the solution.
\end{excclassdesc}
\begin{excclassdesc}{gsl_MatrixNotSquare}{}
is derived from  \exception{gsl_Error} .
This exception is raised if the given matrix is not square.
\end{excclassdesc}
\begin{excclassdesc}{gsl_MaximumIterationError}{}
is derived from  \exception{gsl_ArithmeticError} .
This exception is raised if    the maximum number  of iterations is exceeded.
\end{excclassdesc}
\begin{excclassdesc}{gsl_NoHardwareSupportError}{}
is derived from  \exception{gsl_Error} .
This exception is raised if the requested feature is not supported by the hardware.
\end{excclassdesc}
\begin{excclassdesc}{gsl_NoProgressError}{}
is derived from  \exception{gsl_ArithmeticError} .
This exception is raised if the  iteration is not making progress towards solution.
\end{excclassdesc}
\begin{excclassdesc}{gsl_NotImplementedError}{}
is derived from  \exception{gsl_Error} and from  \exception{NotImplementedError} .
This exception is raised if  a requested feature is not (yet) implemented .
\end{excclassdesc}
\begin{excclassdesc}{gsl_OverflowError}{}
is derived from  \exception{gsl_Error} and from  \exception{OverflowError} .
\end{excclassdesc}
\begin{excclassdesc}{gsl_PointerError}{}
is derived from  \exception{gsl_Error} .
This exception is raised if an invalid pointer is found by the C wrapper code
or by the GSL library.
\end{excclassdesc}
\begin{excclassdesc}{gsl_RangeError}{}
is derived from  \exception{gsl_ArithmeticError} .
This exception is raised if     output would be out or range, e.g. exp(1e100)
     .
\end{excclassdesc}
\begin{excclassdesc}{gsl_RoundOffError}{}
is derived from  \exception{gsl_ArithmeticError} .
This exception is raised if  arithmetic failed because of roundoff error.
\end{excclassdesc}
\begin{excclassdesc}{gsl_RunAwayError}{}
is derived from  \exception{gsl_ArithmeticError} .
This exception is raised if   iterative process is out of control.
\end{excclassdesc}
\begin{excclassdesc}{gsl_SanityCheckError}{}
is derived from  \exception{gsl_Error} .
This exception is raised if a sanity check failed - shouldn't happen.
\end{excclassdesc}
\begin{excclassdesc}{gsl_SingularityError}{}
is derived from  \exception{gsl_ArithmeticError} .
This exception is raised if  an   apparent singularity is detected.
\end{excclassdesc}
\begin{excclassdesc}{gsl_TableLimitError}{}
is derived from  \exception{gsl_Error} .
This exception is raised if the table limit is exceeded.
\end{excclassdesc}
\begin{excclassdesc}{gsl_ToleranceError}{}
is derived from  \exception{gsl_ArithmeticError} .
This exception is raised if  the alghorithm failed to reach the specified tolerance.
\end{excclassdesc}
\begin{excclassdesc}{gsl_ToleranceFError}{}
is derived from  \exception{gsl_ArithmeticError} .
This exception is raised if  the alghorithm cannot reach the specified
tolerance in F (typically the variation of the evaluated function).
\end{excclassdesc}
\begin{excclassdesc}{gsl_ToleranceGradientError}{}
is derived from  \exception{gsl_ArithmeticError} .
This exception is raised if  cannot reach the specified tolerance for the gradient.
\end{excclassdesc}
\begin{excclassdesc}{gsl_ToleranceXError}{}
is derived from  \exception{gsl_ArithmeticError} .
This exception is raised if cannot reach the specified tolerance in X
(typically a search result).
\end{excclassdesc}
\begin{excclassdesc}{gsl_UnderflowError}{}
is derived from  \exception{gsl_Error} and from  \exception{OverflowError} .
\end{excclassdesc}
\begin{excclassdesc}{gsl_ZeroDivisionError}{}
is derived from  \exception{gsl_Error} and from  \exception{ZeroDivisionError} .
\end{excclassdesc}

All the above errors are just translations of the errno to python exceptions.

The following two are specific to pygsl:
\begin{excclassdesc}{pygsl.errors.pygsl_NotImplementedError}{}
is derived from  \exception{gsl_Error} and from  \exception{NotImplementedError} .
This exception is raised if a feature is requested but not
implemented. Currently only used if a module requests the debugging enviroment
of the init module, but the init module was not compiled with \code{\#define DEBUG=1}
\end{excclassdesc}
\begin{excclassdesc}{pygsl.errors.pygsl_StrideError}{}
is derived from  \exception{gsl_SanityCheckError} .
GSL uses as strides multiples of the basis type; for a vector or doubles, one
means from one double to the next. Numpy or numarray count the stride in
multiples of the size of a char. Therefore the stride has to be recalculated
before the approbriate \gsl{} function can be called. If that fails this
exception is raised.
\end{excclassdesc}

\section{Warning Classes}

\begin{excclassdesc} {gsl_Warning}{}
The dedicated warning class for \gsl{} has \exception{Warning} as base class.
\end{excclassdesc}

\begin{excclassdesc}{gsl_DomainWarning}{}
derived from \exception{gsl_Warning}, used by some \module{pygsl.histogram} functions
\end{excclassdesc}


% \appendix
% \chapter{...}
% My appendix.
% The \code{\e appendix} markup need not be repeated for additional
% appendices.

\input{\jobname.ind}                    % Index

\end{document}


%% Local Variables:
%% mode: LaTeX
%% mode: auto-fill
%% fill-column: 79
%% indent-tabs-mode: nil
%% ispell-dictionary: "american"
%% reftex-fref-is-default: nil
%% TeX-auto-save: t
%% TeX-command-default: "pdfeLaTeX"
%% TeX-parse-self: t
%% End:
