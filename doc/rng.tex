\declaremodule{standard}{pygsl.rng}
\moduleauthor{Achim G\"adke}{achimgaedke@users.sourceforge.net}

This chapter introduces the random number generator classes provided by \module{pygsl}.

\section{Random Number Generators}

Each random number generator is a derived sperate class, that returns
a pseudo random number sequence. Methods of the common base class \class{rng}
provide the transformation to different probability distributions and
give access to basic properties of random number generators.
\begin{classdesc}{rng}{\texttt{string} typenamme \code{|} \class{rng} r}
This base class can be instantiated by a name string of the desired generator
\begin{verbatim}
import pygsl.rng
my_ran0=pygsl.rng.rng("ran0")
\end{verbatim}
or a clone of an existing generator can be created by:
\begin{verbatim}
clone_ran0=pygsl.rng.rng(my_ran0)
\end{verbatim}
\end{classdesc}
The type of the allocated generator is given by the method
\begin{methoddesc}{name}{}
which returns its name as string.
\end{methoddesc}
All generators can be seeded with
\begin{methoddesc}{set}{seed}
which sets the internal seed according to the positive integer {\tt seed}. Zero as seed
has a special meaning, please read details in the gsl reference.
\end{methoddesc}
The basic returned number type is integer, these are generated by
\begin{methoddesc}{get}{}
which returns the next number of the pseudo random sequence.
\end{methoddesc}
Basic information about these numbers can be obtained by
\begin{methoddesc}{max}{}
maximum number of this sequence and
\end{methoddesc}
\begin{methoddesc}{min}{}
minimum number of this sequence.
\end{methoddesc}
Implemented uniform probability densities are:
\begin{methoddesc}{uniform}{}
returns a real number between $[0,1)$.
\end{methoddesc}
\begin{methoddesc}{uniform_pos}{}
returns a real number between $(0,1)$ --- this excludes 0.
\end{methoddesc}
\begin{methoddesc}{uniform_int}{upper limit}
returns an integer from 0 to the upper limit (exclusive). If this limit is larger than the
number of return values of the underlying generator, \exception{pygsl.gsl_Error} is raised.
\end{methoddesc}
Furthermore a lot of derived probability densities can be used:
\begin{methoddesc}{gaussian}{}
\end{methoddesc}
\begin{methoddesc}{gaussian\_ratio\_method}{}
\end{methoddesc}
\begin{methoddesc}{ugaussian}{}
\end{methoddesc}
\begin{methoddesc}{ugaussian\_ratio\_method}{}
\end{methoddesc}
\begin{methoddesc}{gaussian\_tail}{}
\end{methoddesc}
\begin{methoddesc}{ugaussian\_tail}{}
\end{methoddesc}
\begin{methoddesc}{bivariate\_gaussian}{}
\end{methoddesc}
\begin{methoddesc}{exponential}{}
\end{methoddesc}
\begin{methoddesc}{laplace}{}
\end{methoddesc}
\begin{methoddesc}{exppow}{}
\end{methoddesc}
\begin{methoddesc}{cauchy}{}
\end{methoddesc}
\begin{methoddesc}{rayleigh}{}
\end{methoddesc}
\begin{methoddesc}{rayleigh\_tail}{}
\end{methoddesc}
\begin{methoddesc}{levy}{}
\end{methoddesc}
\begin{methoddesc}{gamma}{}
\end{methoddesc}
\begin{methoddesc}{gamma\_int}{}
\end{methoddesc}
\begin{methoddesc}{flat}{}
\end{methoddesc}
\begin{methoddesc}{lognormal}{}
\end{methoddesc}
\begin{methoddesc}{chisq}{}
\end{methoddesc}
\begin{methoddesc}{fdist}{}
\end{methoddesc}
\begin{methoddesc}{tdist}{}
\end{methoddesc}
\begin{methoddesc}{beta}{}
\end{methoddesc}
\begin{methoddesc}{logistic}{}
\end{methoddesc}
\begin{methoddesc}{pareto}{}
\end{methoddesc}
\begin{methoddesc}{dir\_2d}{}
\end{methoddesc}
\begin{methoddesc}{dir\_2d\_trig\_method}{}
\end{methoddesc}
\begin{methoddesc}{dir\_3d}{}
\end{methoddesc}
\begin{methoddesc}{dir\_nd}{}
\end{methoddesc}
\begin{methoddesc}{weibull}{}
\end{methoddesc}
\begin{methoddesc}{gumbel1}{}
\end{methoddesc}
\begin{methoddesc}{gumbel2}{}
\end{methoddesc}
\begin{methoddesc}{poisson}{}
\end{methoddesc}
\begin{methoddesc}{bernoulli}{}
\end{methoddesc}
\begin{methoddesc}{binomial}{}
\end{methoddesc}
\begin{methoddesc}{negative\_binomial}{}
\end{methoddesc}
\begin{methoddesc}{pascal}{}
\end{methoddesc}
\begin{methoddesc}{geometric}{}
\end{methoddesc}
\begin{methoddesc}{hypergeometric}{}
\end{methoddesc}
\begin{methoddesc}{logarithmic}{}
\end{methoddesc}
\begin{methoddesc}{landau}{}
\end{methoddesc}
\begin{methoddesc}{erlang}{}
\end{methoddesc}


The different generator classes are created according to the output of \code{gsl_rng_types_setup()}
when the \module{pygsl.rng} is loaded. Here is the list of children from \class{rng} for gsl-1.2:
\newline
\class{rng_borosh13},
\class{rng_coveyou},
\class{rng_cmrg},
\class{rng_fishman18},
\class{rng_fishman20},
\class{rng_fishman2x},
\class{rng_gfsr4},
\class{rng_knuthran},
\class{rng_knuthran2},
\class{rng_lecuyer21},
\class{rng_minstd},
\class{rng_mrg},
\class{rng_mt19937},
\class{rng_mt19937_1999},
\class{rng_mt19937_1998},
\class{rng_r250},
\class{rng_ran0},
\class{rng_ran1},
\class{rng_ran2},
\class{rng_ran3},
\class{rng_rand},
\class{rng_rand48},
\class{rng_random128_bsd},
\class{rng_random128_glibc2},
\class{rng_random128_libc5},
\class{rng_random256_bsd},
\class{rng_random256_glibc2},
\class{rng_random256_libc5},
\class{rng_random32_bsd},
\class{rng_random32_glibc2},
\class{rng_random32_libc5},
\class{rng_random64_bsd},
\class{rng_random64_glibc2},
\class{rng_random64_libc5},
\class{rng_random8_bsd},
\class{rng_random8_glibc2},
\class{rng_random8_libc5},
\class{rng_random_bsd},
\class{rng_random_glibc2},
\class{rng_random_libc5},
\class{rng_randu},
\class{rng_ranf},
\class{rng_ranlux},
\class{rng_ranlux389},
\class{rng_ranlxd1},
\class{rng_ranlxd2},
\class{rng_ranlxs0},
\class{rng_ranlxs1},
\class{rng_ranlxs2},
\class{rng_ranmar},
\class{rng_slatec},
\class{rng_taus},
\class{rng_taus2},
\class{rng_taus113},
\class{rng_transputer},
\class{rng_tt800},
\class{rng_uni},
\class{rng_uni32},
\class{rng_vax},
\class{rng_waterman14}, and
\class{rng_zuf}.
\newline
The default generator of \class{rng} is determined by the environment
variable \envvar{GSL_RNG_TYPE} or defaults to {\tt rng_mt19937}.

\section{Probability Density Functions}


\section{Using probability densities with random number generators}


%% Local Variables:
%% mode: LaTeX
%% mode: auto-fill
%% fill-column: 90
%% indent-tabs-mode: nil
%% ispell-dictionary: "american"
%% reftex-fref-is-default: nil
%% TeX-auto-save: t
%% TeX-command-default: "pdfeLaTeX"
%% TeX-master: "pygsl"
%% TeX-parse-self: t
%% End:
