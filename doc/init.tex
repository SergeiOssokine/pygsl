\declaremodule{extension}{pygsl.init}
\moduleauthor{Achim G\"adke}{achimgaedke@users.sourceforge.net}

This module is called the first time when loading \module{pygsl}.
All following procedures are called once and before everything other.

\section{Exception handling}
\index{exception handling!initialisation}
GSL provides a selectable error handler, that is called for occuring errors (like domain errors, division by zero, etc. ).
\module{pygsl.init} installs a handler by calling \cfunction{gsl_set_error_handler} to set an appropiate exception from \module{pygsl.errors}.
A \module{pygsl} interface function should return \code{NULL} in case of an error, so the exception is raised.
If this handler is called more than once before returning to python, only the first set exception is raised.

Here is a python level example:
\begin{verbatim}
import pygsl.histogram
import pygsl.errors
hist=pygsl.histogram.histogram2d(100,100)
try:
   hist[-1,-1]=0
except pygsl.errors.gsl_Error,err:
   print err
\end{verbatim}
Will result
\begin{verbatim}
input domain error: index i lies outside valid range of 0 .. nx - 1
\end{verbatim}

\section{IEEE-mode}
\index{ieee-mode!initialisation}
The IEEE mode is set from the environment variable \envvar{GSL_IEEE_MODE} via \cfunction{gsl_ieee_env_setup()}.
After the initialisation use \module{pygsl.ieee} for manipulation.

\section{random number generators}
\index{random number generator!initialisation}
Also the random number generator can be initialised from the environment variables \envvar{GSL_RNG_TYPE}
and \envvar{GSL_RNG_SEED} using the gsl function \cfunction{gsl_rng_env_setup()}.
Each random number generators are initialised with \envvar{GLS_RNG_SEED}.

The default generator can be created by:\nopagebreak
\begin{verbatim}
import pygsl.rng
my_rng=gsl_rng()
print my_rng.name()
\end{verbatim}
