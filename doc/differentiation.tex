\declaremodule{extension}{pygsl.diff}
\moduleauthor{Jochen K\"upper}{jochen@jochen-kuepper.de}
\modulesynopsis{Numerical Differentiation.}

This chapter describes the available functions for numerical differentiation.

The functions described in this chapter compute numerical derivatives by finite
differencing.  An adaptive algorithm is used to find the best choice of finite
difference and to estimate the error in the derivative.


\begin{funcdesc}{central}{func, x}
   This function computes the numerical derivative of the function \var{func}
   at the point \var{x} using an adaptive central difference algorithm.  A
   tuple \code{(result, error)} is returned with the derivative and its
   estimated absolute error.
\end{funcdesc}

\begin{funcdesc}{backward}{func, x}
   This function computes the numerical derivative of the function \var{func}
   at the point \var{x} using an adaptive backward difference algorithm.  The
   function \var{func} is evaluated only at points smaller than \var{x} and at
   \var{x} itself.  A tuple \code{(result, error)} is returned with the
   derivative and its estimated absolute error.
\end{funcdesc}

\begin{funcdesc}{forward}{func, x}
   This function computes the numerical derivative of the function \var{func}
   at the point \var{x} using an adaptive forward difference algorithm.  The
   function \var{func} is evaluated only at points greater than \var{x} and at
   \var{x} itself.  A tuple \code{(result, error)} is returned with the
   derivative and its estimated absolute error.
\end{funcdesc}


\begin{seealso}
   The algorithms used by these functions are described in the following book:
   \seetext{S.D.\ Conte and Carl de Boor, \emph{Elementary Numerical Analysis:
         An Algorithmic Approach}, McGraw-Hill, 1972.}
\end{seealso}



%% Local Variables:
%% mode: LaTeX
%% mode: auto-fill
%% fill-column: 79
%% indent-tabs-mode: nil
%% ispell-dictionary: "american"
%% reftex-fref-is-default: nil
%% TeX-auto-save: t
%% TeX-command-default: "pdfeLaTeX"
%% TeX-master: "pygsl"
%% TeX-parse-self: t
%% End:
