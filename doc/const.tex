\declaremodule{extension}{pygsl.const}
\moduleauthor{Achim G\"adke}{achimgaedke@users.sourceforge.net}

In this module some usefull constants are defined.
There are four groups of constants:

\begin{itemize}
\item mathematical
\item physical in cgs unit system
\item physical in mks unit system
\item physical number constants (e.g. fine structure)
\end{itemize}

The other modules are created during the initialisation of \module{pygsl.const}.
The mathematical, physical mks constants and number constants are available in the namespace of \module{pygsl.const}, e.g.
\begin{verbatim}
import pygsl.const
import pygsl.const.cgs
print pygsl.const.cgs.speed_of_light/pygsl.const.speed_of_light
\end{verbatim}
Of course the result is 100.0.

\section[\protect\module{pygsl.const.math} --- Mathematical constants]
{\protect\module{pygsl.const.math} \\ Mathematical constants} 
\label{cha:const-math-module}

\section[\protect\module{pygsl.const.cgs} --- Scientific constants in cgs units]
{\protect\module{pygsl.const.cgs} \\ Scientific constants in cgs units} 
\label{cha:const-cgs-module}

\section[\protect\module{pygsl.const.mks} --- Scientific constants in mks units]
{\protect\module{pygsl.const.mks} \\ Scientific constants in mks units} 
\label{cha:const-mks-module}

\section[\protect\module{pygsl.const.num} --- Scientific number constants]
{\protect\module{pygsl.const.num} \\ Scientific number constants} 
\label{cha:const-num-module}

