\section{Status}

\paragraph*{Status of GSL-Library}
The gsl-library is since version 1.0 stable and for general use.
More information about it at \url{http://sources.redhat.com/gsl/}.

\paragraph*{Status of this interface}
We are collecting implementations of parts of gsl.
So the interface is not complete.
The version 0.0.5 is a developement snapshot for \file{pygsl-0.1}.

\paragraph*{Retriving the Interface}
You can download it here: \url{http://sourceforge.net/projects/pygsl}

\section{Requirements}

To build the interface, you will need
\begin{itemize}
\item \ulink{gsl-1.x}{http://sources.redhat.com/gsl},
\item \ulink{python2.2}{http://www.python.org} or better,
\item \ulink{NumPy}{http://numpy.sf.net}, and
\item a c compiler (like \ulink{gcc}{http://gcc.gnu.org}).
\end{itemize}

Supported Platforms are:
\begin{itemize}
\item Linux (Redhat 7.*/SuSE) with python2.* and gsl-1.*
\item SUN
\item cygnwin
\end{itemize}

\section{Installing GSL interface}

\program{gsl-config} must be on your path:\nopagebreak
\begin{verbatim}
# unpack the source distribution
gzip -d -c pygsl-x.y.z.tar.gz|tar xvf-
cd pygsl-x.y.z
# do this with your prefered python version
# to set the gsl location explicitly use setup.py --gsl-prefix=/path/to/gsl
python setup.py build
# change to an user id, that is allowed to do installation
python setup.py install
\end{verbatim}
Ready....

{\bf Do not test the interface in the distribution root or in the directories \file{src} or \file{pygsl}.}

\paragraph*{Uninstall GSL interface}
\code{rm -r }"python install path"\code{/lib/python}"version"\code{/site-packages/pygsl}

\paragraph*{Testing}
the directory \file{tests} will contains several testsuites, based on python \module{unittest}.

\paragraph*{Support}
Please send mails to our mailinglist at \email{pygsl-discuss@lists.sourceforge.net}.

\paragraph*{Developement}
You can browse our cvs tree at \url{http://cvs.sourceforge.net/cgi-bin/viewcvs.cgi/pygsl/pygsl/}.
\\
Type this to check out the actual version:
\begin{verbatim}
cvs -d:pserver:anonymous@cvs.pygsl.sourceforge.net:/cvsroot/pygsl login
#Hit return for no password.
cvs -z3 -d:pserver:anonymous@cvs.pygsl.sourceforge.net:/cvsroot/pygsl co pygsl
\end{verbatim}
The script \program{tools/extract_tool.py} generates most of the special function code.

\paragraph*{ToDo}
Implement other parts:

\begin{itemize}
\item Chebyshev-Series
\item Complex Numbers
\item Matrix
\item Lin Algebra
\item Permutation
\item Sorting
\item Fitting/Minimization
\item FFT
\item Numerical Integration
\end{itemize}

\paragraph*{History}
\begin{itemize}
\item a gsl-interface for python was needed for a project at
\ulink{Center for Applied Informatics Cologne}{http://www.zaik.uni-koeln.de/AFS}.
\item \file{gsl-0.0.3} was released at May 23, 2001
\item \file{gsl-0.0.4} was released at January 8, 2002
\item \file{gsl-0.0.5} is growing since January, 2002
\end{itemize}

\paragraph*{Thanks}
Jochen K\"upper (\email{jochen@jochen-kuepper.de}) for \module{pygsl.statistics} part

\paragraph*{Maintainer}
Achim G\"adke (\email{AchimGaedke@users.sourceforge.net})
